\documentclass[10pt,letterpaper,landscape]{article}

\usepackage[margin=1in]{geometry}
\usepackage{termcal}
\usepackage{palatino}
\usepackage{fontawesome}
\usepackage{fancyhdr}
\usepackage{titling}
\usepackage{lastpage}
\usepackage{datatool}
\usepackage{ifthen}
\usepackage{enumitem}

\setlist{nosep,wide=0.5em}
\newenvironment{details}{\begin{footnotesize}\begin{itemize}}{\end{itemize}\end{footnotesize}}

\newcounter{examnumber}
\newcounter{quiznumber}
\newcounter{labnumber}
\newcounter{labdaynumber}
\newcounter{weekcounter}
\newcommand{\exam}{%
	%\ifthenelse{%
	%	\isodd{%
	%		\theweekcounter%
	%	}%
	%}%
	%{}%
	%{%
	%	\stepcounter{examnumber}\faShield~\textbf{Quest \arabic{examnumber}}
	%}%
	\stepcounter{examnumber}\textbf{\faEdit~Exam~\arabic{examnumber}}%
}
\newcounter{homeworknumber}
\newcommand{\hw}{%
	\ifthenelse{\isodd{\theweekcounter}}{}{\stepcounter{homeworknumber} \faCalculator~\textbf{HW Set \arabic{homeworknumber} Due}}
}

\title{Chemistry for the Sciences 1 Semester Schedule}
\newcommand{\semester}{FA21}
\author{McCurry}

% Turn off 1st, 2nd, 3rd, etc.
\renewcommand{\calprintdate}{%
	\ifnewmonth\framebox{\monthname\ \arabic{date}}%
	\else \arabic{date}\fi
}

\lhead{\sffamily \thetitle}
\chead{}
\rhead{\sffamily \semester\ --- \theauthor}
\lfoot{\sffamily\footnotesize\em Exam dates are set. All other material subject
to change.}
\cfoot{}
\rfoot{\sffamily\thepage\ of \pageref{LastPage}}
\renewcommand\headrulewidth{0.4pt}
\renewcommand\footrulewidth{0.4pt}
\pagestyle{fancy}

\begin{document}

\begin{calendar}{8/23/21}{16}
	%\setlength{\calboxdepth}{4em}

	% Description of the Week
	\calday[Monday]{\classday\stepcounter{weekcounter}}
	\skipday
	%\calday[Tuesday]{\noclassday}%\weeklytext{\stepcounter{labnumber}
	%\faFlask{} Expt~\arabic{labnumber} \\[1em] }}
	%\calday[Tuesday]{\noclassday}
	\calday[Wednesday]{\classday\weeklytext{\hw}}
	\skipday
	%\calday[Thursday]{\noclassday}
	\calday[Friday]{\classday}%\weeklytext{\exam}}
	%\weeklytext{\stepcounter{quiznumber}
	%\emph{Quiz~\arabic{quiznumber}}}}
	\skipday\skipday

	% Holidays
	\options{9/6/21}{\noclassday\weeklytext{}} % Labor Day
	\caltext{9/6/21}{\em Labor Day --- No Class}
	\options{11/24/21}{\noclassday\weeklytext{}\stepcounter{weekcounter}} % Thanksgiving
	\caltext{11/24/21}{\em Thanksgiving Recess --- No Class}
	\options{11/26/21}{\noclassday\weeklytext{}} % Thanksgiving
	\caltext{11/26/21}{\em Thanksgiving Recess --- No Class}

	% Final Exam
	\options{12/6/21}{\noclassday\weeklytext{}}
	\options{12/8/21}{\noclassday\weeklytext{}}
	\options{12/10/21}{\noclassday\weeklytext{}}
	\caltext{12/8/21}{\faEdit~\textbf{Final Exam} \\
		\hspace{1em} 12:30 -- 2:30 pm}
	%\caltext{4/28/20}{\faFlask~\textbf{Lab Final}}

	% Topics
	\caltexton{1}{Essentials \begin{details} \item Scientific Notation \end{details}}
	\caltextnext{Essentials \begin{details} \item Fundamental Units \item Derived Units \end{details}}
	\caltextnext{Essentials \begin{details} \item Reliability of Measurements \item Problem Solving \end{details}}
	\caltextnext{Atoms \begin{details} \item Substances and Compounds \item Atoms: The Building Blocks of Matter \end{details}}
	\caltextnext{Atoms \begin{details} \item Describing Atoms \end{details}}
	\caltextnext{Atoms \begin{details} \item The Mole Concept \end{details}}
	\caltextnext{Quantum Mechanics \begin{details} \item The Nature of
	Light \end{details}}
	\caltextnext{Quantum Mechanics \begin{details} \item The Nature of Light
	(cont'd) \item The Quantum Nature of the Atom \end{details}}
	\caltextnext{Quantum Mechanics \begin{details} \item The Quantum Nature
	of the Atom (cont'd) \item Uncertainty \end{details}}
	\caltextnext{Quantum Mechanics \begin{details} \item Uncertainty
(cont'd) \end{details}}%\\[1em] Periodic Properties}
	\caltextnext{\exam}
	\caltextnext{Elements and Periodic Properties \begin{details} \item
	Organizing and Classifying Elements \end{details}}
	\caltextnext{Elements and Periodic Properties \begin{details} \item
		Higher-Level Electron Configuations \end{details}}
		\caltextnext{Elements and Periodic Properties \begin{details}
	\item Periodic Trends \end{details}}
	\caltextnext{Elements and Periodic Properties \begin{details}
	\item Periodic Trends \end{details}
	Molecules \& Compounds
\begin{details} \item Concept of Bonding \item Representing Ionic Compounds
\end{details}}
\caltextnext{Molecules \& Compounds \begin{details} \item Representing Ionic
Compounds \end{details}}
\caltextnext{Molecules \& Compounds \begin{details} \item Representing Covalent
Compounds \end{details}}
\caltextnext{Molecules \& Compounds \begin{details} \item Representing Covalent
Compounds \end{details}}% \\[1em] \hw}
\caltextnext{Molecules \& Compounds \begin{details} \item Compound Stoichiometry
\end{details}}
\caltextnext{Molecules \& Compounds \begin{details} \item Compound Stoichiometry
\end{details}}
\caltextnext{Chemical Bonding I \begin{details} \item Polarity and Formal Charge
\end{details}}
\caltextnext{Chemical Bonding I \begin{details} \item Resonance and Exceptions to
the Octet Rule \item Covalent Bond Strength \end{details}}
	\caltextnext{\exam}
\caltextnext{Chemical Bonding I \begin{details} \item VSEPR \end{details}}
%	\caltextnext{Chemical Bonding}
	\caltextnext{Chemical Bonding II}
	\caltextnext{Chemical Bonding II}
	\caltextnext{Chemical Bonding II}
	\caltextnext{Chemical Reactions}% \\[1em] \hw}
	\caltextnext{Chemical Reactions}
	\caltextnext{Chemical Reactions}
	\caltextnext{Chemical Reactions}
	\caltextnext{Aqueous Reactions}
	%\caltextnext{\exam}
	\caltextnext{Aqueous Reactions}
\caltextnext{Aqueous Reactions}% \\[1em] \hw}
	\caltextnext{\exam}
	\caltextnext{Aqueous Reactions}
	\caltextnext{Thermo}
	\caltextnext{Thermo}
	\caltextnext{Thermo}
	\caltextnext{Thermo}
\caltextnext{Gas Laws}% \\[1em] \hw}
	\caltextnext{Gas Laws}
%	\caltextnext{\exam}
	\caltextnext{Gas Laws}
	\caltextnext{Gas Laws}
%	\caltextnext{Intermolecular Forces}
%	\caltextnext{Intermolecular Forces}% \\[1em] \hw}
%	\caltextnext{Intermolecular Forces}

\end{calendar}

\end{document}
