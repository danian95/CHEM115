% !TEX program = xelatex
\documentclass[10pt,letterpaper]{article}

\usepackage[letterpaper,margin=0.5in]{geometry}
\usepackage{tabularx}
\usepackage{fontspec}
\usepackage[bf,small]{titlesec}
\usepackage[inline]{enumitem}
\usepackage[colorlinks,allcolors=black,urlcolor=blue]{hyperref}
\usepackage{framed}
\usepackage{fancyhdr}
\usepackage{lastpage}

\pagestyle{fancy}
\fancyhead[L,C,R]{}
\fancyfoot[L,C]{}
\fancyfoot[R]{Page \thepage{} of \pageref{LastPage}}
\renewcommand{\headrulewidth}{0pt}
\renewcommand{\footrulewidth}{0pt}


\setmainfont{TeX Gyre Heros}
\newfontfamily\copperplate{COPRGTB}[Extension = .ttf]

\urlstyle{same}

\begin{document}

\noindent
\begin{minipage}{0.45\linewidth}
	\copperplate\centering
	Department of Chemistry \\
	and Biochemistry \\
	Bloomsburg University
\end{minipage}
\hfill
\begin{minipage}{0.45\linewidth}
	\bfseries
	\fontsize{14}{16}\selectfont Policies for Laboratories \\
	\fontsize{12}{14}\selectfont Chemistry for the Sciences 2 \\
	\normalfont \fontsize{11}{11}\selectfont Spring 2021
\end{minipage}

\bigskip

\hrule

\bigskip

\noindent
\begin{tabularx}{\linewidth}{@{}>{\bfseries}lcr*{2}{>{\bfseries}lX}}
	Instructor: & \multicolumn{2}{l}{Dr.\ Daniel McCurry} & Section(s): & CHEM116-99A & Office: & HSC 240 \\
		    &      &               &             & CHEM116-99B &         & \\[0.5em]
		    Office hours: & M & 1:30 -- 2:30 p.m.  & Day/Time: & Thursdays & Phone: & (570) 389-5320 \\
				  & T & 8:30 -- 9:30 a.m.  &           & A: 2:00
		    -- 4:50 p.m. &  & \\
		      & & 12:00 -- 1:00 p.m. &           & B: 5:30 -- 8:20 p.m.  & Email: & dmccurry@bloomu.edu       \\
		      & W & 3:00 -- 4:00 p.m. \\
		      & R & 8:30 -- 9:30 a.m. \\
\end{tabularx}

\section*{Materials}
\begin{itemize}[noitemsep]
	\item Lab Manual: \textit{Chemistry for the Sciences 2 Laboratory
		Exercises, \underline{Spring 2021}} available on BOLT.
	\item  A calculator. \textit{Use of calculators in cell phones, PDA’s,
		iPods and similar devices will not be allowed.}
	\item A bound composition notebook to use as a laboratory notebook. (No spiral bindings)
	\item Safety glasses (with side shields, not screens) or goggles.
	\item An approved mask --- it must cover both your mouth and nose.
	\item A permanent marker for writing on glassware.
	\item A pen (No pencils or white-out allowed). Blue or black ink
		\emph{only}!
\end{itemize}

\section*{Communication}
\begin{itemize}[noitemsep]
	\item \textbf{BOLT} --- This is a web-based course communication system
		where instructors can easily post all sorts of things for you
		and communicate with you. Check BOLT regularly for course
		announcements.
	\item \textbf{Zoom} --- Please use the correct meeting ID for your
		section.
		\begin{enumerate}[label={Section \Alph*:},leftmargin=1.25in]
			\item \href{https://bloomu.zoom.us/j/91235148193}{912 3514 8193}
			\item \href{https://bloomu.zoom.us/j/99561171003}{995 6117 1003}
			\item[Office Hours:] Just type 
				\href{https://bloomu.zoom.us/my/dmccurry}{dmccurry}
				for the meeting ID.
		\end{enumerate}
\end{itemize}

\section*{Grading of Lab Materials}
The weighting of the particular lab materials is\ldots

{\renewcommand\arraystretch{1.5}
\begin{tabular} {lr<{\,\%}}
	Lab Reports: & 55 \\
	Participation/Attendance & 10 \\ 
	Recitation: & 30 \\
	Qualitative Equilibrium Quiz: & 5
\end{tabular}
}

\section*{Graded Events}
\begin{itemize}
	\item \textbf{Lab Safety Training} --- You will be
		required to complete the online Lab Safety Training module with
		a score of 80\% or better. The score does not affect your
		laboratory grade but is required to gain admittance into lab.
		Failure to complete the training video or to get the required
		score prevents you from entering lab until the training is
		successfully completed. 
	\item \textbf{Lab Reports} --- These can be found on BOLT and are
		typically due one week after the experiment was conducted,
		unless otherwise specified.  Reports are due by the beginning of
		the laboratory period, unless otherwise specified.  The
		instructor reserves the right to amend the due dates for reports
		based on external circumstances.  Your completed reports will be
		uploaded as .pdf files to the appropriate folder on BOLT. 
		\begin{itemize}
			\item \textbf{Late Penalty:} 25\% deduction per day.
				The deduction begins immediately after the
				instructor calls for the reports to be
				submitted.
		\end{itemize}
	\item \textbf{Lab Cleanliness and Safety} --- It is everyone's responsibility
		to keep the lab a safe working environment.  This entails proper
		extensive cleaning, decontamination, and putting away of all
		materials used during the laboratory session.  You must follow
		the instructor's guidelines for proper up-keep of the lab.
		Additionally, lab safety includes proper wearing of personal
		protective equipment (PPE).  During the entire laboratory
		session, the PPE you must properly wear are: a mask (covering
		both your mouth and nose), goggles, and gloves.  This is in
		addition to the proper lab ``uniform'' (see below).  Failure to
		comply with instructor’s guidelines for both lab cleanliness and
		wearing of PPE will result in a loss of points for each
		infraction.  The points deduced will be at the discretion of the
		instructor based on the severity of the infraction.   
	\item \textbf{Recitation} --- A recitation class is often a subset of a
		lecture-style course, involves a small class size, promotes
		interactive learning opportunities, and allows for personalized
		interactions with the professor. Recitation classes allow
		students to learn and review course material in a small group
		environment and in an in-depth manner. You'll have the
		opportunity to apply what you’ve learned to different scenarios
		and examine concepts in greater detail.  Due to the uncertain
		nature of the COVID-19 situation, recitation sessions will
		provide you clarity and extra practice on course concepts that
		you may find troubling.  Recitation ``problem sets'' will be
		assigned prior to the recitation meeting.  These problem sets
		must be completed and submitted to the appropriate folder on
		BOLT prior to the beginning of the lab period.  Late submissions
		of any sort will not be accepted. 
	\item \textbf{Qualitative Equilibrium Quiz} --- Since this lab does not
		have a post-lab associated with it, a quiz will serve as the
		assessment for the lab.  This quiz can be found on BOLT.  You
		may use your lab notebook for this quiz.  It is due by the start
		of your lab one week following the experiment. 
\end{itemize}

\section*{Lab Attire}
\begin{itemize}[noitemsep]
	\item \textbf{The official ``uniform'' of lab is a t-shirt,
		jeans/khakis, and tennis shoes}
		\begin{itemize}[nosep]
			\item A well-fitting sweatshirt is allowed.
			\item Exposed shoulders, mid-riff, and legs are
				\emph{NEVER} acceptable.
		\end{itemize}
	\item Safety glasses (with side shields, not screens) or goggles
	\item An approved mask --- it must cover both your mouth and nose
	\item Your lab instructor has the right to deny lab admittance for
		clothing or goggles/masks deemed ``unsafe'' for lab
\end{itemize}

\section*{Denial of Admission and/or Removal from Laboratory}
The lab instructor has the authority to deny admission and/or remove from the
lab to any student who
\begin{enumerate*}[label={(\arabic*)}]
	\item does not have the proper materials,
	\item is not properly clothed and equipped with safety
		glasses and an approved mask,
	\item has not properly prepared for lab as evidenced by
		incompletion of assignments,
	\item is in a condition that would pose a safety hazard
		to himself/herself or others in the lab, or 
	\item is not following appropriate safety protocols --
		these protocols include any and all of the
		safety regulations outlined in the lab manual
		or on the laboratory drawer check-in sheet with
		a particular emphasis on wearing safety goggles
		and gloves.
\end{enumerate*}
Failure to wear safety goggles and/or gloves when instructed
will result in your removal from lab.  Arrangements
to do a lab make-up are at the discretion of the instructor and
may not be granted.

\section*{Lab Attendance}
If you are absent with a legitimate excuse, i.e.
\begin{enumerate*}[label={(\arabic*)}]
	\item personal illness sufficiently severe that you
		cannot function, or illness of your child; 
	\item emergency in the immediate family, documented to
		the Vice President for Student Life office, or
	\item experiencing any health-related issues, especially the symptoms of
		COVID-19,
\end{enumerate*}
you must \textbf{notify your instructor within 24 hours} and
make it up at a mutually agreeable time. If absence is caused
by participation in a university sponsored activity, you must
make arrangements by the beginning of the week of the scheduled
lab.

\section*{Classroom and Laboratory Expectations of Student Conduct}
All students are expected to abide by the Laboratory Rules and Guidelines
published in the laboratory manual for the course. Laboratory instructors may
dismiss students from a laboratory period, without credit, for failure to
comply, especially with safety issues. 

Instructors may dismiss students from laboratory, without credit, if the
instructor deems that the student is unprepared for lab, as evidenced by
failure to complete assignments or other obvious unfamiliarity with the tasks
of the day.

All students are expected to read and abide by the Bloomsburg University
Student Disruptive Behavior Policy, PRP 3881, which can be found at
\url{http://www.bloomu.edu/policies_procedures/3881}. While all of the policy is
important and applicable, your attention is drawn in particular to the
following disruptive behaviors:
\begin{itemize}[noitemsep]
	\item Refusal to comply with reasonable instructor directions.
	\item Repeatedly arriving after class has begun or leaving class early.
	\item Distractive talking, including speaking out of turn or monopolizing discussion.
	\item Use of any electronic device not related to class during the class period.
	\item Activities not germane to the content and work of the class in
		session. Examples include activities such as reading the
		newspaper, doing homework for other classes, etc., that are not
		directly related to/appropriate for the class in session.
\end{itemize}
\ldots{} and these expectations:
\begin{itemize}
	\item Attending classes and paying attention. Students are responsible
		for any material presented in class. Students may expect the
		instructor to clarify material already taught but not to
		re-teach the material missed.
	\item Coming to class on time and staying until dismissed. If a student
		has to enter class late, he or she should do so in a manner so
		as not to disrupt the class. Students should not leave a class
		once it has begun unless it is absolutely necessary. This
		applies to testing situations as well, until the student has
		completed the test. 
	\item Respecting the right of others to speak uninterrupted.
	\item Turning off unnecessary electronic devices before class begins.
		Students should ask permission of the instructor for any
		electronic devices used in the classroom, except those
		medically necessary (such as hearing aids, etc.). 
	\item Focusing on class material during class time. Sleeping, talking
		to others, showing audible and visible signs of restlessness or
		boredom, doing work for another class, reading the newspaper,
		checking e-mail, and text messaging are unacceptable classroom
		behaviors. 
	\item Waiting until the instructor has dismissed class to pack class
		materials so as not to miss important closing information.
\end{itemize}

\section*{An important point about lab partners and plagiarism\ldots}
While it is understood that you may be working with a lab partner to conduct the
experiment and gather data, your calculations, explanations,
graphs/charts/figures/etc.\ must be your own work!  If you are working with a lab
partner, your explanations to questions cannot be copied word-for-word --- you
must answer questions in your own words.  A similar thing can be said for
calculations --- you must do them on your own.  Just because you may be allowed to
work with a lab partner does NOT mean you can submit identical work.  This is
plagiarism and will result in a grade of 0 (zero) for the assignment!
Additionally, you may not share graphs, figures, charts, etc.\ with another
person who did not participate in its creation and especially not with other
groups unless instructed by the lab instructor. See
\url{https://intranet.bloomu.edu/policies_procedures/3512} for more information. 

\vfill

\begin{framed}
	\bfseries\noindent
	Continuation in Chem 116 laboratory requires proper safety.  The
	Department of Chemistry and Biochemistry in association with the Dean
	of the College of Science and Technology and the Office of the Provost,
	requires that each semester all students must successfully complete (a
	score of 80\% or better) the ``Chemistry Lab Safety'' program on BOLT
	before second lab or you will not be admitted to lab.
\end{framed}

\end{document}
