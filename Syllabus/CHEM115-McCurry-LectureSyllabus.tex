% !TeX program = lualatex
\documentclass[11pt,letterpaper]{article}

\usepackage{tabularx}
\usepackage{booktabs}
\usepackage{bucolors}
\usepackage[colorlinks=true,allcolors=black,urlcolor=bugold]{hyperref}
\usepackage{mathtools}
\usepackage{mdframed}
\usepackage{titling}
\usepackage{newtxtext}
\usepackage{fancyhdr}
\usepackage{lastpage}
\usepackage[english]{babel}
\usepackage[sf,bf]{titlesec}
\usepackage{enumitem}
\usepackage[margin=1in,letterpaper]{geometry}
\usepackage{multicol}

\title{Chemistry for the Sciences 1}
\author{Dr.\ Daniel A.\ McCurry}
\date{Fall 2021}

\newcommand{\classnum}{CHEM115}

\pagestyle{fancy}
\lhead{}
\chead{}
\rhead{}
\lfoot{\footnotesize\sffamily McCurry --- \classnum\ --- FA21}
\cfoot{}
\rfoot{\footnotesize\sffamily\thepage~of~\pageref{LastPage}}
\renewcommand{\headrulewidth}{0pt}
\renewcommand{\footrulewidth}{0.4pt}

\pretitle{\noindent\color{bumaroon}
	\sffamily\bfseries\Large
	\classnum\newline
	\LARGE\expandafter\MakeUppercase\expandafter}
\posttitle{\par\medskip}
\preauthor{\noindent\sffamily}
\postauthor{ --- }
\predate{\sffamily}
\postdate{}

\setlength{\droptitle}{-2em}

\setcounter{secnumdepth}{0}
\setlist[description]{font=\sffamily\bfseries}
\urlstyle{same}

\begin{document}

\maketitle
\thispagestyle{fancy}

\noindent
\begin{tabularx}{\linewidth} {@{\qquad}>{\bfseries\sffamily}r
	>{\raggedright\arraybackslash}X@{\qquad}}
	\toprule
	Lecture: & MWF, 12:00 pm -- 12:50 pm \\
			    & HSC G42 \\ \\
        Instructor: & Dr.\ Daniel A. McCurry\\
		    & 	Assistant Professor of Chemistry\\
		    & 	HSC 240\\
		    & 	(570) 389-5320\\
		    & 	\href{mailto:dmccurry@bloomu.edu}{\nolinkurl{dmccurry@bloomu.edu}}\\
		    & 	\href{https://bloomu.starfishsolutions.com/starfish-ops/dl/instructor/serviceCatalog.html?bookmark=connection/20001}{HuskySuccess
		     	Profile} \\ \\
	Office Hours: & \begin{minipage}[t]{\linewidth}
		\begin{tabular}[t] {@{}lr@{\,--\,}l@{~@~}l}
			Mon./Tues.  & 4:00 & 6:00\,p.m. & Andruss Library
			(1\textsuperscript{st} floor) \\
			Fri. & 9:00 & 10:00\,a.m. & HSC 240\\
				\end{tabular}
			\end{minipage} \\
		      &   In-Person or via Zoom Meeting ID
                          ``\href{https://bloomu.zoom.us/my/dmccurry}{dmccurry}''\\
                      &    \href{https://bloomu.starfishsolutions.com/starfish-ops/dl/instructor/serviceCatalog.html?bookmark=connection/20001/schedule}{Schedule
		      an Appointment} (not required) \\ \\
        Text: & Tro, N.J. \textit{Chemistry: Structure and Properties},
	2\textsuperscript{nd} Ed. Pearson: Hoboken, NJ, 2018. \\ 
	      & \footnotesize Used in both CHEM115 and CHEM116. This course is
	      participating in inclusive access, so you should have day-1 access
	      to the online version of the text. You may purchase a hard copy
	      for an additional fee or opt out entirely. See BOLT for details.
	      \\ \\
	Lab Text: & Faculty of the Department of Chemistry and
	Biochemistry. \textit{Laboratory Exercises for Chemistry for
			      the Sciences 1}, Fall 2021 Ed. \\
	       	      & \footnotesize Available for purchase during the first
		      week of laboratory. \\ \\
	Materials: & Scientific or graphing calculator (simpler calculators
	are not appropriate). \\
		   & Scrap paper and a pen or pencil to work through problems.
		   \\
	\bottomrule
\end{tabularx}

\section{Course Description}
First course in chemistry for all physical and biological science majors.
Introduces and develops qualitative and quantitative concepts in chemistry
essential for further study in chemistry and application to other science
disciplines. Topics include the elements' atomic and periodic properties,
chemical compound types and properties, chemical reaction types, chemical
stoichiometry, molecular geometry determination and relation to properties,
physical properties of states of matter, qualitative acid-base behavior, and
inter-particle forces. Three hours lecture per week plus three hours per week
laboratory. Successful completion of the course earns 1 GEP toward Goal 3 ---
Analytical and Quantitative Skills and 3 GEPs toward Goal 5 --- Natural
Sciences.


\subsection{Learning Objectives}
We're going to cover a \emph{lot} this semester. You may even wonder if you will
ever use this material in the future. My goal is to not \emph{specifically}
ensure that you have \emph{memorized} all of the material. Rather, you are
expected to be able to:
\begin{itemize}[noitemsep]
	\item Identify the importance of chemicals and chemistry in everyday
		living.
	\item Identify the fundamental concepts of chemistry.
	\item Solve problems using chemical principles and mathematical
		analysis.
	\item Identify the properties of matter and the changes that matter
		undergoes, using logic and chemical principles.
\end{itemize}
Your ability to meet these objectives will be assessed through regular written
evaluations. In preparing for such exams and quizzes, it is therefore imperative
that you understand the \emph{why} and \emph{how} moreso than memorizing the
straight facts covered in the text or examples we try out in lecture.
Approaching the course in this manner will help you develop a skillset
applicable to your specific field of study in the future.

\subsection{Course Community and Communication}
\emph{Your active involvement in the course is important for your success!}
Throughout the semester, I will encourage active participation both during and
outside of lecture. In order to facilitate some community-building, we will be
using the Discussion Boards on BOLT extensively. If you have a question with an
answer that could benefit the class as a whole, please post your question to the
Discussion Board, rather than sending an individual email to directly to me. I
would also be happy to help organize study groups if you are interested. Please
let me know if you would like to chat with other classmates and I will put you
in touch. Learning occurs best when it is driven by your own motivation and
curiosity, not by passively absorbing information I share in a brief 50-minute
period!

All course announcements will be delivered on the BOLT course page. It is your
responsibility to regularly check BOLT. I will \emph{not} send out email
announcements, so BOLT is the \emph{only} location you will need to check for
all important course information --- in my experience, a single location for all
information significantly decreases any confusion. If you would like push
notifications on your phone when announcements are posted, please download the
\href{https://documentation.brightspace.com/EN/brightspace/requirements/all/pulse.htm}{Brightspace
Pulse} app.

\section{Evaluations and Grading}
Unfortunately, we do need to include a metric for assessing your understanding
of the material. This course will be operating on a 100\,\% scale according to
the tables below. Note that there will be 7 homeworks worth 20\,points each and
14 quizzes worth 5\,points each. The sum of the total will be divided by 200,
allowing a little wiggle room for ``off'' quiz days --- we all have them!

\begin{multicols}{2}
	\subsection{Point Distribution}
	\begin{tabular} {l r<{\,\%}}
		Exams & 40 \\
		Homework and Quizzes & 15 \\
		Final Exam & 25 \\
		Laboratory & 20 \\ \midrule
		& 100 \\
	\end{tabular}

	\subsection{Grade Assignment}%
	\begin{tabular} {r@{\,--\,}l<{\,\%} l@{\hspace{0.5in}}r@{\,--\,}l<{\,\%} l}
		\multicolumn{3}{c}{} & 75 & 77 & C+ \\
		91 & 97 & A  & 71 & 74 & C  \\
		88 & 90 & A- & 68 & 70 & C- \\
		\multicolumn{6}{c}{} \\
		85 & 87 & B+ & 65 & 67 & D+ \\
		81 & 84 & B  & 60 & 64 & D  \\
		78 & 80 & B- & 0  & 59 & F
	\end{tabular}
\end{multicols}

\section{Graded Items}
\begin{description}
	\item[Exams:] There will be three (3) 50 minute exams held during our
		regularly scheduled lecture time. Tardiness will not extend your
		exam time. If accommodations must be made, it is your
		responsibility to alert the instructor at least 72 hours
		\emph{before} the exam.
	\item[Quizzes:] There will be a total of fourteen (14) quizzes
		administered in lecture at least once \emph{every} week. Most
		quizzes will be open note, but the instructor reserves the right
		to declare a quiz close book.
	\item[Homework:] There will be seven (7) homework assignments on BOLT
		generally due on Wednesdays. These can be found under the
		Assessments > Quizzes section on BOLT. You will have 10 attempts
		to complete each homework, with the highest grade being taken
		for final grade calculation. On each subsequent attempt, you
		will only need to answer the questions you missed for credit.
		The homework is designed to aid your learning, so use it as a
		chance to test your knowledge before exams. Unless certain
		questions require attention, the homeworks will not be hand
		graded. It is imperative that you follow the instructions
		carefully so BOLT does not score your answers incorrectly.
	\item[Laboratory:] Laboratory attendance is mandatory. See the lab
		syllabus for lab policies and specifics on how the labs will be
		graded. Your laboratory grade is determined solely by your
		laboratory instructor. In cases where I (Dr.\ McCurry) am not
		your laboratory instructor, \emph{I will have no ability to
		modify or adjust items graded by your laboratory instructor}.
	\item[Final Exam:] There will be one (1) \emph{comprehensive} 110 minute
		final exam during the final exam week. This will be a
		standardized exam designed by the American Chemical Society in
		order to ensure that you have learned the competencies expected
		for a first-term undergraduate introductory chemistry course. In
		general, the national average for this exam hovers around
		50\,\% --- you are \emph{not} expected to know everything on
		this exam! Your final grade will be adjusted to account for
		topics that were emphasized a bit differently in our lecture.
		Furthermore, each of the 50-minute exams throughout the semester
		will have a small multiple choice portion to prepare you for
		this final, all multiple choice examination.
%	\item[Textbook Problems:] Homework problems are designed to help you
%		keep current with the material.  Assigned problems in the
%		textbook will \emph{usually} not be collected, but specific
%		questions may require answer submissions on BOLT (generally 1-2
%		per chapter). Please note that the problems are useful only if
%		you conscientiously work on them.
\end{description}

\subsection{Grading Errors}
Did I make a mistake? Let me know within 72 hours from the return of the graded
item so that I may re-grade it. \emph{The entire item will be re-graded}. This
timeline is necessary so I don't get a pile of re-grades during the last week of
class. You have until 4:30\,p.m.\ on the day of the final exam to bring clerical
errors (e.g., I typed a grade incorrectly into BOLT) to my attention.

\section{Important Dates}\label{importantdates}
\begin{center}
	\begin{tabular} {l l l}
		September 6  & Monday    & Labor Day (No Class) \\
		September 17 & Friday    & Exam 1 \\
		October 15   & Friday    & Exam 2 \\
		November 12  & Friday    & Exam 3 \\
		November 24  & Wednesday & Thanksgiving Recess (No Class) \\
		November 26  & Friday    & Thanksgiving Recess (No Class) \\
		December 3   & Friday    & Last Day of Class \\
		December 8   & Friday    & Final Examination (12:30 --
		2:30\,p.m.)
	\end{tabular}
\end{center}

Please note, if you are unable to take an examination or quiz at
its scheduled time due to some unavoidable circumstance, you are obligated to
provide a documented valid excuse to your instructor. \emph{Failure to provide a
valid excuse on these terms will result in a score of zero for the quiz or
exam.}

	\begin{center}
		\renewcommand\arraystretch{1.25}
\begin{tabularx}{\linewidth} {X X}
	\toprule
	\bfseries Valid Excuse & \bfseries Time Frame to Provide Documentation
	\\ \midrule
	Personal illness or accident, illness or accident of a dependent child,
	or death or critical illness of an immediate family member &
	48\,hr after the missed graded event \\
	Participation in a university-sponsored activity &
	48\,hr in advance of the missed graded event \\
	Military duty & As soon as possible \\
	Religious observance & The second Friday of the semester \\
	Others, as merited on a case-by-case basis & 48\,hr after the
	missed graded event \\
	\bottomrule
\end{tabularx}
\end{center}

Immediate technical issues, such as BOLT being inaccessible, can be submitted to
\url{https://helpdesk.bloomu.edu}. At the bottom of the form, you must include
my email address (dmccurry@bloomu.edu) in the ``Additional Email Notifications''
to alert me that you are taking steps to resolve the issue.

\section{Policies and Expectations}
In order to maintain an atmosphere that is conducive to learning, all members
of the class are expected to demonstrate common courtesy during the lecture,
including but not exclusively:
\begin{enumerate}
	\item \textbf{Wear a mask that covers your nose and mouth.} Not all
		students are vaccinated and other, more infectious variants pose
		a health risk to all individuals. In order to ensure the
		wellbeing of your fellow classmates, please wear a well-fitting
		mask while inside.
	\item \textbf{Be on time for class.} If you are unavoidably late, enter
		the room quietly and quickly choose a seat.  Tardiness will not
		extend your exam or quiz time. We will not return to missed
		quizzes.
	\item \textbf{Once in class, stay for the duration.} If you must leave
		early, give me advance warning. If you become ill during an exam
		or quiz, the missed graded event policy will take effect with
		the consent of the professor. In either case, you will not be
		allowed to return that day. Take care of your personal needs
		\emph{before class}.
	\item \textbf{Silence your cell phone and/or other electronic devices.}
		Students with special circumstances need to speak with me right
		away.
	\item \textbf{Refrain from talking to your neighbors excessively.} Stay
		focused on the lecture materials. If you missed something, feel
		free to ask me to repeat it!
\end{enumerate}
\emph{Please review
\href{https://www.bloomu.edu/prp-3881-student-disruptive-behavior-policy}{PRP 3881 Student
Disruptive Behavior Policy} for further details regarding the consequences for
disruptive behavior.}

\section{Academic Dishonesty}
\emph{Academic dishonesty is not tolerated.} If you are unclear about what is
dishonest, please see 
\href{https://www.bloomu.edu/prp-3512-academic-integrity-policy}{PRP 3512 
Academic Integrity Policy} for clarification. If you are unsure about my
specific instructions, ask me.

\begin{mdframed}
	\centering\bfseries The minimum penalty for academic dishonesty is a
	course assignment of ``F'' for \emph{all} students involved.
\end{mdframed}

\section{Class Cancellation}
If class is canceled on the day of an exam, it will be administered the next
time class meets.  If class is canceled the period before an exam, it will
still be administered on its scheduled day. If compression
(see \href{https://www.bloomu.edu/documents/prp5205}{PRP 5205 University Closing
Policy})
occurs on an exam day, the exam will be held as scheduled (with
grading adjusted to allow for the shortened time period).

\section{Fire Alarms}
In the event of a building evacuation, calmly and quickly leave the building via
the nearest exit. In the event of an evacuation during laboratory, ensure all
hot plates or other possible hazards are turned off and unplugged prior to
leaving. Your instructor will point exits out the first week of class.  Gather
with your class on the quad lawn in front of Bakeless. In the case of inclement
weather, we will meet in the Kehr Union Ballroom. Do not re-enter the
building for any reason. Do not gather around the exits. Do not enter a
building that has an alarm sounding. There is one, and only one, drill the first
week of the semester. There are \emph{no} random drills.

%\subsection{From G38}
%\begin{enumerate}[noitemsep]
%	\item Use the door to the front right of the lecture hall and proceed up
%		the staircase to the exit door.
%	\item Proceed behind Sutliff to the sidewalk near Centennial
%		to the quad traveling in front of McCormick to the mustering
%		point near Bakeless
%\end{enumerate}
%
%Treat fire drills as though they are real, evacuate to the lawn
%in front of Bakeless, and all of us get out of the habit of
%standing anywhere near Hartline Science Center.

\section{Last Day to Withdraw}
The Registrar has set the last day of class as the withdrawal deadline.
Students are encouraged to review the latest withdrawal
(\href{https://www.bloomu.edu/prp-3462-undergraduate-course-withdrawal}{PRP 3462 Undergraduate
Course Withdrawal}) and course repeat
(\href{https://www.bloomu.edu/prp-3452-course-repeat}{PRP 3452 Course
Repeat}) policies of the
University prior to that date as there are strict limits on the number of
repeats one can have. 

\begin{mdframed}
	\centering
	All policies, rules, and procedures (PRP) are available at
	\url{https://www.bloomu.edu/about/administration-and-governance/policies}
\end{mdframed}

\section{Accommodations}
Bloomsburg University provides reasonable accommodations to students who have
documented disabilities. If you have a documented disability that requires
academic accommodations and are not registered with 
\href{https://bloomu.prod.acquia-sites.com/offices-directory/disability-services}{University
Disability Services}
please contact this office as soon as possible to establish your
eligibility. Please also provide documentation to Dr.\ McCurry as promptly as
possible.

%\section{How to Succeed in CHEM115}
%\begin{enumerate}
%	\item Read the textbook materials \emph{before class}
%	\item Attend lecture
%	\item Work on the assigned problems in each chapter
%	\item Create an \emph{equally contributing} study group that
%		\emph{actually} studies/works problems
%	\item Synthesize your own questions and answer them
%	\item Use lab to connect lecture topics to the real world
%\end{enumerate}
%
%Do not get behind! If you are not doing as well as you should, be sure to see
%your professor as soon as possible. If you cannot make it to any of my listed
%office hours, please contact me in person, or by phone, or by email to schedule
%an appointment. I will respond to emails as promptly as possible.

\vfill

\begin{mdframed}
	\noindent
	The materials contained in this syllabus, in the lab manual, and the
	BOLT webpage for this course are intended only for those registered for
	the above course/semester. These materials cannot be used without the
	expressed written consent of Dr.\ McCurry.

	\noindent
	The source code for some material has been licensed under the Creative
	Commons Attribution-NonCommercial-ShareAlike 4.0 International (CC
	BY-NC-SA 4.0) license to provide you with an opportunity to view the
	original source materials and contribute to the content. If you would
	like to view the material and suggest improvements, please visit
	\url{https://git.damccurry.com/dan/CHEM115}. Ask Dr.\ McCurry for an
	account!
\end{mdframed}

\end{document}
