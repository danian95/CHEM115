% !TEX program=xelatex
%\documentclass[notes=onlyslideswithnotes,notes=hide]{beamer}
%\documentclass[handout]{beamer}
%\documentclass[notes=hide]{beamer}
%\documentclass[notes=show]{beamer}
\documentclass[notes=only]{beamer}

\usepackage{bucolors}
\usepackage{genchem}
\usepackage{lecture}
\usepackage{bucolors}
\usepackage{ccicons}
\usepackage{amsmath}
\usepackage{mathspec}
\setmainfont[Ligatures=TeX]{Georgia}
\setsansfont[Ligatures=TeX]{Arial}
\usepackage{tabularx}
\usepackage{import}
\usepackage{tikz}
\usepackage{multicol}
\usepackage{elements}
\usepackage{collcell}
\usetikzlibrary{tikzmark}

\newcolumntype{s}{>{\collectcell\unit}l<{\endcollectcell}} % chem left column type

\title{Introduction to Solutions and Aqueous Reactions}
\subtitle{Chapter 8}
\institute{CHEM115 --- Chemistry for the Sciences I \\ Bloomsburg University}
\author{D.A. McCurry}
\date{Spring 2023}

\begin{document}

\maketitle

\section{Solutions}

\begin{frame}{Learning Objectives}
	\begin{itemize}
	\item Explain the components of a solution.
	\item Predict whether a solution will form or not depending on the
		structure of the solvent and solute.
	\item Quantify the amount of solute present in a solution even when the
		amount of solvent changes.
	\end{itemize}
\end{frame}

\begin{frame}{Solution Terminology}
	\begin{columns}
		\column{0.5\textwidth}
		\begin{block}{Solution}
			A homogeneous mixture of two (or more) substances.
		\end{block}

		\bigskip

		An \alert{aqueous solution} is one in which water acts as the
		solvent.
		\begin{reactions*}
			NaNO3\sld{} &->[ H2O ] NaNO3\aq{} \\
			CuSO4\sld{} &->[ H2O ] CuSO4\aq{}
		\end{reactions*}
		\column{0.4\textwidth}
		\begin{center}
			\includegraphics[width=0.9\linewidth]{Borland-Solute-Solvent.jpg}
		\end{center}
	\end{columns}


%	\pause
%
%	How do we \alert{quantify} the amount of \ch{NaNO3} or \ch{CuSO4}
%	\alert{dissolved} in the solvent?
\end{frame}

\begin{frame}{Some Notes on Solutes}
	\begin{columns}
		\column{0.45\textwidth}
		\begin{itemize}
			\item Solutes spread \alert{evenly} throughout the
				solution.
			\item They cannot be separated by filtration.
			\item They \alert{can} be separated by evaporation.
			\item Solutes are not visible (homogeneous mixture), but
				can give color to a solution.
		\end{itemize}
		\column{0.45\textwidth}
		\begin{center}
			\includegraphics[width=\linewidth]{Borland-Solutes.jpg}
		\end{center}
	\end{columns}
\end{frame}

\begin{frame}{``Like Dissolves Like''}
	Two substances form a solution when there is an attraction between the
	particles of the solute and solvent.

	\begin{center}
		\begin{tikzpicture}
			\node[align = center, fill = green!10, minimum
		width=0.45\linewidth](will) {%
				\textbf{Solutions Will Form} \\ \\

		\begin{tabular} {l l}
			\toprule
			\bfseries Solute & \bfseries Solvent \\ \midrule
			Polar & Polar \\
			Nonpolar & Nonpolar \\
			\bottomrule
		\end{tabular}
	};
	\node[right = of will, align = center, fill = red!10, minimum
		width=0.45\linewidth](not) {%
			\textbf{Solutions Will Not Form} \\ \\

		\begin{tabular} {l l}
			\toprule
			\bfseries Solute & \bfseries Solvent \\ \midrule
			Polar & Nonpolar \\
			Nonpolar & Polar \\
			\bottomrule
		\end{tabular}
	};
\end{tikzpicture}
	\end{center}

	\note{%
	\begin{itemize}
		\item \alert{Polar} solvents (such as water) dissolve
			\alert{polar} solutes (such as sugar) and/or \alert{ionic}
			solutes (such as \ch{NaCl}).
		\item \alert{Nonpolar} solvents (such as hexane, \ch{C6H14})
			dissolve \alert{nonpolar} solutes (such as oil or
			grease).
	\end{itemize}
}
\end{frame}

\begin{frame}{Quantifying Solution Concentrations}
	\begin{columns}
		\column{0.375\linewidth}
		The \alert{concentration} of a solution refers to the amount of solute
		contained within
		a specific amount of solvent.
	
		\bigskip
	
		A common measure of concentration is \alert{molarity}.
		\column{0.525\linewidth}
		\begin{center}
			\includegraphics[scale=0.3]{08_02_Figure.jpg}
		\end{center}
	\end{columns}

	\bigskip

	\begin{equation*}
		\text{molarity (\si{\Molar})} = \frac{\text{amount of solute (in
		\si{\mole})}}{\text{volume of solution (in \si{\liter})}}
	\end{equation*}
\end{frame}

\begin{frame}[t]{Calculating Molarity 1}
	If we dissolve \SI{0.522}{\gram}~\ch{NaCl} in
	\SI{100.0}{\milli\liter}~\ch{H2O}, what is the molarity of the resulting
	solution?

	\vspace{10em}

	\note{
		\begin{align*}
			\frac{\SI{0.522}{\gram}}{\SI{100.0}{\milli\liter}}
			\times \frac{\SI{1}{\mole}}{\SI{58.44}{\gram}} \times
			\frac{\SI{1000}{\milli\liter}}{\SI{1}{\liter}} &=
			\SI{0.089322382}{\Molar} \\
			&= \boxed{\SI{0.0893}{\Molar}}
		\end{align*}
		}
\end{frame}

\begin{frame}[t]{Calculating Molarity 2}
	How many grams of \ch{AlCl3} are needed to prepare
	\SI{125}{\milli\liter} of a \SI{0.150}{\Molar} solution?

	\vspace{10em}

	\note{
		We need to first figure out how many moles are needed, then
		convert to grams.
		\begin{align*}
			\SI{125}{\milli\liter} \times
			\frac{\SI{1}{\liter}}{\SI{1000}{\milli\liter}} \times
			\frac{\SI{0.150}{\mole}}{\SI{1}{\liter}} \times
			\frac{\SI{133.332}{\gram}}{\SI{1}{\mole}} &=
			\SI{2.499975}{\gram} \\
			&= \boxed{\SI{2.50}{\gram}~\ch{AlCl3}}
		\end{align*}
		}
\end{frame}

\begin{frame}[t]{Practical Laboratory Skills}
	How do we prepare \SI{1.00}{\liter} of \SI{1.0}{\Molar} \ch{NaCl}?

	\note{
		\begin{enumerate}
			\item Weigh appropriate amount:
				\begin{align*}
					\SI{1.0}{\Molar}~\ch{NaCl} \times
					\SI{1.00}{\liter} \times
					\frac{\SI{58.44}{\gram}}{\SI{1}{\mole}}
					&= \SI{58.44}{\gram} \\
					&= \boxed{\SI{58}{\gram}~\ch{NaCl}}
				\end{align*}
			\item Add water up to \SI{1.0} liter mark on volumetric
				flask.
		\end{enumerate}
		}
\end{frame}

\begin{onyourown}
	If a solution is \SI{0.100}{\Molar}, how many moles of \ch{KBr} are
	contained within \SI{25.00}{\milli\liter}?
\end{onyourown}

\begin{onyourown}
	How many milliliters of \SI{2.00}{\Molar} \ch{HNO3} contain
	\SI{24.0}{\gram} of \ch{HNO3}?
\end{onyourown}

\clearpage

\begin{frame}{Diluting Solutions}
	In a \alert{dilution},
	\begin{tabular}[t] {@{~}l}
		solvent is added. \\
		volume increases. \\
		concentration of solute \alert{decreases}. \\
	\end{tabular}
	
	\bigskip
	
	\begin{center}
		\includegraphics[scale=0.4]{Borland-dilution.jpg}
	\end{center}

	\begin{alertblock}{Key Point}
		In both the initial and diluted solution, the moles of solute
		are the \alert{same}!
	\end{alertblock}

	\note{%
		This allows us to use the following equation:
		\begin{align*}
			M_1V_1 &= M_2V_2
			\shortintertext{because}
			\frac{\si{\mole}_1}{\si{\liter}} \times \si{\liter}_1 &=
			\frac{\si{\mole}_2}{\si{\liter}} \times \si{\liter}_2 \\
			\si{\mole}_1 &= \si{\mole}_2
			\intertext{But this actually applies to
	\alert{any} units of concentration or volume, so} C_1V_1 &= C_2V_2
		\end{align*}
	}
\end{frame}

\begin{frame}{Guide to Calculating Dilution Quantities}
	\begin{enumerate}
		\item Prepare a table of the concentrations and volumes of
			the solutions.
		\item Rearrange the dilution expression ($C_1V_1 = C_2V_2$) to
			solve for the unknown quantity.
		\item Substitute the known quantities into the dilution
			expression and calculate.
	\end{enumerate}
\end{frame}

\begin{frame}[t]{Dilution Calculations 1}
	What is the final volume (in \si{\milli\liter}) of
	\SI{15.0}{\milli\liter} of a \SI{1.80}{\Molar}~\ch{KOH} stock solution
	diluted to give a \SI{0.300}{\Molar} solution?

	\note{
		\begin{tabular} {L S[table-format=2.3] s}
			C_\text{stock} & 1.80 & \Molar \\
			C_\text{dilute} & 0.300 & \Molar \\
			V_\text{stock} & 15.0 & \milli\liter
		\end{tabular}
			
		\begin{align*}
			C_1V_1 &= C_2V_2 \\
			C_\text{stock}V_\text{stock} &=
			C_\text{dilute}V_\text{dilute} \\
			V_\text{dilute} &= \frac{C_\text{stock}V_\text{stock}}{
				C_\text{dilute}} \\
			&=
			\frac{(\SI{1.80}{\Molar})(\SI{15.0}{\milli\liter})}{\SI{0.300}{\Molar}}
			\\
			&= \boxed{\SI{90.0}{\milli\liter}}
		\end{align*}
		}
\end{frame}

\begin{frame}[t]{Dilution Calculations 2}
	What volume of stock \SI{1.00}{\Molar} \ch{HCl} is needed to prepare
	\SI{25}{\milli\liter} of a \SI{0.010}{\Molar} solution?

	\note{
		\begin{tabular} {L S[table-format=2.3] s}
			C_\text{stock} & 1.00 & \Molar \\
			C_\text{dilute} & 0.010 & \Molar \\
			V_\text{dilute} & 25.0 & \milli\liter
		\end{tabular}
			
		\begin{align*}
			C_1V_1 &= C_2V_2 \\
			C_\text{stock}V_\text{stock} &=
			C_\text{dilute}V_\text{dilute} \\
			V_\text{stock} &= \frac{C_\text{dilute}V_\text{dilute}}{
				C_\text{stock}} \\
			&=
			\frac{(\SI{0.010}{\Molar})(\SI{25.0}{\milli\liter})}{\SI{1.00}{\Molar}}
			\\
			&= \boxed{\SI{0.25}{\milli\liter}}
		\end{align*}
		}
\end{frame}

\begin{onyourown}
	What is the concentration of a \SI{40.0}{\milli\liter} solution of
	\ch{FeCl3} prepared from \SI{1.00}{\milli\liter} of a \SI{0.250}{\Molar}
	stock solution?
\end{onyourown}

\section{Precipitation Reactions}

\begin{frame}{Learning Objectives}
	\begin{itemize}
	\item Use stoichiometry to determine the amount of products that will be
		produced or the amount of reactants required.
	\item Predict the solubility of a resulting product based on solubility
		rules.
	\item Identify strong, weak, and nonelectrolytes.
\end{itemize}
\end{frame}

\begin{frame}[t]{Solution Stoichiometry}
	Consider one of the reactions from the Ionic Species Solubility Lab:
	\begin{reaction*}
		3 Ba(NO3)2\aq{} + 2 Na3PO4\aq{} -> Ba3(PO4)2\sld{} + 6
		NaNO3\aq{}
	\end{reaction*}
	Did \emph{all} of the \ch{Ba(NO3)2} and \ch{Na3PO4} react?

	\pause

	\bigskip

	What volume (in \si{\milli\liter}) of a \SI{0.200}{\Molar} \ch{Na3PO4}
	solution is required to completely react with \SI{10.00}{\milli\liter}
	of a \SI{0.025}{\Molar} solution of \ch{Ba(NO3)2}?
	
	\note<2>{\footnotesize
			\begin{enumerate}
				\item How much \ch{Na3PO4} reacts with \ch{Ba(NO3)2}?
					\boxed{\frac{\SI{2}{\mole}~\ch{Na3PO4}}{\SI{3}{\mole}~\ch{Ba(NO3)2}}}
				\item How many moles of \ch{Ba(NO3)2} do we have?
					\begin{align*}
						\frac{\SI{0.025}{\mole}}{\SI{1}{\liter}}
						\times \SI{10.00}{\milli\liter} \times
						\frac{\SI{1}{\liter}}{\SI{1000}{\milli\liter}}
						&= \SI{0.00025}{\mole}~\ch{Ba(NO3)2}
					\end{align*}
				\item How many moles of \ch{Na3PO4} do we need?
					\begin{align*}
						\SI{0.00025}{\mole}~\ch{Ba(NO3)2} \times
						\frac{\SI{2}{\mole}~\ch{Na3PO4}}{\SI{3}{\mole}~\ch{Ba(NO3)2}}
						&= \SI{0.0001666667}{\mole}~\ch{Na3PO4}
					\end{align*}
				\item How many \si{\milli\liter} \ch{Na3PO4} solution do
					we need?
					\begin{align*}
						\SI{0.0001666667}{\mole}~\ch{Na3PO4}
						\times
						\frac{\SI{1}{\liter}}{\SI{0.200}{\mole}~\ch{Na3PO4}}
						\times
						\frac{\SI{1000}{\milli\liter}}{\SI{1}{\liter}}
						&= \SI{0.8333333}{\milli\liter} \\
						&= \boxed{\SI{0.83}{\milli\liter}}
					\end{align*}
			\end{enumerate}
		}
\end{frame}

\clearpage

\begin{onyourown}
	If \SI{22.8}{\milli\liter} of \SI{0.100}{\Molar}~\ch{MgCl2} is needed to
	completely react with \SI{15.0}{\milli\liter} of \ch{AgNO3} solution,
	what is the molarity of the \ch{AgNO3} solution?
	\begin{reaction*}
		MgCl2\aq{} + 2 AgNO3\aq{} -> 2 AgCl\sld{} + Mg(NO3)2\aq{}
	\end{reaction*}
\end{onyourown}

\begin{frame}{Why do some compounds dissolve?}
	Recall ``Like Dissolves Like''\ldots
	\begin{itemize}[<+(1)->]
		\item The \alert{solvent-solute} interactions are able to
			overcome the \alert{solute-solute} interactions.
			\begin{center}
				\includegraphics[scale=0.15,trim={0 0 0
				80pt},clip]{08_05_Figure.jpg}
			\end{center}
		\item Solvent-solute interactions can only overcome
			solute-solute interactions if there is a \alert{strong
			attraction} between the solvent and the solute.
			\begin{center}
				\includegraphics[scale=0.2,trim={0 0 0
				70pt},clip]{08_07_Figure.jpg}
			\end{center}
	\end{itemize}
\end{frame}

\vspace{\stretch{-1}}

\clearpage

\begin{frame}
	\begin{center}
		\mode<presentation>{\includegraphics[scale=0.5]{08_08_Figure.jpg}}
		\mode<article>{\includegraphics[scale=0.3]{08_08_Figure.jpg}}
	\end{center}
\end{frame}

\vspace{\stretch{-1}}

\begin{frame}{What compounds don't dissolve?}
	\begin{center}
		\begin{tabularx}{\linewidth} {@{}l X}
			\toprule
		\bfseries Generally \underline{Soluble} & \bfseries Exceptions
		\\ \midrule
		\ch{Li+}, \ch{Na+}, \ch{K+}, and \ch{NH4+} & None \\
		\ch{NO3-} and \ch{C2H3O2-} & None \\
		\ch{Cl-}, \ch{Br-}, and \ch{I-} & \ch{Ag+}, \ch{Hg2^{2+}},
		\ch{Pb^{2+}} \\
		\ch{SO4^{2-}} & \ch{Sr^{2+}}, \ch{Ba^{2+}}, \ch{Pb^{2+}},
		\ch{Ag+}, \ch{Ca^{2+}} \\
		\\
		\bfseries Generally \underline{Insoluble} & \bfseries Exceptions
		\\ \midrule
		\ch{OH-} & \ch{Ca^{2+}}, \ch{Sr^{2+}}, and \ch{Ba^{2+}} are
		\emph{slightly} soluble \\
		& \ch{Li+}, \ch{Na+}, \ch{K+}, and \ch{NH4+} are soluble \\
		\ch{S^{2-}} & \ch{Li+}, \ch{Na+}, \ch{K+}, \ch{NH4+},
		\ch{Ca^{2+}}, \ch{Sr^{2+}}, and \ch{Ba^{2+}} \\
		\ch{CO3^{2-}} and \ch{PO4^{3-}} & \ch{Li+}, \ch{Na+}, \ch{K+},
		and \ch{NH4+} \\
		\bottomrule
	\end{tabularx}
	\end{center}
\end{frame}

\begin{frame}{Precipitation Reactions}
	\begin{quote}
		A reaction in which a solid forms upon the mixing of two
		solutions.
	\end{quote}
	\begin{reaction*}
		!(soluble)( 3 Ba(NO3)2\aq{} ) + !(soluble)( 2 Na3PO4\aq{} ) ->
		!(insoluble)( Ba3(PO4)2\sld{} ) + !(soluble)( 6
		NaNO3\aq{} )
	\end{reaction*}

	\begin{itemize}
		\item The \alert{insoluble} product is known as the
			\alert{precipitate}.
		\item In some reactions, we can form multiple precipitates.
		\item Only ions that combine to form a precipitate are actually
			participating in the reaction --- the others are
			considered \alert{spectator} ions.
	\end{itemize}
\end{frame}

%\mode<article>{
%\pagebreak
%
%\begin{framed}
%	\centering
%	\sffamily\bfseries\scshape\large
%	Exam IV Material Begins
%\end{framed}}

\begin{frame}{What about the compounds that do dissolve?}
	\begin{block}{Electrolytes}
		Substances that dissolve in water to form solutions that conduct
		electricity. The dissolved ions are \alert{charge carriers}.
	\end{block}

	\pause

	\begin{description}
		\item[Strong electrolytes] conduct electricity \alert{very
	  well}.
	  \begin{reaction*}
	  	!( table~salt )( NaCl\sld{} ) -> Na^{+}\aq{} +
	  	Cl^{-}\aq{}
	  \end{reaction*}

	  \pause

  \item[Weak electrolytes] conduct electricity
	  \alert{poorly}.
	  \begin{reaction*}
	  	!( acetic~acid )( HC2H3O2\aq{} ) <=> H^{+}\aq{}
	  	+ C2H3O2^{-}\aq{}
	  \end{reaction*}

	  \pause

	\item[Nonelectrolytes] do not conduct electricity.
	\begin{reaction*}
		!( sugar )( C12H22O11\sld{} ) ->
		C12H22O11\aq{}
	\end{reaction*}
\end{description}
\end{frame}

\begin{frame}
	\centering
	\includegraphics[scale=0.45]{08_12_Figure.jpg}
\end{frame}

\begin{frame}{Classification of Some Common Substances}
	\centering
	\small
	\begin{tabularx}{\linewidth}{>{\raggedright\arraybackslash\hsize=0.875\hsize\linewidth=\hsize}X
				     >{\hsize=0.875\hsize\linewidth=\hsize}X
			     >{\hsize=1.25\hsize\linewidth=\hsize}X}
		\toprule
		\bfseries Strong Electrolytes & \bfseries Weak
		Electrolytes & \bfseries Nonelectrolytes \\
		\midrule
		HCl, HBr, HI & \ch{CH3CO2H} & \ch{H2O} \\
		\ch{HClO4} & HF & \ch{CH3OH} (methyl~alcohol) \\
		\ch{HNO3} & HCN & \ch{C2H5OH} (ethyl~alcohol) \\
		\ch{H2SO4} & & \ch{C12H22O11} (sucrose) \\
		\ch{KBr} & & Most compounds of \\
		NaCl & & \hspace{1em}carbon \\
		NaOH, KOH & & \hspace{1em}(organic compounds)\\
		Other soluble ionic compounds \\
		\bottomrule
	\end{tabularx}
\end{frame}

\begin{frame}[t]{Writing Electrolyte Dissociation Reactions}
	Complete each of the following equations:

	\vspace{1em}

	\begin{enumerate}
		\item \ch{CaCl2\sld{} ->[ H2O ]}
			\note[item]{\ch{CaCl2\sld{} ->[ H2O ] Ca^{2+}\aq{} + 2
			Cl^{-}\aq{}}}
			\vspace{2em}
		\item \ch{K3PO4\sld{} ->[ H2O ]}
			\note[item]{\ch{K3PO4\sld{} ->[ H2O ] 3
			K^{+}\aq{} + PO4^{3-}\aq{}}}
			\vspace{2em}
		\item \ch{HF\aq{} ->[ H2O ]}
			\note[item]{\ch{HF\aq{} <=>[ H2O ] H^{+}\aq{} +
			F^{-}\aq{}}}
			\vspace{2em}
		\item \ch{C2H5OH\lqd{} ->[ H2O ]}
			\note[item]{\ch{C2H5OH\lqd{} ->[ H2O ] C2H5OH\aq{}}}
			\vspace{2em}
	\end{enumerate}
\end{frame}

%\begin{onyourown}[0em]
%	Complete each of the following equations:
%
%	\vspace{1em}
%
%	\begin{enumerate}
%		\item \ch{Na2SO4\sld{} ->[ H2O ]}
%			\vspace{2em}
%		\item \ch{HCN\aq{} ->[ H2O ]}
%			\vspace{2em}
%		\item \ch{C3H8O3\aq{} ->[ H2O ]}
%			\vspace{2em}
%	\end{enumerate}
%\end{onyourown}

\section{Acid-Base Reactions}

\begin{frame}{Learning Objectives}
	\begin{itemize}
		\item Identify common acids and bases in order to determine the product
			of a neutralization reaction.
		\item Balance acid-base reactions using \ch{H+} and \Hyd{}.
		\item Write the net ionic equation for strong acid-strong base
			reactions.
		\item Calculate stoichiometry of acid-base reactions.
	\end{itemize}
\end{frame}


\begin{frame}[t]{Acid-Base Reactions}
	We have a special name for reactions where \ch{H+} is one of the
	electrolytes\footnote{According to Arrhenius acid-base theory (Chapter 16)}
	--- \alert{acid-base} or \alert{neutralization}
	reactions.

	\begin{description}
		\item[Acid:] produces \ch{H+} ions in water.
			\begin{reaction*}
				HCl\aq{} ->[ H2O ] H^{+}\aq{} +
				Cl^{-}\aq{}
			\end{reaction*}
		\item[Base:] produces \ch{OH-} ions in water.
			\begin{reaction*}
				NaOH\aq{} ->[ H2O ] Na^{+}\aq{} +
				OH^{-}\aq{}
			\end{reaction*}
	\end{description}

	\begin{example}
		Show the reaction of \ch{HCl} and \ch{NaOH}.
	\end{example}

	\note{%
	When an acid and base react together, they \alert{neutralize} each other:
	\begin{reactions*}
		HCl\aq{} + NaOH\aq{} &-> NaCl\aq{} + H2O\lqd{} \\
		H^{+}\aq{} + Cl^{-}\aq{} + Na^{+}\aq{} + OH^{-}\aq{} &->
		Na^{+}\aq{} + Cl^{-}\aq{} + H2O\lqd{} \\
		H^{+}\aq{} + OH^{-}\aq{} &-> H2O\lqd{} \\
	\end{reactions*}
}
\end{frame}

\begin{frame}{Some notes on acids\ldots}
	\ch{H+} normally associates with water to form the
			\alert{hydronium ion}.
			\begin{reaction*}
				!(proton)( H^{+}\aq{} ) + !(water)( H2O\lqd{} )
				-> !(hydronium~ion)( H3O^{+}\aq{} )
			\end{reaction*}
			
			\bigskip

			\pause

		Acids have a sour taste. \hfill
			\parbox{0.5\linewidth}{
				\centering
				\includegraphics[scale=0.15]{Citrus_fruits.jpg}
				}

			\bigskip

			\pause

		\alert{Polyprotic} acids have more than one
			\alert{ionizable} proton.
			\begin{reactions*}
				H2SO4\aq{} + \water\lqd{} &-> \Oxo\aq{} +
				HSO4^{-}\aq{} \\
				HSO4^{-}\aq{} + \water\lqd{} &<=> \Oxo\aq{} +
				SO4^{2-}\aq{}
			\end{reactions*}
			The most common are \alert{diprotic} and
			\alert{triprotic} acids.
\end{frame}

\begin{frame}{Some notes on bases\ldots}
		Basic solutions are sometimes called ``alkaline''.

			\bigskip

			\pause

		Bases have a bitter taste.\hfill
			\parbox{0.5\linewidth}{
				\centering
				\includegraphics[scale=0.5]{BakingSoda.jpg}
				}

			\bigskip

			\pause

		Can turn your fat into soap \textrightarrow\
			\alert{saponification}.
\end{frame}

\begin{frame}{Some Common Acids and Bases}
	\begin{center}
		\small
	\begin{tabular} {l E@{\qquad}l E}
		\toprule
		\bfseries Name of Acid & \bfseries Formula & \bfseries Name of
		Base & \bfseries Formula \\ \midrule
		Hydrochloric acid & HCl & Sodium Hydroxide & NaOH \\
		Hydrobromic acid & HBr & Lithium hydroxide & LiOH \\
		Hydroiodic acid & HI & Potassium hydroxide & KOH \\
		Nitric acid & HNO3 & Calcium hydroxide & Ca(OH)2 \\
		Sulfuric acid & H2SO4 & Barium hydroxide & Ba(OH)2 \\
		Perchloric acid & HClO4 & Ammonia & NH3 \\
		Acetic acid & HC2H3O2 \\
		Hydrofluoric acid & HF \\ \bottomrule
	\end{tabular}
\end{center}
\end{frame}

\begin{frame}{Naming Acids}
		\alert{Binary acids} have just an \ch{H} and a nonmetal.
			\begin{itemize}
				\item They are named with the prefix ``hydro-''
					and the suffix ``-ic acid''.
					\begin{center}
						\begin{tabular}
							{p{3em}p{14em}}
							\ch{HCl} &
							\emph{hydro}chlor\emph{ic
							acid}
						\end{tabular}
					\end{center}
			\end{itemize}

			\bigskip

			\pause
		\alert{Oxyacids} contain \ch{H} and a polyatomic anion
			containing a nonmetal and oxygen.
			\begin{itemize}
				\item If the oxyanion ends in ``-ate'', we
					replace ``-ate'' with ``-ic acid''.
					\begin{center}
						\begin{tabular}
							{p{3em} p{14em}}
							\ch{HNO3} & nitr\emph{ic
							acid}
						\end{tabular}
					\end{center}
				\item If the oxyanion ends in ``-ite'', we
					replace the ``-ite'' with ``-ous acid''.
					\begin{center}
						\begin{tabular}
							{p{3em}p{14em}}
							\ch{HNO2} & nitr\emph{ous
							acid}
						\end{tabular}
					\end{center}
			\end{itemize}
\end{frame}

\begin{frame}{Naming Bases}
		Bases with \Hyd\ ions follow general ionic compound naming
			conventions.

			\begin{center}
				\begin{tabular}{>{\collectcell\ch}p{5em}<{\endcollectcell}p{10em}}
					NaOH    & sodium     \emph{hydroxide} \\
					KOH     & potassium  \emph{hydroxide} \\
					Ba(OH)2 & barium     \emph{hydroxide} \\
					Al(OH)3 & aluminum   \emph{hydroxide} \\
					Fe(OH)3 & iron (III) \emph{hydroxide}
				\end{tabular}
			\end{center}
\end{frame}

%\begin{frame}[t]{Acid-Base Nomenclature Practice}
%	Name the following acids or bases:
%
%	\begin{enumerate}
%		\item \ch{HI} \note[item]{hydroiodic acid}
%		\item \ch{KOH} \note[item]{potassium hydroxide}
%		\item \ch{HClO} \note[item]{hypochlorous acid}
%		\item \ch{HClO2} \note[item]{chlorous acid}
%		\item \ch{HClO3} \note[item]{chloric acid}
%		\item \ch{HClO4} \note[item]{perchloric acid}
%		\item \ch{Sr(OH)2} \note[item]{strontium hydroxide}
%	\end{enumerate}
%\end{frame}
%
%\begin{onyourown}[0em]
%	Name the following acids or bases:
%	
%	\begin{enumerate}
%		\item \ch{H2S}
%		\item \ch{HBr}
%		\item \ch{CsOH}
%		\item \ch{H2SO3} (This one is a bit of a challenge)
%		\item \ch{Cu(OH)2}
%		\item \ch{HC2H3O2}
%	\end{enumerate}
%\end{onyourown}

\begin{frame}{Acid-Base Reactions}
	In the equations for neutralization, an acid and a base produce a salt
	and water.
			\begin{reaction*}
				!(acid)( HCl\aq{} ) + !(base)( NaOH\aq{} ) ->
				!(salt)( NaCl\aq{} ) + !(water)( \water\lqd{} )
			\end{reaction*}

	For \alert{every} acid-base reaction, we should have the general trend,
			\begin{reaction*}
				acid + base -> salt + water
			\end{reaction*}

	The \alert{salt} (the ionic compound produced) is often still dissolved
	in solution. We therefore can prepare a
			\alert{net ionic equation}:
			\begin{reaction*}
				H^{+}\aq{} + \Hyd\aq{} -> \water\lqd{}
			\end{reaction*}
\end{frame}

\begin{frame}[t]{Guide to Balancing Acid-Base Reactions}
	Balance the reaction when aqueous solutions of \ch{HNO3} with
	\ch{Mg(OH)2} are mixed.

	\bigskip

	\begin{enumerate}[<+(1)->]
		\item Write the base and acid formulas.
		\item Balance \Hyd\ and \ch{H^{+}}.
		\item Balance with \water.
		\item Write the salt from remaining ions.
	\end{enumerate}

	\note<.(1)>{%
		\begin{enumerate}
			\item \ch{Mg(OH)2\aq{} + HNO3\aq{}}
			\item \ch{Mg(OH)2\aq{} + "\alert{2}" HNO3\aq{}}
			\item \ch{Mg(OH)2\aq{} + 2 HNO3\aq{} -> salt + 2
				\water\lqd{}}
			\item \ch{Mg(OH)2\aq{} + 2 HNO3\aq{} ->
			Mg(NO3)2\aq{} + 2 \water\lqd{}}
		\end{enumerate}
		}
\end{frame}

\begin{frame}[t]{Balancing Acid-Base Reaction Practice}
	Balance the following reaction:
	\begin{reaction*}
		HCl\aq{} + Al(OH)3\aq{} -> AlCl3\aq{} + H2O\lqd
	\end{reaction*}
\end{frame}

%\begin{onyourown}
%	Balance the following reaction:
%
%	\begin{reaction*}
%		Ba(OH)2\aq{} + H3PO4\aq{} -> Ba3(PO4)2\sld{} + H2O\lqd{}
%	\end{reaction*}
%\end{onyourown}

\begin{frame}{Practical Lab Skills: Titrations}
	\begin{columns}
		\column{0.4\textwidth}
	\begin{block}{Titration}
		A substance in a solution of known concentration is reacted with
		another substance in a solution of unknown concentration.
	\end{block}

	\column{0.5\textwidth}
	\begin{center}
		\includegraphics[scale=0.35,trim={0 0 0 30pt},clip]{08_18_Figure.jpg}
	\end{center}
\end{columns}
\end{frame}

\begin{frame}
	\includegraphics[width=\linewidth]{08_19_Figure.jpg}
\end{frame}

\begin{frame}[t]{Titration Calculations}
	If \SI{32.7}{\milli\liter} of a \SI{0.5000}{\Molar} \ch{KOH} solution
	were required to neutralize \SI{20.0}{\milli\liter} of a \ch{H2SO4}
	solution, what is the concentration of the \ch{H2SO4}?

	\note{
		\begin{reaction*}
			2 KOH\aq{} + H2SO4\aq{} -> K2SO4\aq{} + 2 H2O\lqd{}
		\end{reaction*}

		\begin{align*}
			\intertext{Find moles of KOH:}
			\frac{\SI{0.5000}{\mole}~\ch{KOH}}{\SI{1}{\liter}}
			\times \SI{32.7}{\milli\liter} \times
			\frac{\SI{1}{\liter}}{\SI{1000}{\milli\liter}} &=
			\SI{0.01635}{\mole}~\ch{KOH} \\
			\intertext{From stoichiometry:}
			\SI{0.01635}{\mole}~\ch{KOH} \times
			\frac{\SI{1}{\mole}~\ch{H2SO4}}{\SI{2}{\mole}~\ch{KOH}}
			&= \SI{0.008175}{\mole}~\ch{H2SO4} \\
			\intertext{Solve for concentration:}
			\frac{\SI{0.008175}{\mole}~\ch{H2SO4}}{\SI{20.0}{\milli\liter}}
			\times \frac{\SI{1000}{\milli\liter}}{\SI{1}{\liter}} &=
			\SI{0.40875}{\Molar} \\
			&= \boxed{\SI{0.409}{\Molar}}
		\end{align*}}
\end{frame}

\begin{onyourown}
	If \SI{41.9}{\milli\liter} of a \SI{0.1000}{\Molar} \ch{HCl} solution
	were required to neutralize \SI{25.0}{\milli\liter} of a \ch{Mg(OH)2}
	solution, what is the concentration of the \ch{Mg(OH)2}?
\end{onyourown}

\clearpage

\begin{frame}{Strong Acids and Bases}
	Like electrolytes, acids and bases can be either
	\alert{weak} or \alert{strong}.
	\begin{reactions*}
		HCl\aq{} &-> H^{+}\aq{} + Cl^{-}\aq{} \\
		HF\aq{} &<=> H^{+}\aq{} + F^{-}\aq{}
	\end{reactions*}
	\begin{center}
		\footnotesize
	\begin{tabular} {l l}
		\toprule
		\bfseries Strong Acids & \bfseries Strong Bases
		\\
		\midrule
		Hydrochloric acid, \ch{HCl} & Group 1A metal
		hydroxides \\
		Hydrobromic acid, \ch{HBr} & [\ch{LiOH},
		\ch{NaOH}, \ch{KOH}, \ch{RbOH}, \ch{CsOH}] \\
		Hydroiodic acid, \ch{HI} & Heavy group 2A metal
		hydroxides \\
		Chloric acid, \ch{HClO3} & [\ch{Ca(OH)2},
		\ch{Sr(OH)2}, \ch{Ba(OH)2}] \\
		Perchloric acid, \ch{HClO4} \\
		Nitric acid, \ch{HNO3} \\
		Sulfuric acid (first proton), \ch{H2SO4} \\
		\bottomrule
	\end{tabular}
	\end{center}

	When titrating, we want to use \alert{strong} acids or bases. Why? 
\end{frame}

\section{Gas Evolution Reactions}

\begin{frame}{Learning Objectives}
	\begin{itemize}
		\item Identify when a product will be in a gaseous state.
		\item Write net ionic equations for gas evolution reactions.
		\item Define reaction intermediate.
		\item Balance gas evolution reactions.
	\end{itemize}
\end{frame}

\begin{frame}{Gas-Evolution Reactions}
	In precipitation reactions, we see a solid form:
	\begin{reaction*}
		Na2SO4\aq{} + Ba(NO3)2\aq{} -> BaSO4\sld{} + 2 NaNO3\aq{}
	\end{reaction*}

	\pause

	In \alert{gas-evolution} reactions, we see a \alert{gas} form:
	\begin{reaction*}
		H2SO4\aq{} + Li2S\aq{} -> H2S\gas{} + Li2SO4\aq{}
	\end{reaction*}

	\pause

	A common example is acids reacting with metals to produce hydrogen gas
	and the metal salt:
	\begin{reactions*}
		2 K\sld{} + 2 HCl\aq{} &-> 2 KCl\aq{} + H2\gas{} \\
		Zn\sld{} + 2 HCl\aq{} &-> ZnCl2\aq{} + H2\gas{} \\
	\end{reactions*}

	\pause

	Can we write net ionic equations for gas evolution reactions?
\end{frame}

\begin{frame}[t]{Intermediate Products Leading to Gas Evolution}
	Occasionally, a product is not stable and \alert{decomposes} to form a
	gas.
	\begin{center}
		\small
	\begin{tabular} {l E E}
		\toprule
		\textbf{Reactant} & \textbf{Intermediate} & \textbf{Gas~Evolved}
		\\ \midrule
		Sulfides & None & H2S \\
		Carbonates and bicarbonates & H2CO3 & CO2 \\
		Sulfites and bisulfites & H2SO3 & SO2 \\
		Ammonium & NH4OH & NH3 \\
		\bottomrule
	\end{tabular}
	\end{center}

	\begin{example}
		What is the balanced reaction when \ch{HCl} reacts with
		\ch{NaHCO3}?
	\end{example}

	\note{%
	\begin{center}
		\begin{tabular} {L >{\collectcell\ch}r<{\endcollectcell} @{
				\ch{->} } E}
			& HCl\aq{} + NaHCO3\aq{} & !(intermediate~product)(
			H2CO3\aq{} ) + NaCl\aq{} \\
			+ & H2CO3\aq{} & H2O\lqd{} + CO2\gas{} \\ \midrule
			& HCl\aq{} + NaHCO3\aq{} & H2O\lqd{} + CO2\gas{} +
			NaCl\aq{}
		\end{tabular}
	\end{center}
}
\end{frame}

\vspace{\stretch{-1}}

\begin{frame}[t]{Writing Gas Evolution Reactions}
	Write a balanced molecular equation for the following reactions:

	\begin{enumerate}
		\item \ch{Al\sld{} + HCl\aq{}}
			\note[item]{\ch{2 Al\sld{} + 6 HCl\aq{} -> 2 AlCl3\aq{}
			+ 3 H2\gas{}}}

			\vspace{5em}

		\item \ch{NH4Cl\aq{} + KOH\aq{}}
			\note[item]{\ch{NH4Cl\aq{} + KOH\aq{} -> H2O\lqd{} +
			NH3\gas{} + KCl\aq{}}}

			\vspace{5em}
	\end{enumerate}
\end{frame}

\vspace{\stretch{-1}}

%\begin{onyourown}
%	Write a balanced molecular equation for the following reactions:
%
%	\begin{enumerate}
%		\item \ch{Fe\sld{} + HCl\aq{}}
%			\vspace{5em}
%		\item \ch{K2SO3\aq{} + HCl\aq{}}
%			\vspace{5em}
%	\end{enumerate}
%
%	Write the net ionic equations for the above reactions.
%\end{onyourown}

\section{Oxidation-Reduction Reactions}

\begin{frame}{Learning Objectives}
	\begin{itemize}
		\item Define oxidation and reduction.
		\item Assign oxidation numbers to atoms.
		\item Identify the transfer of electrons in oxidation-reduction reactions.
		\item Identify the oxidizing and reducing agents in a redox
			reaction.
		\item Write balanced half-reactions.
	\end{itemize}
\end{frame}

\begin{frame}{Oxidation-Reduction Reactions}
	When electrons transfer from one reactant to another, we have an
	\alert{oxidation-reduction} or \alert{redox} reaction.
	\begin{description}
		\item[Oxidation] is the \alert{loss} of electrons.
		\item[Reduction] is the \alert{gain} of electrons.
	\end{description}

	\pause

	The gas-evolution reactions just considered are examples of redox
	reactions:
	\begin{reaction*}
		Zn\sld{} + 2 HCl\aq{} -> ZnCl2\aq{} + H2\gas{}
	\end{reaction*}

	\pause

	\begin{center}
		Why? How do we know?
		\pause
		There \emph{must} be a change in charge!
	\end{center}

	\begin{block}{Oxidation State/Number}
		The ``charge'' an atom would have if all shared electrons were
		assigned to the more electronegative atom.
	\end{block}
\end{frame}

\vspace{\stretch{-1}}

\begin{frame}{Rules for Assigning Oxidation Numbers}
	\begin{enumerate}[<+->]
		\item The oxidation state of an atom in a free element
			is 0.
			\note<.>{
				\begin{reaction*}
					!(0~ox~state)( Cu ) 
					\qquad\qquad 
					!(0~ox~state)( Cl2 )
				\end{reaction*}
				}
		\item The oxidation state of a monatomic ion is equal to
			its charge.
			\note<.>{
				\begin{reaction*}
					!(\num{+2}~ox~state)( Ca^{2+} )
					\qquad\qquad
					!(\num{-1}~ox~state)( Cl^{-} )
				\end{reaction*}
				}
		\item The sum of the oxidation states of all atoms in a
			species must match the charge of the species.
			\begin{itemize}
				\item A neutral molecule or formula unit
					is 0.
					\note<.>{
						\begin{reaction*}
							!({$2 (\text{H~ox~state})
							+ 1 (\text{O~ox~state})
							= 0$})( H2O )
						\end{reaction*}
						}
				\item An ion is equal to the charge of
					the ion.
					\note<.>{
						\begin{reaction*}
							!({$1 (\text{N~ox~state})
							+ 3 (\text{O~ox~state})
							= -1$})( NO3- )
						\end{reaction*}
						}
			\end{itemize}
		\item Metals always have positive oxidation states.
			\note<.>{
				\begin{reaction*}
					!(\num{+1}~ox~state)(
					\textbf{\usebeamercolor[fg]{alerted
					text}Na}Cl )
					\qquad\qquad
					!(\num{+2}~ox~state)( \textbf{Ca}F2 )
				\end{reaction*}
				}
		\item Nonmetals are assigned oxidation states as follows:
			\begin{center}
				\small
				\sisetup{table-format=-1}
				\begin{tabular} {l S e | l S e}
					\toprule
					\textbf{Nonmetal} & \textbf{ON} & \textbf{Example} &
					\textbf{Nonmetal} & \textbf{ON} & \textbf{Example} \\
					\midrule
					Fluorine & -1 & MgF2 & Group 7A & -1 & CCl4 \\
					Hydrogen & +1 & H2O  & Group 6A & -2 & H2S \\
					Oxygen & -2 & CO2    & Group 5A & -3 & NH3 \\
					\bottomrule
				\end{tabular}
			\end{center}
	\end{enumerate}
\end{frame}

\begin{frame}[t]{Assigning Oxidation Numbers}
	Assign oxidation numbers for all elements in the following compounds:

	\begin{multicols}{2}
	\begin{enumerate}
		\item \ch{BaCl2}
		\item \ch{PO4^{3-}}
		\item \ch{Zn\sld{}}
		\item \ch{ZnCl2}
		\item \ch{HCl}
		\item \ch{H2}
	\end{enumerate}
	\end{multicols}

	\pause

	What is happening to oxidation numbers in the reaction of Zn with HCl?
	\begin{reaction*}
		Zn\sld{} + 2 HCl\aq{} -> ZnCl2\aq{} + H2\gas{}
	\end{reaction*}

	\pause

	\begin{block}{Identifying Redox Reactions}
		At \emph{any} time in which we have a \alert{change} in oxidation
		number, we know we have a \alert{redox} reaction.
	\end{block}
\end{frame}

\vspace{\stretch{-1}}

\begin{frame}{Intro to Balancing Redox Reactions}
	In the reaction,
	\begin{reaction*}
		Zn\sld{} + 2 HCl\aq{} -> ZnCl2\aq{} + H2\gas{}
	\end{reaction*}

	\begin{itemize}
		\item \ch{Zn} is \alert{oxidized}:
			\begin{reaction*}
				Zn\sld{} -> Zn^{2+}\aq{} + 2 \el{}
			\end{reaction*}
		\item \ch{H+} is \alert{reduced}:
			\begin{reaction*}
				2 H^{+}\aq{} + 2 \el{} -> H2\gas{}
			\end{reaction*}
	\end{itemize}

	\pause

	Note that an oxidation reaction \emph{must} occur for every reduction
	reaction (and vice versa), because we need a \alert{stoichiometric}
	amount of electrons to \alert{transfer}!
	\begin{itemize}
		\item \ch{Zn} is the \alert{reducing agent}.
		\item \ch{H+} is the \alert{oxidizing agent}.
	\end{itemize}
\end{frame}

\begin{frame}[t]{Oxidation-Reduction Identification}
	Identify each of the following as either oxidation or reduction:

	\begin{enumerate}
		\item \ch{Sn\sld{} -> Sn^{4+} + 4 \el{}}

			\bigskip

		\item \ch{Fe^{3+}\aq{} + \el{} -> Fe^{2+}}

			\bigskip

		\item \ch{Cl2\gas{} + 2 \el{} -> 2 Cl^{-}\aq{}}

			\bigskip
	\end{enumerate}
\end{frame}

\vspace{\stretch{-1}}

\begin{onyourown}[0em]
	Identify each of the following as either oxidation or reduction:

	\begin{enumerate}
		\item \ch{3 I^{-}\aq{} -> I3^{-}\aq{} + 2 \el{}}

			\bigskip

		\item \ch{Br2\gas{} + 2 \el{} -> 2 Br^{-}\aq{}}
			\bigskip
	\end{enumerate}
\end{onyourown}

\begin{frame}[t]{Writing Half-Reactions}
	Write the separate oxidation and reduction reactions for the following
	equation:
	\begin{reaction*}
		2 Cs\sld{} + ZnCl2\aq{} -> Zn\sld{} + 2 CsCl\aq{}
	\end{reaction*}

	\note{
		\begin{reactions*}
			2 Cs\sld{} &-> 2 Cs^{+}\sld{} + 2 \el{} \\
			Zn^{2+}\aq{} + 2 \el{} &-> Zn\sld{}
		\end{reactions*}
		}
\end{frame}

%\begin{onyourown}
%	Write the separate oxidation and reduction reactions for the following
%	equation:
%	\begin{reaction*}
%		2 Fe\sld{} + 3 CuBr2\aq{} -> 2 FeBr3\aq{} + 3 Cu\sld{}
%	\end{reaction*}
%\end{onyourown}

\section{Extra Practice}

\begin{frame}[t]{Some Challenging Examples}
	\begin{enumerate}[<+->]
		\item Write the balanced chemical equation for the reaction of
			potassium with water.

			\vfill

			\note<3>[item]{
				\begin{reaction*}
					2 K\sld{} + 2 H2O\lqd{} -> 2 KOH\aq{} +
					H2\gas{}
				\end{reaction*}
				}

		\item Write the balanced chemical reaction for aluminum with
			bromine gas.

			\vfill

			\note<3>[item]{
				\begin{reaction*}
					2 Al\sld{} + 3 Br2\gas{} -> 2 AlBr3\aq{}
				\end{reaction*}
				}

		\item Write the balanced chemical equation for the combustion of
			propane (\ch{C3H8}).

			\vfill

			\note<3>[item]{
				\begin{reaction*}
					C3H8\gas{} + 5 O2\gas{} -> 3 CO2\gas{}
					+ 4 H2O\lqd{}
				\end{reaction*}
				}
	\end{enumerate}
\end{frame}

\end{document}
