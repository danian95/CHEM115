%! TEX program = xelatex
\documentclass[11pt,letterpaper]{article}

\usepackage{genchem}
\usepackage{enumitem}
\usepackage[margin=1in]{geometry}
\usepackage{titling}

\setallmainfonts{TeX Gyre Pagella}

\title{Chapter 8 ``On Your Own'' Solutions}

\begin{document}

\begin{center}
	\bfseries
	\Large
	\thetitle
\end{center}

\begin{enumerate}[itemsep=2em,leftmargin=0pt,label=\textbf{\Alph*.}]
	\item Multiply volume by concentration:
		\begin{align*}
			\SI{25.00}{\milli\liter} \times
			\frac{\SI{0.100}{\mole}}{\SI{1}{\liter}} \times
			\frac{\SI{1}{\liter}}{\SI{1000}{\milli\liter}} &=
			\boxed{\SI{0.00250}{\mole}}
		\end{align*}

	\item First, we need to find how many moles of \ch{HNO3} are in
		\SI{24.0}{\gram}:
		\begin{align*}
			\SI{24.0}{\gram} \times
			\frac{\SI{1}{\mole}}{\SI{63.012}{\gram}} &=
			\SI{0.380879832}{\mole}
		\end{align*}

		Then, divide the number of moles by the concentration to find
		the volume:
		\begin{align*}
			\SI{0.380879832}{\mole} \times
			\frac{\SI{1}{\liter}}{\SI{2.00}{\mole}} \times
			\frac{\SI{1000}{\milli\liter}}{\SI{1}{\liter}} &=
			\boxed{\SI{190.}{\milli\liter}}
		\end{align*}
		
	\item Make a table of values:

		\begin{tabular} {L S[table-format=2.3] s}
			C_\text{stock} & 0.250 & \Molar \\
			V_\text{stock} & 1.00 & \milli\liter \\
			V_\text{dilute} & 40.0 & \milli\liter
		\end{tabular}

		Rearrange dilution equation and solve:
		\begin{align*}
			C_\text{stock}V_\text{stock} &=
			C_\text{dilute}V_\text{dilute} \\
			C_\text{dilute} &=
			\frac{C_\text{stock}V_\text{stock}}{V_\text{dilute}} \\
			&=
			\frac{(\SI{0.250}{\Molar})(\SI{1.00}{\milli\liter})}{\SI{40.0}{\milli\liter}}
			\\
			&= \boxed{\SI{0.00625}{\Molar}}
		\end{align*}

	\item We need to find out how many moles of \ch{MgCl2} we add:
		\begin{align*}
			\frac{\SI{0.100}{\mole}}{\SI{1}{\liter}} \times
			\SI{22.8}{\milli\liter} \times
			\frac{\SI{1}{\liter}}{\SI{1000}{\milli\liter}} &=
			\SI{0.00228}{\mole}
		\end{align*}

		Then, we need to know how many moles of \ch{AgNO3} will react
		completely with this amount:
		\begin{align*}
			\SI{0.00228}{\mole}~\ch{MgCl2} \times
			\underbrace{\frac{\SI{2}{\mole}~\ch{AgNO3}}{\SI{1}{\mole}~\ch{MgCl2}}}_{\mathclap{\text{From
			stoichiometry in reaction.}}}
			&= \SI{0.00456}{\mole}~\ch{AgNO3}
		\end{align*}

		Finally, divide the moles of \ch{AgNO3} needed by the volume of
		the \ch{AgNO3} solution added to get the concentration:
		\begin{align*}
			\SI{0.00456}{\mole}~\ch{AgNO3} \times
			\frac{1}{\SI{15.0}{\milli\liter}} \times
			\frac{\SI{1000}{\milli\liter}}{\SI{1}{\liter}} &=
			\boxed{\SI{0.304}{\Molar}~\ch{AgNO3}}
		\end{align*}

	\item We need to see if each compound in the reactants is a strong
		electrolyte or a weak electrolyte.
		\begin{enumerate}[label={\arabic*.},itemsep=2em]
			\item \ch{Na2SO4\sld{} ->[ H2O ] 2 Na^{+}\aq{} +
				SO4^{2-}\aq{}}

				\ch{Na2SO4} is considered a ``soluble ionic
				compound'', so it is a strong electrolyte.
			\item \ch{HCN\aq{} <=>[ H2O ] H^{+}\aq{} +
				CN^{-}\aq{}}

				\ch{HCN} is a weak electrolyte, so ``equilibrium
				arrows'' (\ch{<=>}) are necessary.
			\item \ch{C3H8O3\aq{} -/>}

				\ch{C3H8O3} is an organic compound, considered a
				nonelectrolyte.
		\end{enumerate}
	\item The names of each acid or base are as follows:

		\begin{tabular} {r E l}
			1. & H2S & hydrosulfuric acid (note that it is
			\emph{not} sulfuric acid) \\
			2. & HBr & hydrobromic acid \\
			3. & CsOH & cesium hydroxide \\
			4. & H2SO3 & sulfurous acid \\
			5. & Cu(OH)2 & copper(II) hydroxide (the II is
			important!) \\
			6. & HC2H3O2 & acetic acid
		\end{tabular}

	\item The balanced equation is
		\begin{reaction*}
			3 Ba(OH)2\aq{} + 2 H3PO4\aq{} -> Ba3(PO4)2\sld{} + 6
			H2O\lqd{}
		\end{reaction*}

	\item We need to consider the balanced reaction:
		\begin{reaction*}
			Mg(OH)2\aq{} + 2 HCl\aq{} -> MgCl2\aq{} + 2 H2O\lqd{}
		\end{reaction*}
		\begin{align*}
			\intertext{Find moles of HCl added:}
			\frac{\SI{0.1000}{\mole}~\ch{HCl}}{\SI{1}{\liter}}
			\times \SI{41.9}{\milli\liter} \times
			\frac{\SI{1}{\liter}}{\SI{1000}{\milli\liter}} &=
			\SI{0.00419}{\mole}~\ch{HCl} \\
			\intertext{From stoichiometry:}
			\SI{0.00419}{\mole}~\ch{HCl} \times
			\frac{\SI{1}{\mole}~\ch{Mg(OH)2}}{\SI{2}{\mole}~\ch{HCl}}
			&= \SI{0.002095}{\mole}~\ch{Mg(OH)2} \\
			\intertext{Solve for concentration:}
			\frac{\SI{0.002095}{\mole}~\ch{Mg(OH)2}}{\SI{25.0}{\milli\liter}}
			\times \frac{\SI{1000}{\milli\liter}}{\SI{1}{\liter}} &=
			\boxed{\SI{0.0838}{\Molar}}
		\end{align*}

		\pagebreak

	\item Balanced molecular and net ionic equations for each reaction are
		as follows:

		\begin{tabular} {r >{\bfseries}r
				>{\collectcell\ch}r<{\endcollectcell} @{ \ch{->}
			} E}
			1. & Molecular: & Fe\sld{} + 2 HCl\aq{} & H2\gas{} +
			FeCl2\aq{} \\
			& Net Ionic: & Fe\sld{} + 2 H^{+}\aq{} &
			H2\gas{} + Fe^{2+}\aq{} \\
			& \multicolumn{3}{c}{\emph{and/or}} \\
			& Molecular: & 2 Fe\sld{} + 6 HCl\aq{} & 3 H2\gas{} +
			2 FeCl3\aq{} \\
			& Net Ionic: & Fe\sld{} + 6 H^{+}\aq{} &
			3 H2\gas{} + 2 Fe^{3+}\aq{} \\[2em]
			2. & Molecular: & K2SO3\aq{} + 2 HCl\aq{} & 
			SO2\gas{} + H2O\lqd{} + 2 KCl\aq{} \\
			& Net Ionic: & SO3^{2-}\aq{} + 2 H^{+}\aq{} &
			SO2\gas{} + H2O\lqd{} \\
			& \multicolumn{3}{c}{\emph{due to}} \\
			& & SO3^{2-}\aq{} + 2 H^{+}\aq{} &
			!(intermediate~product)( H2SO4\aq{} ) ->
			SO2\gas{} + H2O\lqd{}
		\end{tabular}

	\item \begin{enumerate}[label={\arabic*.}]
			\item Oxidation
			\item Reduction
		\end{enumerate}

	\item Half-reactions are as follows:

		\begin{tabular} {>{\bfseries}r
				>{\collectcell\ch}r<{\endcollectcell} @{ \ch{->}
			} E }
			Oxidation: & Fe\sld{} & Fe^{3+} + 3 \el{} \\
			Reduction: & Cu^{2+} + 2 \el{} & Cu\sld{} \\
			\\
			\multicolumn{3}{l}{For balance:} \\
			Oxidation: & 2 Fe\sld{} & 2 Fe^{3+} + 6 \el{} \\
			Reduction: & 3 Cu^{2+} + 6 \el{} & 3 Cu\sld{}
		\end{tabular}


\end{enumerate}

\end{document}
