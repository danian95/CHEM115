% !TEX program = xelatex
\documentclass[notes=only]{beamer}
%\documentclass[notes=hide]{beamer}
%\documentclass[notes=show]{beamer}
%\documentclass[handout]{beamer}

\usepackage{bucolors}
\usepackage{newtxtext}
\usepackage{genchem}
\usepackage{lecture}
\usepackage{multicol}
\usepackage{elements}
\usepackage{collcell}
\usepackage{chemfig}
\usetikzlibrary{tikzmark}

\renewcommand*\printangularmomentum[1]{#1}

\usepackage{orbitalfilling}[2019/02/14]

\title{Chemical Bonding I}
\subtitle{Chapter 5}
\institute[CHEM115 Bloomsburg University]{CHEM115 --- Chemistry for the Sciences I \\ Bloomsburg University}
\author{D.A.\ McCurry}
\date{Fall 2021}

\begin{document}

\maketitle
\mode<article>{\thispagestyle{fancy}}

\begin{frame}{Beyond the Lewis Model}
	\begin{itemize}
		\item We have represented molecules previously using Lewis
			Electron Dot Structures:

			\begin{center}
				\begin{tikzpicture}[node distance=-1ex]
					\node(C) at (0,0) {\chlewis{0:90:180:270}{C}};
					\node[left=of C] {\ch{H}};
					\node[below=of C] {\ch{H}};
					\node[right=of C] {\ch{H}};
					\node[above=of C] {\ch{H}};
					\node(N) at (2,0) {\chlewis{0:90:180:270}{N}};
					\node[left=of N] {\ch{H}};
					\node[below=of N] {\ch{H}};
					\node[right=of N] {\ch{H}};
					\node(O) at (4,0) {\chlewis{0:90:180:270}{O}};
					\node[left=of O] {\ch{H}};
					\node[right=of O] {\ch{H}};
					\node(F) at (6,0) {\chlewis{0:90:180:270}{F}};
					\node[left=of F] {\ch{H}};
				\end{tikzpicture}
			\end{center}

		\item But such structures only tell part of the story.
		\item Both \ch{H2} and \ch{HF} \alert{share} the same number of
			electrons:

			\begin{center}
				\begin{tikzpicture}[node distance=-1ex]
					\node(H) at (0,0) {\chlewis{0:}{H}};
					\node[right=of H] {\ch{H}};
					\node(F) at (2,0) {\chlewis{0:90:180:270}{F}};
					\node[left=of F] {\ch{H}};
				\end{tikzpicture}
			\end{center}

			But they have very different properties.
	\end{itemize}
\end{frame}

\frame{\section{Electronegativity and the Lewis Model}
	\begin{learningobjectives}
	\item Show how electronegativity differences can lead to polar or
		nonpolar molecules.
	\item Determine the most stable Lewis structures based on formal charge.
	\item Understand how resonance plays a part in stability and bond
		polarity.
	\end{learningobjectives}
}

\begin{frame}[t]{Bond Polarity}
	Let's take a look at the \alert{electrostatic potential map} for
	\ch{HF}:
	\begin{center}
		\includegraphics[width=0.75\linewidth]{05_02_Figure.jpg}
	\end{center}

	\note{%
	\begin{itemize}
		\item Many more electrons are surrounding F in HF -- an
			\alert{electrostatic potential map} appears as
		\item F has a much higher attraction towards electrons, thus the
			bonding pair is \alert{unequally shared} between the
			atoms.

			\begin{center}
				\chemfig{
					\chemabove[3pt]{\ch{H}}{\delp}(-[::270,0.5,,,draw=none]@{a})-
					\chemabove[3pt]{\ch{F}}{\delm}(-[::270,0.5,,,draw=none]@{b})
			        }
				\chemmove{
			          \draw[|->,thick] (a)--(b);
				  \draw[-,thick] (a) -- ++(-0.75ex,0);
			         } 
			\end{center}

			\medskip

			We represent this unequal sharing with a \alert{dipole
			moment vector} or \alert{partial charges}.
		\item Atoms that attract electrons moreso than other atoms in
			the same molecule will form \alert{polar covalent bonds}.
	\end{itemize}
}
\end{frame}

%\begin{frame}
%	\begin{block}{Electronegativity}
%		The ability of an atom to attract electrons to itself in a
%		chemical bond. Results in polar covalent and ionic bonds.
%	\end{block}
%
%	\begin{center}
%		\includegraphics[scale=0.40,trim={0 0 0
%		30pt},clip]{05_03_Figure.jpg}
%	\end{center}
%\end{frame}
%
%\begin{frame}[t]{Electronegativity Example}
%	Arrange the following elements in order of \alert{decreasing}
%	electronegativity:
%
%	\begin{center}
%		\ch{Na} \qquad \ch{Al} \qquad{Mg}
%	\end{center}
%
%	\mode<article>{\vspace*{5em}}
%
%	\note{$\ch{Al} > \ch{Mg} > \ch{Na}$}
%\end{frame}
%
%\begin{onyourown}{5em}
%	Arrange the following elements in order of \alert{increasing}
%	electronegativity:
%	
%	\begin{center}
%		\ch{C} \qquad \ch{Al} \qquad \ch{Si}
%	\end{center}
%\end{onyourown}
%
%\clearpage

\begin{frame}[allowframebreaks]{Degree of Polarity}
	Different elements will exhibit different \alert{degrees of
	polarity}
	\begin{itemize}
		\item A \alert{larger} electronegativity
			difference will lead to a \alert{more
			polar} bond.
		\item A \alert{smaller} electronegativity
			difference will lead to a \alert{less
			polar} bond.
	\end{itemize}

	\bigskip

	\begin{example}
		What is the electronegativity difference in \ch{HF}?
		$\text{EN}_\text{F} = 4.0$ and $\text{EN}_\text{H} = 2.1$.
	\end{example}

	\note<1>{
		\begin{equation*}
			\Delta \text{EN} = 4.0 - 2.1 = 1.9
		\end{equation*}
	}

	\framebreak

	We then classify this difference according to this (fairly arbitrary)
	scale:
	\begin{center}
		\includegraphics[scale=0.3,trim={0 0 0
		50pt},clip]{05_04_Figure.jpg}

		Note that the degree of polarity is a
		\alert{continuum}!
	\end{center}
\end{frame}

\begin{frame}[t]{Polarity Examples}
	% McMurray Problem 4.4
	Classify the following compounds as having nonpolar covalent, polar
	covalent, or ionic bonds:

	\begin{columns}
		\column{0.45\linewidth}
		\begin{enumerate}
			\item \ch{SiCl4}
				\note[item]{$\Delta\text{EN} = 3.0 - 1.8 = 1.2$,
				polar covalent}
			\item \ch{CsBr}
				\note[item]{$\Delta\text{EN} = 2.8 - 0.7 = 2.1$,
				ionic}
			\item \ch{Cl2}
				\note[item]{$\Delta\text{EN} = 3.0 - 3.0 = 0.0$,
				nonpolar covalent}
			\item \ch{FeBr3}
				\note[item]{$\Delta\text{EN} = 2.8 - 1.8 = 1.0$,
				polar covalent}
		\end{enumerate}
		\column{0.45\linewidth}
		\centering
		\begin{tabular} {c S[table-format=1.1]}
			\textbf{Element} & \textbf{EN} \\ \midrule
			Si & 1.8 \\
			Cl & 3.0 \\
			Cs & 0.7 \\
			Br & 2.8 \\
			Fe & 1.8 \\
		\end{tabular}
	\end{columns}

	\mode<article>{\vspace*{4em}}
\end{frame}

%\begin{onyourown}
%	% McMurray Problem 4.5
%	Order the following compounds according to the increasing ionic
%	character of their bonds:
%
%	\begin{enumerate}
%		\item \ch{CCl4}
%		\item \ch{BaCl2}
%		\item \ch{TiCl3}
%		\item \ch{Cl2O}
%	\end{enumerate}
%\end{onyourown}

%\begin{frame}{The Dipole Moment}
%	A dipole moment represents both \alert{magnitude} and \alert{direction}.
%	\begin{itemize}
%		\item The \alert{magnitude} can be calculated as
%			\begin{equation*}
%				\mu = \tikzmark{charge}qr
%			\end{equation*}
%			where
%			\begin{description}
%				\item[$\mu$] is the dipole moment measured in
%					debye ($\SI{1}{\debye} =
%					\SI{3.34e-30}{\coulomb\meter}$),
%				\item[$q$] is the \alert{magnitude} of equal (but
%					opposite) charges, and
%				\item[$r$] is the distance over which the
%					charges are separated.
%			\end{description}
%		\item The \alert{direction} is always \alert{towards} the more
%			negative charge.
%	\end{itemize}
%
%	\pause
%
%	\begin{tikzpicture}[remember picture, overlay]
%		\node[right,text width=10em,align=left,font=\footnotesize](note)
%		at ($(pic cs:charge) + (1,-0.3)$) {In our case, $q$ is the charge
%		of an electron, \SI{1.6e-19}{\coulomb}.};
%		\draw[thick,<-,shorten <=3pt] ($(pic cs:charge) + (1ex,0)$) to
%		[in=180,out=300] (note);
%	\end{tikzpicture}
%\end{frame}
%
%\begin{frame}{Percent Ionic Character}
%	If the degree of polarity is a continuum rather than finite ionic or
%	covalent bonds, we can estimate the \alert{percent ionic character}:
%
%	\begin{equation*}
%		\footnotesize
%		\text{\% ionic character} = \frac{\text{measured
%		dipole moment}}{\text{dipole moment assuming
%		full transfer of \el{}}} \times 100\%
%	\end{equation*}
%
%	\bigskip
%
%	In other words, $\mu = qr$ assumes the bond is ionic. After
%	experimentally measuring $\mu$, how close to an ionic compound is it?
%
%	\pause
%
%	\begin{block}{Note:}
%		Don't worry about calculating this. We'll never use it.
%	\end{block}
%\end{frame}

%\begin{frame}[t]{Percent Ionic Character Example}
%	\ch{HBr} has a bond length of \SI{141}{\pico\meter} and a dipole moment
%	of \SI{0.82}{\debye}. What is the percent ionic character of the
%	\chemfig{\ch{H}-\ch{Br}} bond?
%
%	\vspace*{10em}
%
%	\note{
%		\begin{align*}
%			\mu &= qr \\
%			&= (\SI{1.6e-19}{\coulomb})(\SI{141}{\pico\meter}) \\
%			&= \SI{2.256e-17}{\coulomb\pico\meter} \times
%			\frac{\SI{1}{\meter}}{\SI{1e12}{\pico\meter}} \times
%			\frac{\SI{1}{\debye}}{\SI{3.34e-30}{\coulomb\meter}} \\
%			&= \SI{6.75}{\debye} \\
%			\text{\% ionic character} &= \frac{\text{measured
%			dipole moment}}{\text{dipole moment assuming
%			full transfer of \el{}}} \times 100\% \\
%			&= \frac{\SI{0.82}{\debye}}{\SI{6.75}{\debye}} \times
%			100\% \\
%			&= \boxed{\SI{12}{\percent}}
%		\end{align*}
%	}
%\end{frame}
%	
%\begin{onyourown}{10em}
%	\ch{HI} has a dipole moment of \SI{0.44}{\debye} and the percent ionic
%	character of the \chemfig{\ch{H}-\ch{I}} bond is \SI{5.7}{\percent}.
%	What is the length of the bond?
%\end{onyourown}
%
\begin{frame}{Back to Lewis Structures!}
	Considering electroneutrality now provides us with a little more
	direction in drawing Lewis structures. Specifically,
	\begin{itemize}
		\item Hydrogen atoms are always terminal.
		\item The \alert{least} electronegative element is generally in
			the central position.
	\end{itemize}

	For the Spartan lab,
	\begin{center}
		\footnotesize
		\renewcommand\arraystretch{1.5}
		\begin{tabular} {>{\centering\arraybackslash}m{0.5in}
			>{\centering\arraybackslash}m{1.5in} @{ NOT }
		>{\centering\arraybackslash}m{1.5in}}
			\ch{HCN} & \chemfig{\ch{H}-\ch{C}~\chlewis{0:}{N}} &
			\chemfig{\ch{H}-\ch{N}~\chlewis{0:}{C}} \\
			\ch{CH2O} &
			\chemfig{\ch{H}-[:30]\ch{C}(=[:90]\chlewis{30:150:}{O})-[:-30]\ch{H}}
			&
			\chemfig{\ch{H}-[:30]\ch{O}(=[:90]\chlewis{30:150:}{C})-[:-30]\ch{H}}
			\\
			\ch{OCS} &
			\chemfig{\chlewis{120:240:}{O}=\ch{C}=\chlewis{60:300:}{S}} &
			\chemfig{\chlewis{120:240:}{C}=\ch{O}=\chlewis{60:300:}{S}}
		\end{tabular}
	\end{center}
\end{frame}

\begin{frame}{Why is the least electronegative element central?}
	\begin{itemize}
		\item Electronegative elements don't like to share electrons --
			central atoms have to share the most.
		\item The most negative \alert{formal charge} should be on the
			most electronegative element.
	\end{itemize}

	\begin{block}{Formal Charge}
		The charge an atom would have if all bonding electrons were
		equally shared. Ignores electronegativity.
		\begin{equation*}
			\text{FC} = \text{\# valence \el} -
			\left(\text{\# nonbonding \el} + \frac{1}{2}\text{\# bonding
			\el}\right)
		\end{equation*}
	\end{block}
\end{frame}

\begin{frame}{Another Way of Thinking About Formal Charge}
	\begin{itemize}
		\item If an atom has \alert{more} electrons than it does in its
			valence shell when \alert{isolated}, then the formal
			charge is going to be \alert{negative}.
		\item If an atom has \alert{less} electrons than it does in its
			valence shell when \alert{isolated}, then the formal
			charge is going to be \alert{positive}.
	\end{itemize}

	%\mode<article>{\clearpage}

	\noindent
	For example, \ch{H2O} has
	\begin{description}
		\item[8] electrons total
		\item[4] are in lone pairs and belong only to \ch{O}
		\item[4] are shared between \ch{O} and \ch{H}
		\item[2] of those 4 therefore ``belong'' to \ch{O}
	\end{description}
	As oxygen then has 6 of its ``own'' electrons and oxygen
	normally has 6 valence electrons, its formal charge is 0.
\end{frame}

\begin{frame}[t]{Calculating Formal Charge}
	What are the formal charges for every atom in a water (\water) molecule?
	\note{%	
	\begin{columns}
		\column{0.25\linewidth}
		\centering
		\mode<presentation>{\Large}
		\chemfig{\ch{H}-[:45]\chlewis{45:135:}{O}(-[:90,0.001,,,draw=none]@{c})-[:-45]\ch{H}}
		\chemmove{
			\draw<2->[dashed] (c) circle (1.65em);
			}
		\column{0.65\linewidth}
		\centering
		\begin{align*}
			\text{FC} &= \text{\# valence} - \left(\text{\# nonbonding}
			+ \frac{1}{2}\text{\# bonding}\right) \\
			&= 6 - \left(4 + \frac{1}{2}4\right) \\
			&= 6 - 6 = 0
		\end{align*}

		\textit{OR}

		\bigskip
	\end{columns}

	\begin{enumerate}
		\item Draw a circle around the atom.
		\item Note how the circle cuts bonds in half.
		\item Count the electrons in the circle.
		\item Subtract the counted electrons from the \# of
			valence electrons.
	\end{enumerate}
}
\end{frame}

\begin{frame}{Rules of Formal Charge}
	\begin{enumerate}
		\item Sum of all formal charges in a neutral molecule must be
			zero.

			\begin{center}
				\ch{CO2} \qquad $\text{FC} = 0$
			\end{center}

		\item Sum of all formal charges in an ion must equal the charge
			of the ion.

			\begin{center}
				\ch{SO4^{2-}} \qquad $\text{FC} = -2$
			\end{center}

		\item Small (or zero) formal charge on individual atoms are
			preferred.

		\item When formal charges cannot be avoided, negative formal
			charges should be on the most electronegative atom.
	\end{enumerate}
\end{frame}

\begin{frame}[t]{Formal Charge Example}
	Calculate the formal charge on each atom in the thiocyanate (\ch{SCN-})
	polyatomic ion.

	\mode<article>{\vspace*{10em}}

	\note{
		\begin{enumerate}
			\item Draw the lewis structure.
			\item Calculate the formal charges.
			\item Which is correct?
		\end{enumerate}

		\bigskip

		\begin{tabular} {*{3}{>{\qquad}c S[table-format=-1]@{\qquad}}}
			\multicolumn{2}{c}{$\chemleft[
				\chemfig{\chlewis{90:180:270:}{S}-\ch{C}~\chlewis{0:}{N}}
				\chemright]^{-}$}
				&
			\multicolumn{2}{c}{$\chemleft[
				\chemfig{\chlewis{180:}{S}~\ch{C}-\chlewis{270:90:0:}{N}}
				\chemright]^{-}$}
				&
			\multicolumn{2}{c}{$\chemleft[
				\chemfig{\chlewis{120:240:}{S}=\ch{C}=\chlewis{60:300:}{N}}
				\chemright]^{-}$} \\ [2em]
				S & -1 & S & +1 & S & 0 \\
				C & 0 & C & 0 & C & 0 \\
				N & 0 & N & -2 & N & -1
		\end{tabular}}
\end{frame}

%\begin{onyourown}
%	Calculate the formal charge on each atom in the sulfate (\ch{SO4^{2-}})
%	polyatomic ion.
%\end{onyourown}

\begin{frame}{Resonance}
	\begin{itemize}[<+->]
		\item When a compound has multiple possible arrangements of
			electrons (but the same skeletal structure), it is said
			to demonstrate \alert{resonance}.

			\bigskip

			\begin{center}
				\schemestart
				\chemfig{\chlewis{150:270:}{O}=[:30]\chlewis{90:}{O}-[:330]\chlewis{60:240:330:}{O}}
				\arrow{<->}
				\chemfig{\chlewis{120:210:300:}{O}-[:30]\chlewis{90:}{O}=[:330]\chlewis{270:30:}{O}}
				\schemestop
			\end{center}

			\bigskip

		\item This is \alert{convenient} to draw, but the \alert{actual}
			structure is an intermediate between the two -- a
			\alert{resonance hybrid}.

			\bigskip

			\begin{center}
				\chemfig{\ch{O}-[:30,,,,lddbond]\ch{O}-[:330,,,,lddbond]\ch{O}}
			\end{center}

			\bigskip
		\item Electrons are \alert{delocalized}.
	\end{itemize}
\end{frame}

\begin{frame}{Non-Equivalent Resonance Structures}
	\begin{center}
		\schemestart[0,0.6]
		$\chemleft[
		\chemfig[atom sep=2em]{\chlewis{90:180:270:}{S}-C~\chlewis{0:}{N}}
		\chemright]^{-}$
		\arrow{<->}
		$\chemleft[
		\chemfig[atom sep=2em]{\chlewis{180:}{S}~C-\chlewis{270:90:0:}{N}}
		\chemright]^{-}$
		\arrow{<->}
		$\chemleft[
		\chemfig[atom sep=2em]{\chlewis{120:240:}{S}=C=\chlewis{60:300:}{N}}
		\chemright]^{-}$
		\schemestop
	\end{center}

	\bigskip

	\begin{itemize}[<+->]
		\item Many compounds have resonance structures that are not
			equivalent in terms of stability.
		\item N is more electronegative than S --- how does this affect
			the electron distribution?
		\item All structures are correct, but one is lower in energy
			than the others (most stable).
	\end{itemize}
\end{frame}

\begin{frame}[t]{Determining the Most Stable Lewis Structure}
	Draw the possible resonance structures for phosgene (\ch{COCl2}). Which
	contributes the most to the correct structure?

	\mode<article>{\vspace*{14em}}

	\mode<presentation>{
	\bigskip

	\begin{columns}
		\column{0.5\linewidth}
		\column{0.4\linewidth}
		\includegraphics[scale=0.3]{phosgene.jpg}
	\end{columns}}

	\note{
		\begin{tabular} {*{3}{>{\qquad}c S[table-format=-1]@{\qquad}}}
			\multicolumn{2}{c}{
				\chemfig{\chlewis{90:180:270:}{Cl}-C(=[:90]\chlewis{30:150:}{O})-\chlewis{0:90:270:}{Cl}}}
				&
			\multicolumn{2}{c}{
				\chemfig{\chlewis{120:240:}{Cl}=C(-[:90]\chlewis{0:90:180:}{O})-\chlewis{270:90:0:}{Cl}}}
				&
			\multicolumn{2}{c}{
				\chemfig{\chlewis{90:180:270:}{Cl}-C(-[:90]\chlewis{0:90:180:}{O})=\chlewis{60:300:}{Cl}}}
			\\[2em]
				Cl & 0 & Cl & +1 & Cl & 0 \\
				Cl & 0 & Cl & 0 & Cl & +1 \\
				C & 0 & C & 0 & C & 0 \\
				O & 0 & O & -1 & O & -1
		\end{tabular}}
\end{frame}

%\begin{onyourown}
%	What are the possible Lewis structures for \ch{CO2}? Which is lowest in
%	energy?
%\end{onyourown}

\frame{\section{Advanced Lewis Structures}
	\begin{learningobjectives}
	\item Build Lewis structures with odd numbers of electrons.
	\item Build Lewis structures without complete octets.
	\item Build Lewis structures with more than 8 electrons in the
		``octet''.
	\end{learningobjectives}
}

\begin{frame}{Exceptions to the Octet Rule}
	\begin{enumerate}[<+(1)->]
		\item Odd numbers of electrons
			\begin{itemize}[<1->]
				\item Some species do not have even
					\alert{pairs}.
				\item The odd electron is known as a \alert{free
					radical}.
				\item These species are often very reactive.
			\end{itemize}
		\item Incomplete octets
			\begin{itemize}[<1->]
				\item Some atoms do not require a full octet.
				\item \ch{H}, \ch{B}, \ldots
			\end{itemize}
		\item Expanded octets
			\begin{itemize}[<1->]
				\item Some atoms can have more than 8 electrons.
				\item The electrons often occupy $d$ orbitals.
			\end{itemize}
	\end{enumerate}
\end{frame}

\begin{frame}{Free Radicals}
	A major concern regarding depletion of the ozone layer in the 1970s lead
	to the banning of many consumer products that contained
	chlorofluorocarbons (CFCs) in 1989.

	\bigskip

	\begin{columns}
		\column{0.45\linewidth}
		\begin{reactions*}
			\text{CFCs} &->[$hv$] Cl^. \\
			Cl^. + O3 &->[\hphantom{$hv$}] ClO^. + O2 \\
			O3 &->[$hv$] O + O2 \\
			O + ClO^. &->[\hphantom{$hv$}] Cl^. + O2 \\
			2 O3 &->[$hv$] 3 O2
		\end{reactions*}
		\column{0.45\linewidth}
		\centering
		\includegraphics[scale=0.15]{ozone-hole.png}
	\end{columns}

	\bigskip

	What is the Lewis structure for \ch{ClO^.}?
\end{frame}

\begin{frame}{Incomplete Octets}
	\begin{itemize}[<+->]
		\item Boron does not require a complete octet for an appropriate
			Lewis structure.

			\begin{center}
				\tikzmark{textbookfix}\includegraphics[scale=0.18]{05_Pg218_UnFigure_1.jpg}
			\end{center}
		
		\item For a compound such as \ch{BF3}, we \alert{could} create a
			double bond to form a complete octet around boron.

			\begin{center}
				\alt<1>{\includegraphics[scale=0.1]{05_Pg218_UnFigure_2.jpg}}{\includegraphics[scale=0.1]{05_Pg218_UnFigure_3.jpg}}
			\end{center}
		\item This would create a positive formal charge on \ch{F},
			which is \alert{very unfavorable}.
	\end{itemize}

	\begin{tikzpicture}[remember picture, overlay]
		\node[align=right,text width=8em,font=\footnotesize](note) at
		($(pic cs:textbookfix) + (-2,0.5)$) {Fix this figure on Pg. 218
		of your textbook!};
		\draw[<-,thick,shorten <=3pt] ($(pic cs:textbookfix) + (0,0.5)$)
		to (note.east);
	\end{tikzpicture}
\end{frame}

\begin{frame}[t]{Expanded Octets}
	\begin{itemize}
		\item Third period elements and below can hold more than 8
			electrons.
		\item Electrons fill in to $d$ orbitals.
		\item In drawing Lewis structures, make sure to note formal
			charges.
	\end{itemize}

	\pause

	\bigskip

	Let's take a look at the Lewis structure for \ch{H2SO4}\ldots

	\note<2>{%
		\begin{center}
			\begin{tabular} {>{\centering\arraybackslash}m{1.5in}@{ \ch{<->}
				}>{\centering\arraybackslash}m{1.5in}}
				\includegraphics[scale=0.2]{05_Pg219_UnFigure_3.jpg}
				&
				\includegraphics[scale=0.2]{05_Pg219_UnFigure_4.jpg}
			\end{tabular}
		\end{center}
	}

	\vfill

	\pause

	How do we know which structure is correct experimentally?

\end{frame}

\frame{\section{Translating Structure to Physical Properties}
	\begin{learningobjectives}
	\item Describe how different elements and different bond orders can
		affect the overall stability of a molecule.
	\item Explain the different geometries of molecules according to the
		VSEPR model.
	\item Relate the above phenomenon to predict the overall polarity of molecules.
	\end{learningobjectives}
}

\begin{frame}[allowframebreaks=1]{Bond Energies and Bond Lengths}
	\begin{columns}
		\column{0.35\textwidth}
		\begin{block}{Bond Energy}
			The energy required to break \SI{1}{\mole} of the chemical bonds
			in the gas phase. \alert{Always} positive --- energy must be
			added to break bonds.
		\end{block}
		\column{0.65\textwidth}
		\begin{center}
			\includegraphics[scale=0.3]{05_03_Table.jpg}
		\end{center}
	\end{columns}

	\begin{itemize}
		\item The stability of a bond can be measured in terms of the
			energy required to break the bond.
	\end{itemize}

	\framebreak

	\begin{columns}
	\column{0.35\textwidth}
	\begin{block}{Bond Length}
		The length of a chemical bond between two particular atoms.
	\end{block}
\column{0.65\textwidth}
	\begin{center}
		\includegraphics[scale=0.3]{05_04_Table.jpg}
	\end{center}
\end{columns}
	\begin{itemize}
		\item For \ch{H2SO4}, it has been experimentally shown that the
			\chemfig{[,0.75]S=O} bonds are shorter than the
			\chemfig{[,0.75]S-O}
			bonds.
		\item Both bond length and bond energy depend on the type of
			elements present in the compound.
	\end{itemize}
\end{frame}

\begin{frame}[t]{Bond Energies Example}
	Rank the following in order of \alert{increasing} average bond energies:

	\begin{itemize}
		\item \chemfig{[,0.75]C-C} \mode<presentation>{\qquad
			\alt<+(1)->{\alert{\SI{347}{\kilo\joule\per\mole}}}{}}
		\item \chemfig{[,0.75]C=C} \mode<presentation>{\qquad
			\alt<+(1)->{\alert{\SI{611}{\kilo\joule\per\mole}}}{}}
		\item \chemfig{[,0.75]C~C} \mode<presentation>{\qquad
			\alt<+(1)->{\alert{\SI{837}{\kilo\joule\per\mole}}}{}}
	\end{itemize}
\end{frame}

\begin{frame}[t]{Bond Lengths Example}
	Rank the following in order of \alert{decreasing} average bond length:

	\begin{itemize}
		\item \chemfig{[0,0.75]H-F}~  \mode<presentation>{\qquad
			\alt<5->{\alert{\SI{92}{\pico\meter}}}{}}
		\item \chemfig{[0,0.75]H-Cl}  \mode<presentation>{\qquad
			\alt<4->{\alert{\SI{127}{\pico\meter}}}{}}
		\item \chemfig{[0,0.75]H-Br}  \mode<presentation>{\qquad
			\alt<3->{\alert{\SI{141}{\pico\meter}}}{}}
		\item \chemfig{[0,0.75]H-I}~   \mode<presentation>{\qquad
			\alt<2->{\alert{\SI{161}{\pico\meter}}}{}}
	\end{itemize}
\end{frame}

%\begin{onyourown}
%	Rank the following in order of \alert{increasing} stability. Use trends
%	in expected bond lengths to guide your reasoning.
%
%	\begin{itemize}
%		\item \chemfig{[0,0.75]Cl-Cl}
%		\item \chemfig{[0,0.75]Br-Br}
%		\item \chemfig{[0,0.75]I-I}
%	\end{itemize}
%\end{onyourown}

\begin{frame}{Molecular Shape: The VSEPR Theory}

	\textbf{VSEPR: V}alence-\textbf{S}hell \textbf{E}lectron-\textbf{P}air
	\textbf{R}epulsion

	\begin{itemize}
		\item Electrons repel each other
		\item Electron dense areas surrounding an atom (the bonds or
			lone pairs) try to stay as far apart as possible
		\item Molecules assume specific shapes according to these
			repulsive forces (think Coulomb's Law)
	\end{itemize}

	\pause
	\begin{center}
		We can use \alert{Lewis structures} to help determine
		molecular \alert{shape}!
	\end{center}
\end{frame}

\begin{frame}{Predicting Molecular Geometry}
	Predict the shape of the polyatomic anion, \ch{CrO4^{2-}}.

	\bigskip

	\begin{description}[<+(1)->]
		\item[Step 1:] Draw the Lewis electron-dot structure.
		\item[Step 2:] Count the number of lone pairs and bonds.
			\begin{itemize}[<1->]
				\item Double and triple bonds will be counted as
					one bond.
				\item We will consider this number as the number
					of \alert{electron groups}.
			\end{itemize} 
		\item[Step 3:] Assume each charge cloud is as far away from the
			other charge clouds as possible.
	\end{description}
\end{frame}

\vspace{\stretch{-1}}

\begin{frame}{Two Electron Groups}
	\begin{center}
		\includegraphics[scale=0.4]{McMurray/05_Pg169_UnFigure.jpg}
	\end{center}

	\bigskip
	
	\begin{itemize}
		\item \alert{Linear} molecules.
		\item Bond angles of \SI{180}{\degree}.
	\end{itemize}
\end{frame}

\vspace{\stretch{-1}}

\begin{frame}{Three Electron Groups}
	\begin{columns}
		\column{0.55\linewidth}
		\begin{center}
			\includegraphics[scale=0.35]{McMurray/05_Pg170_UnFigure.jpg}
		\end{center}
		\column{0.35\linewidth}
		\begin{itemize}
			\item Molecule can be either \textbf{trigonal planar} or
				\textbf{bent}.
			\item Bond angles are \alert{approximately}
				\SI{120}{\degree}.
		\end{itemize}
	\end{columns}
\end{frame}

\vspace{\stretch{-1}}

\begin{frame}[allowframebreaks]{Four Electron Groups}
	\begin{center}
		\includegraphics[scale=0.4]{McMurray/05_01_Figure.jpg}
	\end{center}

	\bigskip
	
	\begin{itemize}
		\item Molecule can be \textbf{tetrahedral},
			\textbf{trigonal pyramidal}, or \textbf{bent}
		\item Bond angles are \alert{approximately}
			\SI{109.5}{\degree}
	\end{itemize}

	\framebreak

	\begin{center}
		\includegraphics[scale=0.4]{McMurray/05_Pg171_UnFigure_1.jpg}
	\end{center}
\end{frame}

\vspace{\stretch{-1}}

\begin{frame}[allowframebreaks]{Five Electron Groups}
	\begin{center}
		\includegraphics[scale=0.4]{McMurray/05_Pg171_UnFigure_2.jpg}
	\end{center}

	\bigskip

	\begin{itemize}
		\item Molecules can be \textbf{trigonal bipyramidal},
			\textbf{seesaw}, \textbf{T-shaped},
			or \textbf{linear}
		\item Bond angles are \SI{120}{\degree} between
			\alert{equatorial} positions and \SI{90}{\degree}
			between \alert{equatorial} and \alert{axial}
			positions
	\end{itemize}

	\framebreak

	\begin{center}
		\includegraphics[scale=0.4]{McMurray/05_Pg172_UnFigure_1.jpg}

		\bigskip

		\includegraphics[scale=0.4]{McMurray/05_Pg172_UnFigure_2.jpg}
	\end{center}

	\framebreak

	\begin{center}
		\includegraphics[scale=0.4]{McMurray/05_Pg172_UnFigure_3.jpg}

		\bigskip

		\includegraphics[scale=0.2]{McMurray/05_Pg172_UnFigure_4.jpg}
	\end{center}
\end{frame}

\vspace{\stretch{-1}}

\begin{frame}[allowframebreaks]{Six Electron Groups}
	\begin{center}
		\includegraphics[scale=0.4]{McMurray/05_Pg173_UnFigure_1.jpg}
	\end{center}

	\bigskip
	
	\begin{itemize}
		\item Molecules are \textbf{octahedral}, \textbf{square
			pyramidal}, or \textbf{square planar}
		\item \alert{All} bond angles are \SI{90}{\degree}
	\end{itemize}

	\framebreak

	\begin{center}
		\includegraphics[scale=0.4]{McMurray/05_Pg173_UnFigure_2.jpg}
	\end{center}

	\framebreak

	\begin{center}
		\includegraphics[scale=0.4]{McMurray/05_Pg173_UnFigure_3.jpg}

		\bigskip
		
		\includegraphics[scale=0.4]{McMurray/05_Pg173_UnFigure_4.jpg}
	\end{center}
\end{frame}

\mode<presentation|article:0>{
\begin{frame}{Predicting Molecular Geometry}
	Predict the shape of the polyatomic anion, \ch{CrO4^{2-}}.

	\bigskip

	\begin{description}
		\item[Step 1:] Draw the Lewis electron-dot structure.
		\item[Step 2:] Count the number of lone pairs and bonds.
			\begin{itemize}
				\item Double and triple bonds will be counted as
					one bond.
				\item We will consider this number as the number
					of \alert{electron groups}.
			\end{itemize} 
		\item[Step 3:] Assume each charge cloud is as far away from the
			other charge clouds as possible.
	\end{description}
\end{frame}}

\begin{frame}{Shapes of Larger Molecules}

	\begin{itemize}
		\item Not all molecules include just one central atom
		\item More complicated geometries can be predicted by combining 
			the shapes based on their electron cloud distribution
	\end{itemize}

	\bigskip

	\begin{columns}
		\column{0.45\linewidth}
		\centering
		\includegraphics[scale=0.7]{largemolecule.png}
		\column{0.45\linewidth}
		\centering
		\chemfig{
			\Charge{235=\:}{N}
			~[:45]C-[:45]Si(-[:45]
			\Charge{45=\:,135=\:,315=\:}{F})(-[:135]
			\Charge{45=\:,135=\:,225=\:}{F})(<[:260]
			\Charge{0=\:,180=\:,270=\:}{F})(>:[:80]
			\Charge{0=\:,90=\:,180=\:}{F})-[:315]C
			(-[:225]H)(-[:315]H)-[:45]C(=[:90]
			\Charge{45=\:,135=\:}{O})-[:315]
			\Charge{225=\:,315=\:}{O}-[:45]H
}
	\end{columns}
\end{frame}

\begin{frame}[t]{Molecular Shape Practice}
	State the number of electron groups and lone pairs. Use VSEPR theory to
	determine the electronic and molecular shape of the following molecules
	or ions.

	\begin{enumerate}
		\item \ch{PF3} \note[item]{trigonal pyramidal}
		\item \ch{H2S} \note[item]{bent}
		\item \ch{CCl4} \note[item]{tetrahedral}
	\end{enumerate}

	\vspace{10em}
\end{frame}

%\begin{onyourown}
%	% Ebbing 8th Ed. 10.27
%	Predict the electronic and molecular shape of the following molecules:
%
%	\begin{enumerate}
%		\item \ch{SiF4}
%		\item \ch{SF2}
%		\item \ch{COF2}
%		\item \ch{PCl3}
%	\end{enumerate}
%
%	What bond angles do you expect for each?
%\end{onyourown}

\begin{frame}{Molecular Shape and Polarity}
	\begin{itemize}
		\item If a molecule has polar bonds, the molecule as a whole may
			also be polar. \pause
		\item For diatomic molecules, if the bond is polar, the molecule
			is polar.

			\begin{center}
				\includegraphics[scale=0.3]{05_Pg235_UnFigure_4.jpg}
			\end{center}

			\pause

		\item For polyatomic molecules, we must consider the
			\alert{geometry} of the molecule.

			\begin{columns}
				\column{0.45\linewidth}
				\centering
				\includegraphics[scale=0.2]{05_Pg236_UnFigure_1.jpg}
				\column{0.45\linewidth}
				\centering
				\includegraphics[scale=0.2]{05_Pg236_UnFigure_2.jpg}
			\end{columns}
	\end{itemize}
\end{frame}

\begin{frame}{Polar Molecules}
	\begin{itemize}
		\item Contain \alert{polar} bonds.
		\item Has a separation of positive and negative charge called a
			dipole, indicated with \delp\ and \delm.
		\item Has dipoles that \alert{do not} cancel.
	\end{itemize}

	\bigskip

	\begin{center}
		\chemfig{H-Cl}
		\hspace{5em}
		\chemfig{H>:[:20]\Charge{[extra sep=3pt]90=\:}{N}(<[:240]H)-[:330]H}
	\end{center}
\end{frame}

\begin{frame}{Nonpolar Molecules}
	\begin{itemize}
		\item \alert{May} contain \alert{nonpolar} bonds.

			\begin{center}
				\chemfig{Cl-Cl}
				\hspace{5em}
				\chemfig{H-H}
			\end{center}

		\item \alert{May} have a symmetrical arrangement of polar bonds.

			\bigskip

			\begin{center}
				\chemfig{\Charge{135=\:,225=\:}{O}=C=
			\Charge{45=\:,315=\:}{O}}
				\hspace{5em}
				\chemfig{
				\Charge{90=\:,180=\:,270=\:}{Cl}>:[:20]C(<[:240]
		\Charge{135=\:,225=\:,315=\:}{Cl})(-[:90]
\Charge{0=\:,90=\:,180=\:}{Cl})-[:330]\Charge{0=\:,90=\:,270=\:}{Cl}}
			\end{center}
	\end{itemize}
\end{frame}

\begin{frame}[t]{Polarity Practice}
	Which of the following molecules would be expected to have a dipole
	moment of zero?

	\begin{enumerate}
		\item \ch{CS2}
		\item \ch{TeF2}
	\end{enumerate}
	
	\vspace{8em}
\end{frame}

%\begin{onyourown}
%	% Ebbing 8th Ed. 10.39
%	Which of the following molecules would be expected to have a zero dipole
%	moment?
%
%	\begin{enumerate}
%		\item \ch{SeCl4}
%		\item \ch{XeF4}
%	\end{enumerate}
%\end{onyourown}

\begin{frame}{Guide to Predicting Polarity}
\centering
% Define block styles
\tikzstyle{block} = [rectangle, draw, fill=blue!10, 
    text width=8em, text centered, rounded corners, minimum height=2em]
\tikzstyle{line} = [draw,->]

\begin{tikzpicture}[node distance = 8em, auto]
	% Place nodes
	\node [block,text width=6em] (molecule) {Molecule};
	\node [block, below = 2em of molecule] (centrallonepairs) {Only 1 lone pair on central atom?};
	\node [block, left of=centrallonepairs, yshift=-6em,text width=4em](polar) {Polar};
	\node [block, right of=centrallonepairs, yshift=-6em](polarbonds) {Polar bonds?};
	\node [block, right of=polarbonds, yshift=-6em,text width=5em](nonpolar) {Nonpolar};
	\node [block, left of=polarbonds, yshift=-6em](3D) {Dipoles cancel?};
	% Draw edges
	\path [line] (molecule) -- (centrallonepairs);
	\path [line] (centrallonepairs) -| node [above,near start] {Yes} (polar);
	\path [line] (centrallonepairs) -| node [above,near start] {No} (polarbonds);
	\path [line] (polarbonds) -| node [above, near start] {Yes} (3D);
	\path [line] (polarbonds) -| node [above, near start] {No} (nonpolar);
	\path [line] (3D) -- node [above, near start] {Yes} (nonpolar);
	\path [line] (3D) -| node [above, near start] {No} (polar);
%    \path [line] (init) -- (identify);
%    \path [line] (identify) -- (evaluate);
%    \path [line] (evaluate) -- (decide);
%    \path [line] (decide) -| node [near start] {yes} (update);
%    \path [line] (update) |- (identify);
%    \path [line] (decide) -- node {no}(stop);
%    \path [line,dashed] (expert) -- (init);
%    \path [line,dashed] (system) -- (init);
%    \path [line,dashed] (system) |- (evaluate);
\end{tikzpicture}
\end{frame}


\end{document}
