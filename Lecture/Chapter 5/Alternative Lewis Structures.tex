\documentclass[12pt,letterpaper]{article}

\usepackage[margin=1in]{geometry}
\usepackage{chemfig}
\usetikzlibrary{arrows}
\usepackage[version=4]{mhchem}
\usepackage{chemformula}
\usepackage{elements}
\usepackage{siunitx}
\usepackage{mathtools}
\usepackage{amssymb}
\usepackage{cancel}
\usepackage{xlop}
\setlength{\parskip}{\baselineskip}
\setlength{\parindent}{0pt}
\usepackage{array}
\usepackage{xcolor}

\pagestyle{empty}

\begin{document}
\section*{Alternative Method for Drawing Lewis Structures}

Rather than working through the total \# of valence \ce{e-} and ``decorating''
the skeletal structure with lone pairs as we discussed in class, we can also
consider another method for drawing Lewis electron-dot structures. You may find
this method easier but please note that \textbf{this method requires extra
consideration and care when dealing with charged molecules}.

Let's consider the Lewis electron-dot structure for methane, \ce{CH4}:

\begin{enumerate}
	\item Draw the separate, component atoms as complete Lewis structures --
		i.e. surround each component atom with its valence electrons

		\begin{center}
			\chemfig{\lewis{0.2.4.6.,C}
				(-[:0,1.5,,,draw=none]\lewis{4.,H})
				(-[:90,1.5,,,draw=none]\lewis{6.,H})
				(-[:180,1.5,,,draw=none]\lewis{0.,H})
				(-[:270,1.5,,,draw=none]\lewis{2.,H})
			}
		\end{center}

	\item Create bonds by sharing each electron -- ``connect the dots''

		\begin{center}
			\schemestart
			\chemfig{@{C}\lewis{0.2.4.6.,C}
				(-[@{H1b}:0,1.5,,,draw=none]@{H1}\lewis{4.,H})
				(-[@{H2b}:90,1.5,,,draw=none]@{H2}\lewis{6.,H})
				(-[@{H3b}:180,1.5,,,draw=none]@{H3}\lewis{0.,H})
				(-[@{H4b}:270,1.5,,,draw=none]@{H4}\lewis{2.,H})
			}
			\arrow{->}
			\chemfig{C
				(-[:0,1.5]H)
				(-[:90,1.5]H)
				(-[:180,1.5]H)
				(-[:270,1.5]H)
			}
			\schemestop
			% H1
			\chemmove{\draw[blue,thick,-right to,shorten
			<=3pt,shorten >=1pt](C.east)+(2pt,0)..controls
			+(270:6mm) and +(270:6mm)..(H1b);}
			\chemmove{\draw[blue,thick,-right to,shorten
				<=3pt,shorten >=1pt](H1.west)+(-2pt,0)..controls
				+(90:6mm) and +(90:6mm)..(H1b);}
			% H2
			\chemmove{\draw[blue,thick,-right to,shorten
			<=3pt,shorten >=1pt](C.north)+(0,2pt)..controls +(0:6mm)
			and +(0:6mm)..(H2b);}
			\chemmove{\draw[blue,thick,-right to,shorten
				<=3pt,shorten
				>=1pt](H2.south)+(0,-2pt)..controls +(180:6mm)
				and +(180:6mm)..(H2b);}
			% H3
			\chemmove{\draw[blue,thick,-right to,shorten
			<=3pt,shorten >=1pt](C.west)+(-2pt,0)..controls
			+(90:6mm) and +(90:6mm)..(H3b);}
			\chemmove{\draw[blue,thick,-right to,shorten
				<=3pt,shorten >=1pt](H3.east)+(2pt,0)..controls
				+(270:6mm) and +(270:6mm)..(H3b);}
			% H4
			\chemmove{\draw[blue,thick,-right to,shorten
			<=3pt,shorten >=1pt](C.south)+(0,-2pt)..controls
			+(180:6mm) and +(180:6mm)..(H4b);}
			\chemmove{\draw[blue,thick,-right to,shorten
				<=3pt,shorten >=1pt](H4.north)+(0,2pt)..controls
				+(0:6mm) and +(0:6mm)..(H4b);}
		\end{center}

		And we're done!
\end{enumerate}

But what about molecules with double (or triple) bonds? Let's consider
dichloroethene, \ce{C2H2Cl2}:

\begin{enumerate}
	\item Draw the separate, component atoms as complete Lewis structures --
		i.e. surround each component atom with its valence electrons

		\begin{center}
			\chemfig{\lewis{0.2.4.6.,C}
				(-[:180,1.5,,,draw=none]\lewis{0.,H})
				(-[:270,1.5,,,draw=none]\lewis{0:2.4:6:,Cl})
				-[:0,1.5,,,draw=none]\lewis{0.2.4.6.,C}
				(-[:0,1.5,,,draw=none]\lewis{4.,H})
				(-[:270,1.5,,,draw=none]\lewis{0:2.4:6:,Cl})
			}
		\end{center}

	\item Create bonds by sharing each electron -- ``connect the dots''

		\begin{center}
			\schemestart
			\chemfig{@{C1}\lewis{0.2.4.6.,C}
				(-[@{H1b}:180,1.5,,,draw=none]@{H1}\lewis{0.,H})
				(-[@{Cl1b}:270,1.5,,,draw=none]@{Cl1}\lewis{0:2.4:6:,Cl})
				-[@{Cb}:0,1.5,,,draw=none]@{C2}\lewis{0.2.4.6.,C}
				(-[@{H2b}:0,1.5,,,draw=none]@{H2}\lewis{4.,H})
				(-[@{Cl2b}:270,1.5,,,draw=none]@{Cl2}\lewis{0:2.4:6:,Cl})
			}
			\arrow{->}
			\chemfig{\lewis{2.,C}
				(-[:180,1.5]H)
				(-[:270,1.5]\lewis{0:4:6:,Cl})
				-[:0,1.5]\lewis{2.,C}
				(-[:0,1.5]H)
				(-[:270,1.5]\lewis{0:4:6:,Cl})
			}
			\schemestop
			% C
			\chemmove{\draw[blue,thick,-right to,shorten
			<=3pt,shorten >=1pt](C1.east)+(2pt,0)..controls
			+(90:6mm) and +(90:6mm)..(Cb);}
			\chemmove{\draw[blue,thick,-right to,shorten
			<=3pt,shorten >=1pt](C2.west)+(-2pt,0)..controls
			+(270:6mm) and +(270:6mm)..(Cb);}
			% H1
			\chemmove{\draw[blue,thick,-right to,shorten
				<=3pt,shorten >=1pt](C1.west)+(-2pt,0)..controls
				+(270:6mm) and +(270:6mm)..(H1b);}
			\chemmove{\draw[blue,thick,-right to,shorten
			<=3pt,shorten >=1pt](H1.east)+(2pt,0)..controls
			+(90:6mm) and +(90:6mm)..(H1b);}
			% Cl1
			\chemmove{\draw[blue,thick,-right to,shorten
			<=3pt,shorten >=1pt](C1.south)+(0,-2pt)..controls
			+(0:6mm) and +(0:6mm)..(Cl1b);}
			\chemmove{\draw[blue,thick,-right to,shorten
			<=3pt,shorten >=1pt](Cl1.north)+(0,2pt)..controls
			+(180:6mm) and +(180:6mm)..(Cl1b);}
			% H2
			\chemmove{\draw[blue,thick,-right to,shorten
				<=3pt,shorten >=1pt](C2.east)+(2pt,0)..controls
				+(90:6mm) and +(90:6mm)..(H2b);}
			\chemmove{\draw[blue,thick,-right to,shorten
			<=3pt,shorten >=1pt](H2.west)+(-2pt,0)..controls
			+(270:6mm) and +(270:6mm)..(H2b);}
			% Cl2
			\chemmove{\draw[blue,thick,-right to,shorten
			<=3pt,shorten >=1pt](C2.south)+(0,-2pt)..controls
			+(0:6mm) and +(0:6mm)..(Cl2b);}
			\chemmove{\draw[blue,thick,-right to,shorten
			<=3pt,shorten >=1pt](Cl2.north)+(0,2pt)..controls
			+(180:6mm) and +(180:6mm)..(Cl2b);}
		\end{center}

	\item In this case, we are left with 2 unpaired electrons above each C,
		which we can use to create another bond -- the double bond

		\begin{center}
			\schemestart
			\chemfig{@{C1}\lewis{2.,C}
				(-[:180,1.5]H)
				(-[:270,1.5]\lewis{0:4:6:,Cl})
				(-[:0,0.6,,,draw=none]@{C1b})
				-[:0,1.5]@{C2}\lewis{2.,C}
				(-[180,0.6,,,draw=none]@{C2b})
				(-[:0,1.5]H)
				(-[:270,1.5]\lewis{0:4:6:,Cl})
			}
			\arrow{->}
			\chemfig{C
				(-[:180,1.5]H)
				(-[:270,1.5]\lewis{0:4:6:,Cl})
				=[:0,1.5]C
				(-[:0,1.5]H)
				(-[:270,1.5]\lewis{0:4:6:,Cl})
			}
			\schemestop
			% C
			\chemmove{\draw[blue,thick,-right to,shorten
			<=1pt,shorten >=3pt](C1.north)+(0,4pt)..controls
			+(90:6mm) and +(90:8mm)..(C1b);}
			\chemmove{\draw[blue,thick,-left to,shorten
			<=1pt,shorten >=3pt](C2.north)+(0,4pt)..controls
			+(90:6mm) and +(90:8mm)..(C2b);}
		\end{center}
\end{enumerate}

The last problem we will take a look at is if we have a charged molecule, such
as \ce{CN-}:

\begin{enumerate}
	\item Because we have an overall negative charge, we need to add an
		additional electron to \textbf{one} (we have a charge of -1) of
		our atoms -- in this case, we will add it to \ce{N} as \ce{N}
		(EN=3.04) has a larger electronegativity than \ce{C} (EN=2.55)

		\begin{center}
			$\chemleft[
			\chemfig{\lewis{0.2.4.6.,C}
				-[:0,1.5,,,draw=none]\lewis{0:2:4.6.,N}
				}
			\chemright]^-$
		\end{center}

	\item Create bonds by sharing each electron -- ``connect the dots''

		\begin{center}
			\schemestart
			$\chemleft[
				\chemfig{@{C}\lewis{0.2.4.6.,C}
				-[@{CN}:0,1.5,,,draw=none]@{N}\lewis{0:2:4.6.,N}
				}
			\chemright]^-$
			\arrow{->}
			$\chemleft[
			\chemfig{\lewis{2.4.6.,C}
				-[:0,1.5]\lewis{0:2:6.,N}
				}
			\chemright]^-$
			\schemestop
			% CN
			\chemmove{\draw[blue,thick,-right to,shorten
			<=3pt,shorten >=1pt](C.east)+(2pt,0)..controls
			+(90:6mm) and +(90:6mm)..(CN);}
			\chemmove{\draw[blue,thick,-right to,shorten
			<=3pt,shorten >=1pt](N.west)+(-2pt,0)..controls
			+(270:6mm) and +(270:6mm)..(CN);}
		\end{center}

	\item We still have some unpaired electrons so let's make another bond

		\begin{center}
			\schemestart
			$\chemleft[
				\chemfig{@{C}\lewis{2.4.6.,C}
				(-[:0,0.6]@{CN1})
				-[:0,1.5]@{N}\lewis{0:2:6.,N}
				(-[:180,0.6]@{CN2})
				}
			\chemright]^-$
			\arrow{->}
			$\chemleft[
			\chemfig{\lewis{2.4.,C}
				=[:0,1.5]\lewis{0:2:,N}
				}
			\chemright]^-$
			\schemestop
			% CN
			\chemmove{\draw[blue,thick,-left to,shorten
			<=3pt,shorten >=1pt](C.south)+(0,-2pt)..controls
			+(270:6mm) and +(270:8mm)..(CN1);}
			\chemmove{\draw[blue,thick,-right to,shorten
			<=3pt,shorten >=1pt](N.south)+(0,-2pt)..controls
			+(270:6mm) and +(270:8mm)..(CN2);}
		\end{center}

	\item It looks like we won't be able to form any more bonds between
		unpaired electrons -- let's instead combine the two unpaired
		electrons on C to a lone pair and push a lone pair from N into a
		triple bond

		\begin{center}
			\schemestart
			$\chemleft[
				\chemfig{@{C}\lewis{2.4.,C}
				=[@{CN}:0,1.5]@{N}\lewis{0:2:,N}
				}
			\chemright]^-$
			\arrow{->}
			$\chemleft[
			\chemfig{\lewis{4:,C}
				~[:0,1.5]\lewis{0:,N}
				}
			\chemright]^-$
			\schemestop
			% CN
			\chemmove{\draw[blue,thick,-right to,shorten
			<=2pt,shorten >=3pt](C.north)+(0,4pt)..controls
			+(90:6mm) and +(110:8mm)..(C.west);}
			\chemmove{\draw[blue,thick,shorten
			<=3pt,shorten >=1pt](N.north)+(0,2pt)..controls
			+(90:6mm) and +(90:8mm)..(CN);}
		\end{center}


		
\end{enumerate}




\end{document}
