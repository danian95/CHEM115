\documentclass[11pt,letterpaper]{article}

\usepackage{genchem}
\usepackage{enumitem}
\usepackage[margin=1in]{geometry}
\usepackage{titling}
\usepackage{tabularx}
\chemsetup{chemformula/frac-style=nicefrac}

\setallmainfonts{TeX Gyre Pagella}

\title{Chapter 11 ``On Your Own'' Solutions}

\begin{document}

\begin{center}
	\bfseries
	\Large
	\thetitle
\end{center}

\begin{enumerate}[itemsep=2em,leftmargin=0pt,label=\textbf{\Alph*.}]
	\item \begin{enumerate}[label={\arabic*.}]
			\item dispersion
			\item dispersion, hydrogen bonding, dipole-dipole
			\item dispersion, dipole-dipole
			\item dispersion
		\end{enumerate}
	\item We need to first bring ethanol to its boiling point:
		\begin{align*}
			q_1 &= m C_s \Delta T \\
			&=
			(\SI{3.45}{\gram})(\SI{2.04}{\joule\per\gram\per\celsius})(\SI{78}{\celsius}-\SI{18}{\celsius})
			\\
			&= \SI{422.28}{\joule}
			\intertext{We then need to vaporize the ethanol:}
			\Delta H_\text{vap} &= \frac{q_2}{n} \\
			q_2 &= n \Delta H_\text{vap} \\
			&= (\SI{3.24}{\gram})(\SI{841}{\joule\per\gram}) \\
			&= \SI{2901.45}{\joule}
			\intertext{Finally, we sum the two heats together to
			obtain the total energy required:}
			q_\text{total} &= q_1 + q_2 \\
			&= \SI{422.28}{\joule} + \SI{2901.45}{\joule} \\
			&= \SI{3323.73}{\joule} \\
			&= \boxed{\SI{3320}{\joule}}
		\end{align*}
\end{enumerate}

\end{document}
