% !TEX program = xelatex
%\documentclass[notes=onlyslideswithnotes,notes=hide]{beamer}
%\documentclass[notes=only]{beamer}
\documentclass[notes=show]{beamer}
%\documentclass[handout]{beamer}
%\documentclass[10pt,letterpaper]{article}
%\usepackage{beamerarticle}

\usepackage{newtxtext}
\usepackage{genchem}
\usepackage{lecture}
\usepackage{bucolors}
\usepackage{ccicons}

\title{Atoms}
\subtitle{Chapter 1}
\institute[CHEM115 Bloomsburg University]{CHEM115 --- Chemistry for the Sciences I \\ Bloomsburg University}
\author{D.A.\ McCurry}
\date{Fall 2021}

\begin{document}

\maketitle

\mode<presentation|handout:0>{%
	{%
		\usebackgroundtemplate{\includegraphics[scale=0.6]{amazon.jpg}}
		\setbeamercolor{frametitle}{bg=bumaroon!10!black}
		\setbeamercolor{normal text}{fg=white}
		\usebeamercolor*{normal text}
		\setbeamercolor{structure}{fg=white!80!black}
		\begin{frame}[t,plain]{The Amazon is Burning}
			What does this mean globally?
			\begin{itemize}[<+(1)->]
				\item Massive fires release significant amounts of
					\ch{CO2} into the atmosphere.
				\item The \emph{structure} of the \ch{CO2} molecule causes some infrared (IR) radiation
					to be absorbed by the atmosphere.
					\begin{center}
						\includegraphics[width=0.55\linewidth]{co2-ir.png}
					\end{center}
			\end{itemize}

			\begin{footnotesize} Andre Penner/AP \end{footnotesize}
		\end{frame}
	}
}

\frame{%
	\section{Substances and Compounds}
	\begin{learningobjectives}
		\item Know the definition of matter.
		\item Qualitatively understand how states of material relate to atomic structure.
		\item Classify materials based on their composition.
		\end{learningobjectives}
	}

\mode<article>{%
	\begin{block}{What is Chemistry?}
		Chemistry is the science that seeks to understand the properties
		of matter by studying the \alert{structure} of the particles that
		compose it.
	\end{block}

	\begin{columns}
		\column{0.45\linewidth}
		\begin{itemize}[<+(1)->]
		\item What is matter?
			\mode<presentation>{
				\begin{itemize}[<+(1)-|article:0>]
				\item Anything that occupies space
					and has mass.
			\end{itemize}}
		\item What particles compose matter?
			\mode<presentation>{
				\begin{itemize}[<+(1)-|article:0>]
				\item Atoms
			\end{itemize}}
		\item What determines the structure?
	\end{itemize}
		\column{0.35\linewidth}
		\visible<5->{
		\begin{center}
		\includegraphics[scale=0.25]{watermol.jpeg}
		\end{center}}
	\end{columns}
}

\begin{frame}{Properties Follow Structure}
	\begin{center}
		\includegraphics[scale=0.35]{carbonstructure.jpeg}
	\end{center}

	Both graphite and diamond are composed of only carbon, but their
	\alert{structures} differ!
\end{frame}

\begin{frame}[t]{Classifying Matter: States}
	\begin{block}{States of Matter}
		The \alert{state} is determined by the \alert{relative positions}
		of the particles and \alert{how strongly they interact} with one
		another.
	\end{block}

	\only<1>{\begin{center}
		\includegraphics[scale=0.35,trim={0 90pt 0 0},clip]{matterstates.jpeg}
\end{center}}

\only<2>{%
	\begin{center}
		\includegraphics[scale=0.45]{compress.jpeg}
	\end{center}

	We can squeeze gas particles together and \alert{compress} them, but we
	cannot do the same with a solid.
}
\end{frame}

\begin{frame}{Classifying Matter: Composition}
%	\only<1>{
	\begin{center}
		\includegraphics[scale=0.45]{composition.jpeg}
	\end{center}
%
%
%	\pause
%
%	\alert{Pure Substances:} There is only one type of component, invariant
%	between samples.
%	\begin{itemize}[<+(1)->]
%		\item \alert{Elements} cannot be chemically broken down further.
%		\item \alert{Compounds} are composed of two or more elements
%			\alert{in fixed proportions}.
%	\end{itemize}
%
%	\pause
%
%	\alert{Mixtures:} Two or more components with varying composition between
%	samples.
%	\begin{itemize}[<+(1)->]
%		\item \alert{Heterogeneous} mixtures have varying composition
%			from one region to another -- the particles do not mix
%			uniformly.
%		\item \alert{Homogeneous} mixtures have the same composition
%			throughout -- the particles do mix uniformly.
%	\end{itemize}
\end{frame}

\begin{frame}[t]{Classifying Composition Examples}
	Identify whether each substance is a \alert{pure substance} (element or
	compound) or a \alert{mixture} (heterogeneous or homogeneous):
	
	\begin{center}
		\begin{minipage}[t]{0.3\linewidth}
			\centering
			\textbf{Neon} 

			\medskip

			\includegraphics[width=0.8\linewidth,height=5em,keepaspectratio]{NeTube.jpg}

			\medskip
			
			\tiny{\ccbysa\ Pslawinski}
		\end{minipage}
		\qquad
		\begin{minipage}[t]{0.3\linewidth}
			\centering
			\textbf{Gasoline}

			\medskip

			\includegraphics[width=0.8\linewidth,height=5em,keepaspectratio]{GasStationHiroshima.jpg}
		\end{minipage}

		\bigskip

		\begin{minipage}[t]{0.3\linewidth}
			\centering
			\textbf{Oxygen}

			\medskip

			\includegraphics[width=0.8\linewidth,height=5em,keepaspectratio]{astronaut.jpg}
		\end{minipage}
		\qquad
		\begin{minipage}[t]{0.3\linewidth}
			\centering
			\textbf{Cereal}

			\medskip

			\includegraphics[width=0.8\linewidth,height=5em,keepaspectratio]{cereal.jpg}
		\end{minipage}
	\end{center}
	
\end{frame}

%\begin{onyourown}%{0em}
%	Identify whether each substance is a \alert{pure substance} (element or
%	compound) or a \alert{mixture} (heterogeneous or homogeneous):
%	
%	\renewcommand\arraystretch{2}
%	\begin{tabularx}{\linewidth} {l@{\qquad}X}
%		dish soap & \\ \cline{2-2}
%		helium & \\ \cline{2-2}
%		coffee & \\ \cline{2-2}
%	\end{tabularx}
%\end{onyourown}

\frame{\section{Atoms: The Building Blocks of Matter}
	\begin{learningobjectives}
		\item Explain how the atomic theory of matter guides chemical reactions.
		\item Understand the charge distribution in an atom.
		\item Draw and label an atom.
		\end{learningobjectives}
	}

%\begin{frame}{Early Atomic Theory}
%	What is the smallest unit that we can actually break down?
%
%	\medskip
%
%	\begin{columns}
%		\column{0.6\linewidth}
%		\begin{itemize}[<+(1)->]
%			\item All matter is composed of four elements.
%			\item Later aether was added.
%			\item $\sim$450 \textsc{B.C.E.}, Democritus proposed that Matter is composed of small,
%				indivisible parts.  Many different types of atoms (``atomos'') exist.
%		\end{itemize}
%
%		\onslide<+(1)->
%		\begin{center}
%			\bfseries\usebeamercolor[fg]{alerted text} How do these atoms create compounds?
%		\end{center}
%		\column{0.4\linewidth}
%		\onslide<2->
%		\includegraphics<presentation>[width=\linewidth]{4elements.jpg}
%	\end{columns}
%\end{frame}

\begin{frame}{The Law of Conservation of Mass}
	\begin{block}{Early 1700s: Lavoisier}
		Mass is neither created nor destroyed in chemical reactions.
	\end{block}

	\begin{reaction*}
		Hg(NO3)2\aq{} + 2 KI\aq{} -> HgI2\sld{} + 2 KNO3\aq{}
	\end{reaction*}

%	\begin{overlayarea}{\linewidth}{10em}
	%	\only<1>{
	\begin{center}
		\includegraphics[scale=0.4]{conservemass-mcmurry.jpg}
	\end{center}

	%}
	
	%	\only<2->{
%		\begin{align*}
%			\overbrace{\ch{Hg(NO3)2\aq{} + 2
%			KI\aq{}}}^{\mathclap{\SI{3.25}{\gram} + \SI{3.32}{\gram} =
%			\SI{6.57}{\gram}}} &\ch{->} \underbrace{\ch{HgI2\sld{} + 2
%			KNO3\aq}}_{\mathclap{\SI{4.55}{\gram} + \SI{2.02}{\gram} =
%			\SI{6.57}{\gram}}} \\[1em]
%			\text{Reactants} &\ch{->} \text{Products} \\[1em]
%			m_\text{Reactants} &= m_\text{Products}
%		\end{align*}}
%	\end{overlayarea}
\end{frame}

\begin{frame}{The Law of Definite Proportions}
	\begin{block}{1799: Proust}
		All samples of a given compound have the same proportions of
		constituent elements (mass ratio).
	\end{block}

	\begin{center}
		\includegraphics[scale=0.5]{definiteproportions.jpg}
	\end{center}
\end{frame}

\begin{frame}{The Law of Multiple Proportions}
	\begin{block}{1803: John Dalton}
		When two elements (A and B) form two different compounds, the
		mass of B that reacts with A can be expressed as a whole number
		ratio.
	\end{block}

	\begin{center}
		\includegraphics[scale=0.3]{multipleproportions.jpeg}
	\end{center}

	\begin{equation*}
		\frac{2.67}{1.33} = 2 \qquad
		\text{\parbox{0.6\linewidth}{Carbon dioxide has $2 \times$
		more \ch{O} atoms than carbon monoxide.}}
	\end{equation*}
\end{frame}

\begin{frame}{John Dalton and the Atomic Theory of Matter}
	\begin{enumerate}[<+->]
		\item Each element is composed of tiny, indestructible particles
			called atoms.
		\item All atoms of a given element have the same mass and other
			properties that distinguish them from the atoms of other
			elements.
		\item Atoms combine in simple, whole-number ratios to form
			compounds.
		\item Atoms of one element cannot change into atoms of another
			element. In a chemical reaction, the atoms only change
			the way that they are \alert{bound together} with other
			atoms.
	\end{enumerate}
\end{frame}

\begin{frame}{What is an atom?}
	\centering

	\includegraphics[scale=0.8]{stylized-li-atom.pdf}

	\bigskip

	\tiny\ccbysa\ Indolences
\end{frame}

\begin{frame}{The Discovery of the Electron}
	J.J. Thompson (1856 -- 1940) proposes that cathode rays must consist of
	tiny, negatively charged particles, which we now call \alert{electrons}.

	\begin{center}
		\includegraphics[scale=0.45]{cathode-tube-mcmurry.jpg}
	\end{center}
\end{frame}

\begin{frame}{What charge are electrons?}{Millikan's oil drop experiment}
	\begin{center}
		\includegraphics[scale=0.45]{millikan-mcmurry.jpg}
	\end{center}
\end{frame}

\begin{frame}{An Electron-Centric Atomic Structure}
	Electrons were thought to exist within a sphere of
	positive charge.

	\bigskip

	\begin{tabular} {m{0.45\linewidth}@{\qquad}m{0.45\linewidth}}
		\includegraphics[scale=0.2]{superb-english-plum-pudding.jpg}

		\tiny\url{https://www.epicurious.com/}
		& \includegraphics[scale=0.25]{plumpudding.jpeg} \\
	\end{tabular}
\end{frame}

\begin{frame}{The Discovery of the Nucleus}
	{Rutherford's Gold Foil Experiment}

	\begin{center}
		\includegraphics[scale=0.45]{rutherford-mcmurry.jpg}
	\end{center}
\end{frame}

\begin{frame}{The Nuclear Theory of the Atom}
	\begin{enumerate}
		\item Most of the atom's mass and all of its positive charge
			resides in the nucleus.
		\item Most of the atom's volume is empty space, which tiny
			negatively charged electrons dispersed.
		\item In a neutral atom, there are a many negatively charged
			electrons as there are positively charged nuclear
			particles (protons).
	\end{enumerate}

	\begin{center}
		\includegraphics[scale=0.3]{predictedpudding.jpeg}
	\end{center}
\end{frame}

\mode<presentation|handout:0>{{%
	\usebackgroundtemplate{\includegraphics[width=\paperwidth]{baseball-mcmurry.jpg}}
	\begin{frame}[plain]
	\end{frame}}}


\begin{frame}{So\ldots\ What is an atom?}
	\centering

	\includegraphics[scale=0.45]{atom-mcmurry.jpg}
\end{frame}

\mode<article>{Basic Atomic Structure
	\begin{itemize}
		\item \alert{Nucleus}
			\begin{itemize}
				\item Contains positive \alert{protons}
					(\ch{p^{+}} or \ch{^1_1p}).
				\item Contains neutral \alert{neutrons} (\ch{n}
					or \ch{^1_0n}).
				\item Extremely small compared to the size of
					the atom.
			\end{itemize}
		\item \alert{Electrons} (\ch{e^{-}} or \ch{^0_{-1}e}) orbit the
			nucleus.
			\begin{itemize}
				\item Define the size of the atom.
				\item Have energy -- constantly moving.
				\item Remain in atom because of electrostatic
					attraction to the positive nucleus.
			\end{itemize}
	\end{itemize}
}

\begin{frame}{Subatomic Particles}
	\begin{center}
		\small
		\begin{tabular} {@{}l S[table-format=1.5e-2]
			S[table-format=1.5]
			S[table-format=2.5e-2]}
			\toprule & \textbf{Mass (\si{\kilo\gram})} &
			\textbf{Mass (amu)\textsuperscript{*}} & 
			\textbf{Charge (\si{\coulomb})} \\ \midrule
			Proton   & 1.67262e-27 & 1.00727 & +1.60218e-19 \\
			Neutron  & 1.67493e-27 & 1.00866 &  0           \\
			Electron & 0.00091e-27 & 0.00055 & -1.60218e-19 \\
			\bottomrule
			\multicolumn{4}{>{\raggedright\arraybackslash}p{4in}}{\footnotesize *Atomic Mass Units (amu) -- a mass exactly equal to
			$\sfrac{1}{12}$ the mass of one carbon-12 atom.}
		\end{tabular}
	\end{center}

	\begin{itemize}
		\item \alert{Protons}
			\begin{itemize}
				\item The number of protons defines the element
					(\alert{Atomic Number}).
			\end{itemize}
		\item \alert{Neutrons}
			\begin{itemize}
				\item Usually close to or greater than the
					number of protons.
			\end{itemize}
		\item \alert{Electrons}
			\begin{itemize}
				\item Same number of protons ($\# \ch{p^{+}} =
					\# \ch{e^{-}}$)
			\end{itemize}
	\end{itemize}
\end{frame}

% Ended class here on 2020-08-24

\frame{\section{Describing Atoms}
	\begin{learningobjectives}
		\item Read the periodic table to determine atomic structure.
		\item Use atomic notation to describe a specific atom.
		\item Calculate the mass and charge of atoms based on their atomic structure.
	\end{learningobjectives}
}

\begin{frame}{The Number of Protons Defines the Element}
	\centering
	\includegraphics[scale=0.45,trim={0 0 0 30pt},clip]{numprotons.jpeg}
\end{frame}

\mode<presentation>{
	{
	\usebackgroundtemplate{\includegraphics[width=\paperwidth]{ACS-periodic-table-2020.pdf}}
	\begin{frame}[plain]
		\null
\end{frame}}}

\begin{frame}{Atomic Notation}
	\begin{align*}
		\text{\alert{Atomic Number} ($Z$)} &= \text{\# of protons in
		the nucleus} \\
		\text{\alert{Mass Number} ($A$)} &= \text{\# of protons} +
		\text{\# of neutrons} \\
		&= Z + \text{\# of neutrons} \\
	\end{align*}
	\begin{center}
		\includegraphics[scale=0.4]{atomic-notation.jpeg}
	\end{center}

	\begin{block}{Isotopes}
		Atoms of the same element (X) with different numbers of neutrons in their nuclei.
	\end{block}
\end{frame}

\begin{frame}{Complete the table\ldots}
	\mode<presentation>{%
	\begin{tabularx}{\linewidth} {>{\qquad}r@{}l*{2}{>{\centering\arraybackslash}X}}
		\toprule
		\multicolumn{2}{>{\centering\arraybackslash}X}{\bfseries Chemical Symbol} & \bfseries \# Protons & \bfseries \# Neutrons \\
		\midrule
		\ch{^1_1}      & H & 1 & 0 \\[1em] 
		\ch{^2_1}      &&&         \\[1em] 
		\ch{^3_1}      &&&         \\[1em] 
		\ch{^{15}_7}   &&&         \\[1em] 
		\ch{^{238}_92} &&&         \\[1em] 
		\ch{^{235}_92} &&&         \\      
		\bottomrule
	\end{tabularx}
}

	\mode<article>{%
		\begin{tabular} {r@{}l*{2}{c}}
			\toprule \multicolumn{2}{c}{\bfseries Chemical
			Symbol} & \# \bfseries Protons & \# \bfseries Neutrons \\ \midrule
			\ch{^1_1}      & H & 1  & 0   \\[1em] 
			\ch{^2_1}      & H & 1  & 1   \\[1em]
			\ch{^3_1}      & H & 1  & 2   \\[1em]
			\ch{^{15}_7}   & N & 7  & 8   \\[1em]
			\ch{^{238}_92} & U & 92 & 146 \\[1em]
			\ch{^{235}_92} & U & 92 & 143 \\
			\bottomrule
		\end{tabular}}
\end{frame}

%\begin{onyourown}%{0em}
%	Complete the table\ldots
%	\renewcommand\arraystretch{1.5}
%
%	\begin{tabularx}{\linewidth} {@{\qquad}r@{}l*{2}{>{\centering\arraybackslash}X}}
%		\toprule
%		\multicolumn{2}{>{\centering\arraybackslash}X}{\bfseries Chemical Symbol} & \bfseries \# Protons & \bfseries \# Neutrons \\
%		\midrule
%		\ch{^7_3}        \\ 
%		\ch{^{58}_{25}}  \\
%		\ch{^{201}_{83}} \\
%		\bottomrule
%	\end{tabularx}
%\end{onyourown}

\begin{frame}{Isotopes and Average Mass}
	Isotopes are generally referred to by their element name followed by the
	mass:

	\renewcommand\arraystretch{2}
	\begin{center}
		\begin{tabular} {@{}r@{}l@{\qquad} *{2}{c@{\qquad}}}
		\ch{^{238}_{92}} & U  & U-238 & uranium-238 \\
		\ch{^{21}_{10}}  & Ne & Ne-21 & neon-21 \\
		\ch{^{208}_{82}} & Pb & Pb-208 & lead-208
	\end{tabular}
	\end{center}

	\pause

	\begin{center}
		\rmfamily
		\begin{tikzpicture}
			\node[font=\bfseries\LARGE,](Cl) at (0,0) {Cl};
			\node[above = 0.2em of Cl](17) {17};
			\node[below = 0.2em of Cl](35) {35.45};
			\node[font=\small,below = 0.2em of 35](chlorine) {chlorine};
			\draw[thick] (-1.2,-1.65) rectangle (1.2,1);
			\draw[<-,ultra thick,bumaroon] (35) to ++(-2,0) node[left,font=\sffamily,align=right,text
				width=1in,black] {But what is this mass?};
		\end{tikzpicture}
	\end{center}
\end{frame}

\begin{frame}{Calculating the Average Atomic Mass}
	\begin{block}{Atomic Mass}
		Mass of an ``average'' element in atomic mass units (amu) --
		listed on the periodic table -- weighted according to the
		natural abundance of each isotope.
	\end{block}

%	\begin{overlayarea}{\linewidth}{12em}

	\begin{center}
	\begin{tikzpicture}
		\node[fill=orange,circle,draw=orange!60!black](B10) at (0,0) {\ch{^{10}B}};
		\node[right,align=left,font=\footnotesize,xshift=5pt] at (B10.east) {1 \alert{atom} of boron-10
		has \\ a \alert{mass} of \num{10.0129}~\alert{amu}};
		\node[fill=green,circle,draw=green!60!black](B11) at (6,0) {\ch{^{11}B}};
		\node[right,align=left,font=\footnotesize,xshift=5pt] at
		(B11.east) {1 \alert{atom} of boron-11
		has \\ a \alert{mass} of \num{11.0093}~\alert{amu}};
		\foreach \x in {0,...,49}
			\draw[fill=orange,draw=orange!60!black,line width=0.3pt] (\x/20,-1.5) circle (0.4cm);
		\foreach \x in {50,...,200}
			\draw[fill=green,draw=green!60!black,line width=0.3pt] (\x/20,-1.5) circle (0.4cm);
	\end{tikzpicture}
	\end{center}

	In every 1000 atoms of boron, 199 of them are boron-10 atoms and the
	other 801 are boron-11 atoms. What is the atomic mass of the average
	boron atom?

%	\only<2>{
%	\begin{align*}
%		\parbox{\widthof{Atomic}}{\centering Atomic mass} &= \sum_n \bigg[ (\text{fraction of isotope\ } n) \times (\text{mass of isotope\ } n) \bigg] \\
%				   &= (\text{fraction of isotope 1}) \times (\text{mass of isotope 1}) \\
%				   &\qquad {} + (\text{fraction of isotope 2}) \times (\text{mass of isotope 2}) \\
%				   &\qquad {} + (\text{fraction of isotope 3}) \times (\text{mass of isotope 3}) \\
%				   &\qquad {} + \ldots
%	\end{align*}}
%
%	\end{overlayarea}
\end{frame}

\begin{frame}[t]{What is the atomic mass of the average B atom?}
	\centering
	\begin{tabular} {l S[table-format=2.4]
		S[table-format=2.1]}
		\toprule
		\bfseries Isotope & \textbf{Mass (amu)} & \textbf{Natural
		Abundance (\%)} \\ \midrule
		boron-10 & 10.0129 & 19.9 \\
		boron-11 & 11.0093 & 80.1 \\ \bottomrule
	\end{tabular}

	\mode<article>{%
		\begin{align*}
			\parbox{\widthof{Atomic}}{\centering Atomic\\mass} &= \sum_n \bigg[ (\text{fraction of isotope\ } n) \times (\text{mass of isotope\ } n) \bigg] \\
			&= (0.199 \times 10.0129) + (0.801 \times 11.0093) \\
			&= 10.8110164 \\
			&= \fbox{\SI{10.8}{amu}}
		\end{align*}}
\end{frame}

\begin{frame}[t]{What is the atomic mass of the average Mg atom?}
	\centering
	\begin{tabular} {l S[table-format=2.7]
		S[table-format=2.2]}
		\toprule
		\bfseries Isotope & \textbf{Mass (amu)} & \textbf{Natural
		Abundance (\%)} \\ \midrule
		magnesium-24 & 23.9850423 & 78.99 \\
		magnesium-25 & 24.9858374 & 10.00 \\ 
		magnesium-26 & 25.9825937 & 11.01 \\\bottomrule
	\end{tabular}

	\mode<article>{%
		\begin{align*}
			\parbox{\widthof{Atomic}}{\centering Atomic\\mass} &= \sum_n \bigg[ (\text{fraction of isotope\ } n) \times (\text{mass of isotope\ } n) \bigg] \\
			&= (0.7899 \times 23.9850423) + (0.1000 \times
			24.9858374) \\ &\qquad {} + (0.1101 \times 25.9825937) \\
			&= 24.30505222 \\
			&= \fbox{\SI{24.31}{amu}}
		\end{align*}}
\end{frame}

%\begin{onyourown}%{10em}
%	What is the atomic mass of the average O atom?
%
%	\begin{center}
%		\begin{tabular} {l S[table-format=2.4] S[table-format=2.2]}
%			\toprule
%			\bfseries Isotope & {\bfseries Mass (amu)} & {\bfseries Natural Abundance (\%)} \\
%			\midrule
%			oxygen-16 & 15.9949 & 99.76 \\
%			oxygen-17 & 16.9991 & 0.04 \\
%			oxygen-18 & 17.9992 & 0.20 \\
%			\bottomrule
%		\end{tabular}
%	\end{center}
%\end{onyourown}

\begin{frame}[allowframebreaks]{What about the electrons?}{Ionization}
	In order to keep atoms and molecules \alert{neutral}, the number of
	electrons must be \alert{equal to} to the number of protons.

	\begin{center}
	\begin{tabular} {l *{4}{c}}
		\toprule
		\bfseries Atom & \bfseries Protons & \bfseries Neutrons &
		\bfseries Electrons & \bfseries Charge     \\ \midrule
		H    & 1       & 0        & 1         & 0          \\
		Li   & 3       & 4        & 3         & 0          \\
		Be   & 4       & 5        & 4         & 0          \\
		F    & 9       & 10       & 9         & 0          \\
		Cl   & 17      & 18       & 17        & 0          \\
		O    & 8       & 8        & 8         & 0          \\
		\bottomrule
	\end{tabular}
	\end{center}

	\framebreak

	Atoms (or molecules) can \alert{gain} or \alert{lose} electrons
	in a process known as \alert{ionization} to become \alert{ions}.

	\begin{center}
		\begin{tabular} {l *{3}{c} S[table-format=-1]}
		\toprule
		\bfseries Ion & \bfseries Protons & \bfseries Neutrons &
		\bfseries Electrons & \bfseries Charge     \\ \midrule
		\ch{H+}      & 1       & 0        & 0         & +1         \\
		\ch{Li+}     & 3       & 4        & 2         & +1         \\
		\ch{Be^{2+}} & 4       & 5        & 2         & +2         \\
		\ch{F-}      & 9       & 10       & 10        & -1         \\
		\ch{Cl-}     & 17      & 18       & 18        & -1         \\
		\ch{O^{2-}}  & 8       & 8        & 10        & -2         \\
		\bottomrule
	\end{tabular}
	\end{center}

	\begin{center}
		\begin{minipage}{0.3\linewidth}
		\begin{tabular} {c @{ \textrightarrow\ } l}
			$+$ & cation \\
			$-$ & anion
		\end{tabular}
		\end{minipage}
		\qquad
		\begin{minipage}{0.3\linewidth}
			\centering\bfseries
			\usebeamercolor[fg]{alerted text}
			Does the mass change?
		\end{minipage}
	\end{center}
\end{frame}

\begin{frame}{Interpreting Atomic Symbols}
	Complete the table\ldots 

	\bigskip
	
	\noindent
	{\mode<presentation>{\footnotesize}
		\renewcommand\arraystretch{1.5}
		\begin{tabularx}{\linewidth} {@{}*{7}{>{\centering\arraybackslash}X}}
		\toprule
		\bfseries Atom /Ion & \bfseries Atomic \# & \bfseries Mass \# & \bfseries Charge & \bfseries \# Protons & \bfseries \# Neutrons & \bfseries \# Electrons \\ \midrule
		\ch{^{23}_{11}Na+} \\
		\ch{^{35}_{16}S^{2-}} \\
		\ch{^{197}_{79}Au^{3+}} \\
		\ch{^{88}_{38}Sr} \\ \bottomrule
	\end{tabularx}}
\end{frame}

%\begin{onyourown}%{0em}
%	Complete the table\ldots
%
%	\renewcommand\arraystretch{1.5}
%	\begin{tabularx}{\linewidth} {@{}*{7}{>{\centering\arraybackslash}X}}
%		\toprule
%		\bfseries Atom /Ion & \bfseries Atomic \# & \bfseries Mass \# & \bfseries Charge & \bfseries \# Protons & \bfseries \# Neutrons & \bfseries \# Electrons \\ \midrule
%		\ch{^{87}_{37}Rb+} \\
%		\ch{^{71}_{32}Ge^{2-}} \\
%		\ch{^{198}_{77}Ir^{4+}} \\
%		\ch{^{119}_{50}Sn} \\ \bottomrule
%	\end{tabularx}
%\end{onyourown}

\begin{frame}{Mass Spectrometry}
	\begin{center}
		\includegraphics[scale=0.45]{massspec.jpeg}
	\end{center}

	\begin{minipage}{0.3\linewidth}
		\includegraphics[scale=0.15]{massspectrum.jpeg}
	\end{minipage}
	\hfill
	\begin{minipage}{0.65\linewidth}
		\begin{itemize}
			\item \alert{Ions} are affected by electric and magnetic
				fields.
			\item \alert{Mass} affects how easy it is to direct the
				ions.
		\end{itemize}
	\end{minipage}
\end{frame}

\begin{frame}{Graham Cooks: Purdue University}
	\centering
	\includegraphics[scale=0.35]{mini11-1.png}
\end{frame}

{\usebackgroundtemplate{\includegraphics[width=\paperwidth]{johnbygc.jpg}}
\begin{frame}
		\frametitle{Bloomsburg University GC-MS}
\end{frame}
}

\frame{\section{The Mole Concept}
	\begin{learningobjectives}
	\item Know the relation between moles and absolute amount.
	\item Use appropriate conversions to switch between moles and mass.
	\item Calculate the mass required for a specific number of moles and vice versa.
	\end{learningobjectives}
}

\begin{frame}{The Mole Concept}
	Often, we discuss things in terms of a \alert{collection}.
	\begin{align*}
		\text{1 dozen donuts} &= \text{12 donuts} \\
		\text{1 ream of paper} &= \text{500 sheets of
		paper} \\
		\text{1 case} &= \text{24 cans} \\
		\visible<2->{\intertext{This is also true in chemistry!}
		\SI{1}{\mole} &=
		\underbrace{\SI{6.02214e23}{particles}}_{\mathclap{\text{\alert{Avogadro's
		number}}}}}
	\end{align*}

	\mode<article>{%
	\begin{align*}
		\intertext{This relationship arises from the number of atoms in
		exactly \SI{12}{\gram} of pure carbon-12:}
		\SI{12}{\gram}~\text{C-12} = \SI{1}{\mole}~\text{C-12 atoms} =
		\num{6.022e23}~\text{C-12 atoms} \\
		\intertext{But it extends to \alert{anything}\ldots}
	\end{align*}

	\begin{center}
	\begin{tabular} {>{1 mole of }l @{ = } l}
		C & \num{6.022e23} C atoms \\
		Na & \num{6.022e23} Na atoms \\
		Au & \num{6.022e23} Au atoms \\
		donut & \num{6.022e23} donuts \\
	\end{tabular}
	\end{center}}
\end{frame}

\begin{frame}{Moles as a Conversion Factor}
	\begin{itemize}
		\item Equality:
			\begin{equation*}
				\SI{1}{\mole} = \SI{6.022e23}{particles}
			\end{equation*}
		\item Conversion Factors:
			\begin{align*}
				\frac{\SI{6.022e23}{particles}}{\SI{1}{\mole}}
				\qquad\text{and}\qquad
				\frac{\SI{1}{\mole}}{\SI{6.022e23}{particles}}
			\end{align*}
	\end{itemize}

	\pause
	\bigskip

	How many \ch{Cu} atoms are in \SI{0.50}{\mole} of \ch{Cu}?

	\mode<article>{%
		\begin{equation*}
			\SI{0.50}{\mole}~\ch{Cu} \times
			\frac{\SI{6.022e23}{atoms}}{\SI{1}{\mole}~\ch{Cu}} =
			\SI{3.0e23}{atoms}
		\end{equation*}}

	\bigskip

	\begin{alertblock}{Warning}
		This is the only type of question where Avogadro's number is
		important! Do \textbf{not} automatically apply Avogadro's number
		whenever you see moles!
	\end{alertblock}
\end{frame}

%\begin{onyourown}%{10em}
%	How many \ch{Hg} atoms are in \SI{1.20}{\mole} of \ch{Hg}?
%\end{onyourown}

\begin{frame}{The Utility of Moles}{AKA Why bother?}
	When compounds react, they do so \alert{stoichiometrically}
		-- there is a specific ratio in which they will combine.
		\begin{itemize}[<+(1)->]
			\item Consider the following reaction: %In lab, we looked at
				%\ch{Ca(NO3)2} and \ch{Na2CO3}:
				\begin{align*}
					\ch{Ca(NO3)2\aq{} &+ Na2CO3\aq{} \\ 
					&->
					CaCO3\sld{} + 2 NaNO3\aq{}}
				\end{align*}
				\alert{\SI{1}{\mole}} \ch{Ca(NO3)2}
				reacted with \alert{\SI{1}{\mole}}
				\ch{Na2CO3}
			\item This \alert{does not} mean that \SI{1}{\gram}
				\ch{Ca(NO3)2} reacted with \SI{1}{\gram}
				\ch{Na2CO3}!
			\item Saying 1:1 is much easier than saying
				164.09:84.99.
			\item So how do we convert moles to mass of a compound?
		\end{itemize}
\end{frame}

\begin{frame}{Moles to Mass}
	If \SI{1}{\mole} is defined to be the number of carbon-12 atoms in
	\SI{12}{\gram} of carbon, can we create a conversion factor?

	\begin{columns}
		\column{0.45\linewidth}
		\begin{itemize}[<+(1)->]
			\item Equality:
				\begin{equation*}
					\SI{1}{\mole}~\ch{C}~\text{atoms} = \SI{12}{\gram}
				\end{equation*}
			\item Conversion Factor:
				\begin{equation*}
					\frac{\SI{12}{\gram}}{\SI{1}{\mole}~\ch{C}}
				\end{equation*}
		\end{itemize}
		\column{0.45\linewidth}
		\begin{itemize}[<+(1)->]
			\item Equality:
				\begin{equation*}
					\SI{1}{atom}~\ch{C} = \SI{12}{amu}
				\end{equation*}
			\item Conversion Factor:
				\begin{equation*}
					\frac{\SI{12}{amu}}{\SI{1}{atom}~\ch{C}}
				\end{equation*}
		\end{itemize}
	\end{columns}

	\pause

	\begin{equation*}
		\boxed{\frac{\si{\gram}}{\si{\mole}} =
		\frac{\si{amu}}{\si{atom}}}
	\end{equation*}

	The mass in atomic mass units is equivalent to the \alert{molar mass} of
	the element.
\end{frame}

\begin{frame}{Examples: Converting Moles and Mass}
	Complete the table\ldots

	\bigskip

	\mode<presentation|article:0>{%
	\begin{center}
		\begin{tabular} {l S[table-format=1.3]
			S[table-format=2.3]}
			\toprule
			\bfseries Element & \textbf{\# Moles} & \textbf{Mass (\si{\gram})} \\ \midrule
			\ch{Rh} & 0.250    &                  \\[1em]
			\ch{Fr} &          & 23.003           \\[1em]
			\ch{Cd} & 1.033    &                  \\[1em]
			\ch{W}  &          & 0.014            \\ \bottomrule
		\end{tabular}
	\end{center}
}

	\mode<article>{%
	\begin{center}
		\begin{tabular} {l S[table-format=1.3e-1]
			S[table-format=2.3]}
			\toprule
			\bfseries Element & \textbf{\# Moles} & \textbf{Mass (\si{\gram})} \\ \midrule
			\ch{Rh} & 0.250    & 25.7        \\[1em]
			\ch{Fr} & 0.103    & 23.003      \\[1em]
			\ch{Cd} & 1.033    & 116.0       \\[1em]
			\ch{W}  & 7.6e-5   & 0.014       \\ \bottomrule
		\end{tabular}
	\end{center}}
\end{frame}

%\begin{onyourown}%{0em}
%	Complete the table\ldots
%
%	\renewcommand\arraystretch{1.5}
%	\begin{center}
%		\begin{tabular} {l@{\qquad} S[table-format=3.5]@{\qquad} S[table-format=4.4]}
%		\toprule
%		\bfseries Element & {\bfseries \# Moles} & {\bfseries Mass (\si{\gram})} \\
%		\midrule
%		\ch{Y}  & 3.01   & \\
%		\ch{Ru} &        & 72.10 \\
%		\ch{Bi} & 0.0020 & \\
%		\ch{Mo} &        & 204.3 \\
%		\bottomrule
%	\end{tabular}
%	\end{center}
%\end{onyourown}

% Finished on 2020-08-28

\end{document}
