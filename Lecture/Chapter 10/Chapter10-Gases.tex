% !TEX program = xelatex
\documentclass[handout]{beamer}
%\documentclass[notes=only]{beamer}
%\documentclass[notes=hide]{beamer}
%\documentclass[notes=show]{beamer}

\usepackage{bucolors}
\usepackage{newtxtext}
\usepackage{genchem}
\usepackage{lecture}
\usepackage{multicol}
\usepackage{elements}
\usepackage{collcell}
\usetikzlibrary{tikzmark}
%\usepackage{tabularx}

\title{Gases}
\subtitle{Chapter 10}
\institute[CHEM115 Bloomsburg University]{CHEM115 --- Chemistry for the Sciences I \\ Bloomsburg University}
\author{D.A. McCurry}
\date{Fall 2021}

\chemsetup{chemformula/frac-style=nicefrac}

\begin{document}

\maketitle
\mode<article>{\thispagestyle{fancy}}

\frame{\section{States of Matter: Gas}
	\begin{learningobjectives}
	\item Explain the Kinetic Molecular Theory (KMT) of Gases
	\item Describe the origin and common units of pressure.
	\item Relate gas pressure to temperature, volume, and amount.
	\end{learningobjectives}
}

\begin{frame}{What is a gas?}
	\begin{columns}
		\column{0.65\linewidth}
		A gas consists of small particles that\ldots
		\begin{itemize}
			\item move rapidly in straight lines.
			\item have essentially no attractive (or repulsive) forces.
		\end{itemize}
		\column{0.25\linewidth}
		\begin{center}
			\includegraphics[scale=0.12,trim={0 0 0
			50pt},clip]{10_02_Figure.jpg}
		\end{center}
	\end{columns}

	\begin{block}{The Kinetic Molecular Theory of Gases}
		\begin{enumerate}
			\item Gas particles have very small volumes compared to
				the volumes of the containers they occupy.
			\item Gas particles have \alert{kinetic energies} that
				increase with an increase in temperature.
			\item Collisions of particles with each other or with
				the walls of a container is completely
				\alert{elastic}.
		\end{enumerate}
	\end{block}
\end{frame}

\begin{frame}{Properties of Gases}
	\begin{center}
		\mode<presentation>{\small}
	\begin{tabularx}{\linewidth} {l *{2}{>{\raggedright\arraybackslash}X}}
		\toprule
		\bfseries Property & \bfseries Description & \bfseries Unit(s)
		of Measurement \\ \midrule
		Pressure ($P$) & The force exerted by gas against the walls of
		its container. & atmosphere (\si{\atm}); millimeters of mercury
		(\si{\mmHg}); torr; pascal (\si{\pascal}) \\
		Volume ($V$) & The space occupied by a gas. & liter
		(\si{\liter}); milliliter (\si{\milli\liter}); cubic meter
		(\si{\meter\cubed}) \\
		Temperature ($T$) & The factor that determines the kinetic
		energy and rate of motion of gas particles. & degree Celsius
		(\si{\celsius}); Kelvin (\si{\kelvin}) \\
		Amount ($n$) & The quantity of gas present in a container. &
		grams (\si{\gram}) ; moles (\si{\mole}) \\ \bottomrule
	\end{tabularx}
	\end{center}
\end{frame}

\begin{frame}{Gas Pressure}
		\begin{equation*}
			\text{pressure ($P$)} =
			\frac{\text{force}}{\text{area}} = \frac{F}{A}
		\end{equation*}

		\vspace{-2em}

	\begin{center}
		\begin{tikzpicture}
			\node(pressure) {\includegraphics[scale=1.7]{pressure.png}};
			\node(collisions)[right = 5pt of pressure]
				{\includegraphics[scale=0.18]{10_01_Figure.jpg}};
			\node[right = 5pt of collisions]
				{\includegraphics[scale=0.18]{10_05_Figure.jpg}};
		\end{tikzpicture}
	\end{center}
\end{frame}

\begin{frame}
	\begin{center}
		\includegraphics[width=\linewidth]{10_01_Table.jpg}
	\end{center}
\end{frame}

\begin{frame}{Atmospheric Pressure}
		The pressure exerted by a column of air from the top of the atmosphere
		to the surface of the Earth.

		\bigskip
	
		\begin{columns}
		\column{0.55\textwidth}
		Atmospheric pressure\ldots
		\begin{itemize}
			\item is about 1 atmosphere at sea level.
			\item is lower at high altitudes where the density of air is
				less.
			\item falls when poor weather is coming and rises during
				mild weather.
		\end{itemize}
		\column{0.45\textwidth}
		\begin{center}
			\includegraphics[scale=0.37]{atm-pressure.jpg}
		\end{center}
	\end{columns}
\end{frame}

\begin{frame}[allowframebreaks]{Methods of Measuring Pressure}
	\begin{columns}
		\column{0.5\textwidth}
		A \alert{barometer}\ldots
		\begin{itemize}
			\item measures the pressure exerted by the gases in the
				atmosphere.
			\item indicates atmospheric pressure as the height in
				millimeters of the mercury column.
		\end{itemize}
		\column{0.5\textwidth}
		\begin{center}
			\includegraphics[scale=0.4]{10_07_Figure.jpg}
		\end{center}
	\end{columns}

	\framebreak

	\begin{columns}
		\column{0.4\textwidth}
		A \alert{manometer}\ldots
		\begin{itemize}
			\item compares the pressure exerted by an enclosed gas to the
				atmosphere.
			\item indicates atmospheric pressure as the height in
				millimeters of the mercury column.
		\end{itemize}
		\column{0.6\textwidth}
		\begin{center}
			\includegraphics[scale=0.25]{10_08_Figure.jpg}
		\end{center}
	\end{columns}
\end{frame}

\begin{frame}[allowframebreaks]{Boyle's Law}
		\begin{itemize}
			\item The pressure ($P$) of a gas is \alert{inversely} related to
				its volume ($V$) when temperature ($T$) and moles of
				particles ($n$) are constant.
				\begin{equation*}
					P \propto \frac{1}{V} \qquad \text{($T$, $n$
					constant)}
				\end{equation*}
				If the pressure ($P$) increases, then the volume ($V$)
				decreases.
					\bigskip

				\begin{center}
					\includegraphics[scale=0.3,trim={0 100pt
					0 0},clip]{10_09_Figure.jpg}
				\end{center}
			\item The product $P \times V$ is \alert{constant} as long as $T$
				and $n$ \alert{do not change}.
				\begin{align*}
					P_1V_1 &= \SI{8.0}{\atm} \times \SI{2.0}{\liter}
					= \SI{16}{\atm\liter} \\
					P_2V_2 &= \SI{4.0}{\atm} \times \SI{4.0}{\liter}
					= \SI{16}{\atm\liter} \\
					P_3V_3 &= \SI{2.0}{\atm} \times \SI{8.0}{\liter}
					= \SI{16}{\atm\liter}
			\end{align*}
			\item We can therefore write Boyle's Law to compare two
				conditions as
				\begin{equation*}
					P_1V_1 = P_2V_2			
				\end{equation*}
		\end{itemize}
\end{frame}

\begin{frame}{Boyle's Law --- A Diver's View}
		\begin{center}
			\includegraphics[scale=0.5]{breathing.png}
		\end{center}
	
		\footnotesize{%
		Image from Flowers, P.; Theopold, K.; Langley, R.; Neth, E. J.; Robinson, W. R.;
		OpenStax College. Chemistry: Atoms First 2e; 2019.
	}
\end{frame}

\begin{frame}
	\begin{center}
		\includegraphics[width=\linewidth]{10_12_Figure.jpg}
	\end{center}
\end{frame}

\begin{frame}[allowframebreaks]{Charles's Law}
		\begin{itemize}
			\item The temperature ($T$ in \si{\kelvin}) of a gas is
				\alert{directly} related to the volume ($V$) when
				$P$ and $n$ are constant.
				\begin{equation*}
					T \propto V \qquad \text{(constant $P$
					and $n$)}
				\end{equation*}
				When the temperature ($T$) of a gas increases,
				its volume ($V$) increases.
					\bigskip
	
					\begin{center}
						\includegraphics[scale=0.3,trim={0 100pt
						0 50pt},clip]{10_13_Figure.jpg}
					\end{center}
			\item 	The ratio $V \div T$ is \alert{constant} as long as $P$
				and $n$ \alert{do not change}.
				\begin{align*}
					\frac{V_1}{T_1} &=
					\frac{\SI{10.0}{\liter}}{\SI{250.0}{\kelvin}}
					= \SI{0.04}{\liter\per\kelvin} \\
					\frac{V_2}{T_2} &=
					\frac{\SI{5.0}{\liter}}{\SI{125.0}{\kelvin}}
					= \SI{0.04}{\liter\per\kelvin}
			\end{align*}
			\item We can therefore write Charles's Law to compare two
				conditions as
				\begin{equation*}
					\frac{V_1}{T_1} = \frac{V_2}{T_2}
				\end{equation*}
		\end{itemize}
\end{frame}

\begin{frame}{Using Charles's Law to Find Absolute Zero}
	\begin{center}
		\includegraphics[scale=0.5,trim={0 0 0 110pt},clip]{10_14_Figure.jpg}
	\end{center}
\end{frame}

\begin{frame}[allowframebreaks]{Avogadro's Law}
		\begin{itemize}
			\item The amount ($n$) of a gas is \alert{directly}
				related to the volume ($V$) when $P$ and $T$ are
				constant.
				\begin{equation*}
					n \propto V \qquad \text{(constant $P$
					and $T$)}
				\end{equation*}
				When the amount ($n$) of a gas increases,
				its volume ($V$) increases.
				\begin{center}
					\includegraphics[scale=0.4]
					{Standard-balloon-sizes.jpg}
				\end{center}
			\item 	The ratio $V \div n$ is \alert{constant} as long as $P$
				and $T$ \alert{do not change}.
				\begin{align*}
					\frac{V_1}{n_1} &=
					\frac{\SI{10.0}{\liter}}{\SI{2.0}{\mole}}
					= \SI{5.0}{\liter\per\mole} \\
					\frac{V_2}{n_2} &=
					\frac{\SI{20.0}{\liter}}{\SI{4.0}{\mole}}
					= \SI{5.0}{\liter\per\mole}
			\end{align*}
			\item We can therefore write Avogadro's Law to compare
				two conditions as
				\begin{equation*}
					\frac{V_1}{n_1} = \frac{V_2}{n_2}
				\end{equation*}
		\end{itemize}
\end{frame}

\begin{frame}[t]{Simple Gas Laws Practice 1}
	A \SI{2.50}{\liter} volume of hydrogen measured at \SI{-196}{\celsius}
	is warmed to \SI{100}{\celsius}. Calculate the volume of the gas at the
	higher temperature, assuming no change in pressure.

	\vfill

	\note{
		Which law? \qquad \alert{Charles's Law}

		\begin{center}
			\begin{tabular} {L r@{ }s}
				V_1 & 2.50 & \liter \\
				T_1 & -196 & \celsius \\
				T_2 & 100 & \celsius
			\end{tabular}
		\end{center}

		We need temperatures in Kelvin!
		\begin{align*}
			T_1 &= \SI{-196}{\celsius} + \SI{273}{\kelvin} =
			\SI{77}{\kelvin} \\
			T_2 &= \SI{100}{\celsius} + \SI{273}{\kelvin} =
			\SI{373}{\kelvin}
		\end{align*}

		\begin{align*}
			\frac{V_1}{T_1} &= \frac{V_2}{T_2} \\
			V_2 &= \frac{V_1T_2}{T_1}
			=
			\frac{(\SI{2.50}{\liter})(\SI{373}{\kelvin})}{\SI{77}{\kelvin}}
			= \boxed{\SI{12}{\liter}}
		\end{align*}
	}
\end{frame}
	
\begin{frame}[t]{Simple Gas Laws Practice 2}
	A balloon inflated with three breaths of air has a volume of
	\SI{1.7}{\liter}. At the same temperature and pressure, what is the
	volume if five more same-sized breaths are added to the balloon?

	\vfill

	\note{
		Which law? \qquad \alert{Avogadro's Law}

		\begin{center}
			\begin{tabular} {L r@{ }s}
				V_1 & 1.7 & \liter \\
				n_1 & 3 & breaths \\
				n_2 & $3 + 5 = 8$ & breaths
			\end{tabular}
		\end{center}

		\begin{align*}
			\frac{V_1}{n_1} &= \frac{V_2}{n_2} \\
			V_2 &= \frac{V_1n_2}{n_1}
			=
			\frac{(\SI{1.7}{\liter})(\SI{8}{breaths})}{\SI{3}{breaths}}
			= \boxed{\SI{4.5}{\liter}}
		\end{align*}
	}
\end{frame}

\frame{\section{The Ideal Gas Law}
	\begin{learningobjectives}
	\item Unify the three simple gas laws into one equation.
	\item Use the ideal gas law to identify trends in gas properties.
	\item Calculate the molar volume of a gas at standard temperature and
		pressure (STP).
	\item Calculate the density of a gas.
	\end{learningobjectives}
}

\begin{frame}{Simple Gas Laws Summary}
	\begin{center}
		\def\arraystretch{2}
		\def\tabcolsep{2em}
		\begin{tabular} {l L >{(}c<{)}}
			Boyle's Law & V \propto \dfrac{1}{P} & constant $n$ and
			$T$ \\
			Charle's Law & V \propto T & constant $n$ and $P$ \\
			Avogadro's Law & V \propto n & constant $T$ and $P$
		\end{tabular}
	\end{center}

	\bigskip

	\pause

	We can relate each quantity to each other:
	\begin{align*}
		V &\propto \frac{1}{P} \propto T \propto n \\
		V &\tikzmark{propto}\propto \frac{Tn}{P}
	\end{align*}

	\pause

	\begin{tikzpicture}[remember picture, overlay]
		\node(start) at ($(pic cs:propto) + (1.5ex,0)$) {};
		\draw[thick,<-] (start) to [out=270,in=180]($(start) + (0.5,-0.75)$)
		node[right] {What is this
		proportionality factor?};
	\end{tikzpicture}
\end{frame}

\begin{frame}{The Ideal Gas Law}
	The \alert{proportionality constant} between properties is known as the
	\alert{ideal gas constant}:
	\begin{equation*}
		R = \SI{0.08206}{\liter\atm\per\mole\per\kelvin}
	\end{equation*}

	\pause

	\begin{center}
		\usebeamercolor{block body}
		\begin{tikzpicture}[node distance=3em]
			\tikzstyle{arrow}=[thick,->,shorten >=0.5em, shorten
			<=0.5em];
			\tikzstyle{box}=[inner sep=1em,
			fill=bg];
			\node(prop) {$V \propto \dfrac{Tn}{P}$};
			\node[right = of prop](equal)
				{$V = R\dfrac{Tn}{P}$};
			\node[right = of equal,box] (ideal)
				{$PV = nRT$};
%			\node[above = -1pt of ideal, font=\bfseries\footnotesize,box] {Ideal
%			Gas Law:};
			\draw[arrow] (prop) -- (equal);
			\draw[arrow] (equal) -- (ideal);
		\end{tikzpicture}
	\end{center}

	\pause

	At \alert{STP} (standard temperature and pressure, $T =
	\SI{273.15}{\kelvin}$ and $P = \SI{1.00}{\atm}$), \SI{1.00}{\mole} of
	gas occupys \SI{22.4}{\liter}:
	\begin{equation*}
		V = \tfrac{nRT}{P}
		=
		\tfrac{(\SI{1.00}{\mole})(\SI{0.08206}{\liter\atm\per\mole\per\kelvin})(\SI{273.15}{\kelvin})}{\SI{1.00}{\atm}}
		= \boxed{\SI{22.4}{\liter}}
	\end{equation*}
\end{frame}

\begin{frame}[t]{Gas Law Trends}
	Using gas laws, complete the following statements with \emph{increases}
	or \emph{decreases}.

	\begin{enumerate}
		\item Pressure \rule{6em}{0.4pt}, when $V$ decreases.
		\item When $T$ decreases, $V$ \rule{6em}{0.4pt}.
		\item Pressure \rule{6em}{0.4pt} when $V$ changes from
			\SIrange{12}{4}{\liter}.
		\item Volume \rule{6em}{0.4pt} when $T$ changes from
			\SIrange{15}{45}{\celsius}.
	\end{enumerate}
\end{frame}

\vspace{\stretch{-1}}

\newcounter{idealgaslaw}

\stepcounter{idealgaslaw}
\begin{frame}[t]{Ideal Gas Law Practice \arabic{idealgaslaw}}
	The \ch{N2} gas in an airbag, with a volume of \SI{65}{\liter}, exerts a
	pressure of \SI{829}{\mmHg} at \SI{25.0}{\celsius}. How many moles of
	\ch{N2} gas are in the airbag?

	\vfill

	\note{
		\begin{align*}
			PV &= nRT \\
			n &= \frac{PV}{RT} \\
			&=
			\frac{(\SI{829}{\mmHg}\times\frac{\SI{1.00}{\atm}}{\SI{760}{\mmHg}})(\SI{65}{\liter})}
			{(\SI{0.08206}{\liter\atm\per\mole\per\kelvin})((273.15
			+ 25.0)~\si{\kelvin})}\\
			&= \SI{2.89793052}{\mole} \\
			n &= \boxed{\SI{2.9}{\mole}}
		\end{align*}
	}
\end{frame}

\stepcounter{idealgaslaw}
\begin{frame}[t]{Ideal Gas Law Practice \arabic{idealgaslaw}}
	A sample of \ch{CO} gas in a \SI{10.0}{\liter} cylinder at
	\SI{25.0}{\celsius} and \SI{1.00}{\atm} of presure is compressed to a
	final temperature and volume of \SI{50.0}{\celsius} and
	\SI{5.0}{\liter}, respectively. What is the final pressure?

	\vfill

	\note{
		\begin{enumerate}
			\item We need to know
				\alert{how much} \ch{CO} is in the cylinder to
				begin with:
				\begin{align*}
					PV &= nRT \\
					n &= \frac{PV}{RT}
					=
					\frac{(\SI{1.00}{\atm})(\SI{10.0}{\liter})}
					{(\SI{0.08206}{\liter\atm\per\mole\per\kelvin})((273.15
					+ 25.0)\,\si{\kelvin})} \\
					n &= \SI{0.40872733}{\mole}
				\end{align*}
			\item Then we can use the number of moles to calculate
				the new pressure of the gas:
				\begin{align*}
					PV &= nRT \\
					P &= \frac{nRT}{V}
					=
					\frac{(\SI{0.40872733}{\mole})(\SI{0.08206}{\liter\atm\per\mole\per\kelvin})((273.15
					+ 50.0)\,\si{\kelvin})}{\SI{5.0}{\liter}}
					\\
					P &= \SI{2.16770082}{\atm}
					= \boxed{\SI{2.2}{\atm}}
				\end{align*}
		\end{enumerate}
	}
\end{frame}

%\begin{onyourown}
%	What mass of zinc would be needed to prepare \SI{2.5}{\liter} of \ch{H2}
%	gas collected at \SI{1.0}{\atm} and \SI{24}{\celsius}?
%	\begin{reaction*}
%		Zn\sld{} + 2 H^{+}\aq{} -> H2\gas{} + Zn^{2+}\aq{}
%	\end{reaction*}
%\end{onyourown}

\begin{frame}{The Molar Volume of a Gas at STP}
	\begin{columns}
		\column{0.5\textwidth}
		\begin{center}
		\includegraphics[scale=0.275]{10_Pg430_UnFigure.jpg}
	\end{center}
		\column{0.5\textwidth}
		\begin{itemize}
			\item Regardless of the composition of the gas,
				\alert{\SI{1}{\mole}} of a gas occupies a volume of
				\alert{\SI{22.4}{\liter}} at \alert{STP}.
			\item We can use this relationship as a conversion
				factor:
				\begin{align*}
					\frac{\SI{1}{\mole}}{\SI{22.4}{\liter}}
					\textit{\quad{}or\quad}
					\frac{\SI{22.4}{\liter}}{\SI{1}{\mole}}
				\end{align*}
		\end{itemize}
	\end{columns}
\end{frame}

\begin{frame}{The Density of a Gas at STP}
	\begin{align*}
		\mathllap{\text{Recall~}}\text{density ($d$)} &= \frac{\text{mass ($m$)}}{\text{volume
	($V$)}}
		\intertext{If we know the molar masses and molar volume, we can
		calculate the density of a gas:}
		d &= \frac{\text{molar mass}}{\text{molar volume}} \\
		d_{\ch{He}} &=
		\frac{\SI{4.00}{\gram\per\cancel\mole}}{\SI{22.4}{\liter\per\cancel\mole}}\tikzmark{cancelmole}
		= \SI{0.179}{\gram\per\liter}
	\end{align*}

	\begin{tikzpicture}[remember picture,overlay]
		\node(start) at ($(pic cs:cancelmole) + (-2ex,-1ex)$) {};
		\draw[thick,<-,shorten <=5pt] (start) to [out=300,in=180]
		++(1,-0.75)
		node[right] {Note that moles cancel!};
	\end{tikzpicture}

	\pause

	\bigskip

	Because the density of a gas is \alert{directly proportional} to its
	mass, gases with a greater molar mass have a larger density.
\end{frame}

\vspace{\stretch{-1}}

\begin{frame}{The Density of a Gas}
	We are almost never at STP\ldots

	\bigskip

	As such, we need a relationship to describe the density of a gas at
	conditions other than STP:
	\begin{align*}
		PV &= nRT
		\action<+(1)->{
		\intertext{Rearranging to get the molar volume:}
	\frac{V}{n} &= \frac{RT}{P}}
	\action<+(1)->{\intertext{And plugging in to density with molar mass
		($\mathcal{M}$) as $m$:}
		d = \frac{m}{V} &= \frac{\mathcal{M}}{\frac{RT}{P}} = 
	\frac{P\mathcal{M}}{RT}}\tikzmark{gasdensity}
	\end{align*}

	\pause

	\usebeamercolor{block body}
	\begin{tikzpicture}[remember picture, overlay]
		\node[right,inner sep=1em,minimum height=3em,
			fill=bg] at ($(pic cs:gasdensity) + (1,0)$) {$d =
		\dfrac{P\mathcal{M}}{RT}$};
	\end{tikzpicture}
\end{frame}

\vspace{\stretch{-1}}

\begin{frame}[t]{Practicing Gas Density Calculations}
	Calculate the density of xenon gas at a pressure of \SI{742}{\mmHg} and
	a temperature of \SI{45}{\celsius}.

	\note{
		\begin{align*}
			P &= \SI{742}{\mmHg} \times
			\frac{\SI{1}{\atm}}{\SI{760}{\mmHg}} =
			\SI{0.976316}{\atm} \\
			T &= (45 + 273.15)~\si{\kelvin} = \SI{318.15}{\kelvin}
			\\
			d &= \frac{P\mathcal{M}}{RT} \\
			&=
			\frac{(\SI{0.976316}{\atm})(\SI{131.29}{\gram\per\mole})}{(\SI{0.08206}{\liter\atm\per\mole\per\kelvin})(\SI{318.15}{\kelvin})}
			\\
			&= \SI{4.90974}{\gram\per\liter} \\
			&= \boxed{\SI{4.91}{\gram\per\liter}}
		\end{align*}
	}
\end{frame}

\begin{frame}[t]{A Really Complicated Gas Density Calculation}
	Cyanogen, a highly toxic gas, is \SI{46.2}{\percent}~\ch{C} and
	\SI{53.8}{\percent}~\ch{N} by mass. At \SI{25.0}{\celsius} and
	\SI{751}{\torr}, cyanogen has a density of \SI{2.10}{\gram\per\liter}.
	What is the molecular formula of cyanogen?

	\mode<article>{\vspace{20em}}

	\note{
		\begin{enumerate}
			\item We need a plan. Equations:
				\begin{align*}
					d &= \frac{P\mathcal{M}}{RT} \\
					\text{mass \%} &= \frac{\text{mass of $x$}}{\text{total mass}} \times \SI{100}{\percent}
				\end{align*}
				
				What's the relationship? Molar mass
				$\leftrightarrow$ mass \%
			\item Solve for molar mass:
				\begin{align*}
					d &= \frac{P\mathcal{M}}{RT} \\
					\mathcal{M} &= \frac{dRT}{P} \\
					&= \frac{(\SI{2.10}{\gram\per\liter})(\SI{0.08206}{\liter\atm\per\mole\per\kelvin})\big((25.0 + 273.15)~\si{\kelvin}\big)}{\SI{751}{\torr} \times \frac{\SI{1}{\atm}}{\SI{760}{\torr}}} \\
					&= \SI{51.99472}{\gram\per\mole}
				\end{align*}
		\end{enumerate}
	}
\end{frame}

\note{
\begin{enumerate}
	\setcounter{enumi}{2}
			\item Solve for masses of \ch{C} and \ch{N}:
				\begin{align*}
					\text{mass \%} &= \frac{\text{mass of $x$}}{\text{total mass}} \times \SI{100}{\percent} \\
					\text{mass of $x$} &= \frac{\text{mass \%}}{\SI{100}{\percent}} \times \text{total mass} \\
					\text{mass of \ch{C}} &= \frac{\SI{46.2}{\percent}}{\SI{100}{\percent}}	\times \SI{51.99472}{\gram\per\mole} = \SI{24.02156}{\gram\per\mole} \\
					\text{mass of \ch{N}} &= \frac{\SI{53.8}{\percent}}{\SI{100}{\percent}} \times \SI{51.99472}{\gram\per\mole} = \SI{27.97316}{\gram\per\mole}
				\end{align*}
		\end{enumerate}
	}

	\note{
		\begin{enumerate}
			\setcounter{enumi}{3}
			\item Determine \# of \ch{C} and \ch{N}:
				\begin{align*}
					\frac{\SI{24.02156}{\gram\per\mole}}{\SI{12.011}{\gram\per\mole}} &= 1.99964 \approx 2 \\
					\frac{\SI{27.97316}{\gram\per\mole}}{\SI{14.007}{\gram\per\mole}} &= 1.99708 \approx 2
				\end{align*}
			\item Cyanogen is therefore \ch{C2N2} or \ch{(CN)2}
		\end{enumerate}
	}

\frame{\section{Dalton's Law of Partial Pressures}
	\begin{learningobjectives}
	\item Explain how different gases contribute to the measured pressure.
	\item Use mole fractions to quickly calculate partial pressures.
	\end{learningobjectives}
}

\begin{frame}{Partial Pressure}
	From the kinetic molecular theory, the size of gas particles is
	negligible and all collisions are elastic.
	\begin{itemize}
		\item The pressure of each gas in a mixture can be calculated
			independently:
			\begin{equation*}
				P_a = n_a \frac{RT}{V} \qquad P_b = n_b
				\frac{RT}{V} \qquad
				\cdots \qquad P_\infty = n_\infty \frac{RT}{V}
			\end{equation*}
		\item We assume that the pressure exerted by a gas is calculated
			as if it is the only gas in the container.
	\end{itemize}

	\pause

	\bigskip

	\begin{block}{Dalton's Law of Partial Pressures}
		The sum of the pressures of each individual gas in a mixture is
		equal to the total pressure:
		\begin{equation*}
			P = P_a + P_b + \cdots + P_\infty
		\end{equation*}
	\end{block}
\end{frame}

\begin{frame}{Dalton's Law of Partial Pressures}
	\begin{center}
		\includegraphics[scale=0.3]{partial-pressure.png}

		\tiny{\url{https://chem.libretexts.org/Bookshelves/General_Chemistry/Book\%3A_ChemPRIME_(Moore_et_al.)/09Gases/9.11\%3A_Dalton's_Law_of_Partial_Pressures}}
	\end{center}

	\bigskip

	At STP, \SI{1}{\mole} of a pure gas in a \SI{22.4}{\liter} container
	will exert the same pressure as \SI{1}{\mole} of a gas mixture in
	\SI{22.4}{\liter}.
\end{frame}

\begin{frame}[t]{Dalton's Law Practice 1}
	A gas mixture at \SI{322}{\kelvin} in a \SI{10.0}{\liter} container
	has a total pressure of \SI{2.5}{\atm}. Gas A has a partial pressure of
	\SI{1.0}{\atm} and Gas B has a partial pressure of \SI{0.5}{\atm}. How
	many moles of Gas C are present?

	\vspace{15em}

	\note{
		\begin{align*}
			P_\text{total} &= P_\text{A} + P_\text{B} + P_\text{C}
			\\
			P_\text{C} &= P_\text{total} - P_\text{A} - P_\text{B}
			\\
			&= \SI{2.5}{\atm} - \SI{1.0}{\atm} - \SI{0.5}{\atm} \\
			&= \SI{1.0}{\atm} \\
			PV &= nRT \\
			n &= \frac{PV}{RT} \\
			&=
			\frac{(\SI{1.0}{\atm})(\SI{10.0}{\liter})}{(\SI{0.08206}{\liter\atm\per\mole\per\kelvin})(\SI{322}{\kelvin})}
			\\
			&= \SI{0.3784535781272}{\mole} \\
			&= \boxed{\SI{0.38}{\mole}}
		\end{align*}
	}
\end{frame}

\begin{frame}[t]{Dalton's Law Practice 2}
	What is the total pressure of a gaseous mixture containing
	\SI{0.250}{\mole}~\ch{He}, \SI{0.992}{\mole}~\ch{H2}, and
	\SI{0.107}{\mole}~\ch{N2} in a \SI{10.0}{\liter} container at
	\SI{300}{\kelvin}? What is the $P$ of just \ch{He} in the mixture?

	\vfill

	\note<1>{
		\begin{align*}
			n_\text{total} &= \SI{0.250}{\mole} + \SI{0.992}{\mole}
			+ \SI{0.107}{\mole} \\
			&= \SI{1.349}{\mole} \\
			PV &= nRT \\
			P_\text{total} &= \frac{n_\text{total}RT}{V}
			=
			\frac{(\SI{1.349}{\mole})(\SI{0.08206}{\liter\atm\per\mole\per\kelvin})(\SI{300}{\kelvin})}{\SI{10.0}{\liter}}
			\\
			&= \SI{3.3209682}{\atm} \\
			&= \boxed{\SI{3.32}{\atm}} \\
			P_{\ch{He}} &= \frac{n_{\ch{He}}RT}{V} =
			\frac{(\SI{0.250}{\mole})(\SI{0.08206}{\liter\atm\per\mole\per\kelvin})(\SI{300}{\kelvin})}{\SI{10.0}{\liter}}
			\\
			&= \SI{0.61545}{\atm} \\
			&= \boxed{\SI{0.615}{\atm}}
		\end{align*}
	}

	\pause

	What is the percentage of He in the mixture\ldots
	\begin{enumerate}[a)]
		\item in terms of moles?
		\item in terms of pressure?
	\end{enumerate}

	\mode<article>{\vspace{10em}}

	\note<2>{
		\begin{enumerate}[a)]
			\item Moles:
			\begin{align*}
				\frac{\SI{0.250}{\mole}}{\SI{1.349}{\mole}}
				\times \SI{100}{\percent} &= \SI{18.5}{\percent}
				\intertext{\item Pressure:}
				\frac{\SI{0.615}{\atm}}{\SI{3.32}{\atm}}
				\times \SI{100}{\percent} &= \SI{18.5}{\percent}
			\end{align*}
	\end{enumerate}
}
\end{frame}

%\begin{onyourown}[15em]
%	What is the total pressure of a gaseous mixture containing
%	\SI{0.301}{\mole}~\ch{O2}, \SI{0.788}{\mole}~\ch{N2}, and
%	\SI{0.261}{\mole}~\ch{CH4} in a \SI{3.0}{\liter} container at
%	\SI{288}{\kelvin}? What is the pressure of just \ch{CH4} in the mixture?
%\end{onyourown}

\begin{frame}{Mole Fractions and Partial Pressures}
	If we consider the pressure of Gas A in a mixture, we have
	\begin{align*}
		P_\text{A} &= n_\text{A} \frac{RT}{V}
		\action<+(1)->{\intertext{The total pressure of the mixture is given by}
		P_\text{total} &= n_\text{total} \frac{RT}{V}}
		\action<+(1)->{\intertext{The \alert{ratio} of the partial pressure of Gas A to
			the total pressure of the system is taken by dividing
		the values of pressure:}
		\tikzmark{presfrac}\frac{P_\text{A}}{P_\text{total}} &= \frac{n_\text{A}
		\cancel{\frac{RT}{V}}}{n_\text{total} \cancel{\frac{RT}{V}}} =
		\frac{n_\text{A}}{n_\text{total}}\tikzmark{molefrac}}
	\end{align*}

	\pause

	\usebeamercolor{block body}
	\begin{tikzpicture}[remember picture,overlay]
		\tikzstyle{box}=[inner sep=1em,minimum height=3em, fill=bg];
		\node[box,right,align = center] at ($(pic cs:molefrac) + (1,0)$)
		{\textbf{Mole Fraction} \\ $\mathcal{X}_\text{A} = \frac{n_\text{A}}{n_\text{total}}$};
	\end{tikzpicture}

	\pause

	\begin{tikzpicture}[remember picture,overlay]
		\tikzstyle{box}=[inner sep=1em,minimum height=3em, fill=bg];
		\node[box,left,align = center] at ($(pic cs:presfrac) + (-1,0)$)
		{$P_\text{A} = \mathcal{X}_\text{A} P_\text{total}$};
	\end{tikzpicture}
\end{frame}

\vspace{\stretch{-1}}

\begin{frame}{Analysis of a Hydrogen Peroxide Solution}{Experiment 13}
	We needed to correct for the pressure of all gases that can be
	collected in the apparatus.
	\begin{equation*}
		P_\text{total} = P_{\ch{O2}} + P_{\ch{H2O}}
	\end{equation*}

	\begin{center}
	\begin{tikzpicture}
		\node(vapor)
			{\includegraphics[width=0.4\linewidth]{10_18_Figure.jpg}};

		\node[right = of vapor, align=left, text width = 0.4\linewidth] {
		\begin{align*}
			\intertext{When the volumes of the bulb and buret were
			level,}
			P_\text{total} &= P_{\text{bar}}
			\shortintertext{thus}
			P_{\text{bar}} &= P_{\ch{O2}} + P_{\ch{H2O}} \\
			P_{\ch{O2}} &= P_{\text{bar}} - P_{\ch{H2O}} 
		\end{align*}
	};
\end{tikzpicture}
\end{center}
\end{frame}

\vspace{\stretch{-1}}

%\begin{onyourown}
%	When a certain amount of \ch{Zn\sld{}} is reacted with an excess of
%	\ch{HCl\aq{}} at \SI{25}{\celsius}, a total of \SI{0.951}{\liter} of gas
%	was collected with a total pressure of \SI{748.0}{\mmHg}. What mass of
%	\ch{H2} gas was collected? The vapor pressure of water at this
%	temperature is \SI{23.78}{\mmHg}.
%
%	\vspace{15em}
%
%	What would be the minimum amount of \ch{Zn} that would needed to produce
%	the \ch{H2\gas{}} calculated above?
%\end{onyourown}

\frame{\section{Gas Velocity}
	\begin{learningobjectives}
	\item Explain how temperature contributes to the molecular velocity of
		gases.
	\item Describe how gas composition affects gas velocity.
	\end{learningobjectives}
}

\begin{frame}{Temperature and Molecular Velocities}
	Recall from Kinetic Molecular Theory, that gases have \alert{kinetic
	energies} that are dependent on temperature.
	\begin{equation*}
		\text{KE} = \frac{1}{2} mV^2
	\end{equation*}

	\pause

	As such, the \emph{only} way that gases with different compositions (and
	therefore different masses) can have the same kinetic energy is if the
	gas particles have different \alert{velocities}.
	\begin{center}
		\begin{tikzpicture}
			\node(graph) {\includegraphics[scale=0.28,trim={0 0 0
				40pt},clip]{10_19_Figure.jpg}};
			\node[right = 5pt of graph, align = center, text width =
				0.25\linewidth] {
		\begin{equation*}
			\mu_\text{rms} = \sqrt{\frac{3RT}{\mathcal{M}}}
		\end{equation*}
	};
	\end{tikzpicture}
\end{center}
\end{frame}

\vspace{\stretch{-1}}

\begin{frame}[t]{Practicing Gas Velocity Calculations}
	What is the root mean square velocity of \ch{CO2} at \SI{25}{\celsius}?

	\note{
		\begin{align*}
			\mu_\text{rms} &= \sqrt{\frac{3RT}{\mathcal{M}}} \\
			&=
			\sqrt{\frac{(3)(\SI{8.314}{\joule\per\mole\per\kelvin})\big((25
				+
298.15)~\si{\kelvin}\big)}{\SI{44.0095}{\gram\per\mole}}} \\
&=
\sqrt{\frac{(3)(\SI{8.314}{\kilo\gram\meter\squared\per\second\squared\per\mole\per\kelvin})(\SI{298.15}{\kelvin})}{\SI{0.0440095}{\kilo\gram\per\mole}}}
	\\
	&= \boxed{\SI{411}{\meter\per\second}}
\end{align*}
}
\end{frame}

%\begin{onyourown}
%	What is the root mean square velocity of \ch{Cl2} at \SI{45}{\celsius}?
%\end{onyourown}

\begin{frame}{Mean Free Path, Diffusion, and Effusion}
	Gases travel in a straight line until they collide with something else
	-- just because gases are travelling at \SI{>400}{\meter\per\second}
	does not mean they move faster than sound (\SI{\sim
	343}{\meter\per\second})!

	{\renewcommand\arraystretch{1.5}
	\begin{tabularx}{\linewidth} {>{\bfseries}r X}
		Mean Free Path: & The average distance that a molecule
		  travels between collisions. \\
		Diffusion: & The process in which gases spread out in
		  response to a concentration gradient. \\
		Effusion: & The process by which gases escape containers
			through small holes.
	\end{tabularx}}

	\begin{align*}
		\text{rate of effusion} \propto
		\sqrt{\frac{1}{\mathcal{M}}}\tikzmark{effusion}
		\qquad\qquad
		\overbracket[0pt]{\frac{\text{rate}_\text{A}}{\text{rate}_\text{B}} =
		\sqrt{\frac{\mathcal{M}_\text{B}}{\mathcal{M}_\text{A}}}}^{\textbf{Graham's
		Law of Effusion}}
	\end{align*}
\end{frame}

\vspace{\stretch{-1}}

\begin{frame}{Diffusion vs. Effusion}
	\includegraphics[width=0.4\linewidth,height=15em,keepaspectratio]{10_21_Figure.jpg}
	\hfill
	\includegraphics[width=0.5\linewidth,height=15em,keepaspectratio]{10_22_Figure.jpg}
\end{frame}

\vspace{\stretch{-1}}

\begin{frame}[t]{Graham's Law of Effusion Practice}
	An unknown gas effuses at a rate that is \num{0.672} times that of
	oxygen gas at the same temperature. Calculate the molar mass of the
	unknown gas in \si{\gram\per\mole}.

	\vfill

	\note<1>{
		\begin{align*}
			\frac{\text{rate}_\text{unk}}{\text{rate}_{\ch{O2}}} &=
			\sqrt{\frac{\mathcal{M}_{\ch{O2}}}{\mathcal{M}_\text{unk}}}
			\\
			\bigg(\frac{\text{rate}_\text{unk}}{\text{rate}_{\ch{O2}}}\bigg)^2
			&=
			\frac{\mathcal{M}_{\ch{O2}}}{\mathcal{M}_\text{unk}} \\
			\mathcal{M}_\text{unk} &=
			\frac{\mathcal{M}_{\ch{O2}}}{\big(\frac{\text{rate}_\text{unk}}{\text{rate}_{\ch{O2}}}\big)^2}
			\\
			&= \frac{\SI{31.998}{\gram\per\mole}}{0.672^2} \\
			&= \boxed{\SI{70.9}{\gram\per\mole}}
		\end{align*}

		\ch{Cl2}
	}

	\pause

	What is the gas?
\end{frame}

%\begin{onyourown}
%	An unknown gas effuses at a rate that is \num{1.689} times that of
%	bromine gas at the same temperature. Calculate the molar mass of the
%	unknown gas in \si{\gram\per\mole}. Identify the gas.
%\end{onyourown}

\frame{\section{Practical Gas Calculations}
	\begin{learningobjectives}
	\item Apply stoichiometry to calculate gas quantities in chemical
		reactions.
	\item Qualitatively explain how (and why) the ideal gas law is not perfect.
	\end{learningobjectives}
}

\begin{frame}{Gas Stoichiometry}
	All gas laws can be applied to reactions/stoichiometry!

	\begin{center}
		\includegraphics[scale=0.4]{10_Pg444_UnFigure_2.jpg}

		\bigskip

		\includegraphics[scale=0.4]{10_Pg444_UnFigure_3.jpg}
	\end{center}
\end{frame}

\begin{frame}[t]{Gas Stoichiometry Practice}
	What volume (in \si{\liter}) of hydrogen gas, at a temperature of
	\SI{355}{\kelvin} and a pressure of \SI{738}{\mmHg}, do we need to
	synthesize \SI{35.7}{\gram} of methanol from the reaction below?
	\begin{reaction*}
		CO\gas{} + 2 H2\gas{} -> CH3OH\gas{}
	\end{reaction*}

	\vspace{10em}

	\note{
		\begin{enumerate}
			\item We need pressure in \si{\atm}:
				\begin{align*}
					P &= \SI{738}{\mmHg} \times
					\frac{\SI{1}{\atm}}{\SI{760}{\mmHg}} =
					\SI{0.971052631578947}{\atm}
				\end{align*}
			\item How do we convert mass of methanol to
				\ch{H2\gas{}}?
				\begin{align*}
					\SI{35.7}{\gram}~\ch{CH3OH} \times
					\frac{\SI{1}{\mole}~\ch{CH3OH}}{\SI{32.042}{\gram}~\ch{CH3OH}}
					\times
					\frac{\SI{2}{\mole}~\ch{H2}}{\SI{1}{\mole}~\ch{CH3OH}}
					&= \SI{2.2283253}{\mole}~\ch{H2}
				\end{align*}
			\item Convert moles of \ch{H2} to volume:
				\begin{align*}
					PV &= nRT \\
					V &= \frac{nRT}{P} \\
					&=
					\frac{(\SI{2.2283253}{\mole})(\SI{0.08206}{\liter\atm\per\mole\kelvin})(\SI{355}{\kelvin})}{\SI{0.9710526}{\atm}}
					\\
					&= \SI{66.84912}{\liter}
					= \boxed{\SI{66.8}{\liter}}
				\end{align*}
		\end{enumerate}
	}
\end{frame}

%\begin{onyourown}[10em]
%	What is the pressure (in \si{\atm}) of NO gas, at a temperature of
%	\SI{50.0}{\celsius} and in a \SI{2.0}{\liter} container, if
%	\SI{42.0}{\milli\liter}~\ch{H2O} ($d =
%	\SI{1.00}{\gram\per\milli\liter}$) are produced from the reaction below?
%	\begin{reaction*}
%		4 NH3\gas{} + 5 O2\gas{} -> 4 NO\gas{} + 6 H2O\lqd{}
%	\end{reaction*}
%\end{onyourown}

\begin{frame}{Real Gases}
	So far, we have really only considered \alert{ideal} gases:
	\begin{equation*}
		PV = nRT
	\end{equation*}

	\pause

	What differs between ideal gases and real gases?
	\begin{itemize}[<+(1)->]
		\item Volume of gas particles.
		\item Intermolecular forces.
	\end{itemize}

	\pause

	\begin{equation*}
		\textbf{Van der Waals Equation:} \qquad \underbrace{\bigg[P +
		a\big(\frac{n}{V}\big)^2\bigg]}_{\mathclap{\text{intermolecular
		forces}}} \times \underbrace{[V-nb]}_{\mathclap{\text{particle
		volume}}} = nRT
	\end{equation*}
\end{frame}

\end{document}
