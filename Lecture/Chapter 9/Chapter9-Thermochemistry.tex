% !TEX program = xelatex
\documentclass[handout]{beamer}
%\documentclass[notes=hide]{beamer}
%\documentclass[notes=only]{beamer}
%\documentclass[notes=show]{beamer}

\usepackage{newtxtext}
\usepackage{bucolors}
\usepackage{genchem}
\usepackage{lecture}
\usepackage{multicol}
\usepackage{elements}
\usepackage{collcell}
\usetikzlibrary{tikzmark}
\usepackage{tabularx}

\title{Thermochemistry}
\subtitle{Chapter 9}
\institute[CHEM115 Bloomsburg University]{CHEM115 --- Chemistry for the Sciences I \\ Bloomsburg University}
\author{D.A. McCurry}
\date{Fall 2021}

\chemsetup{chemformula/frac-style=nicefrac}

\begin{document}

\NewChemState{\energy}{symbol = E, unit=\kilo\joule\per\mole}

\maketitle
\mode<article>{\thispagestyle{fancy}}

\frame{\section{Work and Energy}
\begin{learningobjectives}
\item Explain the difference between potential and kinetic energy.
\item Explain where potential and kinetic energy are defined in a chemical
system.
\item Follow the direction of energy flow between the system and surroundings.
\end{learningobjectives}}

\begin{frame}{Key Terms}
	\begin{block}{Thermochemistry}
		The relationship between chemistry and energy.
	\end{block}

	\begin{block}{Energy}
		The ability to do \alert{work} (force acting over distance).
	\end{block}

	\begin{block}{Potential Energy}
		The energy of position or composition. Energy stored for use at
		a later time.
		\note{%
			\begin{itemize}
				\item water behind a dam
				\item a compressed spring
				\item chemical bonds in food
			\end{itemize} 
		}
	\end{block}           

	\begin{block}{Kinetic Energy}
		The energy of motion.
		\note{%
		\begin{itemize}
			\item water flowing over a dam
			\item a bobble head bobbling
			\item food being metabolized
		\end{itemize}
	}
	\end{block}
\end{frame}

\begin{frame}{Kinetic Energy in Chemistry}
	\begin{columns}
		\column{0.4\textwidth}
		Molecules move faster when heated, thus \alert{thermal energy} is a form
		of kinetic energy measured as the \alert{temperature} of an object.
		\column{0.5\textwidth}
		\begin{center}
			\includegraphics[width=\linewidth]{frozen-pizza.jpg}
		\end{center}
	\end{columns}
\end{frame}

\begin{frame}{Potential Energy in Chemistry}
	The relative positions of electrons and nuclei in atoms define the
	\alert{chemical energy}, a form of potential energy, of the system.
	\begin{center}
		\includegraphics[width=0.8\linewidth]{McMurray-covalent2.jpg}
	\end{center}
\end{frame}
	
\begin{frame}{Energy Transfer}
	\begin{columns}
		\column{0.525\textwidth}
		Energy is \alert{exchanged} between a system and its surroundings through
		either \alert{heat} ($q$) or \alert{work} ($w$).

		\begin{block}{System}
			The focus of the study, as defined by the investigator.
		\end{block}

		\begin{block}{Surroundings}
			Everything other than the system.
		\end{block}
	
		\column{0.475\textwidth}
		\begin{center}
			\includegraphics[scale=0.2]{cccalorimeter.jpg}
		\end{center}
	\end{columns}
\end{frame}

\begin{frame}{Thermodynamics}
	\begin{quote}
		The general study of energy and its interconversions.
	\end{quote}

	\bigskip

	\begin{block}{The First Law of Thermodynamics}
		\begin{itemize}
			\item The total energy of the universe is constant.
			\item Energy can neither be created nor destroyed.
		\end{itemize}
	\end{block}
\end{frame}

\vspace{\stretch{-1}}

\begin{frame}{Conservation of Energy}
	The \alert{internal energy} ($E$) of a system must equal the total
	potential (PE) and kinetic (KE) energies of the system.
	\begin{itemize}
		\item Can only transfer E with surroundings via
			heat ($q$) or work ($w$), thus
			\begin{align*}
				\Delta E_\text{sys} &= q + w
			\intertext{\item The energy of the universe is constant
				(1\textsuperscript{st} Law), so}
				\Delta E_\text{sys} &= -\Delta E_\text{surr} \\
				\Delta E_\text{sys} + \Delta E_\text{surr} &= 0
			\end{align*}
		\item The exact amount of energy lost by the
			system is gained by the surroundings
			(and vice versa).
	\end{itemize}
	Reactions that \alert{gain} heat are said to be \alert{endothermic}
	whereas reactions that \alert{lose} heat are said to be
	\alert{exothermic}.
\end{frame}

\vspace{\stretch{-1}}

\begin{frame}{Exothermic Reactions}
	\begin{columns}
		\column{0.45\linewidth}
		\begin{itemize}
			\item Heat is released from the system to the
				surroundings.
			\item The energy of the products is less than the energy
				of the reactants.
			\item $q = \text{negative}$
			\item Heat is a \alert{product}.
		\end{itemize}
		\column{0.45\linewidth}
		\begin{center}
			\includegraphics[scale=0.4]{exothermicrxn.jpg}
		\end{center}
	\end{columns}

	\bigskip

	\begin{reaction*}
		C\sld[graphite]{} + O2\gas{} -> CO2\gas{} +
		"\SI{394}{\kilo\joule}"
	\end{reaction*}
	\begin{equation*}
		\Delta H = \SI{-394}{\kilo\joule\per\mole}~\text{(heat
		released)}
	\end{equation*}
\end{frame}

\vspace{\stretch{-1}}

\begin{frame}{Endothermic Reactions}
	\begin{columns}
		\column{0.45\linewidth}
		\begin{itemize}
			\item Heat is absorbed to the system from the
				surroundings.
			\item The energy of the products is greater than the
				energy of the reactants.
			\item $q = \text{positive}$
			\item Heat is a \alert{reactant}.
		\end{itemize}
		\column{0.45\linewidth}
		\begin{center}
			\includegraphics[scale=0.4]{endothermicrxn.jpg}
		\end{center}
	\end{columns}

	\bigskip

	\begin{reaction*}
		N2\gas{} + O2\gas{} + "\SI{180}{\kilo\joule}" -> 2
		NO\gas{}
	\end{reaction*}
	\begin{equation*}
		\Delta H = \SI{180}{\kilo\joule\per\mole}~\text{(heat
		added)}
	\end{equation*}
\end{frame}

\vspace{\stretch{-1}}

\begin{frame}[t]{Identifying Exothermic and Endothermic Reactions}
	Identify each of the following reactions as exothermic or endothermic.
	\begin{enumerate}
		\item \ch{N2\gas{} + 3 H2\gas{} -> 2 NH3\gas{} +
			"\SI{22}{\kilo\cal}"}
			\note[item]{exothermic}
		\item \ch{CaCO3\sld{} + "\SI{133}{\kilo\cal}" -> CaO\sld{} +
			CO2\gas{}}
			\note[item]{endothermic}
		\item \ch{2 SO2\gas{} + O2\gas{} -> 2 SO3\gas{} + "\text{heat}"}
			\note[item]{exothermic}
	\end{enumerate}
\end{frame}

\vspace{\stretch{-1}}

\begin{onyourown}[0em]
	Identify each of the following reactions as exothermic or endothermic.
	\begin{enumerate}
		\item \ch{2 H2\gas{} + O2\gas{} -> 2 H2O\gas{} +
			"\SI{-483.6}{\kilo\joule\per\mole}"}
		\item \ch{CH4\gas{} + 2 O2\gas{} -> CO2\gas{} + 2 H2O\lqd{} +
			"\text{heat}"}
		\item \ch{H2O\sld{} + "\text{heat}" -> H2O\lqd{}}
	\end{enumerate}
\end{onyourown}

\begin{frame}{A Summary of Energy, Work, and Heat}
	\begin{center}
		\begin{tabular} {>{\text{For~}}L<{\text{,}} L l L l}
			\toprule
			q & + & system gains heat & - & system loses heat \\
			w & + & work done on system & - & work done by system \\
			\Delta E & + & system gains energy & - & system loses
			energy \\ \bottomrule
		\end{tabular}
	\end{center}

	Always consider the ``point of view'' of the system when determining
	sign!

	\pause

	\begin{center}
		\includegraphics[scale=0.5]{money.jpg}
	\end{center}
\end{frame}

\vspace{\stretch{-1}}

%\begin{frame}[t]{Calculating Energy Changes}
%	Two gases, A and B, are confined to a cylinder (closed system) with a
%	movable piston. These gases react to form solid C based on the
%	following:
%	\begin{reaction*}
%		A\gas{} + B\gas{} -> C\sld[graphite]{}
%	\end{reaction*}
%	The reaction is exothermic with a magnitude of \SI{1151}{\joule}. As the
%	reaction progresses, the piston moves downward due to less gas in the
%	system. The magnitude of the work done is \SI{481}{\joule}. What is
%	$\Delta E$ after the reaction is complete?
%
%	\vspace{10em}
%
%	\note{
%		\begin{itemize}
%			\item Exothermic means that the sign of the heat is
%				negative.
%			\item Because the piston moved due to the system, we can
%				say that this is work performed by the system,
%				so w is negative.
%		\end{itemize}
%
%		\begin{align*}
%			\Delta E &= q + w \\
%			&= \SI{-1151}{\joule} + \SI{-481}{\joule} \\
%			&= \boxed{\SI{-1632}{\joule}}
%		\end{align*}
%		}
%\end{frame}
%
%\begin{onyourown}{10em}
%	Self-heating 
%	An endothermic reaction inside of a balloon causes a heat transfer of \SI{210}{\joule}. At
%	the same time
%\end{onyourown}

\frame{\section{State Functions}
	\begin{learningobjectives}
	\item Identify examples of state functions.
	\item Use state functions to calculate the change in energy.
	\item Define and use heat capacities.
	\end{learningobjectives}
}

\vspace{\stretch{-1}}
	
\begin{frame}{State Functions}
	The value only depends on the state of the system, not how the
	system arrived at that state.
	\begin{align*}
		\Delta E &= E_\text{final} - E_\text{initial} &
		\Delta P &= P_\text{final} - P_\text{initial} \\
		\Delta V &= V_\text{final} - V_\text{initial} &
		\Delta T &= T_\text{final} - T_\text{initial} \\
	\end{align*}
	\begin{center}
		\includegraphics[scale=0.25,trim={0 0 0
		120pt},clip]{09_05_Figure.jpg}
	\end{center}
\end{frame}

%\begin{frame}{Where have we seen state functions before?}
%	\pause
%	\centering
%	\includegraphics[scale=0.4]{lattice-energy-cmpd.jpg}
%
%	\note<+>{
%		\sisetup{table-format=-3}
%		\begin{tabular}{>{\collectcell\ch}r<{\endcollectcell}@{\ch{->}}E
%			S<{~\si{\kilo\joule\per\mole}}}
%			Na & Na^{+} + \el{} & +496 \\
%			Cl + \el{} & Cl^{-} & -349 \\
%			Na^{+} + Cl^{-} & NaCl & -558 \\ \midrule
%			Na + Cl & NaCl & -411
%		\end{tabular}
%		}
%\end{frame}

\begin{frame}[allowframebreaks]{Representing Chemical Equations in State Functions}
	\note<1>{%
		We often represent energy changes on a vertical axis, such that species
		that are higher in energy are higher on the axis than those species of
		lower energy.
	}

	For the reaction,
	\begin{reaction*}
		C\sld[graphite]{} + O2\gas{} -> CO2\gas{}
	\end{reaction*}

	\bigskip

	\begin{center}
		\includegraphics[scale=0.3]{09_Pg371_UnFigure_1.jpg}
	\end{center}

	\framebreak
	For the reaction,
	\begin{reaction*}
		C\sld[graphite]{} + O2\gas{} <- CO2\gas{}
	\end{reaction*}

	\bigskip
	
	\begin{center}
		\includegraphics[scale=0.3]{09_Pg371_UnFigure_3.jpg}
	\end{center}
\end{frame}

\begin{frame}{Quantifying Heat}
	Heat is measured in \alert{joules} or \alert{calories}.
	\begin{columns}
		\column{0.55\linewidth}
		\begin{align*}
			\SI{4.184}{\joule} &= \SI{1}{calorie}~\text{(\si{\cal})
			(exact)}
			\\
			\SI{1}{\kilo\joule} &= \SI{1000}{\joule} \\
			\SI{1}{kilocalorie}~\text{(\si{\kilo\cal})} &=
			\SI{1000}{calories}~\text{(\si{\cal})} \\
			&= \SI{1}{Calorie}~\text{(Cal)}
		\end{align*}
		\column{0.35\linewidth}
		\begin{framed}
		\SI{1}{\cal} is the energy required to raise the temperature of
		\SI{1}{\gram} of water by \SI{1}{\celsius}.
		\end{framed}
	\end{columns}

	\pause

	\bigskip

	The \alert{heat capacity} ($C$) is the heat required to change the
	temperature of a substance by \SI{1}{\celsius}.
	\begin{equation*}
		q = C \times \Delta T
	\end{equation*}
	Heat capacity is \alert{dependent} of the mass of the object.
	\begin{itemize}
		\item More mass requires more heat.
	\end{itemize}
\end{frame}

\begin{frame}{Specific Heat Capacity}
	Heat capacity is useful for examining how much heat is necessary to
	change the temperature of \alert{different amounts} of the \alert{same
	substance}.

	\bigskip

	Why do some substances (like a metal frying pan) heat differently than
	others (like the water in said frying pan)?

	\pause

	\bigskip

	\begin{block}{Specific Heat Capacity}
		The amount of heat required to raise the temperature of
		\alert{\SI{1}{\gram}} of the substance by \SI{1}{\celsius}.
	\end{block}

	\begin{center}
		\begin{tabular} {c c c}
			\bfseries Heat Capacity & \bfseries vs & \bfseries
			Specific Heat Capacity \\ \midrule
			$ q = C \times \Delta T $ && $ q = m \times C_\text{s}
			\times \Delta T $ \\
			Units of \si{\joule\per\celsius} && Units of
			\si{\joule\per\gram\per\celsius}
		\end{tabular}
	\end{center}
\end{frame}

\begin{frame}{Specific Heats of Some Substances}
	\begin{center}
		\sisetup{table-format=1.3}
		\begin{tabular} {l E S}
			\toprule
			\multicolumn{2}{l}{\bfseries Substance} & \bfseries
			Specific Heat (\si{\joule\per\gram\per\celsius}) \\
			\midrule
			Aluminum & Al\sld & 0.897 \\
			Copper & Cu\sld & 0.385 \\
			Gold & Au\sld & 0.129 \\
			Iron & Fe\sld & 0.452 \\
			Silver & Ag\sld & 0.235 \\
			Titanium & Ti\sld & 0.523 \\
			Ammonia & NH3\gas & 2.04 \\
			Ethanol & C2H5OH\lqd & 2.46 \\
			Sodium chloride & NaCl\sld & 0.864 \\
			Water & H2O\lqd & 4.184 \\ \bottomrule
		\end{tabular}

		\bigskip

		\pause

		Can you make some generalizations from this table?
	\end{center}
\end{frame}

%\begin{frame}[t]{Identifying General Trends in Specific Heat}
%	\begin{enumerate}
%		\item For the same amount of heat added, a substance with a
%			large specific heat\ldots
%			\begin{list}{\tikz{\draw (0,0) circle (0.4em)}}{}
%				\item has a smaller increase in temperature.
%				\item has a greater increase in temperature.
%			\end{list}
%		\item When ice melts in a drink, the drink\ldots
%			\begin{list}{\tikz{\draw (0,0) circle (0.4em)}}{}
%				\item cools.
%				\item warms.
%				\item stays the same.
%			\end{list}
%		\item Sand in the desert is hot in the day and cool at night.
%			Sand must have a\ldots
%			\begin{list}{\tikz{\draw (0,0) circle (0.4em)}}{}
%				\item high specific heat.
%				\item low specific heat.
%			\end{list}
%	\end{enumerate}
%\end{frame}
%
%\vspace{\stretch{-1}}

\begin{frame}[t]{Practicing Specific Heat Calculations}
	Aluminum (\SI{5.00}{\gram}) has a specific heat of
	\SI{0.897}{\joule\per\gram\per\celsius} and is initially at
	\SI{37}{\celsius}. If \SI{18}{\joule} of energy are added to the metal,
	what is the final temperature?

	\note{
		\begin{align*}
			q &= m C_\text{s} \Delta T \\
			\Delta T &= \frac{q}{m C_\text{s}} \\
			&=
			\frac{\SI{18}{\joule}}{(\SI{5.00}{\gram})(\SI{0.897}{\joule\per\gram\per\celsius})}
			\\
			&= \SI{4.013378}{\celsius} \\
			\SI{4.013378}{\celsius} &= T_\text{f} - T_\text{i} \\
			&= T_\text{f} - \SI{37}{\celsius} \\
			T_\text{f} &= \boxed{\SI{41}{\celsius}}
		\end{align*}
		}
\end{frame}

\begin{onyourown}
	Silver (\SI{3.20}{\gram} has a specific heat of
	\SI{0.235}{\joule\per\gram\per\celsius} and is initially at
	\SI{22}{\celsius}. If the final temperature is \SI{42}{\celsius}, how
	much heat was added to the metal?
\end{onyourown}

\clearpage{}

\begin{frame}[t]{Practicing Thermal Energy Transfer}
	Aluminum (\SI{5.00}{\gram}) has a specific heat of
	\SI{0.897}{\joule\per\gram\per\celsius} and is initially at
	\SI{37}{\celsius}. It is placed in a \SI{100.0}{\gram} cup of water
	($C_\text{s} = \SI{4.184}{\joule\per\gram\per\celsius}$) at
	\SI{22.0}{\celsius}. What is the final temperature when both reach
	thermal equilibrium?

	\note{\scriptsize
		Thermal equilibrium = the temperature of both substances is the
		same.
		\begin{align*}
			q_\text{sys} &= -q_\text{surr} \\
			m \times C_\text{s} \times \Delta T &= - m \times
			C_\text{s} \times \Delta T \\
			m \times C_\text{s} \times (T_\text{f} - T_\text{i}) &=
			- m \times C_\text{s} \times (T_\text{f} - T_\text{i})
			\\ \SI{5.00}{\gram} \times
			\SI{0.897}{\joule\per\gram\per\celsius} \times
			(T_\text{f} - \SI{37}{\celsius}) &= - \SI{100.0}{\gram}
			\times \SI{4.184}{\joule\per\gram\per\celsius} \times
			(T_\text{f} - \SI{22.0}{\celsius}) \\
			\SI{4.485}{\joule\per\celsius} \times T_\text{f} -
			\SI{165.945}{\joule} &= \SI{-418.4}{\joule\per\celsius}
			\times T_\text{f} + \SI{9204.8}{\joule} \\
			\SI{422.885}{\joule\per\celsius} \times T_\text{f} &=
			\SI{9370.745}{\joule} \\
			T_\text{f} &= \SI{22.1590858}{\celsius} \\
			&= \boxed{\SI{22}{\celsius}}
		\end{align*}
		}
\end{frame}

\begin{onyourown}
	Titanium (\SI{8.12}{\gram} has a specific heat of
	\SI{0.523}{\joule\per\gram\per\celsius} and is initially at
	\SI{47}{\celsius}. It is placed in a \SI{50.0}{\gram} cup of water
	($C_\text{s} = \SI{4.184}{\joule\per\gram\per\celsius}$) at
	\SI{18.0}{\celsius}. What is the final temperature when both reach
	thermal equilibrium?
\end{onyourown}

\begin{frame}{Measuring Heat Transfer}
	\begin{block}{Calorimetry}
		The experimental procedure used to measure the heat evolved in a
		chemical reaction.
	\end{block}

	\mode<article>{\clearpage}

	\begin{columns}
		\column{0.55\textwidth}
		\begin{align*}
			q_{\text{surr}} &= -q_\text{sys} \\
			q_{\ch{H2O}} &= -q_\text{rxn} \\
			q_{\ch{surr}} &= m \times
			\underbrace{C_\text{s}}_{\mathclap{\text{or $s$}}}
			\times \Delta T \\
			\intertext{Normalized to amount of reactant used:}
			\Delta H &= \frac{q_\text{sys}}{n}
			\end{align*}
		\column{0.45\textwidth}
		\begin{center}
			\includegraphics[scale=0.3]{09_08_Figure.jpg}
		\end{center}
	\end{columns}
\end{frame}

\frame{\section{Enthalpy}
	\begin{learningobjectives}
	\item Explain the source of enthalpy in a system.
	\item Calculate enthalpy in the context of $PV$ work.
	\item Use Hess's Law to calculate enthalpy of a reaction.
	\end{learningobjectives}
}

\begin{frame}[t]{Enthalpy of Reactions}
	What is this \enthalpy*[superscript=]{} term?

	\onslide<+(1)->

	\begin{block}{Enthalpy}
		The sum of the internal energy and the product of pressure and
		volume of a system.
		\begin{equation*}
			H = E + PV
		\end{equation*}
	\end{block}

	\onslide<+(1)>

	Enthalpy is a \alert{state function}, thus
	\begin{align*}
		\Delta H &= \Delta E + \Delta (PV)
		\shortintertext{where}
		\Delta E &= \text{change in internal energy} \\
		\Delta (PV) &= \text{change in pressure and volume}
	\end{align*}
\end{frame}
	
\begin{frame}{Why does heat = enthalpy at constant pressure?}
	At constant pressure,
	\begin{align*}
		\Delta H &= \Delta E + P\Delta V
		\shortintertext{and}
		\Delta H &= \frac{q_\text{sys}}{n}
	\end{align*}
	\emph{Why?}

	\pause

	\bigskip

	Recall \alert{internal energy}:
	\begin{equation*}
		\Delta E = q + w
	\end{equation*}

	\begin{description}
		\item[$q$] is the heat released or absorbed from the system.
		\item[$w$] is the work performed by or done on the system.
			\begin{itemize}[<+(1)->]
				\item What is this work?
			\end{itemize}
	\end{description}
\end{frame}

\vspace{\stretch{-1}}

\begin{frame}{Pressure-Volume Work}
	\begin{center}
		\includegraphics[scale=0.1]{09_Pg377_UnFigure.jpg}\qquad
		\includegraphics[scale=0.15]{09_06_Figure.jpg}
	\end{center}
	\begin{align*}
		\text{Work ($w$)} &= \text{Force ($F$)} \times \text{Distance
		($\Delta h$)}
		\\
		\text{Pressure ($P$)} &= \frac{\text{Force ($F$)}}{\text{Area
		($A$)}} \\
		\shortintertext{Therefore,}
		\text{Force ($F$)} &= \text{Pressure ($P$)} \times \text{Area
		($A$)} \\
		\text{Work ($w$)} &= \text{Pressure ($P$)} \times \underbrace{\text{Area
		($A$)} \times \text{Distance ($\Delta h$)}}_{\mathclap{\text{Volume ($\Delta
		V$)}}} \\
		w &= \uncover<2>{-}P \Delta V
	\end{align*}
\end{frame}

%\begin{frame}[t]{Calculating Pressure-Volume Work}
%	A gas expands in volume from \SIrange{3.0}{7.0}{\liter} at constant
%	temperature. How much work (in \si{\joule}) is done by the system if it
%	expands 
%	\begin{enumerate}[<+->] 
%		\item against a vacuum?
%			\note<.>{%
%				\begin{align*}
%					w &= -P \Delta V \\
%					&= 0 \times \Delta V \\
%					&= \boxed{\SI{0}{\joule}}
%				\end{align*}
%			}
%		\item against a constant external pressure of \SI{2.0}{\atm}?
%			\note<.>{%
%				\begin{align*}
%					w &= -(\SI{2.0}{\atm})(\SI{7.0}{\liter}
%					- \SI{3.0}{\liter}) \\
%					&= \SI{-8.0}{\liter\atm} \times
%					\frac{\SI{101.3}{\joule}}{\SI{1}{\liter\atm}}
%					\\
%					&= \boxed{\SI{-810.4}{\joule}}
%				\end{align*}
%			}
%	\end{enumerate}
%
%	\begin{description}
%		\item[Note:] $\SI{1}{\liter\atm} = \SI{101.3}{\joule}$
%	\end{description}
%\end{frame}

\begin{frame}{Relating Work to Enthalpy}
	Why does $\Delta H = \frac{q_\text{sys}}{n}$?

	\begin{align*}
		\Delta E &= q + w \\
		w &= -P \Delta V
		\visible<+(1)->{
		\intertext{Therefore,}
		\Delta H &= \Delta E + P \Delta V \\
		&= q + w + P \Delta V \\
		&= q - P \Delta V + P \Delta V \\
		\Delta H &= q}
	\end{align*}
\end{frame}

\begin{frame}{Can we measure ΔE directly?}
	\begin{columns}
		\column{0.5\textwidth}
		\enthalpy*[superscript=]{} (and $q$) can be
		measured at constant \emph{pressure} using
		CCC (constant pressure
		calorimetry).

		\onslide<+(1)->
		\vspace{\baselineskip}

		\energy*[superscript=]{} can be measured at
		constant \emph{volume} using \alert{bomb
		calorimetry}.
		\begin{align*}
			\Delta E &= q + w \\
			&= q - P\Delta V \qquad \Delta V = 0 \\
			\Delta E &= q \\
			q_\text{rxn} &= -q_\text{cal} = -C
			\times \Delta T
		\end{align*}
		
		\column{0.4\textwidth}
		\begin{center}
			\includegraphics[scale=0.3]{09_07_Figure.jpg}
		\end{center}
	\end{columns}
\end{frame}

\vspace{\stretch{-1}}

\begin{frame}[t]{Bomb Calorimetry Example}
	A \SI{1.435}{\gram} sample of naphthalene (\ch{C10H8}) was burned in a
	bomb calorimeter (constant $V$) in which the temperature of the water
	changed from \SIrange{20.28}{25.95}{\celsius}. If the heat capacity of
	the calorimeter (including the water) is
	\SI{10.17}{\kilo\joule\per\celsius}, what is the molar
	\energy*[superscript=]{} of the
	combustion of \ch{C10H8}?

	\note{%
		\begin{align*}
			q_\text{cal} &= C \times \Delta T \\
			&= \SI{10.17}{\kilo\joule\per\celsius}
			\times (\SI{25.95}{\celsius} - \SI{20.28}{\celsius}) \\
			&= \SI{57.6639}{\kilo\joule} \\
			q_\text{rxn} &= -q_\text{cal} \\
			&= \SI{-57.6639}{\kilo\joule} \\
			\Delta E &= \frac{q_\text{rxn}}{n} &\qquad n &=
			\SI{1.435}{\gram} \times
			\frac{\SI{1}{\mole}}{\SI{128.1732}{\gram}} \\
			&= \frac{\SI{-57.6639}{\kilo\joule}}{\SI{0.011195788}{\mole}}
			&&= \SI{0.01196}{\mole} \\
			&= \SI{-5150.499}{\kilo\joule\per\mole} \\
			&= \boxed{\SI{-5150.}{\kilo\joule\per\mole}}
		\end{align*}
		}
\end{frame}

\clearpage

\begin{frame}[allowframebreaks,t]{Relationships Involving ΔH\textsubscript{rxn}}
	If a chemical equation is multiplied by some factor, the
	\enthalpy*[superscript=,subscript-right=rxn]{} is also multiplied by the
	same factor.
	\begin{align*}
		\ch{A + 2 B &-> C} &\quad &\Delta H_1 \\
		\ch{2 A + 4 B &-> 2 C} &\quad &\Delta H_2 = 2
		\times \Delta H_1
	\end{align*}

	\framebreak

	If a chemical equation is reversed, then
	\enthalpy*[superscript=,subscript-right=rxn]{}
	changes sign.
	\begin{align*}
		\ch{A + 2 B &-> C} &\quad &\Delta H_1 \\
		\ch{C &-> A + 2 B} &\quad &\Delta H_2 = -\Delta
		H_1
	\end{align*}

	\framebreak

	If a chemical equation can be expressed as the sum of a series of steps,
	then \enthalpy*[superscript=,subscript-right=rxn]{} for the overall
	equation is the \alert{sum} of the heats of reactions for each step.
	\begin{center}
		\begin{tabular}
			{>{\collectcell\ch}r<{\endcollectcell} @{
				\ch{->} } E L }
				A + 2 B & C & \Delta H_1 \\
				C & 2 D & \Delta H_2 \\ \midrule
				A + 2 B & 2 D & \Delta H_3 =
				\Delta H_1 + \Delta H_2
		\end{tabular}
	\end{center}

	\bigskip

	\begin{block}{Hess's Law}
		The change in enthalpy for a stepwise process is the sum of the
		enthalpy changes of the steps.
	\end{block}
\end{frame}

\begin{frame}[t]{Practicing ΔH Stoichiometry}
	In the reaction,
	\begin{reaction*}
		N2\gas{} + O2\gas{} -> 2 NO\gas{} "\qquad \enthalpy{180}"
	\end{reaction*}
	If \SI{15.0}{\gram} of \ch{NO} are produced, how many \si{\kilo\joule}
	were absorbed?

	\vspace{10em}

	\note{
		\begin{enumerate}
			\item Plan: \si{\gram} \ch{NO} \textrightarrow\
				\si{\mole} \ch{NO} \textrightarrow\
				\si{\kilo\joule}
			\item Find moles \ch{NO}:
				\begin{align*}
					\SI{15.0}{\gram} \times
					\frac{\SI{1}{\mole}}{\SI{30.0061}{\gram}}
					= \SI{0.499898}{\mole}~\ch{NO}
				\end{align*}
			\item How many \si{\kilo\joule} produced per mole of
				\ch{NO}?
				\begin{align*}
					\frac{\SI{2}{\mole}~\ch{NO}}{\SI{180}{\kilo\joule}}
				\end{align*}
			\item Solve:
				\begin{align*}
					\SI{0.499898}{\mole} \times
					\frac{\SI{180}{\kilo\joule}}{\SI{2}{\mole}}
					= \boxed{\SI{45.0}{\kilo\joule}}
				\end{align*}
		\end{enumerate}
		}
\end{frame}

\mode<article>{\pagebreak}

\begin{onyourown}[15em]
	How many grams of \ch{O2} reacted if \SI{306}{\kilo\cal} are released in
	the following reaction?
	\begin{reaction*}
		CH4\gas{} + 2 O2\gas{} -> CO2\gas{} + 2 H2O\lqd{} +
		"\SI{213}{\kilo\cal}"
	\end{reaction*}
\end{onyourown}

\begin{frame}[t]{Practicing Hess's Law}
	\begin{equation*}
		\ch{2 CH3OH\lqd{} + O2\gas{} -> 2 H2CO\lqd{} + 2 H2O\lqd{}}
		\qquad \enthalpy[subscript-right=rxn,unit=?]{}
	\end{equation*}

	Calculate \enthalpy*[subscript-right=rxn]{} for the above reaction based
	on the following:
	\begin{align*}
		\ch{C\sld{} + 2 H2\gas{} + 1/2 O2\gas{} &-> CH3OH\lqd{}}
		&&\enthalpy[subscript-right=f]{-238.6} \\
		\ch{H2\gas{} + C\sld{} + 1/2 O2\gas{} &-> H2CO\lqd{}}
		&&\enthalpy[subscript-right=f]{-150.2} \\
		\ch{H2\gas{} + 1/2 O2\gas{} &-> H2O\lqd{}}
		&&\enthalpy[subscript-right=f]{-285.8}
	\end{align*}

	\vspace{15em}

	\note{
		\begin{tabular}
			{>{\collectcell\ch}r<{\endcollectcell} @{ \ch{->} } E
			>{\collectcell\enthalpy}l<{\endcollectcell}}
			2 CH3OH\lqd{} & 2 C\sld{} + 4 H2\gas{} + O2\gas{} &
			+477.2 \\
			2 H2\gas{} + 2 C\sld{} + O2\gas{} & 2 H2CO\lqd{} &
			-300.4 \\
			2 H2\gas{} + O2\gas{} & 2 H2O\lqd{} & -571.6 \\ \midrule
			2 CH3OH\lqd{} + O2\gas{} & 2 H2CO\lqd{} + 2 H2O\lqd{} &
			-394.8
		\end{tabular}
		}
\end{frame}

\mode<article>{\pagebreak}

\begin{onyourown}[15em]
	\begin{align*}
		\ch{CS2\lqd{} + 3 O2\gas{} &-> CO2\gas{} + 2 SO2\gas{}}
		&&\enthalpy[subscript-right=rxn,unit=?]{}
		\intertext{Calculate \enthalpy*[subscript-right=rxn]{} for the
		above reaction based on the following:}
		\ch{C\sld{} + O2\gas{} &-> CO2\gas{}}
		&&\enthalpy[subscript-right=f]{-393.5} \\
		\ch{S\sld{} + O2\gas{} &-> SO2\gas{}}
		&&\enthalpy[subscript-right=f]{-296.8} \\
		\ch{C\sld{} + 2 S\sld{} &-> CS2\lqd{}}
		&&\enthalpy[subscript-right=f]{87.9}
	\end{align*}
\end{onyourown}

\frame{\section{The Source of Energy}
	\begin{learningobjectives}
	\item Explain possible routes of energy loss (or gain) for a system.
	\item Calculate reaction enthalpies from bond energies.
	\item Calculate reaction enthalpies using standard heats of formation.
	\end{learningobjectives}
}

\begin{frame}{Where is all this energy coming from?}
	\alert{Reaction conditions} for a chemical reaction require\ldots
	\begin{itemize}
		\item collisions between reacting molecules.
		\item collisions with sufficient energy to break the bonds in
			the reactants.
		\item the breaking of bonds between atoms of the reactants.
		\item the forming of new bonds to give products.
	\end{itemize}
\end{frame}

\begin{frame}{Formation of Bonds}
	In the reaction,
	\begin{reaction*}
		H2\gas{} + I2\gas{} -> 2 HI\gas{}
	\end{reaction*}

	\vspace{-\parskip}

	\begin{enumerate}
		\item<2-> the reactants \ch{H2} and \ch{I2} collide.
		\item<3-> the bonds of \ch{H2} and \ch{I2} break.
		\item<4-> the bonds for \ch{HI} form.
	\end{enumerate}

	\bigskip

	\begin{center}
		\begin{tikzpicture}[node distance=4.25em]
			\tikzstyle{atom}=[circle,minimum size=2em];
			\tikzstyle{iod}=[atom,draw=blue,fill=blue!20];
			\tikzstyle{hyd}=[atom,draw=red,fill=red!20];
			\tikzstyle{arrow}=[ultra thick,->,shorten >=1.25em,shorten
			<=1.25em];
			% Pre collision
			\node[minimum height=4em,minimum width=2em](precollide)
				{};
			\node[iod] at (precollide.south west) {I};
			\node[iod] at (precollide.south east) {I};
			\node[hyd] at (precollide.north west) {H};
			\node[hyd] at (precollide.north east) {H};
			% Collision
			\node[minimum height=2em,minimum width=2em,right = of
				precollide](collide) {};
			\node[iod] at (collide.south west) {I};
			\node[iod] at (collide.south east) {I};
			\node[hyd] at (collide.north west) {H};
			\node[hyd] at (collide.north east) {H};
			\draw[arrow] (precollide.east) to (collide.west);
			% Separate
			\onslide<3->
			\node[minimum height=4em,minimum width=5em,right = of
				collide,draw,dashed](separate) {};
			\node[iod] at (separate.south west) {I};
			\node[iod] at (separate.south east) {I};
			\node[hyd] at (separate.north west) {H};
			\node[hyd] at (separate.north east) {H};
			\draw[arrow] (collide.east) to (separate.west);
			% Bonding
			\onslide<4->
			\node[minimum height=2em,minimum width=4em,right = of
				separate](bonding) {};
			\node[iod] at (bonding.south west) {I};
			\node[iod] at (bonding.south east) {I};
			\node[hyd] at (bonding.north west) {H};
			\node[hyd] at (bonding.north east) {H};
			\draw[arrow] (separate.east) to (bonding.west);
		\end{tikzpicture}
	\end{center}
\end{frame}

\begin{frame}{Calculating ΔH from Bond Energies}
	\begin{columns}
		\column{0.45\textwidth}
		According to Hess's Law, we can \alert{sum} the energies of bonds
		broken and bonds formed.
		\begin{itemize}
			\item \alert{Breaking} bonds \alert{requires} energy.
			\item \alert{Forming} bonds \alert{releases} energy.
		\end{itemize}
		\column{0.55\textwidth}
		\begin{center}
			\includegraphics[scale=0.3]{09_03_Table.jpg}
		\end{center}
	\end{columns}

	\pause

	\begin{equation*} \Delta H_\text{rxn} = \sum \underbrace{(\Delta H~\text{bonds
		broken})}_\text{positive} +
		\sum \underbrace{(\Delta H~\text{bonds formed})}_\text{negative}
	\end{equation*}
\end{frame}

\begin{frame}[t]{Estimating ΔH from Bond Energies}
	Consider the reaction,
	\begin{reaction*}
		CH4\gas{} + Cl2\gas{} -> CH3Cl\gas{} + HCl\gas{}
	\end{reaction*}
	What is the value of \enthalpy*[superscript=]{}?

	\note{
		\begin{multicols}{2}
		\begin{itemize}
			\item Which bonds are broken?

				\begin{tabular} {c c}
					\ch{H3C-H} & \ch{Cl-Cl} \\
					$1 \times \ch{C-H}$ & $1 \times
					\ch{Cl-Cl}$ \\
					\SI{414}{\kilo\joule\per\mole} &
					\SI{243}{\kilo\joule\per\mole}
				\end{tabular}

			\item Which bonds are formed?

				\begin{tabular} {c c}
					\ch{H3C-Cl} & \ch{H-Cl} \\
					$1 \times \ch{C-Cl}$ & $1 \times
					\ch{H-Cl}$ \\
					\SI{-339}{\kilo\joule\per\mole} &
					\SI{-431}{\kilo\joule\per\mole}
				\end{tabular}

		\end{itemize}
		\end{multicols}
		\begin{align*}
			\Delta H_\text{rxn} &= \sum (\Delta
			H~\text{bonds broken}) + \sum (\Delta
			H~\text{bonds formed}) \\
			&=
			\bigg[(\SI{1}{\mole})(\SI{414}{\kilo\joule\per\mole})
			+
			(\SI{1}{\mole})(\SI{243}{\kilo\joule\per\mole})\bigg] \\
			&\qquad {}
			+
			\bigg[(\SI{1}{\mole})(\SI{-339}{\kilo\joule\per\mole})
			+
			(\SI{1}{\mole})(\SI{-431}{\kilo\joule\per\mole})
			\bigg] \\
			&= \SI{657}{\kilo\joule\per\mole} +
			\SI{-770}{\kilo\joule\per\mole} \\
			&=
			\boxed{\SI{-113}{\kilo\joule\per\mole}}
		\end{align*}
		}
\end{frame}

\begin{onyourown}
	Hydrogen gas, a potential fuel, can be made by the reaction of
	methane gas and steam. Use bond energies to estimate
	\enthalpy*[superscript=,subscript-right=rxn]{}.

	\begin{reaction*}
		CH4\gas{} + 2 H2O\gas{} -> 4 H2\gas{} + CO2\gas{}
	\end{reaction*}
\end{onyourown}

\begin{frame}{Standard Enthalpy of Formation}
	The formation of \alert{\SI{1}{\mole}} of product from its constituent
	elements in their \alert{standard states} has been tabulated.

	\pause

	\begin{block}{Standard State}
		\begin{description}
			\item[Gas:] Pure gas at \SI{1}{\atm}.
			\item[Solid/Liquid:] Pure solid or liquid at
				\SI{1}{\atm} and \SI{25}{\celsius}.
			\item[Solution:] Exactly \SI{1}{\Molar}.
		\end{description}
	\end{block}

	\mode<article>{\clearpage}

	\begin{itemize}[<+(1)->]
		\item The standard enthalpy \alert{change} (\enthalpy*{}) is for
			a process when \alert{all} reactants and products are in
			their standard states.
		\item The standard enthalpy of \alert{formation}
			(\enthalpy*[subscript-right=f]{}) is the standard
			enthalpy change for the formation of \SI{1}{\mole} of a
			compound from its constituent elements.
			\begin{itemize}
				\item For pure elements in the standard state,
					\enthalpy[subscript-right=f,unit=]{0}.
			\end{itemize}
	\end{itemize}
\end{frame}

\vspace{\stretch{-1}}

\begin{frame}[t]{Writing Formation Reactions}
	Write a formation reaction for:
	\begin{enumerate}
		\item \ch{HCl\gas{}}

			\vspace{5em}

			\note[item]{\ch{H2\gas{} + Cl2\gas{} -> 2 HCl\gas{}}}

		\item \ch{CO\gas{}}

			\vspace{5em}

			\note[item]{\ch{2 C\sld[graphite]{} + O2\gas{} -> 2 CO\gas{}}}

		\item \ch{MgCO3\sld{}}

			\vspace{5em}

			\note[item]{\ch{2 Mg\sld{} + 2 C\sld[graphite]{} + 3 O2\gas{} ->
			2 MgCO3\sld{}}}
	\end{enumerate}
\end{frame}

\vspace{\stretch{-1.15}}

\begin{frame}[t]{Calculating \enthalpy*[subscript-right=rxn]{} using
	\enthalpy*[subscript-right=f]{}}
	Determine \enthalpy*[subscript-right=rxn]{} for the metabolism of
	glucose (\ch{C6H12O6}). \enthalpy*[subscript-right=f]{} values are
	provided below each compound.
	\begin{equation*}
		\ch{!(\SI{-1273.3}{\kilo\joule\per\mole})( C6H12O6\sld{} ) +
			!(\SI{0}{\kilo\joule\per\mole})( 6 O2\gas{} ) ->
			!(\SI{-393.5}{\kilo\joule\per\mole})( 6 CO2\gas{} ) +
			!(\SI{-285.8}{\kilo\joule\per\mole})( 6 H2O\lqd{} )}
		\qquad \enthalpy[subscript-right=rxn,unit=?]{}
	\end{equation*}

	\note{
		\begin{tabular} {>{\collectcell\ch}r<{\endcollectcell} @{
				\ch{->} } E @{
				\enthalpy[subscript-right=f,unit=]{}}
				S[table-format=-4.1]<{
			\si{\kilo\joule\per\mole}}}
			6 C\sld{} + 6 H2\gas{} + 3 O2\gas{} &
			C6H12O6\sld{} & -1273.3 \\
			C\sld{} + O2\gas{} & CO2\gas{} & -393.5 \\
			H2\gas{} + 1/2 O2\gas{} & H2O\lqd{} & -285.8 \\
			\multicolumn{3}{c}{} \\
			C6H12O6\sld{} & 6 C\sld{} + 6 H2\gas{} + 3 O2\gas{}
			& 1273.3 \\
			6 C\sld{} + 6 O2\gas{} & 6 CO2\gas{} & -2361.0 \\
			6 H2\gas{} + 3 O2\gas{} & 6 H2O\lqd{} & -1714.8 \\
			\midrule
			C6H12O6\sld{} + 6 O2\gas{} & 6 CO2\gas{} + 6
			H2O\gas{} & -2802.5
		\end{tabular}
	}
\end{frame}

\clearpage{}

\begin{onyourown}[20em]
	Calculate the standard enthalpy change for the following reaction using
	standard enthalpies of formation. Use your textbook to look up each
	\enthalpy*[subscript-right=f]{}.
	\begin{reaction*}
		2 H2S\gas{} + 3 O2\gas{} -> 2 H2O\lqd{} + 2 SO2\gas{}
	\end{reaction*}
\end{onyourown}

\vfill

\begin{frame}
	\centering
	\includegraphics[scale=0.5]{09_12_Figure.jpg}
\end{frame}

\clearpage

\begin{frame}[t]{Some extra practice\ldots}
	Calculate \enthalpy*[subscript-right=f]{} of \ch{CO} based on the
	following:
	\begin{align*}
		\ch{C\sld[graphite]{} + O2\gas{} &-> CO2\gas{}} \qquad
		&\enthalpy[subscript-right=f]{-393.5} \\
		\ch{CO\gas{} + 1/2 O2\gas{} &-> CO2\gas{}} \qquad
		&\enthalpy[subscript-right=rxn]{-283.0}
	\end{align*}

	\vspace{\stretch{2}}

	How does this compare to the literature value of
	\enthalpy[subscript-right=f]{-110.5}?

	\note{
		\begin{tabular} {>{\collectcell\ch}r<{\endcollectcell} @{
				\ch{->} } E @{
				\enthalpy[subscript-right=f,unit=]{}}
				S[table-format=-3.1]<{
			\si{\kilo\joule\per\mole}}}
			C\sld[graphite]{} + O2\gas{} & CO2\gas{} & -393.5 \\
			CO2\gas{} & CO\gas{} + 1/2 O2\gas{} & +283.0 \\ \midrule
			C\sld[graphite]{} + 1/2 O2\gas{} & CO\gas{} & -110.5
		\end{tabular}
	}
\end{frame}

\vfill

\end{document}
