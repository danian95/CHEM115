\documentclass[11pt,letterpaper]{article}
\usepackage{beamerarticle}

\usepackage{genchem}
\usepackage{lecture}
\usepackage{longtable}

\setlength{\textwidth}{6.5in}

\title{Lewis Electron Dot Structures}
\subtitle{Examples}
\institute[CHEM115 Bloomsburg University]{CHEM115 --- Chemistry for the Sciences I \\ Bloomsburg University}
\author{CHEM115 - Chemistry for the Sciences I}
\date{February 25, 2018}

\begin{document}

\section{Ionic Compounds}

\begin{itemize}
	\item Full transfer of electrons from one atom to another.
\end{itemize}

\subsection{Examples}
\begin{center}
	\includegraphics[scale=0.5]{lewisdot-ionic.pdf}
\end{center}

\section{Covalent Compounds}

\begin{itemize}
	\item Sharing of electrons in the \emph{valence} orbitals.
\end{itemize}

\subsection{Steps}
\begin{enumerate}
	\item \label{step:calc} Calculate number of valence \el{}.
	\item Draw skeleton structure.
	\item Subtract \# bonding \el{}.
	\item Distribute remaining \el{} around terminal (not central) atoms.
	\item Put any remaining \el{} on the central atom.
	\item Check for full octets.
		\begin{itemize}
			\item Create double or triple bonds if needed.
		\end{itemize}
	\item Check for formal charge.
		\begin{itemize}
			\item Place compound in brackets with a superscript
				charge if needed.
		\end{itemize}
	\item Double-check your work.
		\begin{itemize}
			\item Ensure total \# \el{} distributed equals \#
				valence \el{} calculated in Step
				\ref{step:calc}.
		\end{itemize}
\end{enumerate}

\subsection{Examples}

\begin{longtabu} to \linewidth {X[3.25,m] | X[1,m,c]}
						& Try on your own: \\
	\includegraphics[scale=0.5]{lewisdot-ccl4.pdf} & \ch{SiBr4} \\
	\includegraphics[scale=0.5]{lewisdot-hocl.pdf} & \ch{AsF3}  \\
	\includegraphics[scale=0.5]{lewisdot-ch2o.pdf} & \ch{HCN}   \\
	\includegraphics[scale=0.5]{lewisdot-h3o.pdf}  & \ch{OH-}   \\
	\includegraphics[scale=0.5]{lewisdot-icl3.pdf} & \ch{ClO3-} \\
\end{longtabu}

\end{document}
