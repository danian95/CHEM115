\documentclass[12pt,letterpaper]{article}

\usepackage[margin=0.5in]{geometry}
\usepackage{chemfig}
\usepackage[version=4]{mhchem}
\usepackage{chemformula}
\usepackage{elements}
\usepackage{siunitx}
\usepackage{amsmath}
\usepackage{amssymb}
\usepackage{cancel}
\usepackage{xlop}
\setlength{\parskip}{\baselineskip}
\setlength{\parindent}{0pt}
\usepackage{array}
\usepackage{xcolor}

\pagestyle{empty}

\begin{document}
\section*{Chapter 4 Part II Examples}
\subsection*{Drawing Electron-Dot/Lewis Structures}

Draw an electron-dot structure for \ce{CCl4}

{\color{blue}
\begin{tabular}{p{1in} p{2in} p{1in} p{2in}}
	\textbf{Step 1:} &
	\multicolumn{3}{l}{$ \underbrace{4}_{\ce{C}} +
	\underbrace{4(7)}_{4 \times \ce{Cl}} = 
	32 $ valence electrons } \\[2em]
	\textbf{Step 2:} &
	\chemfig{C(-[:0]Cl)(-[:90]Cl)(-[:180]Cl)(-[:270]Cl)} &
	\textbf{Step 3:} &
	\chemfig{C(-[:0]\lewis{0:2:6:,Cl})(-[:90]\lewis{0:2:4:,Cl})
	(-[:180]\lewis{2:4:6:,Cl})(-[:270]\lewis{0:4:6:,Cl})} \\
\end{tabular}
}

Draw an electron-dot structure for \ce{H3O+}

{\color{blue}
\begin{tabular}{p{1in} p{2in} p{1in} p{2in}}
	\textbf{Step 1:} &
	\multicolumn{3}{l}{ $ \underbrace{3(1)}_{3 \times \ce{H}} +
	\underbrace{6}_{\ce{O}} - \underbrace{1}_\text{+1 charge} = 
	8 $ valence electrons } \\[2em]
	\textbf{Step 2:} &
	\chemfig{O(-[:0]H)(-[:90]H)(-[:180]H)} &
	\textbf{Step 3:} &
	$\chemleft[\chemfig{\lewis{6:,O}(-[:0]H)(-[:90]H)
	(-[:180]H)}\chemright]^{+}$ \\
\end{tabular}
}

Draw an electron-dot structure for \ce{CH2O}

{\color{blue}
\begin{tabular}{p{1in} p{2in} p{1in} p{2in}}
	\textbf{Step 1:} &
	\multicolumn{3}{l}{ $ \underbrace{4}_{\ce{C}} +
	\underbrace{2(1)}_{2 \times \ce{H}} + \underbrace{6}_{\ce{O}} = 
	12 $ valence electrons } \\[2em]
	\textbf{Step 2:} &
	\chemfig{C(-[:0]H)(-[:90]O)(-[:180]H)} &
	\textbf{Step 3:} &
	\chemfig{C(-[:0]H)(-[:90]\lewis{0:2:4:,O})
	(-[:180]H)} \\[2em]
	\textbf{Step 5:} &
	\schemestart
		\chemfig{C(-[:0]H)(-[@{db}:90]@{lp}\lewis{0:2:4:,O})
			(-[:180]H)}
		\arrow{->}
		\chemfig{C(-[:0]H)(=[:90]\lewis{1:3:,O})
			(-[:180]H)}
	\schemestop
	\chemmove{\draw[shorten <=5pt,shorten >=2pt](lp)..controls +(0:8mm) and
	+(0:8mm)..(db);} \\
\end{tabular}
}

\clearpage

Draw an electron-dot structure for \ce{SF6}

{\color{blue}
\begin{tabular}{p{1in} p{2in} p{1in} p{2in}}
	\textbf{Step 1:} &
	\multicolumn{3}{l}{ $ \underbrace{6}_{\ce{S}} +
	\underbrace{6(7)}_{6 \times \ce{F}} = 
	48 $ valence electrons } \\[2em]
	\textbf{Step 2:} &
	\chemfig{S(-[:30]F)(-[:90]F)(-[:150]F)(-[:210]F)(-[:270]F)(-[:330]F)} &
	\textbf{Step 3:} &
	\chemfig{S(-[:30]\lewis{1:3:7:,F})(-[:90]\lewis{0:2:4:,F})
	(-[:150]\lewis{1:3:5:,F})(-[:210]\lewis{3:5:7:,F})
	(-[:270]\lewis{0:4:6:,F})(-[:330]\lewis{1:5:7:,F})}
\end{tabular}
}

Draw an electron-dot structure for \ce{ICl3}

{\color{blue}
\begin{tabular}{p{1in} p{2in} p{1in} p{2in}}
	\textbf{Step 1:} &
	\multicolumn{3}{l}{ $ \underbrace{7}_{\ce{I}} +
	\underbrace{3(7)}_{3 \times \ce{Cl}} = 
	28 $ valence electrons } \\[2em]
	\textbf{Step 2:} &
	\chemfig{I(-[:18]Cl)(-[:90]Cl)(-[:162]Cl)} &
	\textbf{Step 3:} &
	\chemfig{I(-[:18]\lewis{0:2:6:,Cl})(-[:90]\lewis{0:2:4:,Cl})
	(-[:162]\lewis{2:4:6:,Cl})} \\[2em]
	\textbf{Step 4:} &
	\chemfig{\lewis{5:7:,I}(-[:18]\lewis{0:2:6:,Cl})
	(-[:90]\lewis{0:2:4:,Cl})(-[:162]\lewis{2:4:6:,Cl})} \\[2em]
\end{tabular}
}

Draw an electron-dot structure for \ce{O3}

{\color{blue}
\begin{tabular}{p{1in} p{2in} p{1in} p{2in}}
	\textbf{Step 1:} &
	\multicolumn{3}{l}{ $ \underbrace{3(6)}_{3 \times \ce{O}} =
	18 $ valence electrons } \\[2em]
	\textbf{Step 2:} &
	\chemfig{O(-[:0]O)(-[:180]O)} &
	\textbf{Step 3:} &
	\chemfig{O(-[:0]\lewis{0:2:6:,O})(-[:180]\lewis{2:4:6:,O})} \\[2em]
	\textbf{Step 4:} &
	\chemfig{\lewis{2:,O}(-[:0]\lewis{0:2:6:,O})
	(-[:180]\lewis{2:4:6:,O})} \\[2em]
	\textbf{Step 5:} &
	\schemestart
		\chemfig{\lewis{2:,O}(-[:0]\lewis{0:2:6:,O})
		(-[@{dba}:180]@{lpa}\lewis{2:4:6:,O})}
		\arrow{->}
		\chemfig{\lewis{2:,O}(-[:0]\lewis{0:2:6:,O})
		(=[:180]\lewis{3:5:,O})}
		\arrow{<->}[-90]
		\chemfig{\lewis{2:,O}(=[:0]\lewis{1:7:,O})
		(-[:180]\lewis{2:4:6:,O})}
		\arrow{<-}[180]
		\chemfig{\lewis{2:,O}(-[@{dbb}:0]@{lpb}\lewis{0:2:6:,O})
		(-[:180]\lewis{2:4:6:,O})}
	\schemestop
	\chemmove{\draw[shorten <=5pt,shorten >=2pt](lpa)..controls +(270:8mm)
	and +(270:8mm)..(dba);}
	\chemmove{\draw[shorten <=5pt,shorten >=2pt](lpb)..controls +(90:8mm)
	and +(90:8mm)..(dbb);}
	\\
\end{tabular}
}

\clearpage

Draw the following molecules and determine the formal charge of each atom. If
there are resonance structures, indicate all possibilities.

\begin{itemize}
	\item \ce{SCN-}
	\item \ce{BF3}
	\item \ce{N2O}
	\item \ce{ClO3^{-}}
	\item \ce{HCN}
	\item \ce{H2SO3}
\end{itemize}

\end{document}
