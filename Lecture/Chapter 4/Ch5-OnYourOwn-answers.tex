% !TEX program = xelatex
\documentclass[11pt,letterpaper]{article}

\usepackage{genchem}
\usepackage{enumitem}
\usepackage[margin=1in]{geometry}
\usepackage{titling}

\setallmainfonts{TeX Gyre Pagella}

\title{Chapter 5 ``On Your Own'' Solutions}

\begin{document}

\begin{center}
	\bfseries
	\Large
	\thetitle
\end{center}

\begin{enumerate}[itemsep=2em,leftmargin=0pt,label=\textbf{\Alph*.}]
	\item $\ch{Al} < \ch{Si} < \ch{C}$
	\item First determine electronegativity difference:
		
		\sisetup{table-format=1.1}
		\begin{tabular} {E @{\qquad$\Delta\text{EN}=$ }
			S @{ $-$ } S @{ $=$ } S}
			CCl4  & 3.0 & 2.5 & 0.5 \\
			BaCl2 & 3.0 & 0.9 & 2.1 \\
			TiCl3 & 3.0 & 1.5 & 1.5 \\
			Cl2O  & 3.5 & 3.0 & 0.5 \\
		\end{tabular}

		Then order in increasing $\Delta\text{EN}$ values:

		$\ch{Cl2O} = \ch{CCl4} < \ch{TiCl3} < \ch{BaCl2}$
%	\item 	We use the same percent ionic character equation, but work
%		backwards:
%		\begin{align*}
%			\text{\% ionic character} &= \frac{\text{measured
%			dipole moment}}{\text{dipole moment assuming
%			full transfer of \el{}}} \times 100\% \\
%			\SI{5.7}{\percent} &=
%			\frac{\SI{0.44}{\debye}}{\mu} \times
%			100\% \\
%			\mu &= \SI{7.719298246}{\debye} \\
%			\mu &= qr \\
%			r & = \frac{\mu}{q} \\
%			&= \frac{\SI{7.719298246}{\debye} \overbrace{\times
%			\frac{\SI{3.34e-30}{\coulomb\meter}}{\SI{1}{\debye}}
%		\times
%\frac{\SI{1e12}{\pico\meter}}{\SI{1}{\meter}}}^{\mathclap{\text{Convert to
%appropriate units!}}}}{\SI{1.6e-19}{\coulomb}} \\
%			&= \SI{161.1403509}{\pico\meter} \\
%			&= \boxed{\SI{160}{\pico\meter}}
%		\end{align*}
	\item We need the lewis structure for sulfate before calculating formal
		charge:

		\begin{minipage}{0.45\linewidth}
			\centering
			$\chemleft[
				\chemfig{\lewis{2:4:6:,O}-S(=[:90]\lewis{1:3:,O})(=[:270]\lewis{5:7:,O})-\lewis{0:2:6:,O}}
			\chemright]^{2-}$
		\end{minipage}
		\hfill
		\begin{minipage}{0.45\linewidth}
			\sisetup{table-format=1}
			\begin{tabular} {e@{ $\text{FC} =$ } S @{ $- ($ } S @{
					${}+\sfrac{1}{2}$ } S[table-format=2] @{ $) =$ }
			S[table-format=+1]}
					S & 6 & 0 & 12 & 0 \\
					O1 & 6 & 6 & 2 & -1 \\
					O2 & 6 & 4 & 4 & 0 \\
					O3 & 6 & 6 & 2 & -1 \\
					O4 & 6 & 4 & 4 & 0
			\end{tabular}
		\end{minipage}

	\item Possible lewis structures for \ch{CO2} are:
		\sisetup{table-format=1}
		\begin{itemize}[label={},itemsep=1em]
			\item \chemfig{\lewis{3:5:,O}=C=\lewis{1:7:,O}} \qquad
				\textleftarrow Most stable due to formal charge
				(lowest in energy)

				\begin{tabular} {e@{ $\text{FC} =$ } S @{ $- ($
					} S @{ ${}+\sfrac{1}{2}$ } S @{ $) =$ }
				S[table-format=+1]}
					C & 4 & 0 & 8 & 0 \\
					O1 & 6 & 4 & 4 & 0 \\
					O2 & 6 & 4 & 4 & 0 \\
				\end{tabular}

			\item \chemfig{\lewis{2:4:6:,O}-C~\lewis{0:,O}}

				\begin{tabular} {e@{ $\text{FC} =$ } S @{ $- ($
					} S @{ ${}+\sfrac{1}{2}$ } S @{ $) =$ }
				S[table-format=+1]}
					C & 4 & 0 & 8 & 0 \\
					O1 & 6 & 6 & 2 & -1 \\
					O2 & 6 & 2 & 6 & +1 \\
				\end{tabular}

			\item \chemfig{\lewis{4:,O}~C-\lewis{0:2:6:,O}}

				\begin{tabular} {e@{ $\text{FC} =$ } S @{ $- ($
					} S @{ ${}+\sfrac{1}{2}$ } S @{ $) =$ }
				S[table-format=+1]}
					C & 4 & 0 & 8 & 0 \\
					O1 & 6 & 2 & 6 & +1 \\
					O2 & 6 & 6 & 2 & -1 \\
				\end{tabular}
		\end{itemize}
	\item Bond lengths:

		$\ch{Cl2} < \ch{Br2} < \ch{I2}$

		Stability:

		$\ch{I2} < \ch{Br2} < \ch{Cl2}$
	\item Shapes:

		\begin{tabular} {r<{.} E
		c c c c}
		\multicolumn{2}{c}{} & \bfseries Lewis Structure & \bfseries Electronic &
			\bfseries Molecular & \bfseries Bond Angles \\
			1 & SiF4 &
			\chemfig{\lewis{2:4:6:,F}-Si(-[:90]\lewis{0:2:4:,F})(-[:270]\lewis{0:4:6:,F})-\lewis{0:2:6:,F}}
			& tetrahedral & tetrahedral & \SI{109.5}{\degree}
			\\[4em]
			2 & SF2 &
			\chemfig{\lewis{2:4:6:,F}-\lewis{2:6:,S}-\lewis{0:2:6:,F}}
			& tetrahedral & bent & \SI{109.5}{\degree} \\[2em]
			3 & COF2 &
			\chemfig{\lewis{2:4:6:,F}-C(=[:90]\lewis{1:3:,O})-\lewis{0:2:6:,F}}
			& trigonal planar & trigonal planar & \SI{120}{\degree}
			\\[2em]
			4 & PCl3 &
			\chemfig{\lewis{3:5:,P}(-[:90]\lewis{0:2:4:,Cl})(-[:270]\lewis{0:4:6:,Cl})-\lewis{0:2:6:,Cl}}
			& trigonal bipyramidal & T-shaped & \SI{90}{\degree} \\
		\end{tabular}

	\item \ch{XeF4} will have a zero dipole moment due to symmetry:

		\begin{tabular} {r<{.} E c c c c}
			\multicolumn{2}{c}{} & \bfseries Lewis Structure & \bfseries Electronic &
			\bfseries Molecular & \bfseries Symmetrical? \\
			1 & SeCl4 &
			\chemfig{\lewis{4:,Se}
				(-[:90]\lewis{0:2:4:,Cl})
				(-[:270]\lewis{0:4:6:,Cl})
				(-[:30]\lewis{0:2:6:,Cl})
				(-[:-30]\lewis{0:2:6:,Cl})}
			& trigonal bipyramidal & see-saw & no
			\\[4em]
			2 & XeF4 &
			\chemfig{\lewis{1:3:5:,F}>:[:-45]\lewis{2:6:,Xe}(<:[:45]\lewis{1:3:7:,F})(<[:225]\lewis{3:5:7:,F})<[:-45]\lewis{1:5:7:,F}}
			& octahedral & square planar & yes \\[2em]
		\end{tabular}
\end{enumerate}

\end{document}
