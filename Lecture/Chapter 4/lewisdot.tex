% !TEX program = xelatex
\documentclass[11pt,letterpaper]{beamer}

%\usepackage[margin=1in]{geometry}
\usepackage{genchem}
\usepackage{lecture}
%\usepackage{mathtools}
%\usepackage{mathspec}
%\usepackage{chemmacros}
%\usepackage{enumframe{\frametitle{}}}
%\usepackage{multicol}
%
%\setallmainfonts{TeX Gyre Pagella}
%
%\pagestyle{empty}

\newcounter{breakout}
\newcommand\breakoutroom{\stepcounter{breakout}\arabic{breakout}.\quad}

\begin{document}

%\section*{Lewis Electron Dot Structures}
%Find your Breakout Room \# and draw the Lewis electron dot structure.

%\textbf{Lewis Structure Lecture Feb 25, 2019}
%
%\begin{enumerate}
%	\frame{\frametitle{ Ionization}}
%		\begin{enumerate}
%			\frame{\frametitle{ What causes atoms to gain/lose electrons?}}
%			\frame{\frametitle{ Where do they go?}}
%		\end{enumerate}
%	\frame{\frametitle{ Electrons from one atom will be transferred to another.}}
%	\frame{\frametitle{ A way to illustrate this is orbital diagrams: \ch{KCl}}}
%	\frame{\frametitle{ An easier way is by dots -- Lewis Electron Dot structures.}}
%	\frame{\frametitle{ For ions: \ch{KCl}, \ch{MgBr2}, \ch{AlF3}}}
%	\frame{\frametitle{ Some electrons aren't fully transferred -- think back to Carbon}}
%	\frame{\frametitle{ How do we represent sharing of electrons}}
%	\frame{\frametitle{ Covalent compounds (go over steps):}}
%		\begin{multicols}{2}
%			\begin{enumerate}[frame{\frametitle{sep=14em,leftmargin=0pt]}}
\frame{\frametitle{Directions} Locate your Breakout Room \# and scroll down to find the assigned compound. With your
	group, draw your Lewis structure.

	\bigskip

	Assign someone from your group to be your ``spokesperson''. The spokesperson will draw the structure when we
	regroup.
}
				\frame{\frametitle{\breakoutroom \ch{CCl4}}}
				\frame{\frametitle{\breakoutroom \ch{SiBr4}}}
				\frame{\frametitle{\breakoutroom \ch{HOCl}}}
				\frame{\frametitle{\breakoutroom \ch{AsF3}}}
				\frame{\frametitle{\breakoutroom \ch{CH2O}}}
				\frame{\frametitle{\breakoutroom \ch{HCN}}}
%				\frame{\frametitle{\breakoutroom \ch{H3O+}}}
%				\frame{\frametitle{\breakoutroom \ch{OH-}}}
				\frame{\frametitle{\breakoutroom \ch{BF3}}}
				\frame{\frametitle{\breakoutroom \ch{ICl3}}}
%				\frame{\frametitle{\breakoutroom \ch{ClO3-}}}
				\frame{\frametitle{\breakoutroom \ch{O3}}}
%				\frame{\frametitle{\breakoutroom \ch{SCN-}}}
				\frame{\frametitle{\breakoutroom \ch{N2O}}}
				\frame{\frametitle{\breakoutroom \ch{SiO2}}}
				\frame{\frametitle{\breakoutroom \ch{SF2}}}
				\frame{\frametitle{\breakoutroom \ch{FNO}}}
%				\frame{\frametitle{\breakoutroom \ch{NO2-}}}
				\frame{\frametitle{\breakoutroom \ch{NO2}}}
				\frame{\frametitle{\breakoutroom \ch{BrF3}}}
%				\frame{\frametitle{\breakoutroom \ch{SO4^{2-}}}}
				\frame{\frametitle{\breakoutroom \ch{XeO2}}}
%				\frame{\frametitle{\breakoutroom \ch{SbF4-}}}
%			\end{enumerate}
%		\end{multicols}
%\end{enumerate}

\end{document}
