% !TEX program = xelatex
%\documentclass[notes=onlyslideswithnotes,notes=hide]{beamer}
%\documentclass[notes=hide]{beamer}
%\documentclass[notes=only]{beamer}
%\documentclass[notes=show]{beamer}
\documentclass[handout]{beamer}

\usepackage{newtxtext}
\usepackage{bucolors}
\usepackage{genchem}
\usepackage{lecture}
\usepackage{tabularx}
\usepackage{import}
\usepackage{tikz}
\usepackage{multicol}
\usetikzlibrary{tikzmark}

\title{Chemical Bonding II}
\subtitle{Chapter 6}
\institute{CHEM115 --- Chemistry for the Sciences I \\ Bloomsburg University}
\author{D.A. McCurry}
\date{Fall 2021}

\chemsetup{chemformula/frac-style=nicefrac}

\newcommand{\electron}[2]{{
	\def\+{\underline{\upharpoonleft}}
	\def\-{\underline{\downharpoonright}}
	\def\0{\underline{\phantom{\downharpoonright}}}
	\setlength\tabcolsep{0pt}
	\begin{tabular}{c}$#2$\\[2pt]#1\end{tabular}
}}

\begin{document}

\maketitle
\mode<article>{\thispagestyle{fancy}}

\frame{\section{Valence Bond Theory}
	\begin{learningobjectives}
	\item Describe how covalent bonds form using valence bond theory.
	\item Relate the formation of chemical bonds to the wave nature of
		electrons.
	\end{learningobjectives}
}

\begin{frame}[c]{Valence Bond Theory}
	\begin{quote}
		A quantum mechanical model that shows how electron pairs are
		shared in a covalent bond.
	\end{quote}

	\bigskip
	Each of the bonded atoms maintains its own atomic orbitals, but the
	electron pair in the \alert{overlapping} orbitals is shared by both atoms.
		
	\begin{center}
		\includegraphics[scale=0.6]{h2.png}
	\end{center}
\end{frame}

\begin{frame}{A Local Minimum in Energy}
	\begin{center}
		\includegraphics[scale=0.4]{06_01_Figure.jpg}
	\end{center}
\end{frame}
	
\begin{frame}{Wavefunctions Sum to Form Bonds}
	The two overlapping lobes must be of the \alert{same phase}.  Recall
	\alert{wavefunctions}!
	\begin{center}
		\includegraphics[scale=0.55]{sigmabond.png}
	\end{center}

	\begin{center}
		\bfseries
		The greater the amount of overlap, the stronger the bond
	\end{center}
\end{frame}

\begin{frame}[t]
	\frametitle{Bonding Within Heteronuclear Compounds}
	How does a bond form between hydrogen and chlorine in \ch{HCl}?

	\vfill

	\note<1>{%
		\begin{center}
		\begin{tabular}{r l l}
			\textbf{H:} & $1s^1$ & \electron{$1s$}{\+\0} \\
			\textbf{Cl:} & [Ne]$3s^23p^5$ & \electron{$3s$}{\+\-}\qquad
			\electron{$3p$}{\+\-\ \+\-\ \+\0}
		\end{tabular}
		\end{center}
	
		\begin{center}
			\includegraphics[scale=0.5]{HClsigma.png}
		\end{center}
	}

	\pause

	How does a bond form between sulfur and hydrogen in \ch{SH6}?

	\vfill

	\note<2>{%
		\begin{center}
			\bfseries
			As long as orbitals can overlap, a $\boldsymbol{\sigma}$ bond
			can form.

			\bigskip

			\includegraphics[scale=0.3]{06_Pg253_UnFigure_2.jpg}
		\end{center}
	}

	\pause

	How does a bond form between carbon and hydrogen in \ch{CH4}?

	\vfill

	\note<3>{%
	\begin{center}
	\includegraphics[scale=0.4]{06_Pg254_UnFigure_2.jpg}
	
	\begin{tabular}{r l l}
		\textbf{C:} & [He]$2s^22p^2$ & \electron{$2s$}{\+\-}\qquad
		\electron{$2p$}{\+\0\ \+\0\ \0\0} \\
		\textbf{H:} & $1s^1$ & \electron{$1s$}{\+\0}
	\end{tabular}
	\end{center}
}
\null
\end{frame}


%\begin{frame}{VBT Summary}
%	\begin{itemize}
%		\item The orbitals can be the standard $s$, $p$,
%			$d$, and $f$ orbitals, or they may be a
%			\alert{hybrid} combination of these.
%		\item A chemical bond results from the overlap
%			of two \alert{half-filled} orbitals and
%			spin-pairing of the two \alert{valence}
%			electrons.
%			\begin{itemize}
%				\item There are also cases where
%					a completely filled
%					orbital overlaps with an
%					empty orbital.
%			\end{itemize}
%		\item The geometry of the overlapping orbitals
%			determines the shape of the molecule.
%	\end{itemize}
%\end{frame}

\frame{\section{Hybridization}
	\begin{learningobjectives}
	\item Identify limitations of using the valence shell electron-pair
		repulsion model.
	\item Explain how carbon-centered molecules (and other tetrahedral
		molecules) have a \SI{109.5}{\degree} bond angle.
	\item Identify when orbital hybridization is required.
	\end{learningobjectives}
}

\begin{frame}{Formation of Hybrid Orbitals}
	\begin{center}
		\includegraphics[scale=0.4]{carbon-ground-state.png}
		\visible<2->{\includegraphics[scale=0.4]{carbon-excited-state.png}}
	\end{center}

	\begin{itemize}[<+->]
		\item With only \alert{2} unpaired electrons, we should only be
			able to form \alert{2} covalent bonds.
		\item Promoting one electron from the $2s$ orbital to the $2p$
			orbital creates \alert{4} unpaired electrons so we can
			now make \alert{4} covalent bonds.
		\item Orbitals need to be the \alert{same energy} (degenerate)
			for appropriate distancing.
	\end{itemize}
\end{frame}

\begin{frame}[allowframebreaks]
	\frametitle{$sp^3$ Hybrid Orbitals}

	\begin{center}
		\includegraphics[scale=0.4,trim={0 0 0 70pt},clip]{06_03_Figure.jpg}
	\end{center}

	\framebreak

	\begin{center}
		\includegraphics[scale=0.55]{methane-sp3-overlap.png}
		
	\end{center}

	Each C $sp^3$ orbital can now form a \textsigma{} bond with the H $1s$
	orbitals

	\framebreak

	\begin{center}
		\includegraphics[scale=0.5]{sp3-tetrahedral.png}
	\end{center}
	
	\bigskip
	
	\begin{center}
		\bfseries
		All tetrahedral geometries will have $\boldsymbol{sp^3}$
		hybridization
	\end{center}
\end{frame}

\begin{frame}[allowframebreaks]
	\frametitle{$sp^2$ Hybrid Orbitals}

	We can consider other geometries from VSEPR theory in
	terms of orbital hybridization
	
	\begin{itemize}
		\item The three charge cloud geometries require \alert{3} bonds
			-- we must have \alert{3} overlapping orbitals
	\end{itemize}
	
	\begin{columns}[c]
		\column{0.30\textwidth}
		\begin{center}
			\includegraphics[scale=0.4]{formaldehyde.png}
		\end{center}
		\column{0.70\textwidth}
		\begin{center}
			\includegraphics[scale=0.3]{06_Pg257_UnFigure_2.jpg}
		\end{center}
	\end{columns}
	
	\begin{center}
		\bfseries
		Note the additional $\boldsymbol{p}$ orbital!
	\end{center}

	\framebreak

	\begin{center}
		\includegraphics[scale=0.4,trim={0 0 0 70pt},clip]{06_04_Figure.jpg}
	\end{center}
\end{frame}

\begin{frame}[allowframebreaks]
	\frametitle{$sp$ Hybrid Orbitals}

	Recall that $p$ orbitals have 2 ``lobes''.
	\begin{itemize}
		\item Each \alert{orbital} is composed of 2 lobes.
		\item We can \alert{only} put a total of 2 electrons in each
			\alert{orbital}.
		\item Linear molecules must therefore also hybridize!
	\end{itemize}

	\bigskip

	\begin{center}
		\includegraphics[scale=0.35]{06_Pg261_UnFigure_1.jpg}
	\end{center}

	\framebreak

	\begin{center}
		\includegraphics[width=\linewidth,trim={0 0 0 50pt},clip]{06_06_Figure.jpg}
	\end{center}
\end{frame}


\frame{\section{Higher Bond Orders}
	\begin{learningobjectives}
	\item Explain the difference between \textsigma{} and \textpi{} bonds.
	\item Explain why double and triple bonds are unable to rotate.
	\item Identify $sp$ hybridization.
	\end{learningobjectives}
}

\begin{frame}[c]
	\frametitle{Types of Bonds}
	\begin{itemize}[<+->]
		\item A \alert{sigma} (\textsigma) bond results when the
			interacting atomic orbitals point along the axis
			connecting the two bonding nuclei.
			\begin{center}
				\includegraphics[scale=0.3,trim={0 0 0
				2.5in},clip]{06_05_Figure.jpg}
			\end{center}
		\item A \alert{pi} (\textpi) bond results when the bonding
			atomic orbitals are \alert{parallel} to each other and
			\alert{perpendicular} to the axis connecting the two
			bonding nuclei.
			\begin{center}
				\includegraphics[scale=0.3,trim={0 2in 0 0},
				clip]{06_05_Figure.jpg}
			\end{center}
		\item The interaction between parallel orbitals is not as strong
			as between orbitals that point at each other.
			\begin{itemize}
				\item \textsigma{} bonds are \alert{stronger}
					than \textpi{} bonds.
			\end{itemize}
	\end{itemize}
\end{frame}

	
\begin{frame}
	\frametitle{Double Bond Formation}

	\begin{center}
		\includegraphics[scale=0.3]{06_Pg258_UnFigure_3.jpg}
	\end{center}

	\begin{itemize}
		\item The \textpi{} bond has \alert{two} regions of overlap
		\item The combined overlap of the $sp^2$ and $p$
			orbitals leads to a \alert{double bond}
	\end{itemize}
\end{frame}

\begin{frame}[t]
	\frametitle{Bond Rotation}
		\begin{itemize}[<+->]
			\item \textsigma{} bonds can rotate freely around the
				bond axis.
			\item \textpi{} bonds must break in order to rotate
				around the double bond axis.
		\end{itemize}

		\bigskip

		\begin{center}
			\includegraphics[scale=0.4,trim={0 0 5in 0},clip]{06_Pg259_UnFigure_2.jpg}
			\visible<2->{\includegraphics[scale=0.4,trim={4.5in 0 0
			0},clip]{06_Pg259_UnFigure_2.jpg}}
		\end{center}
\note<2>{%
	\begin{center}
		\includegraphics[scale=0.35]{06_Pg260_UnFigure_2.jpg}
	\end{center}
}
\end{frame}

\begin{frame}{Triple Bonds}
	\begin{center}
		\includegraphics[width=\linewidth]{06_Pg262_UnFigure_1.jpg}
	\end{center}
\end{frame}


\frame{\section{Further Hybridization}
	\begin{learningobjectives}
	\item Relate all of the different VSEPR geometries to orbital hybridization.
	\item Draw Lewis structures for molecules and identify \textsigma{} and
		\textpi{} bonds.
	\end{learningobjectives}
}

\begin{frame}[allowframebreaks]{$sp^3d$ Hybrid Orbitals}
	Trigonal bipyramidal geometries require a $d$ orbital to hybridize.

	\begin{center}
		\includegraphics[width=\linewidth]{06_07_Figure.jpg}
	\end{center}

	\framebreak

	\begin{center}
		\includegraphics[scale=0.4]{06_Pg262_UnFigure_2.jpg}
	\end{center}
\end{frame}

\begin{frame}[c]{$sp^3d^2$ Hybrid Orbitals}
	\begin{center}
		\includegraphics[width=\linewidth]{06_08_Figure.jpg}
	\end{center}
\end{frame}

\begin{frame}
	\frametitle{Hybrid Orbitals and Their Geometry}

	\begin{center}
	\begin{tabular}{c l l}
		\toprule
		\textbf{Electron Groups} & \textbf{Geometry} &
		\textbf{Hybridization} \\ \midrule
		2 & Linear & $sp$ \\
		3 & Trigonal planar & $sp^2$ \\
		4 & Tetrahedral & $sp^3$ \\
		5 & Trigonal bipyramidal & $sp^3d$ \\
		6 & Octahedral & $sp^3d^2$ \\
		\bottomrule
	\end{tabular}
	\end{center}

	\begin{itemize}
		\item VSEPR theory describes the geometric arrangement of atoms
			in a molecule
		\item Valence bond theory describes how the valence shell
			electrons interact in a bond
	\end{itemize}
\end{frame}

\begin{frame}[t]
	\frametitle{Hybridization and Bonding Scheme}
	Draw the Lewis structure for each of the following, describe the
	molecular geometry and hybridization of each central atom, and label all
	\textsigma{} and \textpi{} bonds using appropriate notation.
	\begin{enumerate}
		\item \ch{ICl2}
		\vfill
		\item \ch{NO2+}
		\vfill
		\item \ch{CH3CO2H}
		\vfill
		\item \ch{N2F2   }
		\vfill
	\end{enumerate}
\end{frame}


%\begin{onyourown}
%	Draw the Lewis structure for each of the following, describe the
%	molecular geometry and hybridization of each central atom, and label all
%	\textsigma{} and \textpi{} bonds using appropriate notation.
%	\begin{enumerate}
%		\item \ch{PCl5}
%			\vspace{10em}
%		\item \ch{NO2-}
%			\vspace{10em}
%		\item \ch{CH3CN  }
%			\vspace{10em}
%		\item \ch{N2F2   }
%	\end{enumerate}
%\end{onyourown}


\frame{\section{Molecular Orbital Theory}
	\begin{learningobjectives}
	\item Identify the limitations of Valence Bond Theory
	\item Explain electron delocalization in molecules.
	\item Draw molecular orbital diagrams to determine the bond order,
		magnetic properties, and stability of molecules.
	\end{learningobjectives}
}

\clearpage

\begin{frame}[t]
	\frametitle{Problems with Valence Bond Theory}
	\begin{itemize}[<+->]
		\item Most properties are predicted better with VB theory than
			with Lewis theory:
			\begin{itemize}[<1->]
				\item Bonding schemes
				\item Bond strengths
				\item Bond lengths
				\item Bond rigidity
			\end{itemize}
		\item Many properties still don't have predictions that align
			with observed values, however:
			\begin{itemize}[<1->]
				\item Magnetic behavior not always correct
				\item Electrons not equally shared across the
					entire molecule
			\end{itemize}
	\end{itemize}
\end{frame}

\begin{frame}[t]
	\frametitle{Molecular Orbital Theory}
	Schrödinger's wave equation is solved (approximately) to calculate a set
	of \alert{molecular} orbitals.
	\begin{itemize}[<+->]
		\item Recall that solving wave equations is only possible for
			the hydrogen atom.
		\item The solution is estimated until the energy of the orbital
			is \alert{minimized}.
	\end{itemize}

	\onslide<+->
	\bigskip

	\begin{block}{Delocalization}
		The electrons belong to the \alert{whole} molecule, so orbitals
		are shared across the \alert{entire} molecule.
	\end{block}
\end{frame}

\begin{frame}[t]
	\frametitle{Linear Combination of Atomic Orbitals}
	\begin{itemize}[<+->]
		\item The \alert{simplest} solution stats with adding atomic
			orbitals together -- the \alert{linear combination of
			atomic orbitals} (LCAO) method.
			\begin{center}
				\includegraphics[scale=0.25]{06_Pg267_UnFigure.jpg}
			\end{center}
		\item Recall \alert{constructive} and \alert{destructive}
			interference.
			\begin{center}
				\includegraphics[scale=0.25]{06_Pg268_UnFigure_1.jpg}
			\end{center}
			Orbitals that are out of phase cancel each other out!
	\end{itemize}
\end{frame}

\begin{frame}[c]
	\frametitle{Interaction of 1\textit{s} Orbitals}
	\begin{center}
		\includegraphics[scale=0.4]{06_09_Figure.jpg}
	\end{center}
\end{frame}

\begin{frame}[c]
	\frametitle{Key Terms and Properties}
	\alert{Bond order} describes the number of electrons
		involved in a bond.
		\begin{itemize}
			\item Generally, what we consider to be single
				(BO = 1)
				double (BO = 2), and triple bonds (BO =
				3).
			\item Only takes \alert{valence} electrons into
				account.
			\item Due to \alert{delocalization}, the bond
				order may be a fraction.
			\item Higher bond orders are \alert{stronger} and
				\alert{shorter}.
			\item A bond order of 0 means \alert{no} bond
				will form!
		\end{itemize}
		\begin{equation*}
			\boxed{
			\text{bond order} = \frac{1}{2}\left(\text{\#
				bonding \el{}} - \text{\# antibonding
			\el{}}\right)}
	\end{equation*}
	\pause
	\null
	\bigskip
	\alert{Magnetism} is the result of \alert{unpaired}
	electrons.
\end{frame}

\begin{frame}[c]
	\frametitle{H\textsubscript{2} MO Diagram}
	\begin{center}
		\includegraphics[scale=0.4]{06_Pg268_UnFigure_2.jpg}
	\end{center}
\end{frame}

\begin{frame}[t,allowframebreaks]
	\frametitle{Building MO Diagrams}
	Draw the MO diagram for \ch{He2}. Is this molecule
	stable?

	\framebreak

	Draw the MO diagram for \ch{He2+}. Is this molecule
	stable?

	\framebreak

	Draw the MO diagram for \ch{H2-}. Is this molecule
	stable?

	\framebreak

	Draw the MO diagram for \ch{Li2}. Is this molecule
	stable?
\end{frame}

\clearpage

\begin{frame}[c,allowframebreaks]
	\frametitle{Interaction of 2\textit{p} Orbitals}
	\begin{center}
		\includegraphics[scale=0.4]{06_Pg271_UnFigure_3.jpg}
	\end{center}

	\framebreak

	\begin{center}
		\includegraphics[scale=0.4]{06_Pg272_UnFigure_1.jpg}
	\end{center}

	\framebreak

	\begin{center}
		\includegraphics[scale=0.4]{06_Pg272_UnFigure_2.jpg}
	\end{center}
\end{frame}

\begin{frame}[c,allowframebreaks]
	\frametitle{Molecular Orbital Energy Ordering}
	\begin{center}
		\includegraphics[scale=0.35]{06_10_Figure.jpg}
	\end{center}

	\framebreak

	\begin{center}
		\includegraphics[scale=0.45]{06_12_Figure.jpg}
	\end{center}
\end{frame}

\begin{frame}[t,allowframebreaks]
	\frametitle{Building MO Diagrams with \textit{p} Orbitals}
	Draw the MO diagram for \ch{N2}. What is the bond order?

	\framebreak

	Draw the MO diagram for \ch{O2}. What is the bond order?

\end{frame}

\begin{frame}{Paramagnetism in Molecular Compounds}
	\begin{center}
		\includegraphics[width=\linewidth]{06_Pg274_UnFigure_1.jpg}
	\end{center}
\end{frame}

%\begin{onyourown}[0em]
%	% Ebbing 10.55
%	Describe the electronic structure of each of the following using
%	molecular orbital theory. Calculate the bond order of each and decide
%	whether it should be stable. For each, state whether the substance is
%	diamagnetic or paramagnetic.
%	\begin{enumerate}
%		\item \ch{B2} \vspace{10em}
%		\item \ch{B2+} \vspace{10em}
%		\item \ch{O2-} \vspace{10em}
%	\end{enumerate}
%\end{onyourown}


\begin{frame}[c,allowframebreaks]
	\frametitle{Heteronuclear Diatomic Molecules and Ions}
	\begin{columns}
		\column{0.5\textwidth}
		\begin{itemize}
			\item When the combining atomic orbitals are idential and of
				equal energy (\alert{homonuclear} diatomic molecules),
				the contribution of each is equal.
			\item If the atomic orbitals are different in type and energy,
				the atomic orbital \alert{closest} in energy to the
				molecular orbital contributes more to the molecular
				orbital.
		\end{itemize}

		\column{0.5\textwidth}

		\begin{center}
			\includegraphics[scale=0.3]{06_Pg276_UnFigure_1.jpg}
		\end{center}
	\end{columns}

	\framebreak

	\begin{columns}
		\column{0.5\textwidth}
		\begin{itemize}
			\item More electronegative atoms have lower
			        energy orbitals.
			\item Lower energy atomic orbitals contribute
			        more to the \alert{bonding} MOs.
			\item Higher energy atomic orbitals contribute
			        more to the \alert{antibonding} MOs.
			\item \alert{Nonbonding} MOs remain localized on
				the atom donating its atomic orbitals.
		\end{itemize}
		\column{0.5\textwidth}
		\begin{center}
			\includegraphics[scale=0.25]{06_Pg276_UnFigure_2.jpg}
		\end{center}
	\end{columns}
\end{frame}

\begin{frame}[t] % Ebbing 10.57
	\frametitle{Building Heteronuclear MO Diagrams}
	Write the molecular orbital diagram and bond order of \ch{CN-}. Is this ion
	diamagnetic or paramagnetic?
\end{frame}


%\begin{onyourown}[14em] % Ebbing 10.58
%	Write the molecular orbital diagram of \ch{BN}. What is the bond
%	order of BN? Is the substance diamagnetic or paramagnetic?
%\end{onyourown}

\clearpage

\begin{frame}[c,allowframebreaks]
	\frametitle{Polyatomic Molecules}
	\begin{itemize}
		\item When many atoms are combined together, the electrons are
			\alert{delocalized} over the entire molecules.
		\item Predictions match MO theory much more than Lewis or VB
			theories.
	\end{itemize}

	\bigskip

	Ozone (\ch{O3}) according to Lewis theory:

	\begin{center}	
		\includegraphics[scale=0.3]{06_Pg278_UnFigure_1.jpg}
	\end{center}

	\framebreak

	Ozone (\ch{O3}) according to MO theory:
	\begin{center}	
		\includegraphics[scale=0.2]{06_Pg278_UnFigure_2.jpg}
	\end{center}

	\framebreak

	Benzene (\ch{C6H6}) according to Lewis theory:
	\begin{center}
		\includegraphics[scale=0.4]{06_Pg278_UnFigure_3.jpg}
	\end{center}

	\framebreak

	Benzene (\ch{C6H6}) according to MO theory:
	\begin{center}
		\includegraphics[scale=0.2]{06_Pg278_UnFigure_4.jpg}
	\end{center}
\end{frame}

\end{document}
