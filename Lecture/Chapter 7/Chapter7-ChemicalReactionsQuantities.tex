% !TEX program=xelatex
%\documentclass[notes=onlyslideswithnotes,notes=hide]{beamer}
%\documentclass[handout]{beamer}
%\documentclass[notes=hide]{beamer}
%\documentclass[notes=show]{beamer}
\documentclass[notes=only]{beamer}

\usepackage{bucolors}
\usepackage{genchem}
\usepackage{lecture}
\usepackage{bucolors}
\usepackage{ccicons}
\usepackage{amsmath}
\usepackage{mathspec}
\setmainfont[Ligatures=TeX]{Georgia}
\setsansfont[Ligatures=TeX]{Arial}
\usepackage{tabularx}
\usepackage{import}
\usepackage{tikz}
\usepackage{multicol}
\usepackage{elements}
\usepackage{collcell}
\usetikzlibrary{tikzmark}

\title{Chemical Reactions and Chemical Quantities}
\subtitle{Chapter 7}
\institute{CHEM115 --- Chemistry for the Sciences I \\ Bloomsburg University}
\author{D.A. McCurry}
\date{Spring 2023}

\begin{document}

\maketitle

\section{Chemical Reactions}

\begin{frame}{Learning Objectives}
	\begin{itemize}
		\item Explain the difference between a physical change and a
			chemical change.
		\item Write chemical equations to show reaction conditions.
		\item Balance chemical equations.
		\item Calculate reactant and product amounts using stoichiometry.
	\end{itemize}
\end{frame}

\begin{frame}{Physical Change}
	\begin{columns}
		\column{0.3\textwidth}
	The molecular structure or composition does not change.
	\column{0.6\textwidth}
	\begin{center}
		\includegraphics[scale=0.3]{07_05_Figure.jpg}
	\end{center}
	\end{columns}
\end{frame}

\begin{frame}{Chemical Change}
	\begin{columns}
		\column{0.4\textwidth}
	Atoms rearrange and the original substances transform into
	different substances.
	\column{0.5\textwidth}
	\begin{center}
		\includegraphics[scale=0.3]{07_04_Figure.jpg}
	\end{center}
\end{columns}

\end{frame}

\begin{frame}{Chemical Reactions}
	A \alert{chemical reaction} takes place where one or more new
		substances are produced.
		\begin{itemize}
			\item Sometimes there is no visible evidence.
			\item Old bonds are broken and new bonds formed.
		\end{itemize}

		\bigskip

		\begin{block}{Chemical Equations}
	\begin{columns}
		\column{0.45\textwidth}
			\begin{reaction*}
				\text{Reactants} -> \text{Products}
			\end{reaction*}

			\column{0.45\textwidth}

			\resizebox{\linewidth}{!}{%
				\begin{tabular} {c l}
					\toprule
					\bfseries
					Symbol & \bfseries Meaning \\
					\midrule
					\ch{+} & Separates two or more formula units \\
					\ch{->} & Reacts to form products \\
					\ch{->[$\Delta$]} & Reactants are heated \\
					\ch{\sld} & Solid \\
					\ch{\lqd} & Liquid \\
					\ch{\gas} & Gas or vapor \\
					\ch{\aq} & Aqueous (dissolved in \ch{H2O}) \\
					\bottomrule
				\end{tabular}
			}
		\end{columns}
	\end{block}

	\note{%
		Symbols used in chemical equations show
		\begin{itemize}
			\item the states of the reactants.
			\item the states of the products.
			\item the reaction conditions.
		\end{itemize}
	}
\end{frame}

\begin{frame}{Balancing Chemical Equations}
	\begin{columns}
		\column{0.45\textwidth}
	A \alert{balanced} chemical equation adheres to the
			\alert{law of conservation of mass}.
	\begin{itemize}
		\item The numbers and kinds of
			atoms must remain the same in both products and
			reactants.
	\end{itemize}

	\column{0.45\textwidth}
	\includegraphics[width=\linewidth]{h2o-balance.jpg}

\end{columns}

\begin{example}
	Balance \ch{Na \sld{} + Cl2 \gas{} -> NaCl\sld{}}
	\end{example}


	\note{%
	\begin{center}
		\begin{tabular}{E e E}
			2 Na\sld{} + Cl2\gas\ & -> & 2 NaCl\sld \\ \midrule
			2 Na & & 2 Na \\
			2 Cl & & 2 Cl
		\end{tabular}
	\end{center}
}
\end{frame}

\begin{frame}[t]{Guide to Balancing a Chemical Reaction}
	\begin{example}
		\begin{reaction*}
			CH4\gas{} + O2\gas{} -> CO2\gas{} + H2O\lqd{}
		\end{reaction*}
	\end{example}

	\begin{enumerate}[<+->]
		\item Write the unbalanced equation using the correct chemical
			formula for each reactant and product.
		\item Count the atoms of each element in the reactants and
			products.
		\item Find suitable \alert{coefficients} (start with the heavy
			atoms, end with hydrogens).
		\item Check your answer by making sure that the numbers and
			kinds of atoms are the same on both sides of the equation
	\end{enumerate}

	\bigskip

	\note<.>{%
		\begin{tabular} {l c l}
			\ch{CH4\gas{}} \ch{+}
			\alert{\ch{2}}
			\ch{O2\gas{}} & \ch{->} &
			\ch{CO2\gas{}} \ch{+}
			\alert{\ch{2}} \ch{H2O\lqd{}}
			\\ \midrule
			1 C & $=$ & 1 C \\
			4 H & $=$ & \alert{4}
			H $\times$ 2 \\
			\alert{4} O
			$\times$ 2 & $=$ &
			\alert{4}\ O
			$\times$ 2
		\end{tabular}
	}
\end{frame}


\begin{frame}[t]{Balance the following equations\ldots}
	\begin{enumerate}[1.\quad]
		\item<+-> \ch{N2\gas{} + H2\gas{} -> NH3\gas{}}
		\note<.>{
			\begin{center}
				\begin{tabular} {E e E}
					N2\gas{} + H2\gas{} & -> & NH3\gas{} \\
					\midrule
					2 N & & 1 N \\
					2 H & & 3 H
				\end{tabular}
			\end{center}
	
			We should balance the heaviest (N) first, then H:
	
			\begin{center}
				\begin{tabular} {E e E}
					N2\gas{} + 3 H2\gas{} & -> & 2 NH3\gas{} \\
					\midrule
					2 N & & 2 N \\
					6 H & & 6 H
				\end{tabular}
			\end{center}
			}

			\vfill

		\item<+-> \ch{Fe3O4\sld{} + H2\gas{} -> Fe\sld{} + H2O\lqd{}}
		\note<.>{
		\begin{center}
			\begin{tabular} {E e E}
				Fe3O4\sld{} + H2\gas{} & -> & Fe\sld{} + H2O\lqd{} \\
				\midrule
				3 Fe & & 1 Fe \\
				4 O & & 1 O \\
				2 H & & 2 H
			\end{tabular}
		\end{center}

		We should balance the heaviest (Fe) first, then O:

		\begin{center}
			\begin{tabular} {E e E}
				Fe3O4\sld{} + 4 H2\gas{} & -> & 3 Fe\sld{} + 4 H2O\lqd{} \\
				\midrule
				3 Fe & & 3 Fe \\
				4 O & & 4 O \\
				8 H & & 8 H
			\end{tabular}
		\end{center}
		}

		\vfill

	\item<+-> \ch{Na3PO4\aq{} + MgCl2\aq{} -> NaCl\aq{} + Mg3(PO4)2\sld{}}
	\note<.>{
		\begin{center}
			\begin{tabular} {E e E}
				Na3PO4\aq{} + MgCl2\aq{} & -> & NaCl\aq{} +
				Mg3(PO4)2\sld{} \\ \midrule
				3 Na+ & & 1 Na \\
				1 PO4^{3-} & & 2 PO4^{3-} \\
				1 Mg^{2+} & & 3 Mg^{2+} \\
				2 Cl- & & 1 Cl- \\
			\end{tabular}
		\end{center}

		We should balance the most complicated (\ch{PO4^{3-}}) first:

		\begin{center}
			\begin{tabular} {E e E}
				2 Na3PO4\aq{} + 3 MgCl2\aq{} & -> & 6 NaCl\aq{} +
				Mg3(PO4)2\sld{} \\ \midrule
				6 Na+ & & 6 Na+ \\
				2 PO4^{3-} & & 2 PO4^{3-} \\
				3 Mg^{2+} & & 3 Mg^{2+} \\
				6 Cl- & & 6 Cl- \\
			\end{tabular}
		\end{center}
		}

		\vfill

	\end{enumerate}
\end{frame}

%\begin{onyourown}[10em]
%	Balance the following equations:
%
%	\begin{itemize}
%		\item \ch{Mg\sld{} + N2\gas{} -> Mg3N2\sld{}}
%
%			\vspace{10em}
%
%		\item \ch{C2H4\gas{} + H2O\lqd{} -> C2H5OH\lqd{}}
%	\end{itemize}
%\end{onyourown}

%\begin{frame}
%			\begin{tabular}{E e E}
%				Hg(NO3)2\aq{} + 2 KI\aq & -> & HgI2\sld{} + 2 KNO3\aq
%				\\ \midrule
%				1 Hg & & 1 Hg \\
%				2 N & & 2 I \\
%				6 O & & 2 K \\
%				2 K & & 2 N \\
%				2 I & & 6 O
%			\end{tabular}
%\end{frame}


\begin{frame}{What do chemical equations tell us?}
	\begin{itemize}[<+(1)->]
		\item Identity of products formed.
		\item Reaction conditions.
		\item Relative \alert{amounts} of reactants needed for the
			reaction.
		\item Relative \alert{amounts} of products expected from the
			reaction.
	\end{itemize}

	\bigskip

	\pause

	For the reaction,
	\begin{reaction*}
		CH4\gas{} + 2 O2\gas{} -> CO2\gas{} + 2 H2O\lqd{}
	\end{reaction*}
	
	\begin{tabular} {@{}l @{ }>{\collectcell\alert}l<{\endcollectcell} @{ } l @{ }
		>{\collectcell\alert}l<{\endcollectcell} @{ } l}
		we need & 2 equivalents & \ch{O2} to react with every & 1 equivalent &
	\ch{CH4}. \\
		& \visible<+(1)->{2 molecules} & & \visible<.(1)->{1 molecule} & \\
		& \visible<+(1)->{2 moles} & & \visible<.(1)->{1 mole} &
	\end{tabular}
\end{frame}

\begin{frame}{Molar Ratios}
	The mole-to-mole ratios in chemical equations can be used to predict how
	much reactant is needed or how much product will be produced.
	\begin{reaction*}
		CH4\gas{} + 2 O2\gas{} -> CO2\gas{} + 2 H2O\lqd{}
	\end{reaction*}

	\begin{center}
		\def\arraystretch{1.75}
	\begin{tabular} {>{\collectcell\ch}r<{\endcollectcell}@{:}E 
		>{\begin{math}}c<{\end{math}}
			@{\quad and \quad} >{\begin{math}}c<{\end{math}}}
				CH4 & O2 &
				\frac{\SI{1}{\mole}~\ch{CH4}}{\SI{2}{\mole}~\ch{O2}}
				&
				\frac{\SI{2}{\mole}~\ch{O2}}{\SI{1}{\mole}~\ch{CH4}}
				\\
				CH4 & CO2 &
				\frac{\SI{1}{\mole}~\ch{CH4}}{\SI{1}{\mole}~\ch{CO2}}
				&
				\frac{\SI{1}{\mole}~\ch{CO2}}{\SI{1}{\mole}~\ch{CH4}}
				\\
				CH4 & H2O &
				\frac{\SI{1}{\mole}~\ch{CH4}}{\SI{2}{\mole}~\ch{H2O}}
				&
				\frac{\SI{2}{\mole}~\ch{H2O}}{\SI{1}{\mole}~\ch{CH4}}
	\end{tabular}
	\end{center}

	\pause

	\begin{block}{Stoichiometry}
		The numerical relationship between chemical amounts in a
		balanced chemical equation.
	\end{block}
\end{frame}

\begin{frame}[t]{Molar Ratio Example}
	How many moles of Fe are needed for the reaction of 12.0 moles of
	\ch{O2}?
	\begin{reaction*}
		Fe\sld{} + O2 \gas{} -> Fe2O3\sld{}
	\end{reaction*}

	\vspace{10em}

	\note{
		\begin{enumerate}
			\item Balance:
				\begin{reaction*}
					4 Fe\sld{} + 3 O2 \gas{} -> 2 Fe2O3\sld{}
				\end{reaction*}
			\item Given \SI{12.0}{\mole}~\ch{O2}, find
				\si{\mole}~\ch{Fe}. Plan:
				\begin{equation*}
					\frac{\SI{4}{\mole}~\ch{Fe}}{\SI{3}{\mole}~\ch{O2}}
				\end{equation*}
			\item Solve:
				\begin{align*}
					\SI{12.0}{\mole}~\ch{O2} \times
					\frac{\SI{4}{\mole}~\ch{Fe}}{\SI{3}{\mole}~\ch{O2}}
					= \boxed{\SI{16.0}{\mole}~\ch{Fe}}
				\end{align*}
			\item Why 3 sig figs?
		\end{enumerate}}
\end{frame}

\begin{onyourown}[10em]
	Using the values from the previous question, how many moles of
	\ch{Fe2O3} will be produced?
\end{onyourown}

\tikzstyle{block} = [draw, text width=5em, text centered, rounded corners,
minimum height=2em]
\tikzstyle{blockA} = [block,fill=blue!10]
\tikzstyle{blockB} = [block,fill=green!10]
\tikzstyle{line} = [draw,->]


\section{Practical Use of Chemical Equations}

\begin{frame}{Learning Objectives}
	\begin{itemize}
	\item Calculate mass amounts of products and reactants based on a
		chemical equation.
	\item Determine which species limit the extent of a reaction.
	\item Calculate the percent yield of a reaction.
	\end{itemize}
\end{frame}

\begin{frame}[t]{Translating Molar Ratios to Experimental Quantities}
		We are unable to ``count'' moles directly.
			\begin{itemize}
				\item How would we measure the moles of \ch{Fe}
					required for the last example?
			\end{itemize}

			\pause

		We must convert to and from units of mass!

		\bigskip

	\begin{example}
		How many \emph{grams} of Fe are needed for the reaction of
		\SI{12.0}{moles} of \ch{O2}?
	\end{example}


		\note<.>{%
			\begin{center}
				\small
				\begin{tikzpicture}[node distance = 2em, auto]
					\node[blockA](massA) {Mass \ch{Fe}};
					\node[blockA,right =of massA](molesA)
					{Moles \ch{Fe}};
					\node[blockB,right =of molesA](molesB)
					{Moles \ch{O2}};
					\node[blockB,right =of molesB](massB)
					{Mass \ch{O2}};
					\path[line] (massA) -- (molesA);
					\path[line] (molesA) -- (molesB);
					\path[line] (molesB) -- (massB);
				\end{tikzpicture}
			\end{center}

		In more generic terms,
			\begin{center}
				\small
				\begin{tikzpicture}[node distance = 2em, auto]
					\node[blockA](massA) {Mass \ch{A}};
					\node[blockA,right =of massA](molesA)
					{Amount \ch{A} \\ (in moles)};
					\node[blockB,right =of molesA](molesB)
					{Amount \ch{B} \\ (in moles)};
					\node[blockB,right =of molesB](massB)
					{Mass \ch{B}};
					\path[line] (massA) -- (molesA);
					\path[line] (molesA) -- (molesB);
					\path[line] (molesB) -- (massB);
				\end{tikzpicture}
			\end{center}
		}

	\note<.>{%
		\begin{align*}
			\SI{16.0}{\mole}~\ch{Fe} \times
			\frac{\SI{55.845}{\gram}~\ch{Fe}}{\SI{1}{\mole}~\ch{Fe}}
			&= \SI{893.52}{\gram} \\
			&= \boxed{\SI{894}{\gram}}
		\end{align*}}
\end{frame}

\begin{frame}[t]{Revisiting Balanced Examples 1}
	What mass of \ch{NH3} is produced from \SI{2.50}{\gram}~\ch{N2}?
	\begin{reaction*}
		N2\gas{} + H2\gas{} -> NH3\gas{}
	\end{reaction*}

	\vspace{10em}

	\note{
		\footnotesize
		\begin{enumerate}
			\item Balance:
				\begin{reaction*}
					N2\gas{} + 3 H2\gas{} -> 2 NH3\gas{}
				\end{reaction*}
			\item Stoichiometric ratio:
				\begin{equation*}
					\frac{\SI{2}{\mole}~\ch{NH3}}{\SI{1}{\mole}~\ch{N2}}
				\end{equation*}
			\item Calculate:
			\begin{center}
				\begin{tikzpicture}[node distance = 2em, auto]
					\node[blockA](massA) {Mass \ch{N2}};
					\node[blockA,right =of massA](molesA)
					{Moles \ch{N2}};
					\node[blockB,right =of molesA](molesB)
					{Moles \ch{NH3}};
					\node[blockB,right =of molesB](massB)
					{Mass \ch{NH3}};
					\path[line] (massA) -- (molesA);
					\path[line] (molesA) -- (molesB);
					\path[line] (molesB) -- (massB);
				\end{tikzpicture}
			\end{center}

				\begin{align*}
					\SI{2.50}{\gram}~\ch{N2} \times
					\frac{\SI{1}{\mole}~\ch{N2}}{\SI{28.014}{\gram}~\ch{N2}}
					\times 
					\frac{\SI{2}{\mole}~\ch{NH3}}{\SI{1}{\mole}~\ch{N2}}
					\times
					\frac{\SI{17.031}{\gram}~\ch{NH3}}{\SI{1}{\mole}~\ch{NH3}}
					&= \SI{3.0397}{\gram} \\
					&= \boxed{\SI{3.04}{\gram}}
				\end{align*}
		\end{enumerate}
		}
\end{frame}

\begin{frame}[t]{Revisiting Balanced Examples 2}
	How many grams of \ch{Na3PO4} are required to produce
	\SI{0.338}{\gram}~\ch{NaCl}?
	\begin{reaction*}
		Na3PO4\aq{} + MgCl2\aq{} -> NaCl\aq{} + Mg3(PO4)2\sld{}
	\end{reaction*}

	\vspace{10em}

	\note{\footnotesize
		\begin{enumerate}
			\item Balance:
				\begin{reaction*}
					2 Na3PO4\aq{} + 3 MgCl2\aq{} -> 6
					NaCl\aq{} + Mg3(PO4)2\sld{}
				\end{reaction*}
			\item Stoichiometric ratio:
				\begin{equation*}
					\frac{\SI{2}{\mole}~\ch{Na3PO4}}{\SI{6}{\mole}~\ch{NaCl}}
				\end{equation*}
			\item Calculate:
			\begin{center}
				\begin{tikzpicture}[node distance = 2em, auto]
					\node[blockA](massA) {Mass \ch{NaCl}};
					\node[blockA,right =of massA](molesA)
					{Moles \ch{NaCl}};
					\node[blockB,right =of molesA](molesB)
					{Moles \ch{Na3PO4}};
					\node[blockB,right =of molesB](massB)
					{Mass \ch{Na3PO4}};
					\path[line] (massA) -- (molesA);
					\path[line] (molesA) -- (molesB);
					\path[line] (molesB) -- (massB);
				\end{tikzpicture}
			\end{center}

				\begin{align*}
					\SI{0.337}{\gram}~\ch{NaCl} \times
					\frac{\SI{1}{\mole}~\ch{NaCl}}{\SI{58.44}{\gram}~\ch{NaCl}}
					\times 
					\frac{\SI{2}{\mole}~\ch{Na3PO4}}{\SI{6}{\mole}~\ch{NaCl}}
					\times
					\frac{\SI{163.94}{\gram}~\ch{Na3PO4}}{\SI{1}{\mole}~\ch{Na3PO4}}
					&= \SI{0.315125371}{\gram} \\
					&= \boxed{\SI{0.315}{\gram}}
				\end{align*}
		\end{enumerate}
		}
\end{frame}

\begin{onyourown}[10em]
	How many grams of water will be produced when
	\SI{1.07}{\gram}~\ch{Fe3O4} reacts with excess \ch{H2}?
\end{onyourown}

\begin{frame}[t]{A More Complicated Example}
	How many grams of \ch{AgCl} are produced when \SI{0.412}{\gram}~\ch{Ag}
	react with \SI{0.793}{\gram}~\ch{Cl2}?
	\begin{reaction*}
		Ag\sld{} + Cl2\gas{} -> AgCl\sld{}
	\end{reaction*}

	\vfill

	\note{
		\begin{enumerate}
			\item Balance:
				\begin{reaction*}
					2 Ag\sld{} + Cl2\gas{} -> 2 AgCl\sld{}
				\end{reaction*}
			\item Stoichiometry:
				$\frac{\SI{2}{\mole}~\ch{AgCl}}{\SI{2}{\mole}~\ch{Ag}}$
				and
				$\frac{\SI{2}{\mole}~\ch{AgCl}}{\SI{1}{\mole}~\ch{Cl2}}$
			\item Solve:
				\begin{itemize}
					\item $\SI{0.412}{\gram}~\ch{Ag} \times
						\frac{\SI{1}{\mole}~\ch{Ag}}{\SI{107.87}{\gram}}
						\times
						\frac{\SI{2}{\mole}~\ch{AgCl}}{\SI{2}{\mole}~\ch{Ag}}
						\times
						\frac{\SI{143.32}{\gram}~\ch{AgCl}}{\SI{1}{\mole}~\ch{AgCl}}
						= \SI{0.547}{\gram}$
					\item $\SI{0.793}{\gram}~\ch{Cl2} \times
						\frac{\SI{1}{\mole}~\ch{Cl2}}{\SI{70.90}{\gram}}
						\times
						\frac{\SI{2}{\mole}~\ch{AgCl}}{\SI{1}{\mole}~\ch{Cl2}}
						\times
						\frac{\SI{143.32}{\gram}~\ch{AgCl}}{\SI{1}{\mole}~\ch{AgCl}}
						= \SI{3.21}{\gram}$
				\end{itemize}
			\item Which is right?
		\end{enumerate}
		}
\end{frame}

\begin{frame}[t]{Limits in Chemical Reactions}
	We can't produce the product if one of the reactants is missing.

	\bigskip

	\pause
	\begin{block}{Limiting Reactant}
		The reactant that is completely consumed in a chemical reaction
		and limits the amount of product.
	\end{block}
	
	\begin{block}{Reactant in Excess}
		Any reactant that occurs in a quantity greater than is required
		to completely react with the limiting reactant.
	\end{block}
	
	\bigskip

	\pause

	For the last example, which is the \emph{limiting
	reactant} and which is the \emph{reactant in excess}?
\end{frame}

\begin{frame}{Limits in Sandwiches}
	\begin{center}
		\includegraphics{CNX_Chem_04_04_sandwich.jpg}
	\end{center}

	\begin{footnotesize}
		Theopold, P. F., Klaus; Richard Langley et al. Reaction Yields
		\url{https://chem.libretexts.org/@go/page/78729} (accessed Oct 28, 2021).
	\end{footnotesize}
\end{frame}

\begin{frame}[t]{Calculating the Reactant in Excess}
	When \SI{0.412}{\gram}~\ch{Ag} reacts with \SI{0.793}{\gram}~\ch{Cl2},
	how much of the reactant in excess is left?
	\begin{reaction*}
		Ag\sld{} + Cl2\gas{} -> AgCl\sld{}
	\end{reaction*}

	\vfill

	\note{
		\begin{enumerate}
			\item Balance:
				\begin{reaction*}
					2 Ag\sld{} + Cl2\gas{} -> 2 AgCl\sld{}
				\end{reaction*}
			\item We had enough reactant to produce
				\SI{0.547}{\gram}~\ch{AgCl}. How much \ch{Cl2}
				did it require to produce this much?
			\item Stoichiometry:
				$\frac{\SI{2}{\mole}~\ch{AgCl}}{\SI{1}{\mole}~\ch{Cl2}}$
			\item Solve for \ch{Cl2} needed:
				\begin{align*}
					\SI{0.547}{\gram}~\ch{AgCl} \times
						\frac{\SI{1}{\mole}~\ch{AgCl}}{\SI{143.32}{\gram}}
						\times
						\frac{\SI{1}{\mole}~\ch{Cl2}}{\SI{2}{\mole}~\ch{AgCl}}
						\times
						\frac{\SI{70.90}{\gram}~\ch{Cl2}}{\SI{1}{\mole}}
						= \SI{0.135}{\gram}
				\end{align*}
			\item Subtract for \ch{Cl2} left:
				\begin{align*}
					\SI{0.793}{\gram} - \SI{0.135}{\gram} &=
					\boxed{\SI{0.658}{\gram}}
				\end{align*}
		\end{enumerate}
		}
\end{frame}

\begin{frame}{Guide to Calculating Product from a Limiting Reactant}
	\begin{center}
	\begin{tikzpicture}[node distance = 1.3em,auto]
		\tikzstyle{blockC} = [block,fill=red!10,text width=10em];
		\node [blockC] (small) {Smaller Moles of Product};
		\node [blockB,text width=6em,above right = of small] (productB) {Moles of Product};
		\node [blockB,text width=6em,above = of productB] (molesB) {Moles of Reactant B};
		\node [blockB,text width=6em,above = of molesB] (gramsB) {Grams of Reactant B};
		\node [blockA,text width=6em,above left = of small] (productA) {Moles of Product};
		\node [blockA,text width=6em,above = of productA] (molesA) {Moles of Reactant A};
		\node [blockA,text width=6em,above = of molesA] (gramsA) {Grams of Reactant A};
		\node [blockC,below = of small] (finalprod) {Grams of
		Product};
		\path [line] (gramsB) -- (molesB);
		\path [line] (molesB) -- (productB);
		\path [line] (productB) -| (small);
		\path [line] (gramsA) -- (molesA);
		\path [line] (molesA) -- (productA);
		\path [line] (productA) -| (small);
		\path [line] (small) -- (finalprod);
	\end{tikzpicture}
	\end{center}
\end{frame}

\begin{frame}{Yields of Reaction}
	We don't always get \alert{exactly} the amount we planned to
	obtain.
	\begin{itemize}
		\item Side reactions (other things happening)
		\item Equilibrium limits (a more advanced topic)
		\item Experimental error (did every grain make it in the
			beaker?)
	\end{itemize}

	\begin{description}[<+(1)->]
		\item[Theoretical Yield]
		The amount of product that can be made in a chemical reaction
		based on the amount of limiting reactant.
		\item[Actual Yield]
		The amount of product actually produced by a chemical reaction.
		\item[Percent Yield]
		The ratio of what we actually produced to what we should
		theoretically have produced.
		\begin{equation*}
			\frac{\text{actual yield}}{\text{theoretical yield}}
			\times 100\%
		\end{equation*}
	\end{description}
\end{frame}

\begin{frame}[t]{Calculating a Percent Yield 1}
	When \SI{0.412}{\gram}~\ch{Ag} reactis with \SI{0.793}{\gram}~\ch{Cl2},
	only \SI{0.498}{\gram}~\ch{AgCl} was produced.
	\begin{reaction*}
		Ag\sld{} + Cl2\gas{} -> AgCl\sld{}
	\end{reaction*}

	\begin{enumerate}
		\item What is the actual yield?
			\note[item]{\SI{0.498}{\gram}~\ch{AgCl}}
		\item What is the theoretical yield?
			\note[item]{\SI{0.547}{\gram}~\ch{AgCl}}
		\item What is the percent yield?
			\note[item]{
				\begin{align*}
					\text{\% Yield} &= \frac{\text{actual
					yield}}{\text{theoretical yield}} \times
					100\% \\
					&=
					\frac{\SI{0.498}{\gram}}{\SI{0.547}{\gram}}
					\times 100\% \\
					&= \boxed{\SI{91.0}{\percent}}
				\end{align*}}
			\note[item]{Can we ever have a negative percent yield?}
	\end{enumerate}

\end{frame}

\begin{frame}[t]{Calculating Percent Yield 2}
	With a limited amount of oxygen, the reaction of carbon and oxygen
	produces carbon monoxide according to the following reaction:
	\begin{reaction*}
		2 C\gas{} + O2\gas{} -> 2 CO\gas{}
	\end{reaction*}
	What is the percent yield if \SI{40.0}{\gram} of \ch{CO} are produced
	when \SI{30.0}{\gram}~\ch{O2} are used?

	\mode<article>{\vfill}

	\note{\footnotesize
		\begin{enumerate}
			\item Is it balanced? \boxed{Yes}
			\item What is the actual yield?
				\boxed{\SI{40.0}{\gram}~\ch{CO}}
			\item What is the theoretical yield?
				\begin{enumerate}
					\item Stoichiometry:
						$\frac{\SI{2}{\mole}~\ch{CO}}{\SI{1}{\mole}~\ch{O2}}$
					\item Grams to moles to moles to grams:
						\begin{equation*}
							\SI{30.0}{\gram}~\ch{O2}
							\times
							\frac{\SI{1}{\mole}}{\SI{31.998}{\gram}~\ch{O2}}
							\times
							\frac{\SI{2}{\mole}~\ch{CO}}{\SI{1}{\mole}~\ch{O2}}
							\times
							\frac{\SI{28.010}{\gram}}{\SI{1}{\mole}~\ch{CO}}
							=
							\boxed{\SI{52.522}{\gram}~\ch{CO}}
						\end{equation*}
				\end{enumerate}
			\item What is the percent yield?
				\begin{align*}
					\text{\% Yield} &= \frac{\text{actual
					yield}}{\text{theoretical yield}} \times
					100\% \\
					&=
					\frac{\SI{40.0}{\gram}}{\SI{52.522}{\gram}}
					\times 100\% \\
					&= \boxed{\SI{76.2}{\percent}}
				\end{align*}
		\end{enumerate}
		}
\end{frame}

\clearpage

\begin{onyourown}[15em]
	When excess \ch{N2} and \SI{5.00}{\gram} of \ch{H2} are mixed, the
	reaction produces \SI{15.0}{\gram}~\ch{NH3}. What is the percent yield
	for the reaction?
	\begin{reaction*}
		N2\gas{} + H2\gas{} -> NH3\gas{}
	\end{reaction*}
\end{onyourown}

\begin{frame}[t]{Calculating Percent Yield 3}
	Acetic acid (\ch{C2H4O2}) reacts with isopentyl alcohol (\ch{C5H12O}) to
	yield isopentyl acetate (\ch{C7H14O2}), which has the odor of bananas.
	If the yield from the reaction is \SI{45}{\percent}, how many grams of
	isopentyl acetate are formed from \SI{3.58}{\gram} acetic acid and
	\SI{4.75}{\gram} isopentyl alcohol?
	\begin{reaction*}
		C2H4O2\lqd{} + C5H12O\lqd{} -> C7H14O2\lqd{} + H2O\lqd{}
	\end{reaction*}

	\note{\tiny
		\begin{enumerate}
			\item Balanced? \boxed{Yes}
			\item What is the theoretical yield?
				\begin{enumerate}
					\item What is the limiting reagent?
						\begin{align*}
							\SI{3.58}{\gram}~\ch{C2H4O2}
							\times
							\tfrac{\SI{1}{\mole}~\ch{C2H4O2}}{\SI{60.0516}{\gram}}
							\times
							\tfrac{\SI{1}{\mole}~\ch{C2H4O2}}{\SI{1}{\mole}~\ch{C7H14O2}}
							&=
							\SI{0.05962}{\mole}~\ch{C7H14O2}
							\\
							\SI{4.75}{\gram}~\ch{C5H12O}
							\times
							\tfrac{\SI{1}{\mole}~\ch{C5H12O}}{\SI{88.1488}{\gram}}
							\times
							\tfrac{\SI{1}{\mole}~\ch{C5H12O}}{\SI{1}{\mole}~\ch{C7H14O2}}
							&=
							\boxed{\SI{0.05389}{\mole}~\ch{C7H14O2}}
						\end{align*}
					\item How much \ch{C7H14O2} should be
						produced?
						\begin{equation*}
							\SI{0.05389}{\mole}~\ch{C7H14O2}
							\times
							\tfrac{\SI{130.1856}{\gram}}{\SI{1}{\mole}~\ch{C7H14O2}}
							=
							\boxed{\SI{7.0152}{\gram}~\ch{C7H14O2}}
						\end{equation*}
				\end{enumerate}
			\item What is the actual yield?
				\begin{align*}
					\text{\% Yield} &= \tfrac{\text{actual
					yield}}{\text{theoretical yield}} \times
					100\% \\
					\SI{45}{\percent} &=
					\tfrac{\text{actual yield}}{\SI{7.0152}{\gram}}
					\times 100\% \\
					\text{actual yield} &=
					\SI{3.1568}{\gram} 
					= \boxed{\SI{3.2}{\gram}~\ch{C7H14O2}}
				\end{align*}
		\end{enumerate}}
\end{frame}

\begin{onyourown}[20em]
	The reaction of \SI{91.3}{\gram}~\ch{C3H6} with excess \ch{O2} produces
	a \SI{81.3}{\percent} yield. How many grams of \ch{CO2} are produced?
	\begin{reaction*}
		C3H6\gas{} + O2\gas{} -> CO2\gas{} + H2O\lqd{}
	\end{reaction*}
\end{onyourown}

%C = 12
%H = 1
%O = 16
%
%12 * 3 = 36
%1 * 6 = 6
%
%
%36 + 6 = 42 grams/mole
%
%12 * 1 = 12
%16 * 2 = 32
%
%12 + 32 = 44 
%
%
%42 * 2 = 84      44 * 6 = 264
%
%264/84 = 3.14
%
%3.14 * 91.3 = 286
%
%81.3 = (actual yield/ 286) * 100
%
%81.3 = (actual yield/ 286) * 100
%
%100
%
%286 * .831 = actual yield
%
%actual yield = 232 grams}

\begin{frame}{Some Common Types of Reactions}
	\begin{block}{Combustion reactions}
		\begin{reaction*}
			CH4\gas{} + 2 O2\gas{} -> CO2\gas{} + 2
			H2O\gas{}
		\end{reaction*}
	\end{block}

	\begin{block}{Alkali metal reactions}
		\begin{reaction*}
			2 Na\sld{} + 2 H2O\lqd{} -> 2
			Na^{+}\aq{} + 2 OH^{-}\aq{} + H2\gas{}
		\end{reaction*}
	\end{block}

	\begin{block}{Halogen reactions}
		\begin{reaction*}
			2 Fe\sld{} + 3 Cl2\gas{} -> 2
			FeCl3\sld{}
		\end{reaction*}
	\end{block}
\end{frame}

\end{document}
