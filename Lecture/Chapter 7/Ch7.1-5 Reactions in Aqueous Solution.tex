\documentclass[handout]{beamer}

\usepackage{genchem}
\graphicspath{{../../Figures/}}

\title[Chapter 7.01-05]{Reactions in Aqueous Solution}
\author{CHEM111}

\begin{document}

\frame{\titlepage}

\begin{frame}[allowframebreaks]{Electrolytes in Aqueous Solution}
	\begin{itemize}
		\item \textbf{Electrolytes:}
			\begin{itemize}
				\item Substances that dissolve in water to
					produce conducting solutions of ions
					
					\begin{center}
						\ce{NaCl(s) ->[H2O] Na+(aq) +
						Cl-(aq)}
					\end{center}
			\end{itemize}
		\item \textbf{Nonelectrolytes:}
			\begin{itemize}
				\item Substances that do not produce ions in
					aqueous solutions
					
					\begin{center}
						\ce{C12H22O11(s) ->[H2O]
						C12H22O11(aq)}
					\end{center}
			\end{itemize}
	\end{itemize}

	\framebreak

	\begin{center}
		\includegraphics[scale=0.35]{07_01_Figure.jpg}
	\end{center}

	\framebreak

	\begin{itemize}
		\item \textbf{Strong Electrolytes:}
			\begin{itemize}
				\item Compounds that dissociate to a
					\emph{large} extent into ions when
					dissolved in water
					
					\begin{center}
						\ce{HCl(g) ->[H2O] H+(aq) +
						Cl-{aq}}
					\end{center}
			\end{itemize}
		\item \textbf{Weak Electrolytes:}
			\begin{itemize}
				\item Compounds that dissociate to a
					\emph{small} extent into ions when
					dissolved in water

					\begin{center}
						\ce{CH3CO2H(aq) <=> H+(aq) +
						CH3CO2-(aq)}
					\end{center}
			\end{itemize}
	\end{itemize}

	\framebreak

	What is the concentration of each ion in a 0.650 M solution of the
	strong electrolyte, \ce{Na2SO4}, assuming complete dissociation?

	\vfill
\end{frame}

\begin{frame}{Dissociation Equations}
	\begin{center}
		\begin{tabular}{r@{ \ce{->[H2O]} }l}
			\ce{Na2SO4(s)} & \ce{2Na+(aq) + SO4^{2-}(aq)} \\[2em]
			\ce{FeBr3(s) & Fe^{3+}(aq) + 3Br-(aq)}
		\end{tabular}
	\end{center}
\end{frame}

\begin{frame}{Electrolyte Classification of Some Common Substances}
	\begin{center}
	\small
	\begin{tabularx}{\textwidth}{p{0.25\textwidth} p{0.25\textwidth} X}
		\toprule
		\bfseries Strong Electrolytes & \bfseries Weak \newline
		Electrolytes & \bfseries Nonelectrolytes \\
		\midrule
		HCl, HBr, HI & \ce{CH3CO2H} & \ce{H2O} \\
		\ce{HClO4} & HF & \ce{CH3OH} (methyl~alcohol) \\
		\ce{HNO3} & HCN & \ce{C2H5OH} (ethyl~alcohol) \\
		\ce{H2SO4} & & \ce{C12H22O11} (sucrose) \\
		\ce{KBr} & & Most compounds of carbon \\
		NaCl & & (organic compounds) \\
		NaOH, KOH \\
		Other soluble ionic compounds \\
		\bottomrule
	\end{tabularx}
	\end{center}

	\begin{itemize}
		\only<1>{\item \textbf{\emph{Strong} Acids:}
			\begin{itemize}
				\item Hydrochloric acid, hydrobromic acid,
					hydroiodic acid, perchloric acid, nitric
					acid, sulfuric acid
			\end{itemize}}
		\only<2>{\item \textbf{Ionic Compounds}}
		\only<3>{\item \textbf{\emph{Weak} Acids:}
			\begin{itemize}
				\item Acetic acid, hydrofluoric acid, hydrogen
					cyanide
			\end{itemize}}
		\only<4>{\item \textbf{Molecular compounds:}
			\begin{itemize}
				\item Anything other than strong or weak
					electrolytes
			\end{itemize}}
	\end{itemize}
\end{frame}

\begin{frame}{Some Ways That Chemical Reactions Occur}
	\begin{itemize}
		\item Precipitation
		\item Acid-Base Neutralization
		\item Oxidation-Reduction (Redox)
	\end{itemize}
\end{frame}

\begin{frame}{Precipitation Reactions}
	Processes in which soluble ionic reactants yield an insoluble solid
	product that falls out of solution

	\begin{center}
		\ce{Pb(NO3)2(aq) + 2KI(aq) -> 2KNO3(aq) + PbI2(s)}
	\end{center}

	\begin{center}
		\includegraphics[scale=0.25]{07_Pg239_UnFigure_2.jpg}
	\end{center}
\end{frame}

\begin{frame}{Acid-Base Neutralization Reactions}
	Processes in which an acid reacts with a base to yield water plus an
	ionic compound called a \emph{salt}

	\begin{center}
		\ce{HCl(aq) + NaOH(aq) -> H2O(l) + NaCl(aq)}
	\end{center}
\end{frame}

\begin{frame}{Oxidation-Reduction (Redox) Reactions}
	Processes in which one or more electrons are transferred between
	reaction partners (atoms, molecules, or ions)

	\begin{center}
		\ce{Mg(s) + 2HCl(aq) -> MgCl2(aq) + H2(g)}
	\end{center}
\end{frame}

\begin{frame}{Aqueous Reactions and Net Ionic Equations}
	\begin{center}
		\only<1>{
			\ce{Pb(NO3)2(aq) + 2KI(aq) -> 2KNO3(aq) + PbI2(s)}
			}
		\only<2-3>{
			\ce{Pb^{2+}(aq) + 2NO3-(aq) + 2K+(aq) + 2I-(aq) ->
			2K+(aq) + 2NO3-(aq) + PbI2(s)}
			}
		\only<4>{
			\ce{Pb^{2+}(aq) + 2I-(aq) -> PbI2(s)}
			}
	\end{center}

	\begin{itemize}
		\item \textbf{Molecular Equation:}
			\begin{itemize}
				\item All substances involved in the reaction
					are written using their complete
					formulas as if they were
					\emph{molecules}
			\end{itemize}
			\pause
		\item \textbf{Ionic Equation:}
			\begin{itemize}
				\item All of the strong electrolytes are written
					as ions
				\pause
				\item Ions that undergo no change during the reaction
					and appear on both sides of the reaction arrow
					are called \textbf{spectator ions}
			\end{itemize}
			\pause
		\item \textbf{Net Ionic Equation:}
			\begin{itemize}
				\item Only the ions undergoing change are shown
			\end{itemize}
	\end{itemize}
\end{frame}

\begin{frame}[allowframebreaks]{Precipitation Reactions and Solubility
	Guidelines}

	\begin{itemize}
		\item \textbf{Solubility:}
			\begin{itemize}
				\item States how much of a compound will
					dissolve in a given amount of solvent at
					a given temperature
			\end{itemize}
	\end{itemize}

	\framebreak

	\begin{enumerate}
		\item A compound is soluble if it contains one of the following
			\emph{cations}:

			\begin{itemize}
				\item Group 1A cation: \ce{Li+}, \ce{Na+},
					\ce{K+}, \ce{Rb+}, \ce{Cs+}
				\item Ammonium ion: \ce{NH4+}
			\end{itemize}

		\item A compound is soluble if it contains one of the following
			\emph{anions}:

			\begin{itemize}
				\item Halide: \ce{Cl-}, \ce{Br-}, \ce{I-}
					\begin{itemize}
						\item \emph{except} \ce{Ag+},
							\ce{Hg2^{2+}},
							\ce{Pb^{2+}}
							halides
					\end{itemize}
				\item Nitrate (\ce{NO3-}), perchlorate
					(\ce{ClO4-}), acetate (\ce{CH3CO2-}),
					sulfate (\ce{SO4^{2-}})
					\begin{itemize}
						\item \emph{except}
							\ce{Sr^{2+}},
							\ce{Ba^{2+}},
							\ce{Hg2^{2+}},
							\ce{Pb^{2+}} sulfates
					\end{itemize}
			\end{itemize}
	\end{enumerate}

	\framebreak

	\begin{itemize}
		\item Note how most soluble compounds have a $\pm 1$ charge
		\item Think lattice energies!
	\end{itemize}

	\framebreak

	Write the molecular, ionic, and net ionic equations for the reaction
	that occurs when aqueous solutions of \ce{AgNO3} and \ce{Na2CO3} are
	mixed.

	\vfill

	\framebreak

	Write the molecular, ionic, and net ionic equations for the reaction
	that occurs when aqueous solutions of \ce{NaCl} and \ce{Fe(NO3)2} are
	mixed.

	\vfill

	\framebreak

	Write the molecular, ionic, and net ionic equations for the reaction
	that occurs when aqueous solutions of \ce{Al2(SO4)3} and \ce{NaOH} are
	mixed.

	\vfill
\end{frame}

\begin{frame}[allowframebreaks]{Acids and Bases}
	\begin{itemize}
		\item \textbf{Acid (Arrhenius):}
			\begin{itemize}
				\item A substance that dissociates in water to
					produce hydrogen ions, \ce{H+}

					\begin{center}
					\begin{tabular}{r@{ \ce{->} }l}
						\ce{HA(aq)} & \ce{H+(aq) +
						A-(aq)} \\
						\ce{HCl(aq)} & \ce{H+(aq) +
						Cl-(aq)}
					\end{tabular}
					\end{center}

				\item In water, acids produce hydronium ions,
					\ce{H3O+}

					\begin{center}
						\ce{HCl(aq) + H2O(aq) ->
						H3O+(aq) + Cl-(aq)}
					\end{center}

				\item Acids can have more than one acidic
					hydrogen (monoprotic, diprotic, and
					triprotic acis)
			\end{itemize}

			\framebreak

		\item \textbf{Base (Arrhenius):}
			\begin{itemize}
				\item A substance that dissociates in water to
					produce hydroxide ions \ce{OH-}

					\begin{center}
					\begin{tabular}{r@{ \ce{->} }l}
						\ce{MOH(aq)} & \ce{M+(aq) +
						OH-(aq)} \\
						\ce{NaOH(aq)} & \ce{Na+(aq) +
						OH-(aq)}
					\end{tabular}
					\end{center}

				\item Ammonia, commonly called ``ammonium
					hydroxide,'' is a base

					\begin{center}
						\ce{NH3(aq) + H2O(aq) <=>
						NH4+(aq) + OH-(aq)}
					\end{center}

					Ammonia does not \emph{contribute}
					\ce{OH-}, but it causes water to
					dissociate, producing \ce{OH-}
			\end{itemize}
	\end{itemize}

	\begin{center}
		\includegraphics[scale=0.4]{07_03_Table.jpg}
	\end{center}

	\bigskip

	\textbf{Strong} acids and \textbf{strong} bases are \textbf{strong}
	electrolytes

	\bigskip

	\textbf{Weak} acids and \textbf{weak} bases are \textbf{weak}
	electrolytes

	\framebreak

	\begin{center}
		\includegraphics[scale=0.4]{07_04_Table.jpg}
	\end{center}

\end{frame}

\begin{frame}[allowframebreaks]{Neutralization Reactions}

	\begin{itemize}
		\item Acids and bases mixed in stoichiometric proportions to
			produce water and a salt
		\item Effectively, these acid-base neutralization reactions are
			double-displacement reactions
	\end{itemize}

	\bigskip

	\begin{center}
		\ce{{\color{blue}H}A + {\color{red}M}OH -> {\color{red}M}A +
		{\color{blue}H}OH}

		\bigskip

		Acid + Base \ce{->} Salt + Water
	\end{center}

	\framebreak

	Write the molecular, ionic, and net ionic equations for the reaction of
	aqueous HBr and aqueous \ce{Ba(OH)2}.
	\vfill

	\framebreak

	Write the molecular, ionic, and net ionic equations for the reaction of
	aqueous NaOH and aqueous HF.
	\vfill

	\framebreak

	Write the molecular, ionic, and net ionic equations for the reaction of
	aqueous \ce{HClO4} and aqueous \ce{Ca(OH)2}.
	\vfill

	\framebreak

	Write the molecular, ionic, and net ionic equations for the reaction of
	aqueous \ce{HC2H3O2} and aqueous \ce{NaOH}.
	\vfill



\end{frame}
\end{document}
