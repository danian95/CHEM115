% !TEX program = xelatex
\documentclass[11pt,letterpaper]{article}

\usepackage{genchem}
\usepackage{enumitem}
\usepackage[margin=1in]{geometry}
\usepackage{titling}

\setallmainfonts{TeX Gyre Pagella}

\title{Chapter 2 ``On Your Own'' Solutions}

\setlist[enumerate,1]{itemsep=2em,leftmargin=0pt,label=\textbf{\Alph*.}}
\setlist[enumerate,2]{label={\arabic*.}}

\begin{document}

\begin{center}
	\bfseries
	\Large
	\thetitle
\end{center}

\begin{enumerate}
	\item Make sure we use an equation that has all our needed variables,
		$\lambda$ (what we're trying to find) and $\nu$ (what we're
		given).
		\begin{align*}
			c &= \lambda \nu \\
			\lambda &= \frac{c}{\nu} \\
			&=
			\frac{\SI{3.00e8}{\meter\per\second}}{\SI{3.4e12}{\per\second}} 
			\times
			\underbrace{\frac{\SI{1}{\nano\meter}}{\SI{e-9}{\meter}}}_{\mathclap{\text{We
						were asked to find it in
			\si{\nano\meter}!}}} \\
			&= \SI{88235.29412}{\nano\meter} \\
			&\approx \boxed{\SI{8.8e4}{\nano\meter}}
		\end{align*}
	\item Again, use the appropriate equation:
		\begin{align*}
			E &= \frac{hc}{\lambda} \\
			&= \frac{(\SI{6.626e-34}{\joule\second})
			(\SI{3.00e8}{\meter\per\second})}{\SI{0.107}{\nano\meter}}
			\times \frac{\SI{1}{\nano\meter}}{\SI{e-9}{\meter}} \\
			&= \SI{1.85776e-15}{\joule} \\
			&\approx \boxed{\SI{1.86e-15}{\joule}}
		\end{align*}
	\item A two-parter. First, let's find the energy of \emph{a single
		photon} using the appropriate equation:
		\begin{align*}
			E &= h\nu \\
			  &=
			  (\SI{6.626e-34}{\joule\second})(\SI{3.42e15}{\per\second})
			  \\
			  &= \SI{2.26609e-18}{\joule} \\
			  &\approx \boxed{\SI{2.27e-18}{\joule}}
		  \end{align*}
		Now, we're asked to find the energy in \SI{0.5}{\mole} of
		photons:
		\begin{align*}
			  \frac{\SI{2.26609e-18}{\joule}}{\SI{1}{photon}} \times
			  \frac{\SI{6.022e23}{photons}}{\SI{1}{\mole}} \times
			  \SI{0.5}{\mole} &=
			  \SI{682320.3012}{\joule} \\
			  &\approx \boxed{\SI{6.82e5}{\joule}}
		  \end{align*}
	  \item \begin{enumerate}
		  \item $E_f - E_i = E_3 - E_1 = \SI{-5.0e-19}{\joule} -
			  \SI{-15.0e-19}{\joule} = \boxed{\SI{10.0e-19}{\joule}}$
		  \item $E_f - E_i = E_1 - E_4 = \SI{-15.0e-19}{\joule} -
			  \SI{-1.0e-19}{\joule} = \SI{14.0e-19}{\joule}$
			  \begin{align*}
				  E &= \frac{hc}{\lambda} \\
				  \lambda &= \frac{hc}{E} \\
				  &=
				  \frac{(\SI{6.626e-34}{\joule\second})(\SI{3.00e8}{\meter\per\second})}{\SI{14.0e-19}{\joule}}
				  \times
				  \frac{\SI{1}{\nano\meter}}{\SI{e-9}{\meter}}
				  \\
				  &= \SI{141.9857143}{\nano\meter} \\
				  &\approx \boxed{\SI{142}{\nano\meter}}
			  \end{align*}
	  \end{enumerate}
  \item Using the de Broglie equation:
	  \begin{align*}
		  \lambda &= \frac{h}{mv} \\
		  v &= \frac{h}{m\lambda} \\
		  &= \frac{\SI{6.626e-34}{\joule\second}}
		  {(\SI{1650}{\kilo\gram})(\SI{3.82e-33}{\nano\meter})} \times
		  \frac{\SI{1}{\kilo\gram\meter\squared\per\second\squared}}{\SI{1}{\joule}}
		  \times
		  \frac{\SI{1}{\nano\meter}}{\SI{e-9}{\meter}} 
		  \times
		  \frac{\SI{1}{\kilo\meter}}{\SI{1e3}{\meter}} \\
		  &= \SI{105.1245439}{\kilo\meter\per\second} \\
		  &\approx \boxed{\SI{105}{\kilo\meter\per\second}}
	  \end{align*}
\end{enumerate}

\end{document}
