\documentclass[12pt,letterpaper]{article}

\usepackage[margin=0.5in]{geometry}
\usepackage{mhchem}
\usepackage{siunitx}
\usepackage{amsmath}
\usepackage{amssymb}
\usepackage{cancel}
\usepackage{xlop}
\setlength{\parskip}{\baselineskip}
\setlength{\parindent}{0pt}
\usepackage{array}
\usepackage{xcolor}
\usepackage{tikz}
\usepackage{multicol}
\usepackage{wrapfig}

\newcommand{\electron}[2]{{% orbital-filling diagrams
	\def\+{\underline{\upharpoonleft}}
	\def\-{\underline{\downharpoonright}}
	\def\0{\underline{\phantom{\downharpoonright}}}
	\setlength\tabcolsep{0pt}
	\begin{tabular}{c}$#2$\\[2pt]#1\end{tabular}
}}

\pagestyle{empty}

\begin{document}
\section*{Chapter 2 Part II Examples}
\subsection*{Converting $\lambda$ to Electron Transitions}

\begin{wrapfigure}{r}{0.4\textwidth}
	\centering
	\begin{tikzpicture}
		\draw[thick] (0,0) -- (4,0) node[right] {$n=1$};
		\draw[thick] (0,4) -- (4,4) node[right] {$n=2$};
		\draw[thick] (0,6) -- (4,6) node[right] {$n=3$};
		\draw[thick] (0,7) -- (4,7) node[right] {$n=4$};
		\draw[thick] (0,7.5) -- (4,7.5) node[right] {$n=5$};
		\draw[thick,->] (2,5.9) -- (2,6.5);
		\draw[thick,->] (-0.5,5) node[rotate=90,left] {Energy (E)} -- (-0.5,7);
	\end{tikzpicture}
\end{wrapfigure}


Given the following energy level diagram for an atom that contains an electron
in the $n=3$ level, answer the following questions:

\begin{enumerate}
	\item Which transition of the electron will emit the lowest frequency
		light?
	\item Using only those levels depicted in the diagram, which transition
		of the electron would require the highest frequency light?
	\item If the transition from the $n=3$ level to the $n=1$ level emits
		green light, what color light is absorbed when an electron makes
		the transition from the $n=1$ to $n=3$ level?
\end{enumerate}

{\color{blue}

1. The lowest frequency light corresponds to a lower energy
($E=h\nu=\frac{hc}{\lambda}$). Therefore, we want the smallest transition. We
can use the Balmer-Rydberg equation to determine and verify this, knowing that
we start at $n=3$ and are \underline{emitting} light (i.e., we are falling to
a inner orbital). As we are considering frequency, let's also convert
$\frac{1}{\lambda}$ to frequency, $\nu$.

\begin{align*}
	\dfrac{1}{\lambda} &= R_\infty \bigg( \dfrac{1}{m^2} - \dfrac{1}{n^2}
	\bigg) \\
	\dfrac{1}{\lambda} = \dfrac{\nu}{c}
	&= R_\infty \bigg( \dfrac{1}{m^2} - \dfrac{1}{3^2} \bigg) 
	= R_\infty \bigg( \dfrac{1}{m^2} - \dfrac{1}{9} \bigg) \\
	\nu &= R_\infty c \bigg( \dfrac{1}{m^2} - \dfrac{1}{9} \bigg) 
\end{align*}

We can either go from 3 to 2 or from 3 to 1. In order to get the lowest
frequency, we need the smallest value for $(\frac{1}{m^2}-\frac{1}{9})$, which
would either be $(\frac{1}{4}-\frac{1}{9})$ (from 3 to 2) or
$(\frac{1}{1}-\frac{1}{9})$ (from 3 to 1). As $m=4$ will provide the smallest
value, we can conclude that the transition \framebox[1.2\width]{from $n=3$ to
$n=2$} has the lowest frequency.

2. Because we're talking about \underline{requiring} light, we can assume that
we will be gaining energy and moving to an outer-orbital. Using the same notes
as above, except considering $m=3$ instead of $n=3$, we should be able to figure
this out. $n=4$ will provide us with $(\frac{1}{9}-\frac{1}{16})$ and $n=5$ will
provide us with $(\frac{1}{9}-\frac{1}{25})$. The transition
\framebox[1.2\width]{from $n=3$ to $n=5$} would therefore required the greatest
frequency of light.

3. The orbital energies are quantized. The emission of green light indicates
that we also require \framebox[1.2\width]{green} light to excite from $n=1$ to
$n=3$.

}

\subsection*{Electron Configurations}

Write the ground state electron configurations and orbital-filling diagrams for
the following atoms: Se, Tc, Be, Al

{\color{blue}
\begin{center}
	\begin{tabular}{c c c c}
		\textbf{Atom} & \textbf{Configuration} & \textbf{Shorthand} &
		\textbf{Orbital-Filling Diagram} \\
		Se & $1s^22s^22p^63s^23p^64s^23d^{10}4p^4$ &
		[Ar]$4s^23d^{10}4p^4$ & \electron{4s}{\+\-} \electron{3d}{\+\-\ 
		\+\-\ \+\-\ \+\-\ \+\-} \electron{4p}{\+\-\ \+\0\ \+\0} \\
		Tc & $1s^22s^22p^63s^23p^64s^23d^{10}4p^65s^24d^5$ &
		[Kr]$5s^24d^{5}$ & \electron{5s}{\+\-} \electron{4d}{\+\0\ 
		\+\0\ \+\0\ \+\0\ \+\0} \\
		Be & $1s^22s^2$ &
		[He]$2s^2$ & \electron{2s}{\+\-} \\
		Al & $1s^22s^22p^63s^23p^1$ &
		[Ne]$3s^23p^1$ & \electron{3s}{\+\-} \electron{3p}{\+\0\ 
		\0\0\ \0\0} \\
	\end{tabular}
\end{center}
}

\end{document}
