\documentclass[12pt,letterpaper]{article}

\usepackage[margin=0.5in]{geometry}
\usepackage{mhchem}
\usepackage{siunitx}
\usepackage{amsmath}
\usepackage{amssymb}
\usepackage{cancel}
\usepackage{xlop}
\setlength{\parskip}{\baselineskip}
\setlength{\parindent}{0pt}
\usepackage{array}
\usepackage{xcolor}

\pagestyle{empty}

\begin{document}
\section*{Chapter 2 Part I Examples}
\subsection*{Relating Frequency to Wavelength}

My laser pointer has a wavelength, $\lambda$, of 532 nm. What is its frequency
in hertz?

{\color{blue}
We know that the speed of light, $c$, is \SI{3.00e8}{\meter\per\second}. In
calculating frequency ($\nu$), we are really asking how many waves pass in 1
second.  This then relates wavelength to the speed of light as:

\begin{center}
	\begin{math}
		\lambda \times \nu = c
	\end{math}
\end{center}

Plugging in our values from above, we get

\begin{center}
	\begin{align*}
		(\SI{532}{\nano\meter})
		\overbrace{\bigg(\dfrac{\SI{1}{\meter}}{\SI{1e9}{\nano\meter}}\bigg)}^{\text{units!}}
		\times \nu &=
		\SI{3.00e8}{\meter\per\second} \\
		\nu &= \SI{5.64e14}{\per\second} =
		\framebox[1.2\width]{\SI{5.64e14}{\hertz}}
	\end{align*}
\end{center}
}

\subsection*{Predicting H Emission}

An energetic hydrogen atom is predicted to emit light at a certain wavelength.
Determine the wavelength if the relaxation involves a transition from $n=4$ to
$m=2$.

{\color{blue}
We'll get into more detail later in the chapter about what $m$ and $n$ actually
mean. For now, let's consider just using the Balmer-Rydberg equation to
determine the wavelength.

\begin{align*}
	\dfrac{1}{\lambda} &= R_\infty \bigg( \dfrac{1}{m^2} - \dfrac{1}{n^2}
	\bigg) \\
	&= (\SI{1.097e-2}{\per\nano\meter}) \bigg( \dfrac{1}{2^2} - \dfrac{1}{4^2} \bigg) \\
	&= \SI{0.002056875}{\per\nano\meter} \\
	\lambda = \bigg(\dfrac{1}{\lambda}\bigg)^{-1} &=
	\framebox[1.2\width]{\SI{486.2}{\nano\meter}}
\end{align*}}

\subsection*{Particle-Like Behavior of Light}

How much energy is a photon in my 532 nm laser?

{\color{blue}
The energy contained in a photon is related through Planck's postulate.
Substituting in the appropriate values will help us arrive at a correct answer.
Remember, $\nu=\frac{c}{\lambda}$.

\begin{align*}
	E = h\nu &= \dfrac{hc}{\lambda} \\
	&= (\SI{6.626e-34}{\joule\second}) \bigg(
	\dfrac{\SI{3.00e8}{\meter\per\second}}{\SI{532}{\nano\meter}} \bigg)
	\bigg( \dfrac{\SI{1e9}{\nano\meter}}{\SI{1}{\meter}}\bigg) \\
	&= \framebox[1.2\width]{\SI{3.74e-19}{\joule}}
\end{align*}

Of course, we rarely talk about single photons, much like we rarely talk about
single atoms. Let's convert to kJ/mol:

\begin{center}
\begin{math}
	\bigg( \dfrac{3.7\underline{3}647 \times 10^{-19}
	\si{\joule}}{\SI{1}{photon}}\bigg)
	\bigg( \dfrac{\SI{6.022e23}{photon}}{\SI{1}{\mole}} \bigg)
	\bigg( \dfrac{\SI{1}{\kilo\joule}}{\SI{1e3}{\joule}} \bigg) 
	= \framebox[1.2\width]{225. \si{\ \kilo\joule\per\mole}}
\end{math}
\end{center}


}
\end{document}
