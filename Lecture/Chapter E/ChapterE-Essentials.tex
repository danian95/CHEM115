% !TEX program = xelatex
%\documentclass[notes=onlyslideswithnotes,notes=hide]{beamer}
%\documentclass[handout,notes=hide]{beamer}
%\documentclass[notes=hide]{beamer}
%\documentclass[notes=show]{beamer}
\documentclass[notes=only]{beamer}
%\documentclass[11pt,letterpaper]{article}
%\usepackage{beamerarticle}

%\usepackage{newtxtext}
\usepackage{bucolors}
\usepackage{genchem}
\usepackage{lecture}
\usepackage{tabularx}
\usepackage{import}
\usepackage{tikz}
\usepackage{multicol}
\usetikzlibrary{tikzmark}
\usepackage{media9}
\colorlet{primary}{bumaroon}

\usepackage{amsmath}
\usepackage{mathspec}
\setmainfont[Ligatures=TeX]{Georgia}
\setsansfont[Ligatures=TeX]{Arial}


\DeclareSIUnit\inch{in.}
\DeclareSIUnit\mile{mi.}
\DeclareSIUnit\foot{ft}
\DeclareSIUnit\pound{lb}

% Allow printing just prefixes by using a "noop" unit
\DeclareSIUnit\noop{\relax}

\title{Essentials}%: Units, Measurement, and Problem Solving}
\subtitle{Chapter E}
\institute[CHEM115 Bloomsburg University]{CHEM115 --- Chemistry for the Sciences I \\ Bloomsburg University}
\author{D.A. McCurry}
\date{Spring 2023}

\begin{document}

\maketitle

\begin{frame}
	\frametitle{Learning Objectives}

	\begin{itemize}
		\item Define chemistry
		\item Represent numbers in scientific notation
		\item Use numerical prefixes in numeric conversions
		\item Use dimensional analysis to convert units
		\item Report numbers to the correct significant digits
		\item Perform simple math operations with significant digits
		\item Apply dimensional analysis to word problems
	\end{itemize}
\end{frame}

\part{Essentials Part 1}

\section{What is Chemistry?}

\begin{frame}
	\frametitle{We Need to Know the Language}
	\begin{itemize}
		\item Qualitative vs Quantitative
		\item Scientific Notation
		\item Dimensional Analysis
		\item Precision vs Accuracy
		\item Significant Figures
	\end{itemize}
\end{frame}

\section{Representing Quantities}

\note{%
	\begin{itemize}
		\item We normally aren't using numbers that are easy to work with.

		\item Atoms are very small (about \qty{0.0000000001}{\meter}) and we
	work with a lot of them (over \num{600000000000000000000000}).

\item Losing count of zeroes is easy. There should be
	%\tikzmarknode{zeroes}{23}
	23 zeroes in the example above after the 6.
	
\item 	Let's instead just say how many zeroes we want to use:
	\begin{equation*}
		\num{600000000000000000000000} =
		6 \times 10^23
		%{\tikzmarknode{scinot}{23}}
	\end{equation*}

%	\begin{tikzpicture}[remember picture,overlay]
%		\draw[thick,->,shorten >=5pt,shorten <=5pt,primary]
%			(zeroes.south)
%			to[bend left=55] (scinot);
%	\end{tikzpicture}
\end{itemize}
}

\subsection{Scientific Notation}

\begin{frame}{Writing in Scientific Notation}
	\begin{enumerate}
		\item Rewrite the number so that only the \alert{ones}
			place is occupied to the left of the decimal
			point.
		\item Count how many places your decimal point had to
			move. This will be your exponent.
			\note{%
			The sign of the exponent is \alert{very} important!
			\begin{itemize}
				\item Positive ($+$) if the number is bigger than 1.
				\item Negative ($-$) if the number is smaller than 1.
			\end{itemize}
		}
	\end{enumerate}

	\begin{example}
		Write the following in scientific notation:
		\begin{itemize}
			\item \num{371039}
			\item \num{82909}
			\item \num{40}
			\item \num{0.0003319}
		\end{itemize}
	\end{example}
\end{frame}

\begin{frame}{Converting Scientific to Standard Notation}
	\begin{example}
		Write the following in standard notation:
		\begin{itemize}
			\item \num{3.001e-4}
			\item \num{7.19e9}
			\item \num{1e-1}
			\item \num{8.00e7}
		\end{itemize}
	\end{example}
\end{frame}

%\begin{onyourown}%[0em]
%	\label{oyo:scientificnotation}
%	\begin{enumerate}
%		\item Write the following in scientific notation:
%
%			\begin{tabularx}{\linewidth} {X X}
%				\qty{201}{\gram}     & \\
%				\qty{4330}{\milli\liter}       & \\
%				\qty{0.00100}{\second}         & \\
%			\end{tabularx}
%
%		\item Write the following in standard notation:
%
%			\begin{tabularx}{\linewidth} {X X}
%				\qty{3.107e2}{\liter} & \\
%				\qty{6.0e-2}{\gram}    & \\
%				\qty{9.0002e6}{\second}    & \\
%			\end{tabularx}
%	\end{enumerate}
%\end{onyourown}

%\mode<article>{%
%Solve the following:
%\begin{align*}
%	\num{173800} \times \num{80300000} &=
%	\intertext{Instead of plugging this in right away, let's
%	convert each number to scientific notation:}
%	\num{1.738e5} \times \num{8.03e7} &=
%	\intertext{Some rearrangement:}
%	\underbrace{1.738 \times 8.03}_{13.95614} \times
%	\underbrace{10^{5} \times 10^{7}}_{10^{5+7}=10^{12}} &=
%	\num{13.95614e12} \\
%	&= \boxed{\num{1.395614e13}}
%\end{align*}
%
%\subsection{Division with Scientific Notation}
%Solve the following:
%\begin{align*}
%	\frac{\num{173800}}{\num{80300000}} =
%	\frac{\num{1.738e5}}{\num{8.03e7}} &=
%	\intertext{We can separate the like terms as in
%	multiplication!}
%	\underbrace{\frac{1.738}{8.03}}_{0.216438356} \times
%	\underbrace{\frac{10^5}{10^7}}_{10^{5-7}=10^{-2}} &=
%	\num{0.216438356e-2} \\
%	&= \boxed{\num{2.16438356e-3}}
%\end{align*}

%\subsection{The Big Advantage of Scientific Notation}
%We can \emph{estimate} the order of magnitude of an answer. This lets us
%quickly check if our calculation is set up correctly.
%
%\begin{mdframed}
%	If we wanted to calculate the size of a gold atom, we might
%	put together something that looks like this:
%	\begin{equation*}
%		\frac{19.3}{196.97} =
%	\end{equation*}
%	Will this give us a \emph{reasonable} answer?
%
%	\bigskip
%	
%	No, we get $\frac{\num{e1}}{\num{e2}} = \num{e-1}$ which is very
%	far off from \num{e-10}.
%\end{mdframed}
%}

\begin{frame}
	\frametitle{Scientific Notation Helps with Estimations!}

	We can \emph{estimate} the order of magnitude of an answer. This lets us
	quickly check if our calculation is set up correctly.

	\bigskip

	\begin{example}
	Solve the following:
		\begin{itemize}
			\item $48013 \times 0.1391 =$
				\note[item]{\num{6.6786e3}}

				\bigskip

			\item $0.43 \div 1002 =$
				\note[item]{\num{4.2914e-4}}
		\end{itemize}
	\end{example}
\end{frame}


%	\begin{onyourown}%[7em]
%	Solve the following:
%	\begin{enumerate}
%		\item $0.0758 \times 53.54$
%			\vspace{7em}
%		\item $678 \div 0.001333$
%	\end{enumerate}
%\end{onyourown}

\subsection{Numerical Prefixes}

\begin{frame}{Common Orders of Magnitude}
%		Some orders of magnitude are used more often than others. Rather than
%		express all values in terms of scientific notation, we can use
%		\alert{numerical prefixes} with our units.
	
		\begin{center}
		\begin{tabular} {l c S[table-format=13]
			S[retain-unity-mantissa=false,table-format=1e-2]}
			\toprule
			\bfseries Prefix & \bfseries Symbol &
			\multicolumn{2}{c}{\bfseries Multiplier} \\ \midrule
			tera  & \unit{\tera\noop} & 1000000000000        & 1e12 \\
			giga  & \unit{\giga\noop} & 1000000000           & 1e9 \\
			mega  & \unit{\mega\noop} & 1000000              & 1e6 \\
			kilo  & \unit{\kilo\noop} & 1000                 & 1e3 \\
			\bottomrule
		\end{tabular}
		\end{center}

		\bigskip

		\begin{example}
			Represent \qty{2e3}{\meter} in \unit{\kilo\meter}
		\end{example}
\end{frame}

%	\begin{math}
	%		\textbf{\color{primary}Example:} \hfill
	%		\qty{2000}{\meter} = 2 \times
	%				\underbrace{10^3}_{\text{kilo,
	%				\unit{\kilo\noop}}}
	%			\unit{\meter} = \qty{2}{\kilo\meter} \hfill\null
	%	\end{math}

	%	\framebreak


\begin{frame}{Common Orders of Magnitude in Chemistry}
	\begin{center}
		\begin{tabular} {l c
			S[table-format=1.18]
			S[retain-unity-mantissa=false,table-format=1e-2]}
			\toprule
			\bfseries Prefix & \bfseries Symbol &
			\multicolumn{2}{c}{\bfseries Multiplier} \\ \midrule
			deci  & \unit{\deci\noop} & 0.1                  & 1e-1 \\
			centi & \unit{\centi\noop} & 0.01                 & 1e-2 \\
			milli & \unit{\milli\noop} & 0.001                & 1e-3 \\
			micro & \unit{\micro\noop} & 0.000001             & 1e-6 \\
			nano  & \unit{\nano\noop} & 0.000000001          & 1e-9 \\
			pico  & \unit{\pico\noop} & 0.000000000001       & 1e-12 \\
			femto & \unit{\femto\noop} & 0.000000000000001    & 1e-15 \\
			atto  & \unit{\atto\noop} & 0.000000000000000001 & 1e-18 \\
			\bottomrule
		\end{tabular}
	\end{center}
\end{frame}

\begin{frame}[t]
	\frametitle{Let's Practice}
	Complete the following unit conversions:
	\begin{enumerate}
		\item \qty{23}{\pico\mole} to \unit{\mole}
			\vspace{\stretch{1}}
		\item \qty{4e7}{\second} to \unit{\mega\second}
			\vspace{\stretch{1}}
		\item \qty{23}{\milli\ampere} to \unit{\giga\ampere}
			\vspace{\stretch{1}}
	\end{enumerate}
\end{frame}

%\begin{onyourown}%[10em]
%	Revisit On Your Own \ref{oyo:scientificnotation} and find suitable
%	prefixes for each value. Assume all of the numbers are values of length.
%\end{onyourown}

\subsection{Percentages}

\begin{frame}{Percentages Represent a Ratio}
	Percentages provide a \alert{ratio} between the number of parts to the
	whole.
	\begin{equation*}
		\unit{\percent} = \frac{\text{Parts}}{\text{Whole}} \times \qty{100}{\percent}
	\end{equation*}
	As such, they use the same units for both the parts and the whole.

	\bigskip

	\begin{example}
		Ground beef in the grocery store is often labeled with the
		percentage of fat content, e.g. 80/20. If you purchase
		\qty{2}{\pound} of ground beef that contains \qty{20}{\percent}
		fat, how much beef are you actually getting?
	\end{example}

	\note{%
		\begin{equation*}
			\frac{\qty{2}{\pound}}{1} \times
			\frac{\qty{80}{\pound}}{\qty{100}{\pound}} =
			\boxed{\qty{1.6}{\pound}}
		\end{equation*}
	}
\end{frame}

\section{Problem Solving}
%	\begin{itemize}
%	\item Use conversion factors to find the solution to numerical problems.
%	\item Understand the utility of dimensional analysis in verifying answers.
%	\item Be familiar with the ``Give \textrightarrow\ Plan \textrightarrow\ Find'' method of problem solving.
%	\end{itemize}

\begin{frame}[t]{Dimensional Analysis}
	By 9:00 AM, there will only be \qty{15}{\minute} left of lecture. How many
	seconds is this?

	\note{%
		\begin{equation*}
			\frac{\qty{15}{\minute}}{1} \times
			\frac{\qty{60}{\second}}{\qty{1}{\minute}} =
			\boxed{\qty{900}{\second}}
		\end{equation*}
	}
\end{frame}

\mode<article>{%
\begin{mdframed}
	Using units as a guide to solving problems.
\end{mdframed}

\begin{itemize}
	\item We have \alert{\qty{600}{\second}}.
	\item We want \alert{time in minutes}.
	\item We need \alert{a conversion factor}.
\end{itemize}

\begin{equation*}
	\overbrace{\qty{60}{\second} =
	\qty{1}{min}}^{\mathclap{\text{Equality}}}
	\qquad\longrightarrow\qquad
	\underbrace{\frac{\qty{60}{\second}}{\qty{1}{min}}}_{\mathclap{\text{Conversion
	factor}}} = 1 \qquad \text{\emph{or}} \qquad
	\underbrace{\frac{\qty{1}{min}}{\qty{60}{\second}}}_{\mathclap{\text{Conversion
	factor}}} = 1
\end{equation*}

\begin{enumerate}
	\item We need a final unit of time, so we should start with a
		unit of time (our given value) in the \alert{numerator}.
	\item We must multiply by our conversion factor so that
		the units \alert{cancel out}.
	\item Ensure the units cancel \alert{first} before doing any
		math.
	\item If you get the correct unit at the end, calculate the
		answer.
\end{enumerate}

\begin{equation*}
	\frac{\qtyunitcancel<3->{600}{\second}}{1} \visible<2->{\times
	\frac{\qty{1}{min}}{\qtyunitcancel<3->{60}{\second}}}
	\visible<3->{=} \visible<4->{\num{10}}~\visible<3->{\unit{min}}
\end{equation*}
}

%\begin{onyourown}%[5em]
%	A very tasty, yet incredibly unhealthy recipe requires \qty{42}{lbs} of
%	butter. Your friend in the UK wants the recipe, but he needs the amount
%	of butter expressed in terms of grams. How many grams of butter is this?
%\end{onyourown}

%\begin{frame}{Some Common Equalities}
%	\centering
%	\scriptsize
%	\begin{tabu} to \linewidth {X *{3}{X[-1,r]@{ }X[-1,l] @{ = } X[-1,r]@{
%			}X[-1,l]}}
%		\toprule\rowfont{\bfseries}
%		Quantity & \multicolumn{4}{c}{U.S.} & \multicolumn{4}{c}{Metric	(SI)}                & \multicolumn{4}{c}{Metric - US}     \\ \midrule
%		Length   & 1 & ft & 12 & in.        & 1 & \unit{\kilo\meter} & 1000 & \unit{\meter}      & 2.54 & \unit{\centi\meter} & 1 & in.  \\
%			 & 1 & yard & 3 & ft        & 1 & \unit{\meter} & 1000 & \unit{\milli\meter}     & 1 & \unit{\meter} & 39.4 & in.        \\
%			 & 1 & mile & 5280 & ft     & 1 & \unit{\centi\meter} & 10 & \unit{\milli\meter} & 1 & \unit{\kilo\meter} & 0.621 & mile \\ \midrule
%		Volume   & 1 & qt & 4 & cups        & 1 & \unit{\liter} & 1000 & \unit{\mL}              & 946 & \unit{\mL} & 1 & qt             \\
%		         & 1 & qt & 2 & pt          & 1 & \unit{\deci\liter} & 100 & \unit{\mL}          & 1 & \unit{\liter} & 1.06 & qt         \\
%			 & 1 & gal & 4 & qt         & 1 & \unit{\mL} & 1 & \unit{\centi\meter\cubed}     &                                     \\ \midrule
%		Mass     & 1 & lb & 16 & oz         & 1 & \unit{\kilo\gram} & 1000 & \unit{\gram}        & 1 & \unit{\kilo\gram} & 2.20 & lb     \\
%			 & \multicolumn{4}{l}{}     & 1 & \unit{\gram} & 1000 & \unit{\milli\gram}       & 454 & \unit{gram} & 1 & lb            \\ \midrule
%		Time     & \multicolumn{4}{l}{}     & 1 & h & 60 & min                               &                                     \\
%		         & \multicolumn{4}{l}{}     & 1 & min & 60 & \unit{\second}                    &                                     \\
%		\bottomrule
%	\end{tabu}
%\end{frame}

\mode<article>{%
It is often \alert{convenient} to assume the part is out of 100 parts of the
whole.

	Learning to interpret problems will help your understanding of
	chemistry.

	\begin{center}
		\settowidth{\leftmargini}{\usebeamertemplate{itemize item}}
		\addtolength{\leftmargini}{\labelsep}
		\begin{tikzpicture}[every node/.style={minimum width=0.25\linewidth,
			minimum height=3em}]
			\node[draw=primary,thick,font=\sffamily\bfseries,anchor=south
			west](given) at (0,0) {Given};
			\draw[draw=primary,thick,->] (given.east) to ++(1,0)
			node[draw=primary,font=\sffamily\bfseries,anchor=west,align=center](concept) {Conceptual \\ Plan};
			\draw[draw=primary,thick,->] (concept.east) to ++(1,0)
			node[draw=primary,font=\sffamily\bfseries,anchor=west](find) {Find};
			\node[below,anchor=north
			west,align=left,font=\scriptsize\raggedright] at (given.south west) {%
				\begin{minipage}{0.25\linewidth}
					\begin{itemize}[<+(1)->]
						\item Sort provided values.
						\item Make note of units.
					\end{itemize}
				\end{minipage}
				};
			\node[below,anchor=north
			west,align=left,font=\scriptsize\raggedright] at (find.south west) {%
				\begin{minipage}{0.25\linewidth}
					\begin{itemize}[<+(1)->]
						\item What value is the question
							asking for?
						\item Determine final units.
					\end{itemize}
				\end{minipage}
				};
			\node[below,anchor=north
			west,align=left,font=\scriptsize\raggedright] at (concept.south west) {%
				\begin{minipage}{0.25\linewidth}
					\begin{itemize}[<+(1)->]
						\item Determine conversions
							factors needed.
						\item List possibly useful
							equations.
					\end{itemize}
				\end{minipage}
				};
		\end{tikzpicture}
	\end{center}

	\begin{center}
		Make sure to check that units \alert{cancel}!
	\end{center}
}

\begin{frame}[t]
	\frametitle{Using Dimensional Analysis for Unit Conversion}
	How many centimeters are in \qty{6.51}{miles}?
	\begin{center}
		\qty{5280}{\foot} = \qty{1}{\mile} \qquad
		\qty{12}{\inch} = \qty{1}{\foot} \qquad
		\qty{2.54}{\centi\meter} = \qty{1}{\inch}
	\end{center}

	\note{%
		\begin{equation*}
		\qty{6.51}{\mile} \times
		\frac{\qty{5280}{\foot}}{\qty{1}{\mile}} \times
		\frac{\qty{12}{\inch}}{\qty{1}{\foot}} \times
		\frac{\qty{2.54}{\centi\meter}}{\qty{1}{\inch}} =
		\boxed{\qty{1.05e6}{\centi\meter}}
		\end{equation*}
	}

\end{frame}

\begin{frame}[c]
	\frametitle{The Gimli Glider: An Exercise in Unit Conversions}
	\begin{center}
		\includegraphics[scale=0.5]{gimlix.jpg}
	\end{center}
	\footnotetext{\tiny\url{https://web.archive.org/web/20200225013028/http://www.wadenelson.com/gimli.html}}
\end{frame}

\part{Essentials Part 2}

\section{Error}

\subsection{Uncertainty in Measurements}

\begin{frame}{What temperature is it?}
	\includegraphics[width=\linewidth]{thermometer.jpeg}

	\bigskip

	{\Large\sffamily\renewcommand\arraystretch{1.25}
		\begin{tabularx}{\linewidth}
			{*{3}{>{\centering\arraybackslash}X}}
		\qty{102.9}{\celsius} & \qty{103.0}{\celsius} &
		\qty{103.1}{\celsius} \\
		\bfseries A & \bfseries B & \bfseries C
	\end{tabularx}
}

	
%	How certain are we with these measurements?
%
%	\begin{center}
%		\includegraphics[scale=0.4,trim={0 30pt 0 0},clip]{volumescales.jpeg}
%	\end{center}
%
%	\pause
%
%	\begin{center}
%	\begin{minipage}{0.9\linewidth}
%		\centering
%		Scientific measurements are reported so that \alert{every} digit
%		is certain, except the \alert{last}.
%	\end{minipage}
%	\end{center}
\end{frame}

\begin{frame}[t]
	\frametitle{What defines ``good'' data?}
	\begin{description}
		\item[Accuracy]

		       \vfill

	     \item[Precision]

		       \vfill

	     \item[Uncertainty]

			\vfill
	\end{description}

	\null

	\note{%
	\begin{description}
		\item[Accuracy:] How close to the true value a given measurement
		  is.
	  \item[Precision:] How well a number of independent measurements
		  agree with each other.
	  \item[Uncertainty:] The degree to which one is confident in the
			measurement.
	\end{description}
}
\end{frame}

\begin{frame}{A Graphical Approach to Precision and Accuracy}
	\centering
	\includegraphics[width=\linewidth,trim={0 0 0 40pt},clip]{accuracyprecision.jpeg}
\end{frame}

%\begin{frame}{An Example\ldots}
%	Three students weigh the same unknown mass 3 times and record the
%	following values:
%
%	\begin{center}
%		\sisetup{table-align-text-post=false}
%		\begin{tabular}
%			{>{\bfseries}l*{2}{S<{\,\unit{\kilo\gram}}}S[table-format=4.1]<{\,\unit{\kilo\gram}}}
%			\toprule
%			& \multicolumn{1}{c}{\bfseries Student A} &
%			\multicolumn{1}{c}{\bfseries Student B} &
%			\multicolumn{1}{c}{\bfseries Student C} \\ \midrule
%			Trial 1 & 988.5 & 999.3 & 872.1 \\
%			Trial 2 & 989.4 & 999.4 & 1150.3 \\
%			Trial 3 & 987.0 & 999.6 & 976.5 \\ \midrule
%			Average & 988.3 & 999.4 & 1000.3 \\
%			\bottomrule
%		\end{tabular}
%	\end{center}
%	\mode<article>{
%		\begin{description}
%			\item[Accuracy:] C > B > A
%			\item[Precision:] B > A > C
%		\end{description}
%	}
%
%	Assuming the mass actually weighs \qty{1000.0}{\kilo\gram}, rank the students in terms of accuracy
%	and precision.
%\end{frame}

\begin{frame}[c]
	\frametitle{Why is reliability important?}
	\centering
	\includegraphics[width=0.85\linewidth,trim={0.75in 4.25in 0.75in 1.5in},clip]{COVIDsensor.pdf}
\end{frame}

\begin{frame}[t]
	\frametitle{Types of Error}
	\begin{description}
		\item[Systematic]

			\vfill

		\item[Random]

			\vfill

		\item[Gross]

			\vfill
	\end{description}

	\null

	\note{%
	\begin{description}
		\item[Systematic:] Measurements that are \alert{consistently} too
			high or too low.
		\item[Random:] Measurements that have equal probability of being
			too high or too low.
		\item[Gross:] You did something very wrong\ldots
	\end{description}
}
\end{frame}

\begin{frame}<presentation>[c]
	\frametitle{An Example of Gross Error}
	\begin{center}
		\includegraphics[width=\linewidth]{OJ.png}
	\end{center}
\end{frame}

% Ended here on 2020-08-19

\section{Significant Figures}

\subsection{Determining Significant Figures}

\begin{frame}{How precise can we be with our measurements?}
	\begin{block}{Significant Figures}
		The total number of digits recorded for a measurement or
		calculated quantity. All digits but the last are certain.
	\end{block}

	\begin{itemize}
		\item Reporting of significant figures in an answer is dependent
			on the \alert{precision} of the \alert{measured} values.
		\item The last digit is an estimate. An error of plus or minus
			one (\num{\pm 1}) is assumed.
		\item Exact numbers effectively have an infinite number of
			significant figures.
	\end{itemize}

	\begin{center}
		\bfseries\color{primary} Is the answer 8.89 or 8.888888889?
	\end{center}
\end{frame}

\begin{frame}{Rules for Significant Figures (SF)}
	\begin{enumerate}[<+->]
		\item All nonzero values are significant.
		\item Zeroes between nonzero digits are significant.
		\item Zeroes at the beginning of a number are not significant;
			they act only to locate the decimal point.
		\item Zeroes at the end of a number and after the decimal point
			are always significant.
		\item Zeroes at the end of a number and before the decimal point
			may or may not be significant.
		\item \alert{Exact numbers}, such as conversions between units or
			counting, have \alert{infinite} significant figures (SF
			rules do not apply).
	\end{enumerate}
\end{frame}

\begin{frame}{Rounding Numbers}
	\begin{enumerate}
		\item<1-> If the first digit you remove is less than 5, round
			down by dropping it and all following digits.
		\item<2-> If the first digit you remove is 5 or greater, round
			up by adding 1 to the digit on the left.
	\end{enumerate}

	\visible<3->{
		\begin{example}
		What are each of the following numbers to 3 sig figs?
		\begin{multicols}{2}
		\begin{itemize}
			\item 0.01273 
			\item 182.145
			\item 981.83 
			\item 10.7
			\item 0.000218
			\item 4.000005
		\end{itemize}
	\end{multicols}
	\end{example}
		}
\end{frame}

% Ended here on 1-24-2020
\subsection{Using Significant Figures in Calculations}

\begin{frame}[t]{SF in Multiplication or Division}
	The answer can't have more \alert{significant figures} than either of
	the original numbers.

	\bigskip

	\begin{example}
		Solve the following:
		\begin{itemize}
			\item $110.5 \times 0.048 =$ \note[item]{$5.304 \approx
				\boxed{5.3}$}

				\bigskip

			\item $1008.02 \div 0.012 =$
				\note[item]{$84001.667 \approx
				\boxed{\num{8.4e4}}$}
		\end{itemize}
	\end{example}
\end{frame}

\begin{frame}[t]{SF in Addition or Subtraction}
	The answer can't have more digits to the \alert{right} of the
	\alert{decimal point} than either of the original numbers.

	\bigskip


	\begin{example}
		Solve the following:
		\begin{itemize}
			\item $25.2 + 1.34$

				\bigskip

			\item $235.05 + 19.6 - 2$
		\end{itemize}
		\note{%
	\begin{tabular} {c r@{.}l}
		& 25 & 2 \\
		+ & 1 & 34 \\ \midrule
		& 26 & 54 \\
		$\approx$ & 26 & 5
	\end{tabular}
		\qquad
	\begin{tabular} {c r@{.}l}
		& 235 & 05 \\
		+ & 19 & 6 \\
		- & 2 \\ \midrule
		& 252 & 65 \\
		$\approx$ & 253 &
	\end{tabular}
}
	\end{example}
\end{frame}

%\begin{onyourown}%{0em}
%	What are each of the following numbers to 4 sig figs?
%	\begin{center}
%		\renewcommand\arraystretch{2}
%		\begin{tabularx}{\linewidth} {*{4}{X}}
%			103.101 & & 10.308 & \\
%			0.0188 & & 11.205 & \\
%			0.000111112 & & 6.000009 & \\
%		\end{tabularx}
%	\end{center}
%\end{onyourown}

%\begin{frame}[allowframebreaks]{Additional Practice}
%	\begin{enumerate}
%		\item Write the following in scientific notation:
%			\begin{enumerate}[a.]
%				\item \qty{0.00000008}{\meter}
%				\item \qty{72000}{\liter}
%			\end{enumerate}
%		\item Write the following in standard notation:
%			\begin{enumerate}[a.]
%				\item \qty{2.0e-2}{\second}
%				\item \qty{1.8e5}{\gram}
%			\end{enumerate}
%		\item State the number of significant figures in each of the
%			following measurements:
%			\begin{enumerate}[a.]
%				\item \qty{0.030}{\meter}
%				\item \qty{4.050}{\liter}
%				\item \qty{0.0008}{\gram}
%				\item \qty{2.80}{\meter}
%			\end{enumerate}
%			\framebreak
%		\item Rank the following in order from greatest to least amount:
%			\begin{center}
%				\begin{tabular} {l>{\raggedleft\arraybackslash}p{0.7in}@{\,}l}
%					\toprule 
%					\bfseries Nutrient &
%					\multicolumn{2}{c}{\bfseries Recommended
%					Amount} \\ \midrule
%					Protein                     & 44   & \unit{\gram      } \\
%					Vitamin C                   & 60   & \unit{\milli\gram} \\
%					Vitamin B\textsubscript{12} & 6    & \unit{\micro\gram} \\
%					Calcium                     & 1    & \unit{\gram      } \\
%					Iron                        & 18   & \unit{\milli\gram} \\
%					Iodine                      & 150  & \unit{\micro\gram} \\
%					Sodium                      & 2400 & \unit{\milli\gram} \\
%					Zinc                        & 15   & \unit{\milli\gram} \\ \bottomrule
%				\end{tabular}
%			\end{center}
%
%			\mode<article>{%
%				\begin{enumerate}
%					\item Protein
%					\item Sodium
%					\item Calcium
%					\item Vitamin C
%					\item Iron
%					\item Zinc
%					\item Iodine
%					\item Vitamin B\textsubscript{12}
%				\end{enumerate}
%			}
%	\end{enumerate}
%\end{frame}

\end{document}
