% !TEX program = xelatex
%\documentclass[notes=onlyslideswithnotes]{beamer}
%\documentclass[notes=only]{beamer}
%\documentclass[notes=hide]{beamer}
\documentclass[10pt,letterpaper,twoside]{article}
\usepackage{beamerarticle}

\usepackage{genchem}
\usepackage{lecture}
\usepackage{multirow}
\usetikzlibrary{tikzmark}
\usepackage{media9}

\DeclareSIUnit\inch{in.}
\DeclareSIUnit\mile{mi.}
\DeclareSIUnit\pound{lb}

\title{Essentials}%: Units, Measurement, and Problem Solving}
%\subtitle{Chapter E}
\institute[CHEM115 Bloomsburg University]{CHEM115 --- Chemistry for the Sciences I \\ Bloomsburg University}
\author{D.A. McCurry}
\date{Fall 2020}

\begin{document}

\maketitle
\mode<article>{\thispagestyle{fancy}}

%\begin{frame}{The Central Science}
%	\begin{minipage}{\linewidth}
%		\centering
%		\includegraphics[scale=0.3]{centralscience.pdf}
%
%		\footnotesize M stone.
%	\url{https://en.wikipedia.org/wiki/The_central_science}
%	\end{minipage}
%\end{frame}

%\begin{frame}{The Scientific Method}
%	\tabulinesep=1em
%	\begin{tabu} to \linewidth {>{\color{primary}\bfseries}r X}
%		\savetabu{description}
%		Observation: & What did you see? \\
%		Hypothesis: & What is a possible interpretation of what you
%		  saw? \\
%		Experiments: & What can we do to test the hypothesis? \\
%		Theory: & Is the hypothesis confirmed consistently? \\
%	\end{tabu}
%\end{frame}

%\begin{frame}{Scientists Use Consistent Terminology to Ease Communication}
%	We must have a standard way to \alert{measure} things so we can
%	communicate our ideas to others. There are two categories in which
%	measurements can fall:
%
%	\begin{enumerate}[<+->]
%		\item \alert{Qualitative:} Descriptive, things you
%			\alert{see}.
%			\begin{itemize}[<1->]
%				\item Changes in color and physical state.
%			\end{itemize}
%		\item \alert{Quantitative:} Measured or counted values.
%			\begin{itemize}[<1->]
%				\item Values recorded from instrumentation or
%					glassware.
%				\item The number of students in this class.
%				\item Often associated with a \alert{dimensional
%					unit}.
%			\end{itemize}
%	\end{enumerate}
%
%%	\visible<2->{
%%	\begin{center}
%%		\bfseries \color{primary} What measurements did we do in 
%%		lab yesterday?
%%	\end{center}}
%\end{frame}

\begin{frame}{}
	\section{Scientific Notation}
	\begin{learningobjectives}
		\item Write small and large numbers in scientific notation.
		\item Use scientific notation to facilitate division.
		\item Use scientific notation for order of magnitude estimates.
	\end{learningobjectives}
\end{frame}

\begin{frame}{Why do we need a special notation?}
	We normally aren't using numbers that are easy to work with.
	\begin{example}
		Atoms are very small (about \SI{0.0000000001}{\meter}) and we work with a lot of them (over
		\num{600000000000000000000000}).
	\end{example}

	\pause

	Losing count of zeroes is easy. There should be
	\tikzmarknode{zeroes}{23} zeroes in the example above after the 6.
	
	\pause

	Let's instead just say how many zeroes we want to use:
	\begin{equation*}
		\num{600000000000000000000000} =
		6 \times 10^{\tikzmarknode{scinot}{23}}
	\end{equation*}

	\begin{tikzpicture}[remember picture,overlay]
		\draw[thick,->,shorten >=5pt,shorten <=5pt,primary]
			(zeroes.south)
			to[bend left=55] (scinot);
	\end{tikzpicture}


%	\begin{description}[<+->]
%		\item[Example:] One \alert{mole} of atoms contains
%			\SI{602200000000000000000000}{atoms}.
%		\item[Problem:] My hand is tired from writing all those zeroes.
%		\item[Solution:] We can use \alert{scientific notation} to
%			simplify writing this number: 
%			\begin{equation*}
%				\num{6.022e23}
%			\end{equation*}
%			This indicates that there are \alert{23} digits after the
%			6 before the actual decimal point.
%	\end{description}
\end{frame}

\begin{frame}[t]{Writing in Scientific Notation}
	\begin{enumerate}
		\item Rewrite the number so that only the \alert{ones}
			place is occupied to the left of the decimal
			point.
		\item Count how many places your decimal point had to
			move. This will be your exponent.
		\item The sign of the exponent is \alert{very} important!
			\begin{itemize}
				\item Positive ($+$) if the number is bigger than 1.
				\item Negative ($-$) if the number is smaller than 1.
			\end{itemize}
	\end{enumerate}

	\pause
	\mode<article>{\clearpage}

	Let's try a few examples:
	\begin{enumerate}[<only@+>]
		\renewcommand\arraystretch{1.9}
		\item Write the following in scientific notation:

			\begin{tabularx}{\linewidth} {X X}
				\num{371039}     &
				\num{82909}       \\
				\num{40}         & 
				\num{0.0003319} \\
			\end{tabularx}

		\item Write the following in standard notation:

			\begin{tabularx}{\linewidth} {X X}
				\num{3.001e-4} &
				\num{7.19e9} \\
				\num{1e-1} &
				\num{8.00e7} \\
			\end{tabularx}
	\end{enumerate}
\end{frame}

\begin{onyourown}%[0em]
	\label{oyo:scientificnotation}
	\begin{enumerate}
		\item Write the following in scientific notation:

			\begin{tabularx}{\linewidth} {X X}
				\SI{201}{\gram}     & \\
				\SI{4330}{\milli\liter}       & \\
				\SI{0.00100}{\second}         & \\
			\end{tabularx}

		\item Write the following in standard notation:

			\begin{tabularx}{\linewidth} {X X}
				\SI{3.107e2}{\liter} & \\
				\SI{6.0e-2}{\gram}    & \\
				\SI{9.0002e6}{\second}    & \\
			\end{tabularx}
	\end{enumerate}
\end{onyourown}

\begin{frame}[c]
	\frametitle{Multiplication with Scientific Notation}

	Solve the following:
	\begin{align*}
		\num{173800} \times \num{80300000} &=
		\onslide<+(1)->{
			\intertext{Instead of plugging this in right away, let's
			convert each number to scientific notation:}
			\num{1.738e5} \times \num{8.03e7} &=
		}
		\onslide<+(1)->{
			\intertext{Some rearrangement:}
			\underbrace{1.738 \times 8.03}_{13.95614} \times
			\underbrace{10^{5} \times 10^{7}}_{10^{5+7}=10^{12}} &=
		}
		\onslide<+(1)->{
			\num{13.95614e12} \\
		}
		\onslide<+(1)->{
			&= \boxed{\num{1.395614e13}}
		}
	\end{align*}
\end{frame}

\begin{frame}[c]
	\frametitle{Division with Scientific Notation}
	Solve the following:
	\begin{align*}
		\frac{\num{173800}}{\num{80300000}} =
		\onslide<+(1)->{
			\frac{\num{1.738e5}}{\num{8.03e7}} &=
		}
		\onslide<+(1)->{
			\intertext{We can separate the like terms as in
			multiplication!}
			\underbrace{\frac{1.738}{8.03}}_{0.216438356} \times
			\underbrace{\frac{10^5}{10^7}}_{10^{5-7}=10^{-2}} &=
		}
		\onslide<+(1)->{
			\num{0.216438356e-2} \\
		}
		\onslide<+(1)->{
			&= \boxed{\num{2.16438356e-3}}
		}
	\end{align*}
\end{frame}

\begin{frame}[t]
	\frametitle{Let's Practice}
	Solve the following:
	\begin{enumerate}
		\item $48013 \times 0.1391$
			\vspace{\stretch{1}}
			\note[item]{$\approx\num{6.6786e3}$}
		\item $0.43 \div 1002$
			\vspace{\stretch{1}}
			\note[item]{\num{4.3086e2}}
	\end{enumerate}
\end{frame}

	\begin{onyourown}%[7em]
	Solve the following:
	\begin{enumerate}
		\item $0.0758 \times 53.54$
			\vspace{7em}
		\item $678 \div 0.001333$
	\end{enumerate}
\end{onyourown}

\begin{frame}[c]
	\frametitle{The Big Advantage of Scientific Notation}
	We can \alert{estimate} the order of magnitude of an answer. This lets us
	quickly check if our calculation is set up correctly.

	\pause
	\mode<presentation>{\bigskip}

	\begin{example}
		If we wanted to calculate the size of a gold atom, we might
		put together something that looks like this:
		\begin{equation*}
			\frac{19.3}{196.97} =
		\end{equation*}
		Will this give us a \emph{reasonable} answer?
		\note<+>{No, we get $\frac{\num{e1}}{\num{e2}} = \num{e-1}$ which is very
		far off from \num{e-10}.}
	\end{example}
\end{frame}

\begin{frame}{}
	\section{Fundamental Units}
	\begin{learningobjectives}
		\item Identify common units used in the sciences.
		\item Understand the utility of unit prefixes.
		\item Convert between different unit prefixes.
	\end{learningobjectives}
\end{frame}

\begin{frame}[c]
	\frametitle{The Gimli Glider: An Exercise in Unit Conversions}
	\begin{center}
		\includegraphics[scale=0.5]{gimlix.jpg}
	\end{center}
	\footnotetext{\tiny\url{https://web.archive.org/web/20200225013028/http://www.wadenelson.com/gimli.html}}
\end{frame}

\begin{frame}{Système Internationale d'Unités (SI)}
	For consistency, scientists have agreed to use \alert{SI units} and the
	\alert{metric system} for quantitative measurements.

	\begin{center}
		\renewcommand\arraystretch{1.1}
		\begin{tabular}{l@{\qquad}l@{\qquad}l}
			\toprule
			\bfseries Physical Quantity & \bfseries Name of Unit &
			\bfseries Abbreviation \\ \midrule
			Mass                & kilogram & \si{\kilo\gram }\\
			Length              & meter    & \si{\meter     }\\
			Temperature         & kelvin   & \si{\kelvin    }\\
			Amount of substance & mole     & \si{\mole      }\\
			Time                & second   & \si{\second    }\\
			Electric current    & ampere   & \si{\ampere    }\\
			Luminous intensity  & candela  & \si{\candela   }\\
			\bottomrule
		\end{tabular}
	\end{center}

	\begin{center}
		All other units are derived from these \alert{fundamental} units.
	\end{center}
\end{frame}

\begin{frame}[c]
	\frametitle{Fundamental Units were Defined from Physical Objects}

	\begin{center}
		\begin{tikzpicture}
			\node(length) {\includegraphics[width=0.4\linewidth]{US_National_Length_Meter.JPG}};
			\node[right=of
				length](mass) {\includegraphics[width=0.4\linewidth]{National_prototype_kilogram_K20_replica.jpg}};
		\end{tikzpicture}
	\end{center}
\end{frame}

%\begin{frame}{Fundamental Units: Measuring Length}
%	The \alert{meter} has gone through a number of revisions to its
%	definition\ldots
%
%	\begin{tabularx}{\linewidth} {>{\usebeamercolor[fg]{structure}\bfseries}r<{:} >{\raggedright\arraybackslash}X}
%		1790 & One ten-millionth of the distance from the equator to the North Pole. \\
%		1889 & \begin{tabularx}{\linewidth}[t] {@{}>{\raggedright\arraybackslash}X >{\centering\arraybackslash}X}
%				Distance between two thin lines on a bar of platinum-iridium alloy stored near Paris,
%				France. &
%				\raisebox{-7em}{\includegraphics[scale=0.4]{US_National_Length_Meter.JPG}}
%				\end{tabularx} \\
%		1983 & Distance traveled by light through a vacuum in $\sfrac{1}{\num{299792458}}$~\si{\second}.
%	\end{tabularx}
%\end{frame}

%\begin{frame}{Fundamental Units: Measuring Mass}
%	Likewise, the SI unit of mass (\si{\kilo\gram}) has undergone a few
%	changes.
%
%	\begin{center}
%		\begin{tabularx}{\linewidth} {c >{\usebeamercolor[fg]{structure}\bfseries}r<{:}@{\quad}>{\raggedright\arraybackslash}X}
%			\multirow{4}{0.2\linewidth}{\includegraphics[width=\linewidth]{National_prototype_kilogram_K20_replica.jpg}} &
%			1795 & Mass of one liter of water. \\
%			     & 1799 & Mass of a platinum standard weight. \\
%			     & 1879 & Mass of a platinum-iridium standard weight. \\
%			     & 2019 & Defined in terms of the Planck constant,
%				\SI{6.62607015e-34}{\kilo\gram\meter\squared\per\second\squared}
%	\end{tabularx}
%	\end{center}
%
%	\begin{block}{Mass vs Weight}
%		The \alert{mass} of an object is the quantity of matter within
%		it whereas weight is a measure of gravitational pull ($m \times g$).
%		Weight differs on different planets, but mass is always the
%		same.
%	\end{block}
%\end{frame}

\begin{frame}[c]
	\frametitle{Fundamental Units are Defined from Physical Constants}

	\begin{center}
		\renewcommand\arraystretch{1.1}
		\sisetup{unit-color=primary,
			table-format=1.3e+2,
			table-align-text-post=false,
			per-mode=fraction,
			round-mode=places,
			round-precision=3,
			table-space-text-post=\,\si{\kilo\gram\meter\squared\per\second\squared\per\kelvin}
		}
		\begin{tabular}
			{>{\raggedright\arraybackslash}m{0.4\linewidth}cS}
			\toprule
			\bfseries Quantity & \bfseries Symbol & \bfseries Value \\ \midrule
			Elementary charge & $e$ & 1.602176634e-19 \,\si{\coulomb} \\
			Speed of light in vacuum & $c$ & 2.99792458e8 \,\si{\meter\per\second} \\
			Planck's constant & $h$ & 6.62607015e-34 \,\si{\kilo\gram\meter\squared\per\second} \\
			Avogadro's number & $N_{\textrm{A}}$ & 6.02214076e23 \,\si{\per\mole} \\
			Boltzmann's constant & $k$ & 1.380649e-23 \,\si{\kilo\gram\meter\squared\per\second\squared\per\kelvin} \\
			Unperturbed ground-state hyperfine transition frequency of \ch{^{133}Cs} & $\Delta v_{\ch{Cs}}$
												 & 9.192631770e9
												 \,\si{\per\second}
			\\
			\bottomrule
		\end{tabular}
	\end{center}
\end{frame}

\begin{frame}{Fundamental Unit Conversions from Familiar Units}
	\only<+>{
		\begin{align*}
			\textbf{Length:} &&  \SI{1}{\inch} &=
			\SI{0.0254}{\meter} & \\
					 &&  \SI{1}{\mile} &\approx
			\SI{1610}{\meter} & \\ \\
			\textbf{Weight/Mass:\textsuperscript{*}} &&
				\SI{1}{\pound} &\approx \SI{0.4545}{\kilo\gram}
					       &
		\end{align*}

		\begin{block}{\textsuperscript{*}Mass vs Weight}
			The \alert{mass} of an object is the quantity of matter within
			it whereas weight is a measure of gravitational pull ($m \times g$).
			Weight differs on different planets, but mass is always the
			same.
		\end{block}
	}

	\mode<article>{\clearpage}

	\only<+>{
		\begin{flalign*}
			\textbf{Temperature:} &&
			T (\si{\celsius}) &= \frac{T (\si{\fahrenheit})
					- 32}{1.8} &
					\\ \bigskip
						   && T (\si{\kelvin}) &=
					T (\si{\celsius}) + 273.15 &
		\end{flalign*}
		
	\begin{center}
		\begin{tikzpicture}
			\node(img)
				{\includegraphics[width=0.48\linewidth,trim={0 0
				0 40pt},clip]{tempscales.jpeg}};
			\node[right=of img,text
				width=0.3\linewidth,align=center]{\bfseries\usebeamercolor[fg]{alerted
				text} At \SI{0}{\kelvin}, molecular motion
			ceases.};
		\end{tikzpicture}
	\end{center}
	}
\end{frame}

% Ended here on 2020-08-17

\begin{frame}{Numerical Prefixes}
	\only<+>{
		Some orders of magnitude are used more often than others. Rather than
		express all values in terms of scientific notation, we can use
		\alert{numerical prefixes} with our units.
	
		\begin{center}
		\begin{tabular} {l c S[table-format=13]
			S[retain-unity-mantissa=false,table-format=1e-2]}
			\toprule
			\bfseries Prefix & \bfseries Symbol &
			\multicolumn{2}{c}{\bfseries Multiplier} \\ \midrule
			tera  & \si{\tera } & 1000000000000        & 1e12 \\
			giga  & \si{\giga } & 1000000000           & 1e9 \\
			mega  & \si{\mega } & 1000000              & 1e6 \\
			kilo  & \si{\kilo } & 1000                 & 1e3 \\
			\bottomrule
		\end{tabular}
		\end{center}
	
		\begin{math}
			\textbf{\color{primary}Example:} \hfill
			\SI{2000}{\meter} = 2 \times
					\underbrace{10^3}_{\text{kilo, \si{\kilo}}}
				\si{\meter} = \SI{2}{\kilo\meter} \hfill\null
		\end{math}
	}

	\only<+>{
		There are prefixes for small numbers as well\ldots
	
		\begin{center}
		\begin{tabular} {l c
			S[table-format=1.18]
			S[retain-unity-mantissa=false,table-format=1e-2]}
			\toprule
			\bfseries Prefix & \bfseries Symbol &
			\multicolumn{2}{c}{Multiplier} \\ \midrule
			deci  & \si{\deci } & 0.1                  & 1e-1 \\
			centi & \si{\centi} & 0.01                 & 1e-2 \\
			milli & \si{\milli} & 0.001                & 1e-3 \\
			micro & \si{\micro} & 0.000001             & 1e-6 \\
			nano  & \si{\nano } & 0.000000001          & 1e-9 \\
			pico  & \si{\pico } & 0.000000000001       & 1e-12 \\
			femto & \si{\femto} & 0.000000000000001    & 1e-15 \\
			atto  & \si{\atto } & 0.000000000000000001 & 1e-18 \\
			\bottomrule
		\end{tabular}
		\end{center}
	}
\end{frame}

\begin{frame}{What's more convenient?}
	\begin{center}
	``I ran \SI{10000}{\meter} today.'' \par
		\textbf{vs} \par
	``I ran \SI{10}{\kilo\meter} today.''
		
		\bigskip

	``My phone has \SI{32000000000}{B} of storage.'' \par
		\textbf{vs} \par
	``My phone has \SI{32}{\giga B} of storage.''
	\end{center}
\end{frame}

\begin{frame}[c]
	\frametitle{Converting Between Prefixes}

	We can think of numerical prefixes as a \alert{replacement} for the
	``$\times 10^x$'' term of scientific notation:
	\begin{equation*}
		1 \tikzmarknode{multiplier}{\times 10^3}\,\si{\meter} =
		1 \,\tikzmarknode{prefix}{\si{\kilo}}\si{\meter}
	\end{equation*}

	\begin{tikzpicture}[remember picture,overlay]
		\draw[thick,primary,->,shorten <=5pt,shorten >=5pt]
			(multiplier.south) to[bend right=45] (prefix.south);
	\end{tikzpicture}

	When we need to convert between different prefixes, we need to multiply
	by the correct \alert{conversion factor}.

	\begin{example}
		Convert \SI{2}{\micro\gram} to \si{\gram}.
		\begin{equation*}
			\SI{2}{\micro\gram} \times
			\begin{array} {c|c|c}
				1 & \times 10^{-6} & \si{\gram} \\ \midrule
				1 & \si{\micro} & \si{\gram}
			\end{array}
			= \SI{2e-6}{\gram}
		\end{equation*}
	\end{example}
\end{frame}

\begin{frame}[t]
	\frametitle{Let's Practice}
	Complete the following unit conversions:
	\begin{enumerate}
		\item \SI{23}{\pico\mole} to \si{\mole}
			\vspace{\stretch{1}}
		\item \SI{4e7}{\second} to \si{\mega\second}
			\vspace{\stretch{1}}
		\item \SI{23}{\milli\ampere} to \si{\giga\ampere}
			\vspace{\stretch{1}}
	\end{enumerate}
\end{frame}

\begin{onyourown}%[10em]
	Revisit On Your Own \ref{oyo:scientificnotation} and find suitable
	prefixes for each value. Assume all of the numbers are values of length.
\end{onyourown}

\begin{frame}{}
	\section{Derived Units}
	\begin{learningobjectives}
		\item Understand how other units are related to the fundamental
			units.
		\item Recognize the differences between intensive and extensive
			properties.
		\item Know some common derived units in chemistry relating to
			volume, density, and energy.
	\end{learningobjectives}
\end{frame}

\begin{frame}{Derived Units are Composed of Fundamental Units}
	Often, we need a combination of units to adequately describe a
	measurement.

	\begin{center}
	\begin{tabular} {@{}*{3}{l}l}
		\toprule \bfseries
		Quantity     & \bfseries Definition                 &
		\multicolumn{2}{l}{\bfseries Derived Unit}                                                   \\
		\midrule
		Area         & Length $\times$ length     & \si{\meter\squared                                  }                              \\
		Volume       & Area $\times$ length       & \si{\meter\cubed                                    }                              \\
		Density      & Mass / unit volume         & \si{\kilo\gram\per{\meter\cubed}                    }                              \\
		Speed        & Distance / unit time       & \si{\meter\per{\second}                             }                              \\
		Acceleration & Δ speed / unit time        & \si{\meter\per{\second\squared}                     }                              \\
		Force        & Mass $\times$ acceleration &
		\si{\kilo\gram.\meter\per{\second\squared}          } & \textrightarrow\ \si{\newton} \\
		Pressure     & Force / unit area          &
		\si{\kilo\gram\per(\meter.\second\squared)          } & \textrightarrow\ \si{\pascal} \\
		Energy       & Force $\times$ distance    &
		\si{(\kilo\gram.\meter\squared)\per{\second\squared}} & \textrightarrow\ \si{\joule}  \\
		\bottomrule
	\end{tabular}
	\end{center}
\end{frame}

% Ended here on 1-22-2020

\begin{frame}{Measuring Volume}
	\begin{center}
	\begin{tikzpicture}
	\node(list)[text width=0.55\linewidth]{
	\noindent
	\textbf{Why is volume a derived unit?}

	\medskip

		\begin{itemize}
			\item We can describe volume as the length 3 dimensions.
			\item A cube has a \alert{height}, \alert{width}, and
				\alert{depth}.
				\begin{align*}
					V &= h \times w \times d \\
					&= m \times m \times m
				\end{align*}
			\item All are measurements of length, thus the unit is a
				measure of length to the third power,
				\si{\meter\cubed}.
		\end{itemize}
	};
	\node(img)[anchor=west,align=center] at (list.east)
		{\includegraphics[scale=0.3]{lengthandvolume.jpeg} \\[1em]
	\fbox{$\SI{1}{\milli\liter} = \SI{1}{\centi\meter\cubed}$}
	};
\end{tikzpicture}
\end{center}
\end{frame}


\begin{frame}{Measuring Density}
	Density is really asking ``How much can fit in this space?''

	\begin{align*}
		\text{Density} &= \frac{\text{How much?}}{\text{In what space?}}
		\\
		\intertext{With mass as a measure of amount and volume as a
		measure of space,}
		\text{Density} &= \frac{\text{Mass (\si{\gram})}}{\text{Volume
		(\si{\mL} or \si{\centi\meter\cubed})}}
	\end{align*}

	Density is an \alert{intensive property} -- it is independent of the
	amount of substance. An \alert{extensive property}, such as mass, depends
	on the amount present.
\end{frame}

\begin{frame}{Density of Some Common Substances at \SI{20}{\celsius}}
	\begin{center}
		\begin{tabular}{l S[table-format=2.3]}
			\toprule
			\bfseries Substance & \textbf{Density (\si{\gram\per\centi\meter\cubed})} \\
			\midrule
			Charcoal (from oak) & 0.57 \\
			Ethanol & 0.789 \\
			Sugar (sucrose) & 1.58 \\
			Table salt (sodium chloride) & 2.16 \\
			Glass & 2.6 \\
			Aluminum & 2.70 \\
			Iron & 7.86 \\
			Copper & 8.96 \\
			Lead & 11.4 \\
			Gold & 19.3 \\
			Platinum & 21.4 \\
			\bottomrule
		\end{tabular}
	\end{center}
\end{frame}


\begin{frame}{Density is temperature dependent!}
	\centering
	\includegraphics[scale=0.4]{dH2O.jpeg}
\end{frame}

\begin{frame}{Measuring Energy}
	\begin{description}
		\item[Energy] is the capacity to supply heat or do work.
		\item[Work] is the action of force through a distance.
	\end{description}

	\begin{center}
		\includegraphics[scale=0.4,trim={0 70pt 0 0},clip]{work.jpeg}
	\end{center}

	The \alert{total energy} of a system is the sum of its \alert{kinetic
	energy} (motion) and \alert{potential energy} (position/composition)
\end{frame}

\begin{frame}{Units of Energy}
	\begin{center}
		\begin{tabularx}{\linewidth} {l l X}
		\toprule
		\bfseries Unit & \bfseries Symbol & \bfseries Derivation \\
		\midrule
		joule & J & From kinetic energy, $E = \frac{1}{2}mV^2$ \\
		& & \SI{1}{\joule} =
		\SI{1}{\kilo\gram\meter\squared\per\second\squared} \\
		calorie & cal & Amount of energy to raise \SI{1}{\gram} of water
		by \SI{1}{\celsius} \\
		Calorie & Cal & From nutrition, \SI{1}{Cal} =
		\SI{1}{\kilo cal} \\
		\bottomrule
	\end{tabularx}
	\end{center}

	Joules and calories are common units, so we need to keep a
	\alert{conversion factor} in mind:
	\begin{equation*}
		\SI{1}{cal} = \SI{4.184}{\joule}
	\end{equation*}
\end{frame}

\begin{frame}{The Directionality of Energy}
	Energy can flow into or out of a \alert{system} from or to the
	\alert{surroundings}.

	\begin{description}
		\item[Exothermic:] Heat is lost from a system to the
			surroundings. The change in energy is \alert{negative}.
		\item[Endothermic:] Heat is gained by the system from the
			surroundings. The change in energy is
				\alert{positive}.
	\end{description}
\end{frame}

\begin{frame}{}
	\section{The Reliability of Measurements}
	\begin{learningobjectives}
		\item Understand the difference between accuracy and precision.
		\item Note possible sources of error in experiments.
		\item Round answers appropriately based on possible errors in
			values.
	\end{learningobjectives}
\end{frame}

\begin{frame}{What temperature is it?}
	\includegraphics[width=\linewidth]{thermometer.jpeg}

	\bigskip

	{\Large\sffamily\renewcommand\arraystretch{1.25}
		\begin{tabularx}{\linewidth}
			{*{3}{>{\centering\arraybackslash}X}}
		\SI{102.9}{\celsius} & \SI{103.0}{\celsius} &
		\SI{103.1}{\celsius} \\
		\bfseries A & \bfseries B & \bfseries C
	\end{tabularx}
}

	
%	How certain are we with these measurements?
%
%	\begin{center}
%		\includegraphics[scale=0.4,trim={0 30pt 0 0},clip]{volumescales.jpeg}
%	\end{center}
%
%	\pause
%
%	\begin{center}
%	\begin{minipage}{0.9\linewidth}
%		\centering
%		Scientific measurements are reported so that \alert{every} digit
%		is certain, except the \alert{last}.
%	\end{minipage}
%	\end{center}
\end{frame}

\begin{frame}{What defines ``good'' data?}
	As important as units are to measurements, we must also make sure that
	our measurements are accurate and precise!

	\begin{description}
		\item[Accuracy:] How close to the true value a given measurement
		  is.
	  \item[Precision:] How well a number of independent measurements
		  agree with each other.
	  \item[Uncertainty:] The degree to which one is confident in the
			measurement.
	\end{description}
\end{frame}

\begin{frame}{A Graphical Approach to Precision and Accuracy}
	\centering
	\only<1>{\includegraphics[scale=0.9]{dartboard.png}}

	\only<2>{\includegraphics[scale=0.45]{accuracyprecision.jpeg}}
\end{frame}

\begin{frame}{An Example\ldots}
	Three students weigh the same unknown mass 3 times and record the
	following values:

	\begin{center}
		\sisetup{table-align-text-post=false}
		\begin{tabular}
			{>{\bfseries}l*{2}{S<{\,\si{\kilo\gram}}}S[table-format=4.1]<{\,\si{\kilo\gram}}}
			\toprule
			& \multicolumn{1}{c}{\bfseries Student A} &
			\multicolumn{1}{c}{\bfseries Student B} &
			\multicolumn{1}{c}{\bfseries Student C} \\ \midrule
			Trial 1 & 988.5 & 999.3 & 872.1 \\
			Trial 2 & 989.4 & 999.4 & 1150.3 \\
			Trial 3 & 987.0 & 999.6 & 976.5 \\ \midrule
			Average & 988.3 & 999.4 & 1000.3 \\
			\bottomrule
		\end{tabular}
	\end{center}
	\note{
		\begin{description}
			\item[Accuracy:] C > B > A
			\item[Precision:] B > A > C
		\end{description}
	}

	Assuming the mass actually weighs \SI{1000.0}{\kilo\gram}, rank the students in terms of accuracy
	and precision.
\end{frame}

\begin{frame}<presentation>[c]
	\frametitle{Why is reliability important?}
	\includegraphics[width=\textwidth,trim={0.75in 4.25in 0.75in 1.5in},clip]{COVIDsensor.pdf}
\end{frame}

\begin{frame}{Types of Error}
	\begin{description}
		\item[Systematic:] Measurements that are \alert{consistently} too
			high or too low.
		\item[Random:] Measurements that have equal probability of being
			too high or too low.
		\item[Gross:] You did something very wrong\ldots
	\end{description}

	\mode<presentation>{%
		\begin{center}
			\includemedia[%
				width=0.6\linewidth,
				activate=pageopen,
				addresource=OJ.mp4,
				flashvars={source=OJ.mp4&loop=true}
			]{\includegraphics[width=\linewidth]{OJ.png}}{VPlayer.swf}
		\end{center}
	}
\end{frame}

% Ended here on 2020-08-19

\begin{frame}{How precise can we be with our measurements?}
	\begin{block}{Significant Figures}
		The total number of digits recorded for a measurement or
		calculated quantity. All digits but the last are certain.
	\end{block}

	\begin{itemize}
		\item Reporting of significant figures in an answer is dependent
			on the \alert{precision} of the \alert{measured} values.
		\item The last digit is an estimate. An error of plus or minus
			one (\num{\pm 1}) is assumed.
		\item Exact numbers effectively have an infinite number of
			significant figures.
	\end{itemize}

	\begin{center}
		\bfseries\color{primary} Is the answer 8.89 or 8.888888889?
	\end{center}
\end{frame}

\begin{frame}{Rules for Significant Figures (SF)}
	\begin{enumerate}[<+->]
		\item All nonzero values are significant.
		\item Zeroes between nonzero digits are significant.
		\item Zeroes at the beginning of a number are not significant;
			they act only to locate the decimal point.
		\item Zeroes at the end of a number and after the decimal point
			are always significant.
		\item Zeroes at the end of a number and before the decimal point
			may or may not be significant.
		\item \alert{Exact numbers}, such as conversions between units or
			counting, have \alert{infinite} significant figures (SF
			rules do not apply).
	\end{enumerate}
\end{frame}

% Ended here on 1-24-2020

\begin{frame}[t]{Significant Figures in Calculations}
	\begin{description}
		\item<+->[Multiplication or Division:] The answer can't have more
			\alert{significant figures} than either of the original numbers.
			\only<.>{
				\begin{align*}
					110.5 \times 0.048 = 5.304 &\approx \\
					\frac{1008.02}{0.012} = 84001.667
					&\approx
				\end{align*}}
		\item<+->[Addition or Subtraction:] The answer can't have more
		digits to the \alert{right} of the \alert{decimal point} than
			either of the original numbers.
			\only<.>{
				\begin{center}
				\begin{tabular} {c r@{.}l}
					& 25 & 2 \\
					+ & 1 & 34 \\ \midrule
					& 26 & 54 \\
					$\approx$
				\end{tabular}
					\qquad
				\begin{tabular} {c r@{.}l}
					& 235 & 05 \\
					+ & 19 & 6 \\
					+ & 2 \\ \midrule
					& 256 & 65 \\
					$\approx$
				\end{tabular}
				\end{center}}
	\end{description}

	\visible<+->{
	\begin{center}
		What happens to the non-significant figures?
	\end{center}}
\end{frame}

\begin{frame}{Rounding Numbers}
	\begin{enumerate}
		\item<1-> If the first digit you remove is less than 5, round
			down by dropping it and all following digits.
		\item<2-> If the first digit you remove is 5 or greater, round
			up by adding 1 to the digit on the left.
	\end{enumerate}

	\visible<3->{
		What are each of the following numbers to 3 sig figs?

		\begin{center}
			\renewcommand\arraystretch{2}
			\begin{tabularx}{\linewidth} {*{4}{X}}
			0.01273 & & 182.145 & \\
			981.83 & & 10.7 & \\
			0.000218 & & 4.000005 & \\
		\end{tabularx}
		\end{center}
		}
\end{frame}

\begin{onyourown}%{0em}
	What are each of the following numbers to 4 sig figs?
	\begin{center}
		\renewcommand\arraystretch{2}
		\begin{tabularx}{\linewidth} {*{4}{X}}
			103.101 & & 10.308 & \\
			0.0188 & & 11.205 & \\
			0.000111112 & & 6.000009 & \\
		\end{tabularx}
	\end{center}
\end{onyourown}

\begin{frame}{}
	\section{Problem Solving}
	\begin{learningobjectives}
	\item Use conversion factors to find the solution to numerical problems.
	\item Understand the utility of dimensional analysis in verifying answers.
	\item Be familiar with the ``Give \textrightarrow\ Plan \textrightarrow\ Find'' method of problem solving.
	\end{learningobjectives}
\end{frame}


\begin{frame}{Converting Between Units}
	\begin{block}{Dimensional Analysis}
		Using units as a guide to solving problems.
	\end{block}

	By 1:20 PM, there will only be \SI{600}{\second} left of lecture. How many
	minutes is this?

	\begin{itemize}
		\item<2-> We have \alert{\SI{600}{\second}}.
		\item<3-> We want \alert{time in minutes}.
		\item<4-> We need \alert{a conversion factor}.
	\end{itemize}

	\visible<5->{
		\begin{equation*}
			\overbrace{\SI{60}{\second} =
			\SI{1}{min}}^{\mathclap{\text{Equality}}}
			\qquad\longrightarrow\qquad
			\underbrace{\frac{\SI{60}{\second}}{\SI{1}{min}}}_{\mathclap{\text{Conversion
			factor}}} = 1 \qquad \text{\emph{or}} \qquad
			\underbrace{\frac{\SI{1}{min}}{\SI{60}{\second}}}_{\mathclap{\text{Conversion
			factor}}} = 1
		\end{equation*}
		}
\end{frame}

\begin{frame}{Dimensional Analysis Example}
	Let's try it!

	\begin{enumerate}[<+->]
		\item We need a final unit of time, so we should start with a
			unit of time (our given value) in the \alert{numerator}.
		\item We must multiply by our conversion factor so that
			the units \alert{cancel out}.
		\item Ensure the units cancel \alert{first} before doing any
			math.
		\item If you get the correct unit at the end, calculate the
			answer.
	\end{enumerate}

	\begin{equation*}
		\frac{\SIunitcancel<3->{600}{\second}}{1} \visible<2->{\times
		\frac{\SI{1}{min}}{\SIunitcancel<3->{60}{\second}}}
		\visible<3->{=} \visible<4->{\num{10}}~\visible<3->{\si{min}}
	\end{equation*}
\end{frame}

\clearpage

\begin{onyourown}%[5em]
	A very tasty, yet incredibly unhealthy recipe requires \SI{42}{lbs} of
	butter. Your friend in the UK wants the recipe, but he needs the amount
	of butter expressed in terms of grams. How many grams of butter is this?
\end{onyourown}

\vspace*{\stretch{1}}

%\begin{frame}{Some Common Equalities}
%	\centering
%	\scriptsize
%	\begin{tabu} to \linewidth {X *{3}{X[-1,r]@{ }X[-1,l] @{ = } X[-1,r]@{
%			}X[-1,l]}}
%		\toprule\rowfont{\bfseries}
%		Quantity & \multicolumn{4}{c}{U.S.} & \multicolumn{4}{c}{Metric	(SI)}                & \multicolumn{4}{c}{Metric - US}     \\ \midrule
%		Length   & 1 & ft & 12 & in.        & 1 & \si{\kilo\meter} & 1000 & \si{\meter}      & 2.54 & \si{\centi\meter} & 1 & in.  \\
%			 & 1 & yard & 3 & ft        & 1 & \si{\meter} & 1000 & \si{\milli\meter}     & 1 & \si{\meter} & 39.4 & in.        \\
%			 & 1 & mile & 5280 & ft     & 1 & \si{\centi\meter} & 10 & \si{\milli\meter} & 1 & \si{\kilo\meter} & 0.621 & mile \\ \midrule
%		Volume   & 1 & qt & 4 & cups        & 1 & \si{\liter} & 1000 & \si{\mL}              & 946 & \si{\mL} & 1 & qt             \\
%		         & 1 & qt & 2 & pt          & 1 & \si{\deci\liter} & 100 & \si{\mL}          & 1 & \si{\liter} & 1.06 & qt         \\
%			 & 1 & gal & 4 & qt         & 1 & \si{\mL} & 1 & \si{\centi\meter\cubed}     &                                     \\ \midrule
%		Mass     & 1 & lb & 16 & oz         & 1 & \si{\kilo\gram} & 1000 & \si{\gram}        & 1 & \si{\kilo\gram} & 2.20 & lb     \\
%			 & \multicolumn{4}{l}{}     & 1 & \si{\gram} & 1000 & \si{\milli\gram}       & 454 & \si{gram} & 1 & lb            \\ \midrule
%		Time     & \multicolumn{4}{l}{}     & 1 & h & 60 & min                               &                                     \\
%		         & \multicolumn{4}{l}{}     & 1 & min & 60 & \si{\second}                    &                                     \\
%		\bottomrule
%	\end{tabu}
%\end{frame}

\begin{frame}{Percentages}
	Percentages provide a \alert{ratio} between the number of parts to the
	whole.

	\begin{equation*}
		\si{\percent} = \frac{\text{Parts}}{\text{Whole}} \times \SI{100}{\percent}
	\end{equation*}

	As such, they use the same units for both the parts and the whole.
	\begin{description}
		\item[Example:] A food contains 18\% (by mass) fat.
			\sisetup{inter-unit-product={~}}
			\begin{equation*}
				\frac{\SI{18}{\gram~fat}}{\SI{100}{\gram~food}}
				\qquad\text{and}\qquad
				\frac{\SI{100}{\gram~food}}{\SI{18}{\gram~fat}}
			\end{equation*}
	\end{description}

	It is often \alert{convenient} to assume the part is out of 100 parts of
	the whole.
\end{frame}

\begin{frame}{Problem Solving}
	Learning to interpret problems will help your understanding of
	chemistry.

	\begin{center}
		\settowidth{\leftmargini}{\usebeamertemplate{itemize item}}
		\addtolength{\leftmargini}{\labelsep}
		\begin{tikzpicture}[every node/.style={minimum width=0.25\linewidth,
			minimum height=3em}]
			\node[draw=primary,thick,font=\sffamily\bfseries,anchor=south
			west](given) at (0,0) {Given};
			\draw[draw=primary,thick,->] (given.east) to ++(1,0)
			node[draw=primary,font=\sffamily\bfseries,anchor=west,align=center](concept) {Conceptual \\ Plan};
			\draw[draw=primary,thick,->] (concept.east) to ++(1,0)
			node[draw=primary,font=\sffamily\bfseries,anchor=west](find) {Find};
			\node[below,anchor=north
			west,align=left,font=\scriptsize\raggedright] at (given.south west) {%
				\begin{minipage}{0.25\linewidth}
					\begin{itemize}[<+(1)->]
						\item Sort provided values.
						\item Make note of units.
					\end{itemize}
				\end{minipage}
				};
			\node[below,anchor=north
			west,align=left,font=\scriptsize\raggedright] at (find.south west) {%
				\begin{minipage}{0.25\linewidth}
					\begin{itemize}[<+(1)->]
						\item What value is the question
							asking for?
						\item Determine final units.
					\end{itemize}
				\end{minipage}
				};
			\node[below,anchor=north
			west,align=left,font=\scriptsize\raggedright] at (concept.south west) {%
				\begin{minipage}{0.25\linewidth}
					\begin{itemize}[<+(1)->]
						\item Determine conversions
							factors needed.
						\item List possibly useful
							equations.
					\end{itemize}
				\end{minipage}
				};
		\end{tikzpicture}
	\end{center}

	\pause

	\begin{center}
		Make sure to check that units \alert{cancel}!
	\end{center}
\end{frame}

\clearpage

\begin{frame}[t]{Applying Dimensional Analysis}
	Greg has been treating his headache with Ibuprofen. He has already had 6
	pills today. The recommended maximum dose is \SI{100}{\milli\gram} every
	\SI{1}{h} as needed in a single day. If each pill is
	\SI{200}{\milli\gram}, can Greg have any more?\footnote{Please do not
	refer to this lecture for medical advice.}

	\note{
%	\begin{tabu} to \linewidth {>{\bfseries}X[-1,l] X}
%		Given: & \SI{100}{\milli\gram\per\hour} \\
%		       & \SI{200}{\milli\gram\per pill} \\
%		Find:  & \# of pills in 1 day \\
%		Plan:  & \SI{24}{h\per d} \\
%		       & Is the max. \# of pills $\geq$ 6?
%	\end{tabu}

	\bigskip

	\begin{equation*}
		\frac{\SI{1}{pill}}{\SI{200}{\milli\gram}} \times
		\frac{\SI{100}{\milli\gram}}{\SI{1}{\hour}} \times
		\frac{\SI{24}{\hour}}{\SI{1}{\day}} =
		\frac{\SI{12}{pills}}{\SI{1}{\day}}
	\end{equation*}

	\begin{enumerate}
		\item We want units of pills per day, so pills must be in the
			numerator.
		\item Cancel out the \si{\milli\gram} using the conversion
			factor.
		\item Cancel out the \si{\hour} using the conversion factor.
		\item Do all units cancel?
		\item Does the answer make sense?
	\end{enumerate}
	}
\end{frame}

%\mode<article>{
%	\setlength{\textwidth}{\paperwidth}
%	\pagebreak
%	\section*{Additional Practice}
%
%	\begin{enumerate}
%		\item Write the following in scientific notation:
%
%			\begin{tabu} to \linewidth {X X X X}
%				\SI{0.00000008}{\meter} & & \SI{72000}{\liter} &
%			\end{tabu}
%
%		\item Write the following in standard notation:
%
%			\begin{tabu} to \linewidth {X X X X}
%				\SI{2.0e-2}{\second} & & \SI{1.8e5}{\gram} & \\
%			\end{tabu}
%
%		\item \begin{samepage}
%			Rank the following in order from greatest to least amount:
%			\begin{center}
%				\begin{tabu} spread 10pt {X[l]
%					X[-1]{S[table-format=4]}@{ }X[-1,l]}
%					\toprule\rowfont{\bfseries} & \multicolumn{2}{l}{Amount} \\
%					\rowfont{\bfseries} Nutrient & \multicolumn{2}{l}{Recommended} \\ \midrule
%					Protein                     & 44   & \si{\gram      } \\
%					Vitamin C                   & 60   & \si{\milli\gram} \\
%					Vitamin B\textsubscript{12} & 6    & \si{\micro\gram} \\
%					Calcium                     & 1    & \si{\gram      } \\
%					Iron                        & 18   & \si{\milli\gram} \\
%					Iodine                      & 150  & \si{\micro\gram} \\
%					Sodium                      & 2400 & \si{\milli\gram} \\
%					Zinc                        & 15   & \si{\milli\gram} \\ \bottomrule
%				\end{tabu}
%			\end{center}
%			\end{samepage}
%
%		\item State the number of significant figures in each of the following measurements:
%			\begin{enumerate}
%				\item \SI{0.030}{\meter}
%				\item \SI{4.050}{\liter}
%				\item \SI{0.0008}{\gram}
%				\item \SI{2.80}{\meter}
%			\end{enumerate}
%	\end{enumerate}
%	}
%

\end{document}
