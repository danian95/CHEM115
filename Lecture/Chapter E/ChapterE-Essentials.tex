% !TEX program = xelatex
%\documentclass[notes=onlyslideswithnotes,notes=hide]{beamer}
%\documentclass[handout,notes=hide]{beamer}
\documentclass[notes=hide]{beamer}
%\documentclass[notes=show]{beamer}
%\documentclass[11pt,letterpaper]{article}
%\usepackage{beamerarticle}

\usepackage{newtxtext}
\usepackage{bucolors}
\usepackage{genchem}
\usepackage{lecture}
\usepackage{tabularx}
\usepackage{import}
\usepackage{tikz}
\usepackage{multicol}
\usetikzlibrary{tikzmark}
\usepackage{media9}
\colorlet{primary}{bumaroon}

\DeclareSIUnit\inch{in.}
\DeclareSIUnit\mile{mi.}
\DeclareSIUnit\pound{lb}

\title{Essentials}%: Units, Measurement, and Problem Solving}
%\subtitle{Chapter E}
\institute[CHEM115 Bloomsburg University]{CHEM115 --- Chemistry for the Sciences I \\ Bloomsburg University}
\author{D.A. McCurry}
\date{Fall 2021}

\begin{document}

\maketitle

\begin{frame}
	\section{What is Chemistry?}
\end{frame}

\mode<article>{%
	\begin{multicols}{2}
		\begin{enumerate}
			\item[E.] Essentials
			\item Atoms
			\item Quantum Mechanics
			\item Elements and Periodic Properties
			\item Molecules \& Compounds
			\item Chemical Bonding I
			\item Chemical Bonding II
			\item Chemical Reactions
			\item Aqueous Reactions
			\item Thermo
			\item Gas Laws
		\end{enumerate}
	\end{multicols}
}

\begin{frame}[t]
	\frametitle{We Need to Know the Language}
	\begin{itemize}
		\item Qualitative vs Quantitative
		\item Scientific Notation
		\item Dimensional Analysis
		\item Precision vs Accuracy
		\item Significant Figures
	\end{itemize}
\end{frame}

\mode<article>{%
\section{Scientific Notation}
\subsection{Why do we need a special notation?}
	We normally aren't using numbers that are easy to work with.
	\begin{mdframed}
		Atoms are very small (about \SI{0.0000000001}{\meter}) and we work with a lot of them (over
		\num{600000000000000000000000}).
	\end{mdframed}

	Losing count of zeroes is easy. There should be
	\tikzmarknode{zeroes}{23} zeroes in the example above after the 6.
	
	Let's instead just say how many zeroes we want to use:
	\begin{equation*}
		\num{600000000000000000000000} =
		6 \times 10^{\tikzmarknode{scinot}{23}}
	\end{equation*}

	\begin{tikzpicture}[remember picture,overlay]
		\draw[thick,->,shorten >=5pt,shorten <=5pt,primary]
			(zeroes.south)
			to[bend left=55] (scinot);
	\end{tikzpicture}
}

\begin{frame}[t]{Writing in Scientific Notation}
	\begin{enumerate}[<+->]
		\item Rewrite the number so that only the \alert{ones}
			place is occupied to the left of the decimal
			point.
		\item Count how many places your decimal point had to
			move. This will be your exponent.
		\item The sign of the exponent is \alert{very} important!
			\begin{itemize}[<1->]
				\item Positive ($+$) if the number is bigger than 1.
				\item Negative ($-$) if the number is smaller than 1.
			\end{itemize}
	\end{enumerate}
\end{frame}

\begin{frame}[t]{Let's try a few examples}
	\begin{enumerate}[<+->]
		\renewcommand\arraystretch{1.9}
		\item Write the following in scientific notation:

			\begin{tabularx}{\linewidth} {X X}
				\num{371039}     &
				\num{82909}       \\
				\num{40}         & 
				\num{0.0003319} \\
			\end{tabularx}

			\bigskip

		\item Write the following in standard notation:

			\begin{tabularx}{\linewidth} {X X}
				\num{3.001e-4} &
				\num{7.19e9} \\
				\num{1e-1} &
				\num{8.00e7} \\
			\end{tabularx}
	\end{enumerate}
\end{frame}

%\begin{onyourown}%[0em]
%	\label{oyo:scientificnotation}
%	\begin{enumerate}
%		\item Write the following in scientific notation:
%
%			\begin{tabularx}{\linewidth} {X X}
%				\SI{201}{\gram}     & \\
%				\SI{4330}{\milli\liter}       & \\
%				\SI{0.00100}{\second}         & \\
%			\end{tabularx}
%
%		\item Write the following in standard notation:
%
%			\begin{tabularx}{\linewidth} {X X}
%				\SI{3.107e2}{\liter} & \\
%				\SI{6.0e-2}{\gram}    & \\
%				\SI{9.0002e6}{\second}    & \\
%			\end{tabularx}
%	\end{enumerate}
%\end{onyourown}

\mode<article>{%
\subsection{Multiplication with Scientific Notation}
Solve the following:
\begin{align*}
	\num{173800} \times \num{80300000} &=
	\intertext{Instead of plugging this in right away, let's
	convert each number to scientific notation:}
	\num{1.738e5} \times \num{8.03e7} &=
	\intertext{Some rearrangement:}
	\underbrace{1.738 \times 8.03}_{13.95614} \times
	\underbrace{10^{5} \times 10^{7}}_{10^{5+7}=10^{12}} &=
	\num{13.95614e12} \\
	&= \boxed{\num{1.395614e13}}
\end{align*}

\subsection{Division with Scientific Notation}
Solve the following:
\begin{align*}
	\frac{\num{173800}}{\num{80300000}} =
	\frac{\num{1.738e5}}{\num{8.03e7}} &=
	\intertext{We can separate the like terms as in
	multiplication!}
	\underbrace{\frac{1.738}{8.03}}_{0.216438356} \times
	\underbrace{\frac{10^5}{10^7}}_{10^{5-7}=10^{-2}} &=
	\num{0.216438356e-2} \\
	&= \boxed{\num{2.16438356e-3}}
\end{align*}

\subsection{The Big Advantage of Scientific Notation}
We can \emph{estimate} the order of magnitude of an answer. This lets us
quickly check if our calculation is set up correctly.

\begin{mdframed}
	If we wanted to calculate the size of a gold atom, we might
	put together something that looks like this:
	\begin{equation*}
		\frac{19.3}{196.97} =
	\end{equation*}
	Will this give us a \emph{reasonable} answer?

	\bigskip
	
	No, we get $\frac{\num{e1}}{\num{e2}} = \num{e-1}$ which is very
	far off from \num{e-10}.
\end{mdframed}
}

\begin{frame}[t]
	\frametitle{Scientific Notation Helps with Estimations!}
	Solve the following:
	\begin{enumerate}
		\item $48013 \times 0.1391$
			\vspace{\stretch{1}}
			\mode<article>{$\approx\num{6.6786e3}$}
		\item $0.43 \div 1002$
			\vspace{\stretch{1}}
			\mode<article>{\num{4.3086e2}}
	\end{enumerate}
\end{frame}


%	\begin{onyourown}%[7em]
%	Solve the following:
%	\begin{enumerate}
%		\item $0.0758 \times 53.54$
%			\vspace{7em}
%		\item $678 \div 0.001333$
%	\end{enumerate}
%\end{onyourown}

\begin{frame}[allowframebreaks]{Numerical Prefixes}
		Some orders of magnitude are used more often than others. Rather than
		express all values in terms of scientific notation, we can use
		\alert{numerical prefixes} with our units.
	
		\begin{center}
		\begin{tabular} {l c S[table-format=13]
			S[retain-unity-mantissa=false,table-format=1e-2]}
			\toprule
			\bfseries Prefix & \bfseries Symbol &
			\multicolumn{2}{c}{\bfseries Multiplier} \\ \midrule
			tera  & \si{\tera } & 1000000000000        & 1e12 \\
			giga  & \si{\giga } & 1000000000           & 1e9 \\
			mega  & \si{\mega } & 1000000              & 1e6 \\
			kilo  & \si{\kilo } & 1000                 & 1e3 \\
			\bottomrule
		\end{tabular}
		\end{center}
	
		\begin{math}
			\textbf{\color{primary}Example:} \hfill
			\SI{2000}{\meter} = 2 \times
					\underbrace{10^3}_{\text{kilo, \si{\kilo}}}
				\si{\meter} = \SI{2}{\kilo\meter} \hfill\null
		\end{math}

		\framebreak

		There are prefixes for small numbers as well\ldots
	
		\begin{center}
		\begin{tabular} {l c
			S[table-format=1.18]
			S[retain-unity-mantissa=false,table-format=1e-2]}
			\toprule
			\bfseries Prefix & \bfseries Symbol &
			\multicolumn{2}{c}{Multiplier} \\ \midrule
			deci  & \si{\deci } & 0.1                  & 1e-1 \\
			centi & \si{\centi} & 0.01                 & 1e-2 \\
			milli & \si{\milli} & 0.001                & 1e-3 \\
			micro & \si{\micro} & 0.000001             & 1e-6 \\
			nano  & \si{\nano } & 0.000000001          & 1e-9 \\
			pico  & \si{\pico } & 0.000000000001       & 1e-12 \\
			femto & \si{\femto} & 0.000000000000001    & 1e-15 \\
			atto  & \si{\atto } & 0.000000000000000001 & 1e-18 \\
			\bottomrule
		\end{tabular}
		\end{center}
\end{frame}

\begin{frame}[t]
	\frametitle{Let's Practice}
	Complete the following unit conversions:
	\begin{enumerate}
		\item \SI{23}{\pico\mole} to \si{\mole}
			\vspace{\stretch{1}}
		\item \SI{4e7}{\second} to \si{\mega\second}
			\vspace{\stretch{1}}
		\item \SI{23}{\milli\ampere} to \si{\giga\ampere}
			\vspace{\stretch{1}}
	\end{enumerate}
\end{frame}

%\begin{onyourown}%[10em]
%	Revisit On Your Own \ref{oyo:scientificnotation} and find suitable
%	prefixes for each value. Assume all of the numbers are values of length.
%\end{onyourown}

\mode<article>{%
	\section{Problem Solving}
	\begin{itemize}
	\item Use conversion factors to find the solution to numerical problems.
	\item Understand the utility of dimensional analysis in verifying answers.
	\item Be familiar with the ``Give \textrightarrow\ Plan \textrightarrow\ Find'' method of problem solving.
	\end{itemize}
}

\begin{frame}[t]{Dimensional Analysis}
	By 12:40 PM, there will only be \SI{600}{\second} left of lecture. How many
	minutes is this?
\end{frame}

\mode<article>{%
\begin{mdframed}
	Using units as a guide to solving problems.
\end{mdframed}

\begin{itemize}
	\item We have \alert{\SI{600}{\second}}.
	\item We want \alert{time in minutes}.
	\item We need \alert{a conversion factor}.
\end{itemize}

\begin{equation*}
	\overbrace{\SI{60}{\second} =
	\SI{1}{min}}^{\mathclap{\text{Equality}}}
	\qquad\longrightarrow\qquad
	\underbrace{\frac{\SI{60}{\second}}{\SI{1}{min}}}_{\mathclap{\text{Conversion
	factor}}} = 1 \qquad \text{\emph{or}} \qquad
	\underbrace{\frac{\SI{1}{min}}{\SI{60}{\second}}}_{\mathclap{\text{Conversion
	factor}}} = 1
\end{equation*}

\begin{enumerate}
	\item We need a final unit of time, so we should start with a
		unit of time (our given value) in the \alert{numerator}.
	\item We must multiply by our conversion factor so that
		the units \alert{cancel out}.
	\item Ensure the units cancel \alert{first} before doing any
		math.
	\item If you get the correct unit at the end, calculate the
		answer.
\end{enumerate}

\begin{equation*}
	\frac{\SIunitcancel<3->{600}{\second}}{1} \visible<2->{\times
	\frac{\SI{1}{min}}{\SIunitcancel<3->{60}{\second}}}
	\visible<3->{=} \visible<4->{\num{10}}~\visible<3->{\si{min}}
\end{equation*}
}

%\begin{onyourown}%[5em]
%	A very tasty, yet incredibly unhealthy recipe requires \SI{42}{lbs} of
%	butter. Your friend in the UK wants the recipe, but he needs the amount
%	of butter expressed in terms of grams. How many grams of butter is this?
%\end{onyourown}

%\begin{frame}{Some Common Equalities}
%	\centering
%	\scriptsize
%	\begin{tabu} to \linewidth {X *{3}{X[-1,r]@{ }X[-1,l] @{ = } X[-1,r]@{
%			}X[-1,l]}}
%		\toprule\rowfont{\bfseries}
%		Quantity & \multicolumn{4}{c}{U.S.} & \multicolumn{4}{c}{Metric	(SI)}                & \multicolumn{4}{c}{Metric - US}     \\ \midrule
%		Length   & 1 & ft & 12 & in.        & 1 & \si{\kilo\meter} & 1000 & \si{\meter}      & 2.54 & \si{\centi\meter} & 1 & in.  \\
%			 & 1 & yard & 3 & ft        & 1 & \si{\meter} & 1000 & \si{\milli\meter}     & 1 & \si{\meter} & 39.4 & in.        \\
%			 & 1 & mile & 5280 & ft     & 1 & \si{\centi\meter} & 10 & \si{\milli\meter} & 1 & \si{\kilo\meter} & 0.621 & mile \\ \midrule
%		Volume   & 1 & qt & 4 & cups        & 1 & \si{\liter} & 1000 & \si{\mL}              & 946 & \si{\mL} & 1 & qt             \\
%		         & 1 & qt & 2 & pt          & 1 & \si{\deci\liter} & 100 & \si{\mL}          & 1 & \si{\liter} & 1.06 & qt         \\
%			 & 1 & gal & 4 & qt         & 1 & \si{\mL} & 1 & \si{\centi\meter\cubed}     &                                     \\ \midrule
%		Mass     & 1 & lb & 16 & oz         & 1 & \si{\kilo\gram} & 1000 & \si{\gram}        & 1 & \si{\kilo\gram} & 2.20 & lb     \\
%			 & \multicolumn{4}{l}{}     & 1 & \si{\gram} & 1000 & \si{\milli\gram}       & 454 & \si{gram} & 1 & lb            \\ \midrule
%		Time     & \multicolumn{4}{l}{}     & 1 & h & 60 & min                               &                                     \\
%		         & \multicolumn{4}{l}{}     & 1 & min & 60 & \si{\second}                    &                                     \\
%		\bottomrule
%	\end{tabu}
%\end{frame}

\begin{frame}[t]{Percentages}
	Percentages provide a \alert{ratio} between the number of parts to the
	whole.

	\begin{equation*}
		\si{\percent} = \frac{\text{Parts}}{\text{Whole}} \times \SI{100}{\percent}
	\end{equation*}

	As such, they use the same units for both the parts and the whole.
	\begin{example}
		Ground beef in the grocery store is often labeled with the
		percentage of fat content, e.g. 80/20. If you purchase
		\SI{2}{\pound} of ground beef that contains \SI{20}{\percent}
		fat, how much beef are you actually getting?
	\end{example}
\end{frame}

\mode<article>{%
It is often \alert{convenient} to assume the part is out of 100 parts of the
whole.

	Learning to interpret problems will help your understanding of
	chemistry.

	\begin{center}
		\settowidth{\leftmargini}{\usebeamertemplate{itemize item}}
		\addtolength{\leftmargini}{\labelsep}
		\begin{tikzpicture}[every node/.style={minimum width=0.25\linewidth,
			minimum height=3em}]
			\node[draw=primary,thick,font=\sffamily\bfseries,anchor=south
			west](given) at (0,0) {Given};
			\draw[draw=primary,thick,->] (given.east) to ++(1,0)
			node[draw=primary,font=\sffamily\bfseries,anchor=west,align=center](concept) {Conceptual \\ Plan};
			\draw[draw=primary,thick,->] (concept.east) to ++(1,0)
			node[draw=primary,font=\sffamily\bfseries,anchor=west](find) {Find};
			\node[below,anchor=north
			west,align=left,font=\scriptsize\raggedright] at (given.south west) {%
				\begin{minipage}{0.25\linewidth}
					\begin{itemize}[<+(1)->]
						\item Sort provided values.
						\item Make note of units.
					\end{itemize}
				\end{minipage}
				};
			\node[below,anchor=north
			west,align=left,font=\scriptsize\raggedright] at (find.south west) {%
				\begin{minipage}{0.25\linewidth}
					\begin{itemize}[<+(1)->]
						\item What value is the question
							asking for?
						\item Determine final units.
					\end{itemize}
				\end{minipage}
				};
			\node[below,anchor=north
			west,align=left,font=\scriptsize\raggedright] at (concept.south west) {%
				\begin{minipage}{0.25\linewidth}
					\begin{itemize}[<+(1)->]
						\item Determine conversions
							factors needed.
						\item List possibly useful
							equations.
					\end{itemize}
				\end{minipage}
				};
		\end{tikzpicture}
	\end{center}

	\begin{center}
		Make sure to check that units \alert{cancel}!
	\end{center}
}

\begin{frame}[t]{Another Dimensional Analysis Example!}
	Greg has been treating his headache with Ibuprofen. He has already had 6
	pills today. The recommended maximum dose is \SI{100}{\milli\gram} every
	\SI{1}{h} as needed in a single day. If each pill is
	\SI{200}{\milli\gram}, can Greg have any more?\footnote{Please do not
	refer to this lecture for medical advice.}

	\mode<article>{%
%	\begin{tabu} to \linewidth {>{\bfseries}X[-1,l] X}
%		Given: & \SI{100}{\milli\gram\per\hour} \\
%		       & \SI{200}{\milli\gram\per pill} \\
%		Find:  & \# of pills in 1 day \\
%		Plan:  & \SI{24}{h\per d} \\
%		       & Is the max. \# of pills $\geq$ 6?
%	\end{tabu}

	\begin{equation*}
		\frac{\SI{1}{pill}}{\SI{200}{\milli\gram}} \times
		\frac{\SI{100}{\milli\gram}}{\SI{1}{\hour}} \times
		\frac{\SI{24}{\hour}}{\SI{1}{\day}} =
		\frac{\SI{12}{pills}}{\SI{1}{\day}}
	\end{equation*}

	\begin{enumerate}
		\item We want units of pills per day, so pills must be in the
			numerator.
		\item Cancel out the \si{\milli\gram} using the conversion
			factor.
		\item Cancel out the \si{\hour} using the conversion factor.
		\item Do all units cancel?
		\item Does the answer make sense?
	\end{enumerate}
	}
\end{frame}

\begin{frame}[c]
	\frametitle{The Gimli Glider: An Exercise in Unit Conversions}
	\begin{center}
		\includegraphics[scale=0.5]{gimlix.jpg}
	\end{center}
	\footnotetext{\tiny\url{https://web.archive.org/web/20200225013028/http://www.wadenelson.com/gimli.html}}
\end{frame}

\begin{frame}{The Reliability of Measurements}
	There is often a degree of \alert{uncertainty} in our measurements.

	\medskip

	What temperature is it?

	\includegraphics[width=\linewidth]{thermometer.jpeg}

	\bigskip

	{\Large\sffamily\renewcommand\arraystretch{1.25}
		\begin{tabularx}{\linewidth}
			{*{3}{>{\centering\arraybackslash}X}}
		\SI{102.9}{\celsius} & \SI{103.0}{\celsius} &
		\SI{103.1}{\celsius} \\
		\bfseries A & \bfseries B & \bfseries C
	\end{tabularx}
}

	
%	How certain are we with these measurements?
%
%	\begin{center}
%		\includegraphics[scale=0.4,trim={0 30pt 0 0},clip]{volumescales.jpeg}
%	\end{center}
%
%	\pause
%
%	\begin{center}
%	\begin{minipage}{0.9\linewidth}
%		\centering
%		Scientific measurements are reported so that \alert{every} digit
%		is certain, except the \alert{last}.
%	\end{minipage}
%	\end{center}
\end{frame}

\begin{frame}{What defines ``good'' data?}
	As important as units are to measurements, we must also make sure that
	our measurements are accurate and precise!

	\begin{description}
		\item[Accuracy:] How close to the true value a given measurement
		  is.
	  \item[Precision:] How well a number of independent measurements
		  agree with each other.
	  \item[Uncertainty:] The degree to which one is confident in the
			measurement.
	\end{description}
\end{frame}

\begin{frame}{A Graphical Approach to Precision and Accuracy}
	\centering
	\mode<article>{includegraphics[scale=0.9]{dartboard.png}}

	\includegraphics[scale=0.45]{accuracyprecision.jpeg}
\end{frame}

\begin{frame}{An Example\ldots}
	Three students weigh the same unknown mass 3 times and record the
	following values:

	\begin{center}
		\sisetup{table-align-text-post=false}
		\begin{tabular}
			{>{\bfseries}l*{2}{S<{\,\si{\kilo\gram}}}S[table-format=4.1]<{\,\si{\kilo\gram}}}
			\toprule
			& \multicolumn{1}{c}{\bfseries Student A} &
			\multicolumn{1}{c}{\bfseries Student B} &
			\multicolumn{1}{c}{\bfseries Student C} \\ \midrule
			Trial 1 & 988.5 & 999.3 & 872.1 \\
			Trial 2 & 989.4 & 999.4 & 1150.3 \\
			Trial 3 & 987.0 & 999.6 & 976.5 \\ \midrule
			Average & 988.3 & 999.4 & 1000.3 \\
			\bottomrule
		\end{tabular}
	\end{center}
	\mode<article>{
		\begin{description}
			\item[Accuracy:] C > B > A
			\item[Precision:] B > A > C
		\end{description}
	}

	Assuming the mass actually weighs \SI{1000.0}{\kilo\gram}, rank the students in terms of accuracy
	and precision.
\end{frame}

\begin{frame}<presentation>[c]
	\frametitle{Why is reliability important?}
	\includegraphics[width=\textwidth,trim={0.75in 4.25in 0.75in 1.5in},clip]{COVIDsensor.pdf}
\end{frame}

\begin{frame}{Types of Error}
	\begin{description}
		\item[Systematic:] Measurements that are \alert{consistently} too
			high or too low.
		\item[Random:] Measurements that have equal probability of being
			too high or too low.
		\item[Gross:] You did something very wrong\ldots
	\end{description}

	\mode<presentation>{%
		\begin{center}
			\includemedia[%
				width=0.6\linewidth,
				activate=pageopen,
				addresource=OJ.mp4,
				flashvars={source=OJ.mp4&loop=true}
			]{\includegraphics[width=\linewidth]{OJ.png}}{VPlayer.swf}
		\end{center}
	}
\end{frame}

% Ended here on 2020-08-19

\begin{frame}{How precise can we be with our measurements?}
	\begin{block}{Significant Figures}
		The total number of digits recorded for a measurement or
		calculated quantity. All digits but the last are certain.
	\end{block}

	\begin{itemize}
		\item Reporting of significant figures in an answer is dependent
			on the \alert{precision} of the \alert{measured} values.
		\item The last digit is an estimate. An error of plus or minus
			one (\num{\pm 1}) is assumed.
		\item Exact numbers effectively have an infinite number of
			significant figures.
	\end{itemize}

	\begin{center}
		\bfseries\color{primary} Is the answer 8.89 or 8.888888889?
	\end{center}
\end{frame}

\begin{frame}{Rules for Significant Figures (SF)}
	\begin{enumerate}[<+->]
		\item All nonzero values are significant.
		\item Zeroes between nonzero digits are significant.
		\item Zeroes at the beginning of a number are not significant;
			they act only to locate the decimal point.
		\item Zeroes at the end of a number and after the decimal point
			are always significant.
		\item Zeroes at the end of a number and before the decimal point
			may or may not be significant.
		\item \alert{Exact numbers}, such as conversions between units or
			counting, have \alert{infinite} significant figures (SF
			rules do not apply).
	\end{enumerate}
\end{frame}

% Ended here on 1-24-2020

\begin{frame}{Significant Figures in Calculations}
	\begin{description}
		\item<+->[Multiplication or Division:] The answer can't have more
			\alert{significant figures} than either of the original numbers.
			\mode<article>{
				\begin{align*}
					110.5 \times 0.048 = 5.304 &\approx \\
					\frac{1008.02}{0.012} = 84001.667
					&\approx
				\end{align*}}

			\mode<presentation>{\bigskip}
		\item<+->[Addition or Subtraction:] The answer can't have more
		digits to the \alert{right} of the \alert{decimal point} than
			either of the original numbers.
			\mode<article>{
				\begin{center}
				\begin{tabular} {c r@{.}l}
					& 25 & 2 \\
					+ & 1 & 34 \\ \midrule
					& 26 & 54 \\
					$\approx$
				\end{tabular}
					\qquad
				\begin{tabular} {c r@{.}l}
					& 235 & 05 \\
					+ & 19 & 6 \\
					+ & 2 \\ \midrule
					& 256 & 65 \\
					$\approx$
				\end{tabular}
				\end{center}}
	\end{description}

	\mode<presentation>{\bigskip}

	\visible<+->{
	\begin{center}
		What happens to the non-significant figures?
	\end{center}}
\end{frame}

\begin{frame}{Rounding Numbers}
	\begin{enumerate}
		\item<1-> If the first digit you remove is less than 5, round
			down by dropping it and all following digits.
		\item<2-> If the first digit you remove is 5 or greater, round
			up by adding 1 to the digit on the left.
	\end{enumerate}

	\visible<3->{
		\begin{example}
		What are each of the following numbers to 3 sig figs?
		\begin{multicols}{2}
		\begin{itemize}
			\item 0.01273 
			\item 182.145
			\item 981.83 
			\item 10.7
			\item 0.000218
			\item 4.000005
		\end{itemize}
	\end{multicols}
	\end{example}
		}
\end{frame}

%\begin{onyourown}%{0em}
%	What are each of the following numbers to 4 sig figs?
%	\begin{center}
%		\renewcommand\arraystretch{2}
%		\begin{tabularx}{\linewidth} {*{4}{X}}
%			103.101 & & 10.308 & \\
%			0.0188 & & 11.205 & \\
%			0.000111112 & & 6.000009 & \\
%		\end{tabularx}
%	\end{center}
%\end{onyourown}

\begin{frame}[allowframebreaks]{Additional Practice}
	\begin{enumerate}
		\item Write the following in scientific notation:
			\begin{enumerate}[a.]
				\item \SI{0.00000008}{\meter}
				\item \SI{72000}{\liter}
			\end{enumerate}
		\item Write the following in standard notation:
			\begin{enumerate}[a.]
				\item \SI{2.0e-2}{\second}
				\item \SI{1.8e5}{\gram}
			\end{enumerate}
		\item State the number of significant figures in each of the
			following measurements:
			\begin{enumerate}[a.]
				\item \SI{0.030}{\meter}
				\item \SI{4.050}{\liter}
				\item \SI{0.0008}{\gram}
				\item \SI{2.80}{\meter}
			\end{enumerate}
			\framebreak
		\item Rank the following in order from greatest to least amount:
			\begin{center}
				\begin{tabular} {l>{\raggedleft\arraybackslash}p{0.7in}@{\,}l}
					\toprule 
					\bfseries Nutrient &
					\multicolumn{2}{c}{\bfseries Recommended
					Amount} \\ \midrule
					Protein                     & 44   & \si{\gram      } \\
					Vitamin C                   & 60   & \si{\milli\gram} \\
					Vitamin B\textsubscript{12} & 6    & \si{\micro\gram} \\
					Calcium                     & 1    & \si{\gram      } \\
					Iron                        & 18   & \si{\milli\gram} \\
					Iodine                      & 150  & \si{\micro\gram} \\
					Sodium                      & 2400 & \si{\milli\gram} \\
					Zinc                        & 15   & \si{\milli\gram} \\ \bottomrule
				\end{tabular}
			\end{center}

			\mode<article>{%
				\begin{enumerate}
					\item Protein
					\item Sodium
					\item Calcium
					\item Vitamin C
					\item Iron
					\item Zinc
					\item Iodine
					\item Vitamin B\textsubscript{12}
				\end{enumerate}
			}
	\end{enumerate}
\end{frame}

\end{document}
