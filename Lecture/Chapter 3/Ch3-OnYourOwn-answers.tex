% !TEX program = xelatex
\documentclass[11pt,letterpaper]{article}

\usepackage{genchem}
\usepackage{enumitem}
\usepackage[margin=1in]{geometry}
\usepackage{titling}

\usepackage{orbitalfilling}

\setallmainfonts{TeX Gyre Pagella}

\title{Chapter 3 ``On Your Own'' Solutions}

\setlist[enumerate,1]{itemsep=2em,leftmargin=0pt,label=\textbf{\Alph*.}}
\setlist[enumerate,2]{label={\arabic*.}}

\begin{document}

\begin{center}
	\bfseries
	\Large
	\thetitle
\end{center}

\begin{enumerate}
	\item The completed table:

		\begin{tabularx}{\linewidth} {c c X}
			\bfseries Element & \bfseries \ch{e^{-}} Config & \bfseries Orbital Diagram \\ \midrule
			\ch{Na} & $1s^22s^22p^63s^1$ & $1s:$\electronboth\quad $2s:$\electronboth\quad
			$2p:$\electronboth\electronboth\electronboth\quad $3s:$\electronup \\                                                                       
			\ch{F} & $1s^22s^22p^5$ & $1s:$\electronboth\quad $2s:$\electronboth\quad
			$2p:$\electronboth\electronboth\electronup\\                                                                        
                \ch{Cl} & $1s^22s^22p^63s^23p^6$ & $1s:$\electronboth\quad $2s:$\electronboth\quad
			$2p:$\electronboth\electronboth\electronboth \\
			&& $3s:$\electronboth\quad
			$3p:$\electronboth\electronboth\electronup
        \end{tabularx}                                                                                 
\item The completed table:
	
	\begin{tabularx}{\linewidth} {c c X}
			\bfseries Element & \bfseries \ch{e^{-}} Config & \bfseries Orbital Diagram \\ \midrule
			\ch{C} & $1s^22s^22p^2$ & $1s:$\electronboth\quad $2s:$\electronboth\quad
			$2p:$\electronup\electronup\electronnone \\
			\ch{S} & $1s^22s^22p^63s^23p^4$ & $1s:$\electronboth\quad $2s:$\electronboth\quad
			$2p:$\electronboth\electronboth\electronboth \\                                                                        
			&& $3s:$\electronboth\quad $3p:$\electronboth\electronup\electronup \\
                \ch{Mg} & $1s^22s^22p^63s^2$ & $1s:$\electronboth\quad $2s:$\electronboth\quad
			$2p:$\electronboth\electronboth\electronboth\quad $3s:$\electronboth
	\end{tabularx}

\item \begin{itemize}
	\item[\ch{Se}:] [\ch{Ar}]$4s^23d^{10}4p^4$
	\item[\ch{Se^{2-}:}] [\ch{Kr}]
	\item[\ch{Sr}:] [\ch{Kr}]$5s^2$
	\item[\ch{Sr^{2+}}:] [\ch{Kr}]
	\end{itemize}

\item \begin{itemize}
	\item[Si:] $Z_\text{eff} = 14 - 10 = \boxed{+4}$, could lose or gain an electron (similar to
		carbon)
	\item[Fr:] $Z_\text{eff} = 87 - 86 = \boxed{+1}$, will lose an electron
	\item[I:] $Z_\text{eff} = 53 - 46 = \boxed{+7}$, will gain an electron (remember the $d$
		orbitals were shielding!)
	\item[As:] $Z_\text{eff} = 33 - 28 = \boxed{+5}$, probably will gain
	\end{itemize}

	\clearpage

\item Rank: $\ch{K+} > \ch{Ca^{2+}} > \ch{Ga^{3+}}$
	
	\begin{tabular} {>{\bfseries}l c c c}
	        \toprule
		& \bfseries \ch{K+} & \bfseries \ch{Ca^{2+}} & \bfseries \ch{Ga^{3+}} \\
	        \midrule
		\ch{e-} config & [\ch{Ar}] & [\ch{Ar}] & [\ch{Ar}]$3d^{10}$ \\
		\# protons & 19 & 20 & 31 \\
		\# electrons & 18 & 18 & 28 \\
		$\bm{Z_\text{eff}}$ & +9 & +10 & +21 \\
	        \bottomrule
	\end{tabular}

	This is a \emph{really} difficult problem. Don't expect something like this on an exam, but
	if you can understand how $Z_\text{eff}$ was calculated above (subtract the \emph{shielding}
	electrons), you have a great grasp on effective nuclear charge!

\item The completed table (with full diagrams):

	\begin{tabularx}{\linewidth} {l c X}
		\ch{Ni^{2+}} & [\ch{Ar}]$3d^8$ & $4s:$\electronnone\quad
		$4p:$\electronnone\electronnone\electronnone\quad
		$4d:$\electronnone\electronnone\electronnone\electronnone\electronnone \\
		&& $3s:$\electronboth\quad
		$3p:$\electronboth\electronboth\electronboth\quad
		$3d:$\electronboth\electronboth\electronboth\electronup\electronup \\
		&& $2s:$\electronboth\quad $2p:$\electronboth\electronboth\electronboth \\
		&& $1s:$\electronboth\hfill \fbox{paramagnetic} \\
		\ch{Ag^{+}} & [\ch{Kr}]$5s^14d^9$ & $5s:$\electronup\quad
		$5p:$\electronnone\electronnone\electronnone\quad
		$5d:$\electronnone\electronnone\electronnone\electronnone\electronnone \\
		&& $4s:$\electronboth\quad $4p:$\electronboth\electronboth\electronboth\quad
		$4d:$\electronboth\electronboth\electronboth\electronboth\electronup \\
		&& $3s:$\electronboth\quad
		$3p:$\electronboth\electronboth\electronboth\quad
		$3d:$\electronboth\electronboth\electronboth\electronboth\electronboth \\
		&& $2s:$\electronboth\quad $2p:$\electronboth\electronboth\electronboth \\
		&& $1s:$\electronboth\hfill according to this table (the \emph{systematic} way),
		\fbox{paramagnetic} \\
		&& If you looked up the true filling diagram, the 4 and 5
		levels would probably look like this: \\
		&& $5s:$\electronnone\quad
		$5p:$\electronnone\electronnone\electronnone\quad
		$5d:$\electronnone\electronnone\electronnone\electronnone\electronnone \\
		&& $4s:$\electronboth\quad $4p:$\electronboth\electronboth\electronboth\quad
		$4d:$\electronboth\electronboth\electronboth\electronboth\electronboth \\
		&& and your final answer would be \fbox{diamagnetic}. \ch{Ag+} is one of those
		exceptions that you \emph{do not} need to know. \\
		\ch{Ca^{2+}} & [\ch{Ar}] & $3s:$\electronboth\quad
		$3p:$\electronboth\electronboth\electronboth\quad
		$3d:$\electronboth\electronboth\electronboth\electronboth\electronboth \\
		&& $2s:$\electronboth\quad $2p:$\electronboth\electronboth\electronboth \\
		&& $1s:$\electronboth \hfill \fbox{diamagnetic}
        \end{tabularx}
\end{enumerate}

\end{document}
