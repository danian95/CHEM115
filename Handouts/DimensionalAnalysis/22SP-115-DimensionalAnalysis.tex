\documentclass{exam}

\usepackage{newtxtext}
\usepackage[letterpaper,margin=1in]{geometry}
\usepackage{mathtools}
\usepackage{amssymb}
\usepackage{siunitx}

\DeclareSIUnit{\mile}{mi.}

\title{Dimensional Analysis}
\author{D.A.\ McCurry}

\begin{document}

\textbf{Dimensional analysis} is a technique we use quite often in chemistry
where we use the units (or ``dimensions'') to lead our calculations (or
``analysis''). In other words, we do not need to know \emph{how} to solve a
problem; we just need to know how to cancel out units.\footnote{Of course,
knowing what you're solving for does help and is important, but the point is
that we don't \emph{need} to know.}

We're going to start with a few simple examples and work our way up to something
challenging.

\begin{questions}
	\question
	Assume you're driving on I-80 at \SI{65}{\mile\per\hour}. How long will
	it take you to get to 

\end{questions}

\end{document}
